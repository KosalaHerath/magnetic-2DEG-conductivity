% ****** Start of file response.tex ****** %
\documentclass{article}

%===============================================================================
% Import packages
%===============================================================================

\usepackage[top=0.8in,bottom=1in,left=1in,right=1in]{geometry}
% Physics package
\usepackage{physics}
% Paragraph skip package
\usepackage[parfill]{parskip}
% AMS packages
\usepackage{amsmath}
\usepackage{amssymb}
\usepackage{mathtools}
% Color package
\usepackage[dvipsnames]{xcolor}
% Acronym package
\usepackage[acronym]{glossaries}
% SI units package
\usepackage{siunitx}
% Hypertext package
\usepackage[hidelinks]{hyperref}
% Bold math
\usepackage{bm}
% Other packages
\usepackage{graphicx}
\usepackage{textcomp}
\usepackage{float}
\usepackage{cite}
\usepackage{multicol}
\usepackage{multirow}
% \usepackage{kpfonts}

% Bibliography style
\usepackage[numbers,sort&compress]{natbib}
\bibliographystyle{apsrev4-1}

\begin{document}

% Define acronyms
\newacronym{tls}{TLS}{two-level system}

%===============================================================================
% Add date
%===============================================================================
\today\\

Ashot Melikyan,\\
Associate Editor,\\
Physical Review B.\\

Dear Professor Melikyan,

Thank you very much for your effort in managing the review process of our manuscript. We are also thankful for the reviewer comments, and believe the second version of the manuscript we submit herewith has been significantly improved by the constructive criticism we received. Additionally, the subsequent sections of this document discuss the reviewer comments and our responses to them.

Please note that in the following sections, the statements in {\color{RoyalBlue} \textbf{blue}} are the comments of the reviewers. Our responses are shown in black letters, and the modifications we have done to the manuscript are given in {\color{Red} red}.

\subsection*{General changes to the manuscript}


We have made minor changes in language and presentation to improve clarity, and to match the rest of the manuscript better to the changes done to address the reviewers' comments.
\begin{itemize}
    \item Section I - seventh paragraph (page 2): \\
        {\color{Red} In Sec. VII, we discuss the physical significance of our theoretical results and their possible employments in future nanoelectronic devices.
        Finally, we summarize our findings and present our conclusions in Sec. VIII.}
    \item Section VI - fourth paragraph (page 8): \\
        {\color{Red} Considering the effects of the applied dressing field on the longitudinal conductivity of 2DEG, we can identify that the dressing field has sharpened the conductivity peaks.}
    \item Section VIII - first paragraph (page 10): \\
        {\color{Red} Finally, we derived analytical expressions for the diagonal components of electric conductivity tensor concerning the 2DEG quantum Hall system under low temperatures.}
\end{itemize}

\subsection*{Response to the comments of Reviewer 1}

We would like to thank the reviewer for bringing the deficiencies of our manuscript to our attention and providing constructive feedback to improve the quality of our work. We have considered all of your suggestions seriously and revised our manuscript as described below.

\subsubsection*{Comment 1 -
\color{RoyalBlue} My concern is that the manuscript is heavily skewed towards a purely mathematical formulation of the problem. It has a minimal connection to realistic two-dimensional electron systems. The manuscript does not discuss how the results can be applied to understanding mechanisms of charge transport in nanoelectronic devices and can be used to optimize device performance. Without such discussion, the manuscript will have a minimal impact on the community working on developing nanoelectronics.}

We agree that our work has a theoretical bias, and the reader may benefit from some application perspective from our theoretical results to characterize or design nanoelectronic devices. Therefore, we have discussed the physical significance of our theoretical results and their possible use in optimizing nanoelectronic device performance. First, we added a detailed comparison between our theoretical results and experiment observation when no radiation is present.  We also added a new section (Section VII) on this very aspect.
,

\begin{itemize}
  \item Section VI - third paragraph (page 8):\\
  {\color{Red}
  By comparing the theoretical \cite{ando72,ando74_1,ando74_2,ando74_3,ando74_4,ando82,endo09} and experimental \cite{endo09,wakabayashi78,ochiai190,mancoff96,arapov02,grbic04,caviglia10} studies on the magnetoresistance of 2DEG quantum Hall systems when no radiation is present against our results, we can identify that longitudinal conductivity oscillations in Fig. 4 are a repetition of the Shubnikov–de Haas(SdH) oscillations.
  As observed in the experimental work of Caviglia \textit{et al.} \cite{caviglia10}, the SdH oscillations period depends only on the perpendicular component of the magnetic field to the plane of the 2DEG.
  Therefore, we can identify that many experiments done in low-temperatures \cite{endo09,ochiai190,mancoff96,arapov02,grbic04,caviglia10} have analyzed these SdH oscillations against the applied magnetic field amplitude for different types of 2DEGs.
  The cyclotron frequency ($\omega_0$) of the system depends on the magnetic field's perpendicular component to the plane of the 2DEG.
  Therefore we can achieve the same oscillatory behavior for longitudinal conductivity in 2DEG quantum Hall system by changing the gate voltage.
  The applied gate voltage to the system modifies the Fermi level of the system.
  Since the Landau level energy is only dependent on the cyclotron frequency, this gate voltage chnages generates SdH oscillations.
  This oscillatory behavior with applied gate voltage has been observed in an  experiment done by Wakabayashi \textit{et al.} \cite{wakabayashi78} in an inversion layer on a silicon surface at low temperature.
  By comparing these observations with our results in Fig. 4, we can identify that our oscillations also show the same characteristic behavior; i.e  $\hbar\omega_0$ periodic SdH oscillations against the Fermi energy.
  Furthermore, when we increase the gate voltage, the Fermi level rises. As illustrated in Fig. 4, this will result in a higher conductivity peak value at the higher-order Landau levels. This same behavior was also observed in the experimental observations presented in Ref. \cite{wakabayashi78}.
  In our work, we provide analytical results describing the controllability of these conductivity regions using a dressing field.
  }
  \item Section VII (page 9):\\
  {\color{Red}
  \subsection*{VII. PHYSICAL SIGNIFICANCE OF THE OUTCOMES}

  With the realization of 2DEGs in Si-MOSFETs (Metal Oxide Semiconductor Field Effect Transistors) \citep{fowler66}, Klitzing \textit{et al.} \cite{klitzing80} made the first transport measurements on such systems to reveal the quantum Hall effect. The empirical discovery of these unusual properties marked the beginning of a whole new realm in condensed matter physics that continues to produce phenomenal advancements in electronic systems. The quantum Hall effects in a 2DEG under a static magnetic field are described by plateaus quantized to integer values of the conductivity quantum ($\flatfrac{e^2}{\hbar}$) in the off-diagonal conductivity, with simultaneously peaks at inter-plateau transition for the diagonal conductivity \cite{endo09}. This is due to the applied magnetic field and it changes the energy spectrum of 2DEG dramatically. The magnetic field causes the density of states in 2DEG to split up into a sequence of delta functions, separated by an energy $\hbar\omega_0$, with $\omega_0$ the cyclotron frequency which depends on the applied magnetic field.
  However, experimental results demonstrate that these Landau levels are broadened and the main source of these broadening in low-temperatures is the disorders in materials \cite{ando85,dial07}. The broaden sequence of delta functions of density of states implies the oscillating behavior in the experimental measurements of longitudinal conductivity which is known as SdH oscillations. \cite{endo09,wakabayashi78}.

  Our theoretical analysis on longitudinal conductivity behavior of dressed quantum Hall system developed by considering low-temperature limit with gaussian impurity broadening assumptions.
  As illustrated in Fig. 4, we can describe the experimentally observed SdH oscillation results in Refs.\cite{endo09,wakabayashi78} through our model.
  Under the undressed condition, our results overlap with the conductivity measurement of quantum Hall systems \cite{endo09}. Strikingly, we show in Fig. 5 that we can manipulate the broadening of these conductivity peaks using an external dressing field. In low-temperatures, the principal cause of broadening of these conductivity peaks is impurity scattering and using an external dressing field we can suppress the scattering which results in shrinkage of both the scattering-induced broadening and the longitudinal conductivity peaks.

  Research on novel states of matter has driven the evolution of present-day nanoelectronic devices. In particular, controllable manipulation of material properties through a gate electric field has revolutionized the development of material science and technology \cite{ahn03,deng18}.
  The charge carrier concentration of a system is an imperative parameter that defines its conductivity properties.
  As we can manipulate the charge carrier concentration using an electrostatic field-effect, we exert significant control over the conductivity externally.
  A 2DEG under static magnetic field with quantum Hall effects is an excellent example that illustrates how the gate electric field controls conductivity.
  We also observed that a considerable number of studies exists on charge transport in the quantum limit that employed different types of 2D field-effect transistors (FETs) in magnetic fields \cite{wakabayashi78,yang18,long20,li14}. Yang \textit{et al.} \cite{yang18} study shows quantized Hall plateaus and SdH oscillations for longitudinal conductivity against gate voltage in black phosphorus FET under static magnetic fields in low-temperatures. Since the Fermi level of a system can be altered with the applied gate voltage, our Fig. 4 depicts this result.
  In particular, our analysisshoes that we can manipulate the broadening of the conductivity regions using an external dressing field. Although Yang \textit{et al.} \cite{yang18}, achieved broadening in longitudinal conductivity peaks by changing the temperature in a low range, in this study we presented a general theory on manipulating longitudinal conductivity broadening using a high-intensity electromagnetic field.

  The realization of the underlying mechanism of 2D FETs in the quantum realm promises its potential in next-generation nanoelectronic applications. In a particular application that uses the switching mechanism of the above-discussed FETs with quantum Hall effects, we can achieve high and low output conductivities by changing the input gate voltage. As a result of manipulating the broadening of conductivity regions, we can limit the broadening of conductivity peaks around Landau levels using a dressing high-intensity electromagnetic field. This will enhance the sensitivity of FETs which provides the ability to observe narrow changes in gate voltage.
  Furthermore, adopting the mechanism presented in Ref. \cite{hirakawa01}, we can manipulate conductivity peaks into very sensitive, narrowband high-frequency radiation detectors.
  We envision these advances in nanoelectronics from our theoretical model and knowledge generated from the insight it provides to the underlying dependencies.
  Furthermore, this theoretical model will aid for development of simulation tools that will design the quantum effects in magnetotransport properties of 2D nanostructures.
  }
\end{itemize}

\subsubsection*{Comment 2 -
\color{RoyalBlue} Moreover, this current research direction has a significant overlap with previous experimental and theoretical studies of quantum Hall systems that started with the observation of zero-resistance states in high mobility systems [Zudov et al, Phys. Rev. B 64, 201311 (2001), Mani et al, Phys. Rev. Lett. 92, 146801 (2004)] and gave rise to theoretical models for the phenomenon [Durst et al, Phys. Rev. Lett. 91, 086803 (2003), Dmitriev et al., Phys. Rev. B 71, 115316 (2005), Dmitriev et al, Phys. Rev. B 80, 165327 (2009)]. The present manuscript needs to connect to various known phenomena discussed earlier in the literature on quantum Hall systems.}

We agree with the reviewer that it is useful to have a discussion comparing  our theoretical model and previous work on transport properties of quantum Hall systems.
Among the mentioned studies we can identify that experimental work reported in Refs. \cite{zudov01,mani02,zudov03,mani04} are specifically aimed at the unusual oscillations of the magnetoresistance induced by the microwave (millimeter-wave) radiation in 2DEG quantum Hall systems.
These oscillations are known as \textit{microwave-induced resistance oscillations} (MIROs).
To describe these behaviors Durst \textit{et al.} \cite{durst03} introduced a simple theoretical model assuming that the experimentally observed oscillations are a consequence of photoexcited disorder-scattered electrons.
However, later a novel model was proposed in Refs. \cite{dmitriev03,dmitriev05,dmitriev09} considering the changes of the electron distribution function done by the microwave field.
These more generalized models \cite{dmitriev03,dmitriev05,dmitriev09} have successfully described the behavior of MIROs at experimentally relevant temperatures that were missed in the previous model \cite{durst03}.
Furthermore, we can recognize that the underlying mechanism of all these models \cite{durst03,dmitriev03,dmitriev05,dmitriev09} is microwave photon absorption by an electron in the associated system. In our theoretical model, we take into account higher frequencies than microwaves as the dressing field.
Therefore we can identify and quantify significant dissimilarities between our system and MIROs systems \cite{zudov01,mani02,zudov03,mani04,durst03,dmitriev03,dmitriev05,dmitriev09}. They are:
\begin{itemize}
  \item
  in the experiments on magnetoresistance oscillations on 2DEG quantum Hall systems \cite{zudov01,mani02,zudov03,mani04} were performed in the microwave frequency range ($30 -\SI{150}{\giga\hertz}$). This leads to building the theoretical models presented in Refs. \cite{durst03,dmitriev03} by assuming that these oscillations are caused by photoexcited electrons. Since these models consider on relevant frequency range (microwave radiation), it will allow us to acknowledge the photon absorption by electrons. In contrast to that, our consideration is only focused on the systems with high-frequency range dressing fields which will not be associated with any photon absorption.
  \item
  the applied microwave radiation power on the MIROs experiments \cite{mani02,zudov03} are varies around the  $\SI{1}{\milli\watt\per\square\centi\metre}$ range. However, in our analysis, we take the dressing field as a high-intensity electromagnetic field, where we can not consider the dressing field as a perturbation in our analysis. This leads us to recognize the formation of Floquet states together with conventional Landau levels. As we mentioned in our results, we used high-intensity dressing fields of magnitude around the $\SI{100}{\watt\per\square\centi\metre}$ range in our numerical calculations.
  \item
  these fascinating MIROs are only observed under the influence of weak magnetic fields ($B < \SI{0.2}{\tesla}$) in experiments performed in Refs. \cite{zudov01,mani02,zudov03,mani04}. In this range of weak magnetic fields,  we can only observe MIROs as Shubnikov–de Haas (SdH) oscillations reveal only on high-intensity magnetic fields. In comparison to our analysis, we are  interested in the SdH oscillations and manipulation of their characteristics. Therefore in our analysis, we aim at the effects induced by high-intensity magnetic fields ($B \sim \SI{1}{\tesla}$).
  \item
  in our work, we analyzed the 2DEG quantum Hall system with a dressing high-intensity electromagnetic field which does not contribute to a energy exchange between the high-frequency dressing field and electrons. Therefore we have assumed the applied electromagnetic radiation as a pure dressing field. There are two possible absorption mechanisms in a 2DEG quantum Hall system; namely electron transitions between distinct Landau levels, and electron transitions between distinct states of the same broadened Landau level. To avoid these absorptions, we have tuned the dressing field into a high-frequency range in our system under the analysis. Furthermore, due to the high-intensity dressing field, the entire electronic states of the conventional 2DEG quantum Hall system will be restructured. We have addressed these modifications through the Floquet theory. However, the MIROs models \cite{durst03,dmitriev03} are based on low-frequency cases where the system is able to absorb low-frequency photons from the field.
\end{itemize}

Based on the above discussion, we can identify that the high-frequency and low-frequency illumination of 2DEG quantum Hall systems leads to two distinguishable magnetotransport scenarios.

\begin{itemize}
  \item Section I - fifth paragraph (page 1):\\
  {\color{Red}
  Recently, experimental research on the effects of microwave illumination of 2DEG systems revealed microwave-induced resistance oscillations (MIRO) under weak magnetic fields \cite{zudov01,mani02,zudov03,mani04}.
  This inspired investigations on theoretical description of MIROs, and several semiclassical and quantum kinetic equation formalisms have been proposed to address the underlying mechanism of MIROs \cite{durst03,dmitriev03,dmitriev05,dmitriev09}. These formalisms provide a proper explanation for the experimental observations of MIROs. However, these experimental and theoretical works have been linked to photon absorption from low-frequency (microwave) fields.
  In contrast to that, high-frequency external illumination on a 2DEG quantum Hall system needs to be studied as a pure dressing (nonabsorbable) field.
  The influence of induced by a purely dressing field on magnetotransport properties of 2DEG quantum Hall system need to be described by a non absorption mechanism, and it has escaped the researchers’ attention before.
  Lately, Dini \textit{et al.} \cite{dini16} have investigated the one-directional conductivity behavior of dressed quantum Hall system subjected to a high-frequency field.
  }
  \item Section VI - sixth paragraph (page 9):\\
  {\color{Red}
  Lately, several experimental \cite{zudov01,mani02,zudov03,mani04} and theoretical \cite{durst03,dmitriev03,dmitriev05,dmitriev09} studies have uncovered various remarkable magnetotransport properties induced by microwave radiation on 2DEG quantum Hall systems. However, these experiments and theoretical models only examined the behavior of MIROs in 2DEG systems which are based on photon absorption from the applied low-frequency field.
  It is important to state that the difference between our SdH oscillations and MIROs \cite{zudov01,mani02,zudov03,mani04} by considering the frequency range of the applied electromagnetic field. In our case, the applied dressing field is in the off-resonant range, and it will only change the broadening of Landau levels, but avoid any contribution towards the photon absorptions. This will clearly describe the crucial difference between low-frequency illumination and high-frequency illumination effects on 2DEG quantum systems. Therefore our theoretical analysis can help to fill the gap in knowledge of the high-frequency dressing field effects on 2DEG quantum Hall systems.
  }
\end{itemize}

\subsubsection*{Comment 3 -
\color{RoyalBlue} The manuscript will also provide more impact if it demonstrates how the new results can help to improve the future development of nanoelectronic devices. After these questions are addressed, the manuscript will be suitable for publication in Physical Review B. Otherwise, it will fit better to a more mathematically oriented journal.
}

This is a valid comment. As we mentioned in comment 1, considering the importance of discussing the physical significance of our results, we included a new section (Section VII) to address this comment.

\begin{itemize}
  \item Section VII (page 8):\\
  {\color{Red}
  \subsection*{VII. PHYSICAL SIGNIFICANCE OF THE OUTCOMES}

  With the realization of 2DEGs in Si-MOSFETs (Metal Oxide Semiconductor Field Effect Transistors) \citep{fowler66}, Klitzing \textit{et al.} \cite{klitzing80} made the first transport measurements on such systems to reveal the quantum Hall effect. The empirical discovery of these unusual properties marked the beginning of a whole new realm in condensed matter physics that continues to produce phenomenal advancements in electronic systems. The quantum Hall effects in a 2DEG under a static magnetic field are described by plateaus quantized to integer values of the conductivity quantum ($\flatfrac{e^2}{\hbar}$) in the off-diagonal conductivity, with simultaneously peaks at inter-plateau transition for the diagonal conductivity \cite{endo09}. This is due to the applied magnetic field and it changes the energy spectrum of 2DEG dramatically. The magnetic field causes the density of states in 2DEG to split up into a sequence of delta functions, separated by an energy $\hbar\omega_0$, with $\omega_0$ the cyclotron frequency which depends on the applied magnetic field.
  However, experimental results demonstrate that these Landau levels are broadened and the main source of these broadening in low-temperatures is the disorders in materials \cite{ando85,dial07}. The broaden sequence of delta functions of density of states implies the oscillating behavior in the experimental measurements of longitudinal conductivity which is known as SdH oscillations. \cite{endo09,wakabayashi78}.

  Our theoretical analysis on longitudinal conductivity behavior of dressed quantum Hall system developed by considering low-temperature limit with gaussian impurity broadening assumptions.
  As illustrated in Fig. 4, we can describe the experimentally observed SdH oscillation results in Refs.\cite{endo09,wakabayashi78} through our model.
  Under the undressed condition, our results overlap with the conductivity measurement of quantum Hall systems \cite{endo09}. Strikingly, we show in Fig. 5 that we can manipulate the broadening of these conductivity peaks using an external dressing field. In low-temperatures, the principal cause of broadening of these conductivity peaks is impurity scattering and using an external dressing field we can suppress the scattering which results in shrinkage of both the scattering-induced broadening and the longitudinal conductivity peaks.

  Research on novel states of matter has driven the evolution of present-day nanoelectronic devices. In particular, controllable manipulation of material properties through a gate electric field has revolutionized the development of material science and technology \cite{ahn03,deng18}.
  The charge carrier concentration of a system is an imperative parameter that defines its conductivity properties.
  As we can manipulate the charge carrier concentration using an electrostatic field-effect, we exert significant control over the conductivity externally.
  A 2DEG under static magnetic field with quantum Hall effects is an excellent example that illustrates how the gate electric field controls conductivity.
  We also observed that a considerable number of studies exists on charge transport in the quantum limit that employed different types of 2D field-effect transistors (FETs) in magnetic fields \cite{wakabayashi78,yang18,long20,li14}. Yang \textit{et al.} \cite{yang18} study shows quantized Hall plateaus and SdH oscillations for longitudinal conductivity against gate voltage in black phosphorus FET under static magnetic fields in low-temperatures. Since the Fermi level of a system can be altered with the applied gate voltage, our Fig. 4 depicts this result.
  In particular, our analysisshoes that we can manipulate the broadening of the conductivity regions using an external dressing field. Although Yang \textit{et al.} \cite{yang18}, achieved broadening in longitudinal conductivity peaks by changing the temperature in a low range, in this study we presented a general theory on manipulating longitudinal conductivity broadening using a high-intensity electromagnetic field.

  The realization of the underlying mechanism of 2D FETs in the quantum realm promises its potential in next-generation nanoelectronic applications. In a particular application that uses the switching mechanism of the above-discussed FETs with quantum Hall effects, we can achieve high and low output conductivities by changing the input gate voltage. As a result of manipulating the broadening of conductivity regions, we can limit the broadening of conductivity peaks around Landau levels using a dressing high-intensity electromagnetic field. This will enhance the sensitivity of FETs which provides the ability to observe narrow changes in gate voltage.
  Furthermore, adopting the mechanism presented in Ref. \cite{hirakawa01}, we can manipulate conductivity peaks into very sensitive, narrowband high-frequency radiation detectors.
  We envision these advances in nanoelectronics from our theoretical model and knowledge generated from the insight it provides to the underlying dependencies.
  Furthermore, this theoretical model will aid for development of simulation tools that will design the quantum effects in magnetotransport properties of 2D nanostructures.
  }
\end{itemize}

\subsubsection*{Comment 4 -
\color{RoyalBlue} The quantum Quantum Hall effect requires high mobility samples. In these samples, the structure of the disorder is usually complicated and combines both short-length potentials of impurities and long-length electrostatic inhomogeneities. The interplay of these components of disorder opens exciting questions about the transport properties of 2DEGs. What is the structure of disorder considered in the present manuscript and hidden in the notations for Vimp? What are the conditions for validity of eq. (15)?
}

We thank the reviewer for raising this important question. In our previous manuscript, we have presented the detailed derivation of Eq.~[15] with a discussion on models of disorder under Appendix C. However, here we again elucidate our disorder model and approximations made to derive the Eq.~[15] for clarity and to respond to this question.

We modelled the effect caused by impurities in the considered system as a single perturbation potential. Analyzing the electric properties for a specific impurity configuration is a rather formidable task and is not examined in this work since it is unlikely to have exactly the evaluated impurity configuration in an experiment. Therefore, in this study, we consider the statistically averaged properties of 2DEG over impurity configurations.
In addition, we consider a static disorder corresponds only to elastic electron scattering processes.
First, we adapt the Edwards model \cite{akkermans10} to represent the randomly distributed impurities over the associated system and we approximate this into a Gaussian white noise.

Since we are presenting the perturbation potential $V(\vb{r})$ by a group of randomly localized impurities, we assume $N_{imp}$ number of identical single impurity potentials distributed randomly but in fixed positions $\vb{r}_i$. Thus, we can describe the perturbation potential $V(\vb{r})$ as the sum of uncorrelated single impurity potentials $\upsilon(\vb{r})$,
\begin{equation} \label{eq:1}
  V(\vb{r}) =
  \sum_{i=1}^{N_{imp}}
  \upsilon (\vb{r}-\vb{r}_i).
\end{equation}
Furthermore, we approximate this perturbation potential as a Gaussian random potential where one can choose the zero of energy such that the potential is zero on average. This model is characterized by the following two equations \cite{akkermans10}
\begin{subequations}
\begin{equation} \label{eq:2}
  \expval{\upsilon(\vb{r})}_{imp} =0,
\end{equation}
\begin{equation} \label{eq:3}
  \expval{\upsilon(\vb{r})\upsilon(\vb{r'})}_{imp} = \Upsilon(\vb{r}-\vb{r'}),
\end{equation}
\end{subequations}
where $\expval{\cdot}_{imp}$ represents the average over the impurity disorder and $\Upsilon(\vb{r}-\vb{r'})$ is any decaying function which solely depends only on $\vb{r}-\vb{r'}$, and decays with a characteristic length $r_c$. Since this study considers the case where the wavelength of radiation or a scattering electron is much greater than $r_c$, it is a reasonable to approximation its two-point correlation function by
\begin{equation} \label{eq:4}
  \expval{\upsilon(\vb{r})\upsilon(\vb{r'})}_{imp} = \Upsilon_{imp}^2\delta(\vb{r}-\vb{r'}),
\end{equation}
where $\Upsilon_{imp}^2$ is a positive constant. A random potential $V(\vb{r})$ with this property is called white noise \cite{akkermans10}. Then we can approximately write the total scattering potential as
\begin{equation} \label{eq:5}
  V(\vb{r}) =
  \sum_{i=1}^{N_{imp}}
  \Upsilon_{imp} \delta(\vb{r}-\vb{r}_i).
\end{equation}

Using this expression, we can evaluate the Floquet-Fermi golden rule for a dressed quantum Hall system. Furthermore, considering the properties of a Gaussian white noise impurity distribution \cite{wackerl20,akkermans10}, we assume $V_{imp}$ is a constant in the momentum space.
However, throughout the derivation, we use only the first-order contribution (Born approximation) of the impurity potential. All the detailed steps for the derivation of the Floquet-Fermi golden rule for a dressed quantum Hall system are included in Appendix C.

Since previous studies on Floquet-Drude conductivity \cite{wackerl20}, and magneto-transport properties in undressed \cite{endo09} and dressed \cite{dini16} quantum Hall systems have used this particular Gaussian white noise potential, we also selected this pertucular impurity model to describe our system. Such a choise enables us to compare our analytical results with these previous models. As you have mentioned in the comment, consideration of other impurity disorder models and their impact on the magnetotransport properties of dressed quantum Hall system would be an intriguing future research possibility.

Since we have only mentioned the validity conditions we used to derive the Eq.~[15] in Appendix C and not in the main text of the previous manuscript, we have added these validity conditions to the main text of the revised manuscript.

\begin{itemize}
  \item Section IV - first paragraph (page 4):\\
  {\color{Red}
  We have modeled the effect caused by impurities as a single short-range perturbation potential. Analyzing the electric properties for a specific impurity distribution is a rather formidable task. We do not consider a simple impurity distribution here as it is unlikely to represent a measured impurity configuration in an experiment.
  Therefore, in this study, we consider the statistically averaged properties of 2DEG over impurity configurations.
  Furthermore, we have assumed that a group of randomly distributed impurities forms our perturbation potential under the Edwards impurity model  \cite{akkermans10,wackerl20}.
  Essentially, all these assumptions means that we can write the total scattering potential in the 2DEG as a sum of uncorrelated single impurity potentials $\upsilon(\vb{r})$. This enables us to approximate the impurity potential as a Gaussian white noise \cite{akkermans10,wackerl20},
  where $\eta_{imp}$ is the number of impurities in a unit area, $V_{imp} = \expval{|V_{{k'}_x,k_x}|^2}_{imp}$ with $V_{{k'}_x,k_x} = \mel**{k'_x}{\upsilon(x) }{k_x}$, and $\braket{x}{k_x} = e^{-ik_x x}$.
  Moreover, in this analysis, $\expval{\cdot}_{imp}$ represents the average over the impurity disorder. In this derivation, we only considered the first order (the Born approximation) contribution from the impurity potential.
  }
\end{itemize}

\subsubsection*{Comment 5 -
\color{RoyalBlue} If continuous illumination is applied to an electron system, the system will heat indefinitely. The distribution function is stabilized when the electron-electron and electron-phonon scattering mechanisms are included. What are the conditions of validity for eqs. (33-34)? The deviations of the distribution function from its equilibrium result in effects that can significantly overcome the equilibrium contributions, see Dmitriev et al, Phys. Rev. B 80, 165327 (2009). However, to capture these effects, a treatment within Floquet-Drude formalism is insufficient and a complete quantum kinetic equation has to be analyzed. Can the authors argue why their results still present some interest to the community even if they potentially disregard more significant contributions?
}

This is a good, valid question. First, we discuss the validity of the Eq.~[33] and Eq.~[34]. Our analysis derived the longitudinal conductivity of a dressed quantum Hall system without any partial distribution function specification until Eq.~[31].
In contrast to the MIROs analysis \cite{dmitriev03,dmitriev05,dmitriev09}, we consider the dressed quantum Hall system under the off-resonant condition, but photon absorption does not happen. Therefore, we can select the Fermi-Dirac distribution as our particle distribution function for Eq.~[31]
\begin{equation} \tag{5}
  f(\varepsilon) = \frac{1}{\exp[(\varepsilon - \varepsilon_F)/k_B T]+1}.
\end{equation}
Here, $k_B$ is the Boltzmann constant, $T$ is the absolute temperature, and $\varepsilon_F$ is the Fermi energy of the system.
Then we consider a special scenario where the system operates at a low-temperature limit, i.e., the condition $k_BT \ll \varepsilon_F$.
Since previous theoretical studies \cite{wackerl20,dini16,endo09} on magneto-transport properties of 2DEG quantum Hall systems were derived under this assumption, we also adopted the same conditions as it enables us to compare results with those studies.
In addition, as mentioned in previous experimental studies on SdH oscillations \cite{zudov03,mani02,arapov02} in 2DEG undressed quantum Hall systems with increasing temperature, we can identify the decaying behavior of SdH oscillations.
Therefore, if we need to maintain a good oscillatory behavior, we need to operate in low temperatures.
Since in our analysis, we study the manipulation of SdH oscillations using a dressing field, it is reasonable to simplify our general derivation in given Eq.~[31] for low-temperature limit. Under these assumptions, the derivative of the Fermi-Dirac distribution is sharply peaked around the Fermi energy, and  can be represented by the delta function \cite{endo09}
\begin{equation} \tag{6}
  - \pdv{f(\varepsilon)}{\varepsilon} \approx \delta(\varepsilon - \varepsilon_F).
\end{equation}
Applying these adequate conditions, we can simplify our derivation into single expression that can be compared with previous studies \cite{wackerl20,dini16,endo09} on magnetotransport in dressed quantum Hall systems. We have added clarification on the above assumptions in the manuscript.
\begin{itemize}
  \item Section V - second paragraph (page 7):\\
  {\color{Red}
  We consider a dressed quantum Hall system under off-resonant condition, where photon absorption does not happen. Therefore, we can select the Fermi-Dirac distribution as our partial distribution function ($f$) for our system
  \begin{equation} \tag{33}
    f(\varepsilon) = \frac{1}{\exp[(\varepsilon - \varepsilon_F)/k_B T]+1},
  \end{equation}
  where $k_B$ is the Boltzmann constant, $T$ is the absolute temperature, and $\varepsilon_F$ is the Fermi energy of the system. At low-temperatures, i.e. $k_BT \ll \varepsilon_F$, the derivative of this distribution is sharply peaked around the Fermi energy, and can be approximated by a delta function \cite{endo09}
  \begin{equation} \tag{34}
    - \pdv{f(\varepsilon)}{\varepsilon} \approx \delta(\varepsilon - \varepsilon_F).
  \end{equation}
  }
\end{itemize}

Yes, indeed, experiments on quantum hall systems subject to a strong dressing field show elevated temperatures. However, as demonstrated in previous studies by Dini \textit{et al.} \cite{dini16} and Kibis \textit{et al.} \cite{kibis15}, it is possible to maintain low system temperature by limiting the dressing field to narrow pulses.

Although we have extended our general derivation given in Eq.~[31] into special consideration for comparison, we can improve our theoretical model into different types of particle distribution functions. As presented in the previous work done by Dmitriev \textit{et al.} \cite{dmitriev05}, a well-defined analytical discussion needs to address the changes in distribution functions in our derived general longitudinal conductivity formula, and this would be a noteworthy extension to our study.

\subsubsection*{Comment 6 -
\color{RoyalBlue} The authors presented the results for the conductivity tensor's xx- and yy-components. They seem to be identical, apart of a dimensional factor $(eB)^2$ in the denominator. The equality of these two components is expected for isotropic systems. Does the polarization of the electromagnetic field break the isotropy? Do the two components of the conductivity remain equal even for an arbitrary direction of a linearly polarized field? Is there a non-Hall contribution to the xy-component?
}

We express our gratitude towards the reviewer for pointing out this inadvertent oversight done in the $yy$-component of the current operator calculation. This has led to an unexpected normalization factor in the yy-component of the  conductivity expression. Since we have only used the normalized $xx$-component of the conductivity tensor for subsequent analysis, this oversight has no impact on the analysis reported in the  manuscsript. With this correction, we observe that both xx-component and $yy$-component show the same conductivity behaviors. We added these amendments to the current operator calculations done in Appendix D and the longitudinal conductivity calculation under Section V.

\begin{itemize}
  \item Appendix D (page 16):\\
  {\color{Red}
  \subsection*{\label{appendix_d}Appendix D: Current operators for a dressed quantum Hall system}

  In this section, we derive the current density operator for the $N$-th Landau level in a dressed quantum Hall system. We already found the exact solution for the time-dependent Schrödinger equation with the Hamiltonian give in Eq.~(1) and we identified them as the Floquet states in Eq.~(14). For the simplicity of notation, we can represent the Floquet modes derived in Eq.~(10) as quantum states using their corresponding quantum numbers as follows
  \begin{equation} \tag{D1}
    \ket{\phi_{n,m}} = \ket{n,k_x}.
  \end{equation}
  Using this complete set of quantum states \cite{wackerl20,holthaus15,grifoni98}, we can represent the single particle current operator's matrix element as
  \begin{equation} \tag{D2}
    \left(\vb{j} \right)_{nm,n'm'} = \mel{n,k_x}{\;\hat{\vb{j}}\;}{n',k'_x}.
  \end{equation}
  Next, we can identify the particle current operator for our system \cite{mahan00,bruus04} as
  \begin{equation} \tag{D3}
    \hat{\vb{j}} = \frac{1}{\widetilde{m}} \left\{\hat{\vb{p}} - e\left[\vb{A}_s + \vb{A}_d(t) \right]\right\},
  \end{equation}
  where $\widetilde{m}$ is the mass of the considering particle.

  First, we consider the $x$-directional particle current operator component, and we can identify that as
  \begin{equation} \tag{D4}
    \hat{j}_x = \frac{1}{\widetilde{m}} \left(-i\hbar\pdv{x} + eBy \right).
  \end{equation}
  Next, we calculate the matrix elements of $x$-directional current operator against our Floquet mode basis
  \begin{equation} \tag{D5}
    \left({j}_x \right)_{nm,n'm'} =
    \mel{n,k_x}{\;\frac{1}{\widetilde{m}} \left(-i\hbar\pdv{x} + eBy \right)}{n',k'_x},
  \end{equation}
  and we evaluate these using the Floquet modes derived in Eq.~(7) as follows
  \begin{equation} \tag{D6}
    \begin{aligned}
      \left({j}_x \right)_{nm,n'm'} = &
      \frac{1}{{\widetilde{m}}}
      \delta_{k_x,k'_x}
      \int
      \left(\hbar k'_x + eBy\right) \\
      & \times
       \chi_{n}\bm{\left(}y - y_0 - \zeta(t)\bm{\right)}
      \chi_{n'}\bm{\left(}y - y_0 - \zeta(t)\bm{\right)}
      dy.
    \end{aligned}
  \end{equation}
  Let $[y - y_0 - \zeta(t)] = \bar{y}$, and we can obtain
  \begin{equation} \tag{D7}
    \begin{aligned}
      \left({j}_x \right)_{nm,n'm'} =
      \frac{1}{{\widetilde{m}}}
      \delta_{k_x,k'_x}
      \int &
      \left[ \hbar k'_x + eB\bar{y} -\hbar k'_x + eB\zeta(t)\right]
      \chi_{n}(\bar{y})
      \chi_{n'}(\bar{y})
      d\bar{y}.
    \end{aligned}
  \end{equation}
  Using the following integral identities of the Floquet modes that are made up of Gauss-Hermite functions \cite{vedenyapin11,szego59}
  \begin{equation} \tag{D8}
    \int
    \chi_{n}({y})
    \chi_{n'}({y}) d{y} =
    \delta_{n',n},
  \end{equation}
  \begin{equation} \tag{D9}
    \int
    y \chi_{n}({y})\chi_{n'}({y}) d{y} =
    \frac{1}{\kappa}
    \left(\sqrt{\frac{n+1}{2}} \delta_{n',n+1} + \sqrt{\frac{n}{2}}
    \delta_{n',n-1} \right),
  \end{equation}
  we simplify the matrix elements of $x$-directional current operator to obtain
  \begin{equation} \tag{D10}
    \begin{aligned}
      &\left({j}_x \right)_{nm,n'm'} =
      \frac{eB}{{\widetilde{m}\kappa}}
      \delta_{k_x,k'_x}
      \left[
      \left(\sqrt{\frac{n+1}{2}} \delta_{n',n+1} + \sqrt{\frac{n}{2}}
      \delta_{n',n-1}\right)
      + \zeta(t) \delta_{n',n}
      \right].
    \end{aligned}
  \end{equation}
  Due to high complexity of extract solution, in this study we only consider the constant contribution from the Fourier series components of the above expression. Therefore, we can identify the $0$-th component of the Fourier series as
  \begin{equation} \tag{D11}
    \begin{aligned}
        \left({j}_x \right)_{nm,n'm'} =&
        \frac{eB}{\widetilde{m}\kappa}
        \delta_{k_x,k'_x}
        \left(\sqrt{\frac{n+1}{2}} \delta_{n',n+1} + \sqrt{\frac{n}{2}}
        \delta_{n',n-1} \right).
    \end{aligned}
  \end{equation}
  To calculate the electric current operator, we substitute the electron's charge and effective mass to the above derived equation. This leads to
  \begin{equation} \tag{D12}
    \begin{aligned}
        \Big({j}^x_{s=0}\Big)_{nm,n'm'}^{electron}  =&
        \frac{e\hbar}{{m_e}l_0}
        \delta_{k_x,k'_x}
        \left(\sqrt{\frac{n+1}{2}} \delta_{n',n+1} + \sqrt{\frac{n}{2}}
        \delta_{n',n-1} \right).
    \end{aligned}
  \end{equation}
  where $l_0 = \sqrt{\flatfrac{\hbar}{eB}}$ is the magnetic length.

  Moreover, we can identify the $y$-directional current operator component as
  \begin{equation} \tag{D13}
    \hat{j}_y = \frac{1}{\widetilde{m}} \left(-i\hbar\pdv{y} - \frac{eE}{\omega}\cos(\omega t) \right).
  \end{equation}
  Using this operator, we can represent the matrix elements of $y$-directional current operator in Floquet mode basis as
  \begin{equation} \tag{D14}
    \left({j}_y \right)_{nm,n'm'} =
    \mel{n,k_x}{\;\frac{-1}{\widetilde{m}} \left(i\hbar\pdv{y} + \frac{eE}{\omega}\cos(\omega t) \right)}{n',k'_x}.
  \end{equation}
  After following the same steps done for the $x$-directional current operator, and recursion relation of the first derivative of Gauss-Hermite functions
  \begin{equation} \tag{D15}
    \begin{aligned}
      \pdv{\chi_{n}({y})}{y} =
      \kappa\left[
      -  \sqrt{\frac{n+1}{2}} \chi_{n+1}({y})
      +  \sqrt{\frac{n}{2}} \chi_{n - 1}({y})
      \right],
    \end{aligned}
  \end{equation}
  we can identify the $0$-th component of matrix elements for $y$-directional electric current operator as
  \begin{equation} \tag{D16}
    \begin{aligned}
      \Big({j}^y_{s=0}\Big)_{nm,n'm'}^{electron} = &
      \frac{ie\hbar\kappa}{m_e}
      \delta_{k_x,k'_x}
      \left(
      \sqrt{\frac{n}{2}} \delta_{n',n-1}
      - \sqrt{\frac{n+1}{2}} \delta_{n',n+1}
      \right).
    \end{aligned}
  \end{equation}
  \begin{equation} \tag{D17}
    \begin{aligned}
      \Big({j}^y_{s=0}\Big)_{nm,n'm'}^{electron} = &
      \frac{ie \hbar}{m_e l_0}
      \delta_{k_x,k'_x}
      \left(
      \sqrt{\frac{n}{2}} \delta_{n',n-1}
      - \sqrt{\frac{n+1}{2}} \delta_{n',n+1}
      \right).
    \end{aligned}
  \end{equation}
  }
  \item Section V - second paragraph (page 7):\\
  {\color{Red}
  Moreover, let $\Pi = \varepsilon_F$ and the derived expression in Eq.~(31) leads to
  \begin{equation} \tag{35}
    \begin{aligned}
      \sigma^{xx}  =
      \frac{e^2l_0^2}{\pi\hbar A}
      \sum_{n}
      \frac{(n+1)}{\gamma_{n}\gamma_{n+1}}
      \left[
        \frac{1}
        {
          1 + \left(\frac{X_F - n -1}{\gamma_{n+1}}\right)^2
        }
      \right]
      \left[
        \frac{1}
        {
          1 + \left(\frac{X_F - n}{\gamma_{n}}\right)^2
        }
      \right],
    \end{aligned}
  \end{equation}
  where $X_F = \left[\flatfrac{\varepsilon_F}{(\hbar \omega_0)} - \flatfrac{1}{2}\right]$,
  $\gamma_n = \flatfrac{\widetilde{{\Gamma}}(\varepsilon_n)}{(\hbar \omega_0)}$, and $l_0 = \sqrt{\flatfrac{\hbar}{eB}}$.
  Following the same steps as above derivation, we can derive the longitudinal conductivity in the $y$-direction by applying the electric current operator for $y$-direction derived in Appendix D
  \begin{equation} \tag{36}
    \begin{aligned}
      {\sigma}^{yy} =
      \frac{e^2l_0^2}{\pi\hbar A}
      \sum_{n}
      \frac{(n+1)}{\gamma_{n}\gamma_{n+1}}
      \left[
        \frac{1}
        {
          1 + \left(\frac{X_F - n -1}{\gamma_{n+1}}\right)^2
        }
      \right]
      \left[
        \frac{1}
        {
          1 + \left(\frac{X_F - n}{\gamma_{n}}\right)^2
        }
      \right].
    \end{aligned}
  \end{equation}
  }
\end{itemize}

Suppose we applied an arbitrary directional polarized dressing field. In thay case, we could derive an expression for the longitudinal conductivity components by considering dressing field contribution for each direction ($x$ and $y$) with the same steps presented in our analysis. This would change the amplitude of the considering dressing field ($E$) in each direction. There will be a new component in parallel to the considering conductivity component, which will modify the time-dependent Hamiltonian given in Eq.~[2]. Then, we need to solve the Schrödinger equation with the modified Hamiltonian to find the new Floquet modes for the new system. These will define the effect on the inverse scattering time matrix components. As mentioned in Ref. \cite{wackerl20} and our work, applied strong dressing field tends to change the quantum state of the considered system and creates novel states called Floquet states. The properties of these states depend on the characteristics of the applied dressing field. Therefore, the polarization method also changes the behavior of the Floquet states. These polarization-dependent conductivity behaviors in 2DEG systems can be found in references \cite{wackerl20,morina15} and in their work authors have illustrated the conductivity behavior with circular and linear polarized dressing fields. However, in a dressed quantum Hall system with a $y$-directional linear polarized field, we predict the same longitudinal conductivity behaviors in diagonal components of the conductivity tensor.

For analyzing our system, we mainly employ the diagonal components of the Floquet-Drude conductivity tensor introduced by Wackerl \textit{et al.} \cite{wackerl20}. As we mainly focus on SdH oscillations in longitudinal conductivity components, we limit our analysis to the dressing field on two diagonal components of the conductivity tensor. Therefore we are unable to predict the off-diagonal component behavior of the system.
%
% As mentioned in Ref.\cite{wackerl20} and our work, applied strong dressing field tends to change the quantum state of the consideirng system and creates novel states called Floquet states. The properties of these states are depends on the characteristics of applied dressing field. Therefore, the polarization method also change the behavior of the Floquet states.
% These polarization dependent conductivity behaviors in 2DEG systems can be found in references \cite{wackerl20,morina15} and in their work authors have illustrated the conductivity behavior with circular and linear polarized dressing fields. In addition, they have mentioned the impact on the direction of the linearly polarization dressing field and this will break the isotropy behavior of the longitudinal conductivities in the consideirng system.
%
% We also followed the same formalism of Floquet states for 2DEG quantum Hall systems. In our work also we were able to identify that the applied dressing field's polarization direction can be effect the magnitude of the longitudinal conductivity component. A detailed derivation of these two longitudinal conductivity component's current operators are given in Appendix D. If we applied a arbitary directional polarized dressing field, we can derive a expression for the longitudinal conductivity components by considering dressing field contribution for each direction ($x$ and $y$) with same steps in this analysis. This would change the amplitude of the considering dressing field ($E$) in each direction. However, now there will be a new conponent in parallel to the considering conductivity component and we need to solve the Schrödinger equation with the modified Hamiltonian in Eq.~[2] for the new system with these changes.
% However in our analysis, our main objective was to identify the manipulation capabilities of SdH oscillations using dressing field intensity. In addition, we only consider the longitudinal components in this analysis by extending the Floquet-Drude conductivity formula that derived only for the diagonal components in conductivity tensor. Therefore we are unable to predict the off-diagonal component behavior of the system through our analysis and we need a novel formalism to handle these type of behaviors of dressed 2DEG quantum Hall systems.

\newpage
\subsection*{Response to the comments of Reviewer 2}

We would like to thank the reviewer for his/her insightful comments on our work. To address your comments, we had to add new materials and reorganize our text. As a result, we have been able to considerably improve our discussion on the underlying assumptions and corresponding physical significance of our analysis. We hence believe that both the presentation and scientific content of this manuscript are of a much higher quality than our previous submission. We hope our response and the corresponding changes we incorporated into the manuscript would be sufficient to clarify the issues raised.


\subsubsection*{Comment 1 -
\color{RoyalBlue} The paper is very low on comparison with experiments, for which there is a large amount of data available for the zero radiation case. I would like to see detailed analysis of how Figs 4 and 5 compare with available
experimental data. Without this addition, physical relevance of such
detailed calculations is questionable.
}

We thank the reviewer for pointing out the importance of this fact. To overcome this shortage of discussion we made a detailed analysis of our results against the undressed 2DEG quantum Hall system behavior. We have inserted this discussion into our manuscript under Section VI.

\begin{itemize}
  \item Section VI - third paragraph (page 8):\\
  {\color{Red}
  By comparing the theoretical \cite{ando72,ando74_1,ando74_2,ando74_3,ando74_4,ando82,endo09} and experimental \cite{endo09,wakabayashi78,ochiai190,mancoff96,arapov02,grbic04,caviglia10} studies on the magnetoresistance of 2DEG quantum Hall systems under zero radiation with our results, we can identify that longitudinal conductivity oscillations in Fig. 4 are a repetition of the Shubnikov–de Haas(SdH) oscillations.
  As observed in the experimental work done by Caviglia \textit{et al.} \cite{caviglia10}, the SdH oscillations period only depends on the perpendicular component of the applied magnetic field.
  Therefore, we can identify that many experiments done in low-temperatures \cite{endo09,ochiai190,mancoff96,arapov02,grbic04,caviglia10} with a different type of 2DEG have been examined these SdH oscillations against the applied magnetic field amplitude.
  The magnetic field's perpendicular component amplitude defines the cyclotron frequency ($\omega_0$) of the considering system.
  Therefore we can achieve the same oscillatory behavior for longitudinal conductivity in 2DEG quantum Hall system by changing the gate voltage. The  change of the gate voltage applied to the system will modify the Fermi level of the system.
  Since the Landau level energy is only dependent on the cyclotron frequency, this will generate the same SdH oscillations against gate voltage.
  This oscillatory behavior with applied gate voltage has been observed in an  experiment done by Wakabayashi \textit{et al.} \cite{wakabayashi78} in an inversion layer on a silicon surface at low temperature. By comparing these empirical observations with our results given in Fig. 4, we can identify that our oscillations are also shown the same characteristic behavior with $\hbar\omega_0$ periodicity against Fermi level value. Furthermore, when we increase the gate voltage, the Fermi level of the system gets increased. This will locate to the Fermi level on higher-order Landau levels. As illustrated in Fig. 4, this will result in a higher conductivity peak value at the higher-order Landau levels. This same behavior was also observed in the experimental observations presented in Ref. \cite{wakabayashi78} as well. However, in our analysis, we have analytically explained the controllability of these conductivity regions using a dressing field.
  }
\end{itemize}

\subsubsection*{Comment 2 -
\color{RoyalBlue} In the presence of disorder, Anderson localization and topologically protected edge modes are dominant considerations when calculating DC transport coefficients. How are these factors taken into account?
}

We agree with the reviewer that the discussion on the Anderson localization and topologically protected edge modes were missing from the previous manuscript.
We would like to mention to the reviewer that while deriving Eq.~[31], we have assumed that the applied static magnetic field is placed in a lower intensity range where we can neglect several effects known to appear in a 2DEG quantum Hall system. Under this condition, we can neglect the localization in the tails of Landau level peaks, the formation of the edge states, the electron-electron interactions, and the spin splitting as well. As mentioned in the work done by Endo \textit{et al.} \cite{endo09}, under the given magnetic field range, the undressed magnetotransport properties of GaAs/AlGaAs 2DEG quantum Hall system provide an excellent agreement between the experimental observations and theory that was developed by neglecting the above-mentioned effects. We also followed the same theoretical derivation with given assumptions but with an additional dressing field. Our results for the undressed system give a adequate agreement with the results from Ref. \cite{endo09}, while we provide a better description to manipulate the width of the Landau level conductivity peaks. We have added this clarification in the manuscript.

\begin{itemize}
  \item Section I - sixth paragraph (page 2):\\
  {\color{Red}
  It is important to note that, in this analysis, we consider the applied magnetic field at a range where the Anderson localization, the formation of the edge states, the spin splitting, and the electron-electron interactions can be neglected. The use of this assumption can also be found in the study done by Endo \textit{et al.} \cite{endo09}.
  }
  \item Section VIII - first paragraph (page 10):\\
  {\color{Red}
  We assumed the impurities in the material as a Gaussian random scattering potential and the applied magnetic field within an order where we can neglect effects of the Anderson localization, the formation of the edge states, the spin splitting, and the electron-electron interactions.
  }
\end{itemize}


\subsubsection*{Comment 3 -
\color{RoyalBlue} Can the authors reproduce the quantized Hall conductance of filled Landau levels in the presence of disorder? Please discuss.
}

For analyzing our system, we mainly employ the Floquet-Drude conductivity formula introduced by Wackerl \textit{et al.} \cite{wackerl20}. The Floquet-Drude conductivity formula was only developed for the diagonal components of the conductivity tensor of the considering system. In this analysis we were only focused on SdH oscillations in longitudinal conductivity components we analyzed the dressing field effect on two diagonal components of the conductivity tensor through the Floquet-Drude conductivity formula. Therefore we are unable to predict the off-diagonal component behavior of the system through our analysis and we need a novel formalism to handle these types of behaviors of dressed 2DEG quantum Hall systems. However, we can use the already established formalism introduced by Endo \textit{et al.} \cite{endo09} to predict the behavior of off-diagonal components of the conductivity tensor. As mentioned in Ref. \cite{endo09}, this theory on the relation between longitudinal and transverse conductivities highly match with the experiment observations
\begin{equation} \tag{1}
  \dv{\widetilde{\sigma}^{xy}(X_F,I)}{B} \approx
    \pi\mu \frac{\hbar \omega_0}{\varepsilon_F} [\widetilde{\sigma}^{xx}(X_F,I)]^2.
\end{equation}
By examining the connection between longitudinal and transverse conductivities in the 2DEG quantum Hall system, we can expect the same Hall conductivity behavior in a dressed quantum Hall system. However, using the dressing field we can squeeze the conductivity peaks on Landau levels. This will squeeze the region of longitudinal conductivity peaks placed near the Landau levels. This will imply sudden Hall conductivity transitions between one plateau to another near a Landau level. As empirical observation shown in Refs.  \cite{klitzing80,gusynin06}, we can predict a staircase behavior for quantum Hall conductivity agaist the gate voltage. With a external dressing field we will be able to get rapid transitions than the undressed system. However, to derive an analytical expression for the quantum Hall conductivity through Floquet-Drude conductivity needs to be reformulated from the begin and it will be the subject of our future study.

\subsubsection*{Comment 4 -
\color{RoyalBlue} The way some previous works are cited is disappointing. Example from page 8: “Despite this behavior being identified in previous works, their results did not coincide with the more accurate description of conductivity components in undressed quantum Hall systems.” This is not informative. Where, specifically, did results from earlier works fall short of reality, which have been better addressed in this work?
}

We would like to thank the reviewer for pointing out the deficiencies in our manuscript and we have considered your recommendations seriously. We made the following changes in language and presentation to improve the clarity of our manuscript.

\begin{itemize}
  \item Section VIII (page 8):\\
  {\color{Red}
  Despite this behavior being identified in previous works done by Dini \textit{et al.}  \cite{dini16}, their results do not coincide with the experimental observations of longitudinal conductivity components in undressed quantum Hall systems presented in Ref. \cite{endo09}. This is due to the fact that the authors of Ref. \cite{dini16} has been used the conventinal expression of longitudinal conductivity from Refs. \cite{ando74_1,ando82} and this theory yields a semi-elliptical broadening, which significantly deviates by the experimentally observed Landau levels \cite{endo09}.
  However, our generalized analysis on the conductivity of dressed quantum Hall systems provide a well-suited description for empirically observed behaviors of undressed quantum Hall systems as well.
  }
  \item Section III - first paragraph (page 3):\\
  {\color{Red}
  In their study \cite{wackerl20}, the authors have presented more accurate results than the former theoretical descriptions \cite{morina15,pervishko15} for the conductivity of nanoscale systems in the  presence of a dressing field. Therefore, we apply the Floquet-Drude conductivity theory to analyze our 2DEG system which is subjected to both a stationary magnetic field and a dressing field.
  }
  \item Section I - fifth paragraph (page 2):\\
  {\color{Red}
  The authors of Ref. \cite{dini16} used the conductivity models from Refs. \cite{ando74_1,ando82}, and as mentioned in Endo \textit{et al.} \cite{endo09}, those models predict a semi-elliptical broadening against Fermi level for each Landau level and provide less agreement with the empirical results.
  }
  \item Section IV - third paragraph (page 5):\\
  {\color{Red}
  In the study presented in Ref. \cite{endo09}, the authors have assumed that the effective mass of the electron in GaAs-based quantum well system is $m_e \approx 0.07\widetilde{m}_e$ where $\widetilde{m}_e$ is the mass of the electron \cite{endo09,winkler03,wackerl20}.
  }
\end{itemize}

\subsubsection*{Comment 5 -
\color{RoyalBlue} Adding some physical insight into the remarkable observation of radiation-induced narrowing of lineshapes (Figs 4, 5) will help elevate
this work.
}

We agree with the reviewer that a discussion on our theoretical results and their physical significance in modern nanoelectronic devices is a vital requirement. We have added a new Section VII to overcome the above-mentioned deficiency of our manuscript. The total content of the section is given below,

\begin{itemize}
  \item Section VII (page 8):\\
  {\color{Red}
  \subsection*{VII. PHYSICAL SIGNIFICANCE OF THE OUTCOMES}

  With the realization of 2DEGs in Si-MOSFETs (Metal Oxide Semiconductor Field Effect Transistors) \cite{fowler66}, Klitzing \textit{et al.} \cite{klitzing80} made the first transport measurements on such systems to reveal the quantum Hall effect. The empirical discovery of these unusual properties marked the beginning of a whole new realm in condensed matter physics that continues to produce phenomenal advancements in electronic systems. The quantum Hall effects in a 2DEG under a static magnetic field are described by plateaus quantized to integer values of the conductivity quantum ($\flatfrac{e^2}{\hbar}$) in the off-diagonal conductivity, with simultaneously peaks at inter-plateau transition for the diagonal conductivity \cite{endo09}. This is due to the applied magnetic field and it changes the energy spectrum of 2DEG dramatically. The magnetic field causes the density of states in 2DEG to split up into a sequence of delta functions, separated by an energy $\hbar\omega_0$, with $\omega_0$ the cyclotron frequency which depends on the applied magnetic field.
  However, experimental results demonstrate that these Landau levels are broadened and the main source of these broadening in low-temperatures is the disorders in materials \cite{ando85,dial07}. The broaden sequence of delta functions of density of states implies the oscillating behavior in the experimental measurements of longitudinal conductivity which is known as Shubnikov–de Haas (SdH) oscillations. \cite{endo09,wakabayashi78}.

  Our theoretical analysis on longitudinal conductivity behavior of dressed quantum Hall system developed by considering low-temperature limit with gaussian impurity broadening assumptions. As illustrated in Fig. 4, we were also able to demonstrate the same SdH oscillations as experimental results observed in Refs. \cite{endo09,wakabayashi78} through our model. Under the undressed condition, our results overlap with the conductivity measurement for quantum Hall systems \cite{endo09}. However, from our results given in Fig. 5, we demonstrate that we can manipulate the broadening of these conductivity peaks using an external dressing field. In low-temperatures, the principal cause of broadening of these conductivity peaks is impurity scattering and using an external dressing field we can suppress the scattering which results in shrinkage of both the scattering-induced broadening and the longitudinal conductivity peaks.

  Research on novel states of matter has driven the evolution of present-day nanoelectronic devices. In particular, controllable manipulation of material properties through a gate electric field has revolutionized the development of material science and technology \cite{ahn03,deng18}.
  The charge carrier concentration of a considering system is an imperative parameter that defines the conductivity properties of the system. We can manipulate that using the electrostatic field-effect mechanism and it is an ideal tool to control the conductivity in some specific systems.
  A 2DEG under static magnetic field with quantum Hall effects is an excellent example that the gate electric field can be used to manipulate the conductivity. A considerable number of researches have been performed using different types of 2D field-effect transistors (FETs) in magnetic fields to study the charge transport in the quantum limit \cite{wakabayashi78,yang18,long20,li14}. In the study done by Yang \textit{et al.} \cite{yang18}, the authors have observed quantized Hall plateaus and Shubnikov–de Haas oscillations for longitudinal conductivity against gate voltage in black phosphorus FET under static magnetic fields in low-temperatures. Since the Fermi level of the system can be altered with the applied gate voltage, this behavior can be easily mapped into our results given in Fig. 4. However, the specialty of our outcomes is we owned the capability of manipulating the broadening of the conductivity regions using an external dressing field. Although Yang \textit{et al.} \cite{yang18}, achieved broadening in longitudinal conductivity peaks by changing the temperature in a low range, in this study we presented a general mathematical description to manipulate longitudinal conductivity broadening using only a high-intensity electromagnetic field.

  The realization of the underlying mechanism of 2D FETs in the quantum realm promises its potential in next-generation nanoelectronic applications. In a particular application that uses the switching mechanism of the above-discussed FETs with quantum Hall effects, we can achieve high and low output conductivities by changing the input gate voltage. As a result of manipulating the broadening of conductivity regions, we can shrink the broadening of conductivity peaks around Landau levels using a high-intensity dressing field. This will enhance the sensitivity of FETs which provides the ability to observe narrow changes in gate voltage.
  Furthermore, adopting the mechanism presented in Ref. \cite{hirakawa01}, we can utilize our manipulation of conductivity peaks into very sensitive, narrowband high-frequency radiation detectors.
  Based on these ready-to-use nanoelectronic devices and their feasible optimization, we believe that our mathematical description offers the potential to realize advanced nanoelectronic devices. Furthermore, this theoretical model will help to develop simulation tools that will design the quantum effects in magnetotransport properties of 2D nanostructures.
  }
\end{itemize}

\bibliography{response}

\bigskip
\bigskip

Sincerely yours,

\def\s#1#2#3{\vbox{\hsize=4.5cm
		\kern2cm
		\hrule\kern1ex
		\hbox to \hsize{\strut\hfil #1 \hfil}
		\hbox to \hsize{\strut\hfil #2 \hfil}
		\hbox to \hsize{\strut\hfil #3 \hfil}}}

\hbox to \hsize{\s{Malin Premaratne}{(Corresponding Author)}{\href{malin.premaratne@monash.edu}{malin.premaratne@monash.edu}}}


\end{document}

% ****** End of file dressed_quantum_hall.tex ****** %
