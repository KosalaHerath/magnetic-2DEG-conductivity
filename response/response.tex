% ****** Start of file response.tex ****** %
\documentclass{article}

%===============================================================================
% Import packages
%===============================================================================

\usepackage[top=0.8in,bottom=1in,left=1in,right=1in]{geometry}
% Physics package
\usepackage{physics}
% Paragraph skip package
\usepackage[parfill]{parskip}
% AMS packages
\usepackage{amsmath}
\usepackage{amssymb}
\usepackage{mathtools}
% Color package
\usepackage[dvipsnames]{xcolor}
% Acronym package
\usepackage[acronym]{glossaries}
% SI units package
\usepackage{siunitx}
% Hypertext package
\usepackage{hyperref}
% Bold math
\usepackage{bm}
% Other packages
\usepackage{graphicx}
\usepackage{textcomp}
\usepackage{float}
\usepackage{cite}
\usepackage{multicol}
\usepackage{multirow}

% Bibliography style
\bibliographystyle{unsrt}

\begin{document}

% Define acronyms
\newacronym{tls}{TLS}{two-level system}

%===============================================================================
% Add date
%===============================================================================
\today\\

Ashot Melikyan,\\
Associate Editor,\\
Physical Review B.\\

Dear Professor Melikyan,

Thank you very much for your effort in managing the review process of our manuscript. We are also thankful for the reviewer comments, and believe the second version of the manuscript we submit herewith has been significantly improved by the constructive criticism we received. Additionally, the subsequent sections of this document discusses the reviewer comments and our responses to them.

Please note that in the following sections, the statements in {\color{RoyalBlue} \textbf{blue}} are the comments of the reviewers. Our responses are shown in black letters and the modifications we have done to the manuscript are given in {\color{Maroon} red}.

\subsection*{General changes to the manuscript}

\begin{itemize}
    \item We have made minor changes in language and presentation to improve clarity, and to match the rest of the manuscript better to the changes done to address the reviewer's comments.
    \begin{itemize}
        \item Section IV-A first paragraph - \\
            {\color{Maroon} We prepare the thermal reservoirs $B_L$ and $B_R$ so that the temperature $T_L$ of $B_L$ is significantly higher than the temperature $T_R$ of $B_R$ \acrshort{tls}.}
    \end{itemize}
\end{itemize}

\subsection*{Response to the comments of Reviewer 1}

We would like to thank the reviewer for bringing the deficiencies of our manuscript to our attention and providing constructive feedback to improve the quality of our work. We have considered all of your suggestions seriously and revised our paper manuscript as described below.

\subsubsection*{Comment 1 -
\color{RoyalBlue} My concern is that the manuscript is heavily skewed towards a purely mathematical formulation of the problem. It has a minimal connection to realistic two-dimensional electron systems. The manuscript does not discuss how the results can be applied to understanding mechanisms of charge transport in nanoelectronic devices and can be used to optimize device performance. Without such discussion, the manuscript will have a minimal impact on the community working on developing nanoelectronics.}

We strongly agree with the reviewer that a reasoning on our theoritical results and their application in current nanoelectronic devices would be essential to the reader. Therefore, we have made a discussion on physical significance of our theoritical results and their possible employments in the optimization of nanoelectronic device performance. We have inserted a new Section VII to incorporate the above discussion into the manuscript. The total content of the section is given below,

\begin{itemize}
  \item Section VII (page 8):\\
  {\color{Maroon}
  \subsection*{Physical significance of outcomes}

  With the realization of 2DEGs in Si-MOSFETs (Metal Oxide Semiconductor Field Effect Transistors) \cite{fowler66}, Klitzing \textit{et al.} \cite{klitzing80} made the first transport measurements on such systems to reveal the quantum Hall effect. The empirical discovery of these unusual properties marked the beginning of a whole new realm in condensed matter physics that continues to produce phenomenal advancements in electronic systems. The quantum Hall effects in a 2DEG under a static magnetic field are described by plateaus quantized to integer values of the conductivity quantum ($\flatfrac{e^2}{\hbar}$) in the off-diagonal conductivity, with simultaneously peaks at inter-plateau transition for the diagonal conductivity \cite{endo09}. This is due to the applied magnetic field and it changes the energy spectrum of 2DEG in a dramatic way. The magnetic field causes the density of states in 2DEG to split up into a sequence of delta functions, separated by an energy $\hbar\omega_0$, with $\omega_0$ the cyclotron frequency which depends on the applied magnetic field.
  However, experimental results demonstrate that these Landau levels are broadened and the main source of these broadening in low temperatures is the disorders in materials \cite{ando85,dial07}. This behavior implies the oscillating behavior of the experimental measurement of longitudinal conductivity which is known as Shubnikov–de Haas (SdH) oscillations. \cite{endo09,wakabayashi78}.

  Our theoretical analysis on longitudinal conductivity behavior of dressed quantum Hall system developed by considering low-temperature limit with gaussian impurity broadening assumptions. As illustrated in Fig. 4, we were also able to demonstrate the same SdH oscillations as experimental results \cite{endo09,wakabayashi78} through our model. Under the undressed ($I=0$) condition, our results overlap with the conductivity measurement for quantum Hall systems. However, from our results given in Fig. 5, we demonstrate that we can manipulate the broadening of these conductivity peaks using an external dressing field. In low temperatures, the principal cause of broadening of these conductivity peaks is impurity scattering and using an external dressing field we can suppress the scattering which results in shrinkage of both the scattering-induced broadening and the longitudinal conductivity.

  Research on novel states of matter has driven the evolution of present-day nanoelectronic devices. In particular, controllable manipulation of material properties through a gate electric field has revolutionized the development of material science and technology \cite{ahn03,deng18}.
  The charge carrier concentration of a considering system is an imperative parameter that defines the conductivity properties of the system. We can manipulate that using the electrostatic field-effect mechanism and it is an ideal tool to control the conductivity in some specific systems.
  A 2DEG under static magnetic field with quantum Hall effects is an excellent example that the gate electric field can be used to manipulate the conductivity. A considerable number of researches have been performed using different types of 2D field-effect transistors (FETs) in magnetic fields to study the electronic transport in the quantum limit \cite{wakabayashi78,yang18,long20,li14}. In the study done by Yang \textit{et al.} \cite{yang18}, the authors have observed quantized Hall plateaus and Shubnikov–de Haas oscillations for longitudinal conductivity against gate voltage in black phosphorus FET under high magnetic fields in low temperatures. Since the Fermi level of the system can be altered with applied gate voltage, this behavior can be easily mapped into our results given in Fig. 4. However, the specialty of our outcomes is we owned the capability of manipulating the broadening of the conductivity regions using an external dressing field. Although Yang \textit{et al.} \cite{yang18}, achieved this broadening manipulation by changing the temperature in a low range, in this study we present a general mathematical description to perform that using only a high-intensity electromagnetic field.

  The realization of the underlying mechanism of 2D FETs in the quantum realm promises its potential in next-generation nanoelectronic applications. In a particular application that uses the switching operation of the above-discussed FETs with quantum Hall effects, we can achieve high and low output conductivities by changing the input gate voltage. As a result of manipulating the broadening of conductivity regions, we can shrink the broadening of conductivity peaks around Landau levels using a high-intensity dressing field. This will enhance the sensitivity of FETs which provides the ability to observe narrow changes in gate voltage. Based on these available nanoelectronic devices and their feasible optimization, we believe that our mathematical description offers great potential to realize advanced nanoelectronic devices. Furthermore, this theoretical model will help to develop simulation tools that will design the quantum effects in magnetotransport properties of 2D nanostructures.
  }
\end{itemize}

\subsubsection*{Comment 2 -
\color{RoyalBlue} Moreover, this current research direction has a significant overlap with previous experimental and theoretical studies of quantum Hall systems that started with the observation of zero-resistance states in high mobility systems [Zudov et al, Phys. Rev. B 64, 201311 (2001), Mani et al, Phys. Rev. Lett. 92, 146801 (2004)] and gave rise to theoretical models for the phenomenon [Durst et al, Phys. Rev. Lett. 91, 086803 (2003), Dmitriev et al., Phys. Rev. B 71, 115316 (2005), Dmitriev et al, Phys. Rev. B 80, 165327 (2009)]. The present manuscript needs to connect to various known phenomena discussed earlier in the literature on quantum Hall systems.}

We thank the reviewer for pointing out that the nessisity of a comparison between the our theoritical model and previous discussions on transport properties of quantum Hall systems.
Among the mentioned studies we can identify that experimental researches \cite{zudov01,mani02,zudov03,mani04} are specificly aim on the unusual oscillations of the magnetoresistance induced by the microwave (millimeterwave) radiation in 2DEG quantum Hall systems. These oscillations are know as \textit{microwave-induced resistance oscillations} (MIROs). To describe these behaviors Durst \textit{et al.} \cite{durst03} introduced a simple theoretical model assuming that the experimentally observed oscillations are a consequence of photoexcited disorder-scattered electrons.
However, later a novel model was proposed in Ref. \cite{dmitriev03,dmitriev05,dmitriev09} considering the changes of the electron distribution function done by the microwave field. These more generalized models \cite{dmitriev03,dmitriev05,dmitriev09} have succesfully describe the behavior of MIROs in experimentally relevent emperatures which was missed in previous model \cite{durst03}.
Furthermore we can recognize that the underlying mechanism of all these models \cite{durst03,dmitriev03,dmitriev05,dmitriev09} is microwave photon absorbtion by electron in the considering system. However, in our theoritical model we consider higher frequencies than microwaves for the dressing field. Therefore we can clearly identify major dissimilarities between our system under the analysis and MIROs systems \cite{durst03,dmitriev03,dmitriev05,dmitriev09}. We listed these dissimilarities as follows for our comparison.
\begin{itemize}
  \item
  The mentioned experiments on magnetoresistance oscillations on 2DEG quantum Hall systems \cite{zudov01,mani02,zudov03,mani04} were performed in microwave frequency range ($30 -\SI{150}{\giga\hertz}$). This leads to build the theoritical models presented in Ref. \cite{durst03,dmitriev03} by assuming that these oscillations are caused by photoexcited electrons. Since these models consider on relevant frequency range (microwave radiation), it will allow to acknowledge the photon absorbtion by electrons. In cotrast to that, our consideration is only focus on the systems with high frequency range dressing fields which will not be associated to any photon absorbtion.
  \item
  The applied microwave radiation power on the MIROs experiments \cite{mani02,zudov03} are varies around $\SI{1}{\milli\watt\per\square\centi\metre}$ range. However, in our analysis we take the dressing field as a high intensity field, where we can not address the dressing field as a perturbation to our considering system. This leads us to use regonize new Floquet states together with conventinal Landau levels. As we mentioned in our results, we used high intensity dressing fields of magnitude around $\SI{100}{\watt\per\square\centi\metre}$ range in our numerical calculations.
  \item
  These fascinating MIROs are only observed under the weak magnetic fields ($B < \SI{0.2}{\tesla}$) in the experiments performed in Ref. \cite{zudov01,mani02,zudov03,mani04}. In this range of weak magnetic fields we can only observe MIROs and Shubnikov–de Haas (SdH) oscillations reveal only on high intensity magnetic fields. In comaparison to our analysis, we are  interested in the SdH oscillations and manipulation of their characteristics. Thesefore in our analysis we aim on the effects induced by high intensity magnetic fields ($B \sim \SI{1}{\tesla}$).
  \item
  In our work, we analysed the 2DEG quantum Hall system with high-intensity dressing field which does not contribute to the energy exchange between a high-frequency dressing field and electrons. Therefore we have assumed the applied electromagnetic radiation as a purely dressing field. There are two possible absorbtion mechanism in 2DEG quantum Hall system namely electron transitions bewtween distinct Landau levels and electron transitions between distinct states of a same broadened Landau level. To avoid these absorbtions we have tuned the dressing field to a high frequency in our system under the analysis. Furthermore due to high intensity dressing field, it will restructure the entire electronic states of the conventinal 2DEG quantum Hall systems. We have address these modifications throuth the Floquet theory. However, the MIROs models \cite{durst03,dmitriev03} are based on low frequecy case where the system is able to absorb low frequency photons from the field.
\end{itemize}

Examine above mentioned points, we can clearly identify that the high-frequency and low-frequency illumination of 2DEG quantum Hall systems leads to two particular natures of the magnetotransport properties. However, we identify that it is crusial to acknowledge these well-defined findings in MIROs and the their relatioship to the our considered system. We have added this clarification in the manuscript.

\begin{itemize}
  \item Section I - fifth paragraph (page 1):\\
  {\color{Maroon}
  In contrast, we can observe more exciting phenomena by simultaneously applying a dressing field to a quantum Hall system already under a non-oscillating magnetic field.
  Whilst there exist several leading theories for calculating conductivity tensor elements in quantum Hall systems \cite{ando74_1,ando82,endo09}, these studies  have not been utilized to describe the optical manipulation of charge transport. Recently, an experimental research on the effect of microwave illumination of 2DEG systems revealed microwave induced resistance oscillations (MIRO) under weak magnetic fields \cite{zudov01,mani02,zudov03,mani04}.
  This inspired investigations on theoretical description of MIROs and several semiclassical and quantum kinetic equation formalisms have been proposed to address the underlying mechanism of MIROs \cite{durst03,dmitriev03,dmitriev05,dmitriev09}. These analytical formalisms provide a good explanation of the experimental characteristics of MIROs. However, these experimental and theoretical works have been linked to the photon absorbtion from low-frequecy (microwave) fields. In contrast to that, high-frequency external illumination on 2DEG quantum Hall system need to be study as a purely dressing (nonabsorbable) field.
  The impacts induced by a purely dressing field on magnetotransport properties of 2DEG quantum Hall system need to be descriped by a nonabsorbtion mechanism and it has escaped the researchers’ attention before.
  Then, lately Dini \textit{et al.} \cite{dini16} have investigated the one-directional conductivity behavior of dressed quantum Hall systems for high-frequency case. However, they have not adopted the state-of-the-art model to describe the conductivity in a quantum Hall system. In their study, they used the conductivity models from Refs. \cite{ando74_1,ando82}, and as mentioned in Endo \textit{et al.} \cite{endo09}, those models predict a semi-elliptical broadening against Fermi level for each Landau level and provide less agreement with the empirical results.
  }
  \item Section VI - fifth paragraph (page 8):\\
  {\color{Maroon}
  By comparing the theoritical \cite{ando72,ando74_1,ando74_2,ando74_3,ando74_4,ando82,endo09} and experimental \cite{endo09,wakabayashi78} studies on magnetoresistance of 2DEG qunatum Hall systems under zero radiation with our results, we can identify that longitudinal conductivity oscillations in Fig.~[4] are repetition of the Shubnikov–de Haas(SdH) oscillations.
  These SdH oscillations shows $\hbar\omega_0$ periodicity with gate voltage and the periodicity of these oscillations are not effected by appplied dressing field frequency. It is important to state that the difference between SdH oscillations and MIROs \cite{zudov01,mani02,zudov03,mani04} by considering this frequency dependence on periodicity. In our case, the applied dressing field only change the the broadening of Landau levels and avoid any contribution towards the photon absorbtions. This will clearly describe the crusial difference between low-frequency illumination and high-frequency illumination effects on 2DEG quantum systems.
  }
\end{itemize}

\subsubsection*{Comment 3 -
\color{RoyalBlue} The manuscript will also provide more impact if it demonstrates how the new results can help to improve the future development of nanoelectronic devices. After these questions are addressed, the manuscript will be suitable for publication in Physical Review B. Otherwise, it will fit better to a more mathematically oriented journal.
}

We thank the reviewer for pointing out the importance of these facts that will help to elevate the value of our work. As we mentioned in the commnet 1, consideirng the important of discussing physical significance of our results, we added a new section(Section VII) to address these facts.

\begin{itemize}
  \item Section VII (page 8):\\
  {\color{Maroon}
  \subsection*{Physical significance of outcomes}

  With the realization of 2DEGs in Si-MOSFETs (Metal Oxide Semiconductor Field Effect Transistors) \cite{fowler66}, Klitzing \textit{et al.} \cite{klitzing80} made the first transport measurements on such systems to reveal the quantum Hall effect. The empirical discovery of these unusual properties marked the beginning of a whole new realm in condensed matter physics that continues to produce phenomenal advancements in electronic systems. The quantum Hall effects in a 2DEG under a static magnetic field are described by plateaus quantized to integer values of the conductivity quantum ($\flatfrac{e^2}{\hbar}$) in the off-diagonal conductivity, with simultaneously peaks at inter-plateau transition for the diagonal conductivity \cite{endo09}. This is due to the applied magnetic field and it changes the energy spectrum of 2DEG in a dramatic way. The magnetic field causes the density of states in 2DEG to split up into a sequence of delta functions, separated by an energy $\hbar\omega_0$, with $\omega_0$ the cyclotron frequency which depends on the applied magnetic field.
  However, experimental results demonstrate that these Landau levels are broadened and the main source of these broadening in low temperatures is the disorders in materials \cite{ando85,dial07}. This behavior implies the oscillating behavior of the experimental measurement of longitudinal conductivity which is known as Shubnikov–de Haas (SdH) oscillations. \cite{endo09,wakabayashi78}.

  Our theoretical analysis on longitudinal conductivity behavior of dressed quantum Hall system developed by considering low-temperature limit with gaussian impurity broadening assumptions. As illustrated in Fig. 4, we were also able to demonstrate the same SdH oscillations as experimental results \cite{endo09,wakabayashi78} through our model. Under the undressed ($I=0$) condition, our results overlap with the conductivity measurement for quantum Hall systems. However, from our results given in Fig. 5, we demonstrate that we can manipulate the broadening of these conductivity peaks using an external dressing field. In low temperatures, the principal cause of broadening of these conductivity peaks is impurity scattering and using an external dressing field we can suppress the scattering which results in shrinkage of both the scattering-induced broadening and the longitudinal conductivity.

  Research on novel states of matter has driven the evolution of present-day nanoelectronic devices. In particular, controllable manipulation of material properties through a gate electric field has revolutionized the development of material science and technology \cite{ahn03,deng18}.
  The charge carrier concentration of a considering system is an imperative parameter that defines the conductivity properties of the system. We can manipulate that using the electrostatic field-effect mechanism and it is an ideal tool to control the conductivity in some specific systems.
  A 2DEG under static magnetic field with quantum Hall effects is an excellent example that the gate electric field can be used to manipulate the conductivity. A considerable number of researches have been performed using different types of 2D field-effect transistors (FETs) in magnetic fields to study the electronic transport in the quantum limit \cite{wakabayashi78,yang18,long20,li14}. In the study done by Yang \textit{et al.} \cite{yang18}, the authors have observed quantized Hall plateaus and Shubnikov–de Haas oscillations for longitudinal conductivity against gate voltage in black phosphorus FET under high magnetic fields in low temperatures. Since the Fermi level of the system can be altered with applied gate voltage, this behavior can be easily mapped into our results given in Fig. 4. However, the specialty of our outcomes is we owned the capability of manipulating the broadening of the conductivity regions using an external dressing field. Although Yang \textit{et al.} \cite{yang18}, achieved this broadening manipulation by changing the temperature in a low range, in this study we present a general mathematical description to perform that using only a high-intensity electromagnetic field.

  The realization of the underlying mechanism of 2D FETs in the quantum realm promises its potential in next-generation nanoelectronic applications. In a particular application that uses the switching operation of the above-discussed FETs with quantum Hall effects, we can achieve high and low output conductivities by changing the input gate voltage. As a result of manipulating the broadening of conductivity regions, we can shrink the broadening of conductivity peaks around Landau levels using a high-intensity dressing field. This will enhance the sensitivity of FETs which provides the ability to observe narrow changes in gate voltage. Based on these available nanoelectronic devices and their feasible optimization, we believe that our mathematical description offers great potential to realize advanced nanoelectronic devices. Furthermore, this theoretical model will help to develop simulation tools that will design the quantum effects in magnetotransport properties of 2D nanostructures.
  }
\end{itemize}

\subsubsection*{Comment 4 -
\color{RoyalBlue} The quantum Quantum Hall effect requires high mobility samples. In these samples, the structure of the disorder is usually complicated and combines both short-length potentials of impurities and long-length electrostatic inhomogeneities. The interplay of these components of disorder opens exciting questions about the transport properties of 2DEGs. What is the structure of disorder considered in the present manuscript and hidden in the notations for Vimp? What are the conditions for validity of eq. (15)?
}

We thank the reviewer for raising this important question. In our previous manuscript, we have presented the detailed derivation of Eq.~[15] with a discusssion on models of disorder under the Appendix C. However we will discuss our selection of disorder model and approximations made to derive the Eq.~[15] here as well.

We modeled the effect caused by impurities in the considered system as a single perturbation potential. Analyzing the electric propeties for a specific impurity configuration is a rather formidable task and is not examined in this work since it is unlikely to have exactly the evaluated impurity configuration in an experiment. Therefore, in this study we consider on the statistically averaged propeties of 2DEG over impurity configurations. In addition, we consider only the our considering static disorder corresponds to the situation in which the electrons scatter elastically. First, we adapt the Edwards model \cite{akkermans10} to represent the randomly distributed impurities over the considering system and we approximate this into a Gaussian white noise.

Since we are presenting the the perturbation potential $V(\vb{r})$ by a group of randomly distributed impurities, we take into account $N_{imp}$ number of identical single impurity potentials distributed at randomly but in fixed positions $\vb{r}_i$. Thus, we can describe the scattering potential $V(\vb{r})$ as the sum of uncorrelated single impurity potentials $\upsilon(\vb{r})$
\begin{equation} \label{eq:1}
  V(\vb{r}) =
  \sum_{i=1}^{N_{imp}}
  \upsilon (\vb{r}-\vb{r}_i).
\end{equation}
Furthermore, we model the perturbation $V(\vb{r})$ as a Gaussian random potential where one can choose the zero of energy such that the potential is zero on average. This model is characterized by the following two equations \cite{akkermans10}
\begin{subequations}
\begin{equation} \label{eq:2}
  \expval{\upsilon(\vb{r})}_{imp} =0,
\end{equation}
\begin{equation} \label{eq:3}
  \expval{\upsilon(\vb{r})\upsilon(\vb{r'})}_{imp} = \Upsilon(\vb{r}-\vb{r'}),
\end{equation}
\end{subequations}
where $\expval{\cdot}_{imp}$ represents the average over the impurity disorder and $\Upsilon(\vb{r}-\vb{r'})$ is any decaying function depends only on $\vb{r}-\vb{r'}$. In addition, this model assumes that $\upsilon (\vb{r}-\vb{r'})$ only depends on the magnitude of the position difference $|\vb{r}-\vb{r'}|$, and it decays with a characteristic length $r_c$. Since this study considers the case where the wavelength of radiation or scattering electron is much greater than $r_c$, it is a better approximation to make two-point correlation function to be
\begin{equation} \label{eq:4}
  \expval{\upsilon(\vb{r})\upsilon(\vb{r'})}_{imp} = \Upsilon_{imp}^2\delta(\vb{r}-\vb{r'}),
\end{equation}
where $\Upsilon_{imp}^2$ is a constant. A random potential $V(\vb{r})$ with this property is called white noise \cite{akkermans10}. Then we can approximately model the total scattering potential as
\begin{equation} \label{eq:5}
  V(\vb{r}) =
  \sum_{i=1}^{N_{imp}}
  \Upsilon_{imp} \delta(\vb{r}-\vb{r}_i).
\end{equation}

Then we calculate the Floquet-Fermi golden rule for a dressed quantum Hall system using this perturbative impurity potential. In the derivation we have defined the $V_{imp}$ value which is constant in moemntum space. It is a propety of Gaussian white noise impurity distribution \cite{akkermans10,wackerl20}. However, throughout the derivation we use only the first order contribution (Born approximation) of the impurity potential. All the detailed step for the derivation of the Floquet-Fermi golden rule for a dressed quantum Hall system are included in Appendix C.

Since previous studies on Floquet-Drude consuctivity \cite{wackerl20}, and magnetotransport propeties in dressed qunatum Hall systems \cite{dini16} have  used this perticular Gaussian white noise potential, we also selected this pertucular impurity model to describe our system. This open a path to compare our analytical outcomes with these previous models.

Since we have not mentioned the validity conditions we used to derive the Eq.~[15] in the main text of the previous manuscript, we have added validity conditions to the new manuscript.

\begin{itemize}
  \item Section IV - first paragraph (page 4):\\
  {\color{Maroon}
  The Floquet-Fermi golden rule was proposed in Ref. \cite{wackerl20} as an approach to analyze the transport properties of dressed quantum systems with impurities.
  However, this theory has not been applied for a dressed quantum Hall system in the previous studies. In this analysis, we use Floquet-Fermi golden rule to identify the effects induced by impurities on the magneto-transport properties.
  With the help of $t-t'$ formalism \cite{wackerl20,grifoni98,sambe75,peskin93,althorpe97} and applying Floquet states derived in Eq.~(14), we can derive an  expression for $(l,l')$-th element of the inverse scattering time matrix for the $N$-th Landau level as
  \begin{equation} \label{eq:15}
    \begin{aligned}
      \left(\frac{1}{\tau(\varepsilon,k_x)}\right)^{ll'}_N =
      \frac{ \pi \hbar\varrho^2}{e^2B^2} \delta(\varepsilon - \varepsilon_N)
      \int_{-\infty}^{\infty} &
      J_l \bm{\left(} \frac{b\hbar}{eB}({k}_x - k_1) \bm{\right)}
      J_{l'} \bm{\left(} \frac{b\hbar}{eB}({k}_x - k_1) \bm{\right)}
      \\
      & \times
      \left|
      \int_{-\infty}^{\infty}
      \chi_N \left( \frac{\hbar}{eB}k_2 \right)
      \chi_N \bm{\left(} \frac{\hbar}{eB}
      \left( k_1 - {k}_x - k_2 \right) \bm{\right)}
      dk_2 \right|^2 d k_1,
    \end{aligned}
  \end{equation}
  where $\varrho = \eta_{imp} L_x \left[\flatfrac{ V_{imp}}{\pi}\right]^{1/2}$, $\varepsilon$ is a given energy value, $J_l(\cdot)$ are Bessel functions of the first kind with $l$-th integer order, and $\varepsilon_N$ is the energy of the $N$-th Landau level.
  A more detailed derivation is given in Appendix C.
  We modeled the effect caused by impurities in the considered system as a single perturbation potential. Analyzing the electric propeties for a specific impurity configuration is a rather formidable task and is not examined in this work since it is unlikely to have exactly the evaluated impurity configuration in an experiment.
  Furthermore, we assumed that our perturbation potential is formed by a group of randomly distributed impurities under the Edwards impurity model \cite{akkermans10,wackerl20}.
  Thus, we represented the total scattering potential in the 2DEG as a sum of uncorrelated single impurity potentials $\upsilon(\vb{r})$. Then we  approximate the impurity potential into a Gaussian white noise \cite{akkermans10,wackerl20}.
  Here $\eta_{imp}$ is the number of impurities in a unit area, $V_{imp} = \expval{|V_{{k'}_x,k_x}|^2}_{imp}$ with $V_{{k'}_x,k_x} = \mel**{k'_x}{\upsilon(x) }{k_x}$, and $\braket{x}{k_x} = e^{-ik_x x}$.
  Moreover, in this analysis, $\expval{\cdot}_{imp}$ represents the average over the impurity disorder. In this derivation, we only considered the first order (the Born approximation) contribution from the impurity potential and
  }
\end{itemize}

























%
% \subsubsection*{Comment 2 -
% \color{RoyalBlue} As the proposed system is an extension of the one reported in
% reference \cite{joulain2016quantum}, authors should compare their results with the
% corresponding ones reported in that reference. In particular, the
% comparison of the amplification factors would be interesting.}
%
% We agree with the reviewer that a comparison between the our optical-gating system and the device reported in \cite{joulain2016quantum} would be valuable to the reader. However it is important to note that the previous device controlled the $J_L$ and $J_R$ heat flows via changing the temperature $T_M$ of the terminal $M$. In contrast, we employ the driving strength of the field $F$, or in other words its Rabi frequency $\Omega$, to control the same heat flows. Even though the two devices are basically identical in the output side, they are quite different on the input side. This makes a simple, direct comparison of the two devices impossible. Therefore, we had to resort to the following methodology for our comparison.
% \begin{enumerate}
%     \item First we simulate both systems with the same bath temperatures $T_L$ and $T_R$, and same system parameters $\omega_P$s and $\omega_{PQ}$s. -
%     \begin{itemize}
%         \item We first reproduced the results of Joulain \textit{et al.} \cite{joulain2016quantum}. Since we use SI units for our work, we mapped all their parameters to SI units and generated corresponding results.
%         \item We then used the same parameter set as Joulain \textit{et al.} \cite{joulain2016quantum} to simulate our device.
%     \end{itemize}
%
%     \item Then, we find the control-parameter (i.e. $T_M$ for original system, $\Omega$ for our system) ranges within which each system shows the required thermal behaviors. -
%     \begin{itemize}
%         \item For the previous system, the original authors had found that the control range is between the lower bound $T_R$ and the upper bound $T_L$.
%         \item In our system the magnitude of $\Omega$ has the lower bound of zero, but does not have a hard upper bound. However we see from Fig. \ref{fig:energyflows2}, that our system eventually saturates for large $\Omega$ values, after which increasing $\Omega$ further has minimal effect on the heat flows. We use this $\Omega$ value at saturation to define an approximate upper bound, which turns out to be around $3\Delta$ in our simulations.
%     \end{itemize}
%
%     \item Next, we simulate all output heat flows of each device for the full control-parameter range of that particular device. -
%     \begin{itemize}
%         \item For each system, we use a very small step-size and vary the control-parameter from its minimum value to its maximum value, and measure the output heat flows.
%         \item The resulting plots are shown in Fig. \ref{fig:energyflows2} and Fig. \ref{fig:energyefficiency}. Each plot have a single vertical axis, and two horizontal axes. The top horizontal axis measures $T_M$ and corresponds to the device of Joulain \textit{et al.} \cite{joulain2016quantum}, while the bottom axis measures $\Omega$ and corresponds to our system. We plot the values measured from the previous device with dashed lines, and values measured from our device with solid lines.
%     \end{itemize}
% \end{enumerate}

\newpage
\bibliography{response}

\bigskip
\bigskip

Sincerely yours,

\def\s#1#2#3{\vbox{\hsize=4.5cm
		\kern2cm
		\hrule\kern1ex
		\hbox to \hsize{\strut\hfil #1 \hfil}
		\hbox to \hsize{\strut\hfil #2 \hfil}
		\hbox to \hsize{\strut\hfil #3 \hfil}}}

\hbox to \hsize{\s{Malin Premaratne}{(Corresponding Author)}{\href{malin.premaratne@monash.edu}{malin.premaratne@monash.edu}}}


\end{document}

% ****** End of file dressed_quantum_hall.tex ****** %
