% ****** Start of file response.tex ****** %
\documentclass{article}

%===============================================================================
% Import packages
%===============================================================================

\usepackage[top=0.8in,bottom=1in,left=1in,right=1in]{geometry}
% Physics package
\usepackage{physics}
% Paragraph skip package
\usepackage[parfill]{parskip}
% AMS packages
\usepackage{amsmath}
\usepackage{amssymb}
\usepackage{mathtools}
% Color package
\usepackage[dvipsnames]{xcolor}
% Acronym package
\usepackage[acronym]{glossaries}
% SI units package
\usepackage{siunitx}
% Hypertext package
\usepackage{hyperref}
% Bold math
\usepackage{bm}
% Other packages
\usepackage{graphicx}
\usepackage{textcomp}
\usepackage{float}
\usepackage{cite}
\usepackage{multicol}
\usepackage{multirow}

% Bibliography style
\bibliographystyle{unsrt}

\begin{document}

% Define acronyms
\newacronym{tls}{TLS}{two-level system}

%===============================================================================
% Add date
%===============================================================================
\today\\

Ashot Melikyan,\\
Associate Editor,\\
Physical Review B.\\

Dear Professor Melikyan,

Thank you very much for your effort in managing the review process of our manuscript. We are also thankful for the reviewer comments, and believe the second version of the manuscript we submit herewith has been significantly improved by the constructive criticism we received. Additionally, the subsequent sections of this document discusses the reviewer comments and our responses to them.

Please note that in the following sections, the statements in {\color{RoyalBlue} \textbf{blue}} are the comments of the reviewers. Our responses are shown in black letters and the modifications we have done to the manuscript are given in {\color{Maroon} red}.

\subsection*{General changes to the manuscript}

\begin{itemize}
    \item We have made minor changes in language and presentation to improve clarity, and to match the rest of the manuscript better to the changes done to address the reviewer's comments.
    \begin{itemize}
        \item Section IV-A first paragraph - \\
            {\color{Maroon} We prepare the thermal reservoirs $B_L$ and $B_R$ so that the temperature $T_L$ of $B_L$ is significantly higher than the temperature $T_R$ of $B_R$ \acrshort{tls}.}
    \end{itemize}
\end{itemize}

\subsection*{Response to the comments of Reviewer 1}

We would like to thank the reviewer for bringing the deficiencies of our manuscript to our attention and providing constructive feedback to improve the quality of our work. We have considered all of your suggestions seriously and revised our paper manuscript as described below.

\subsubsection*{Comment 1 -
\color{RoyalBlue} My concern is that the manuscript is heavily skewed towards a purely mathematical formulation of the problem. It has a minimal connection to realistic two-dimensional electron systems. The manuscript does not discuss how the results can be applied to understanding mechanisms of charge transport in nanoelectronic devices and can be used to optimize device performance. Without such discussion, the manuscript will have a minimal impact on the community working on developing nanoelectronics.}

We strongly agree with the reviewer that a reasoning on our theoritical results and their application in current nanoelectronic devices would be essential to the reader. Therefore, we have made a discussion on physical significance of our theoritical results and their possible employments in the optimization of nanoelectronic device performance. We have inserted a new Section VII to incorporate the above discussion into the manuscript. The total content of the section is given below,

\begin{itemize}
  \item Section VII (page 8):\\
  {\color{Maroon}
  \subsection*{Physical significance of the outcomes}

  With the realization of 2DEGs in Si-MOSFETs (Metal Oxide Semiconductor Field Effect Transistors) \cite{fowler66}, Klitzing \textit{et al.} \cite{klitzing80} made the first transport measurements on such systems to reveal the quantum Hall effect. The empirical discovery of these unusual properties marked the beginning of a whole new realm in condensed matter physics that continues to produce phenomenal advancements in electronic systems. The quantum Hall effects in a 2DEG under a static magnetic field are described by plateaus quantized to integer values of the conductivity quantum ($\flatfrac{e^2}{\hbar}$) in the off-diagonal conductivity, with simultaneously peaks at inter-plateau transition for the diagonal conductivity \cite{endo09}. This is due to the applied magnetic field and it changes the energy spectrum of 2DEG in a dramatic way. The magnetic field causes the density of states in 2DEG to split up into a sequence of delta functions, separated by an energy $\hbar\omega_0$, with $\omega_0$ the cyclotron frequency which depends on the applied magnetic field.
  However, experimental results demonstrate that these Landau levels are broadened and the main source of these broadening in low temperatures is the disorders in materials \cite{ando85,dial07}. This behavior implies the oscillating behavior of the experimental measurement of longitudinal conductivity which is known as Shubnikov–de Haas (SdH) oscillations. \cite{endo09,wakabayashi78}.

  Our theoretical analysis on longitudinal conductivity behavior of dressed quantum Hall system developed by considering low-temperature limit with gaussian impurity broadening assumptions. As illustrated in Fig. 4, we were also able to demonstrate the same SdH oscillations as experimental results \cite{endo09,wakabayashi78} through our model. Under the undressed ($I=0$) condition, our results overlap with the conductivity measurement for quantum Hall systems. However, from our results given in Fig. 5, we demonstrate that we can manipulate the broadening of these conductivity peaks using an external dressing field. In low temperatures, the principal cause of broadening of these conductivity peaks is impurity scattering and using an external dressing field we can suppress the scattering which results in shrinkage of both the scattering-induced broadening and the longitudinal conductivity.

  Research on novel states of matter has driven the evolution of present-day nanoelectronic devices. In particular, controllable manipulation of material properties through a gate electric field has revolutionized the development of material science and technology \cite{ahn03,deng18}.
  The charge carrier concentration of a considering system is an imperative parameter that defines the conductivity properties of the system. We can manipulate that using the electrostatic field-effect mechanism and it is an ideal tool to control the conductivity in some specific systems.
  A 2DEG under static magnetic field with quantum Hall effects is an excellent example that the gate electric field can be used to manipulate the conductivity. A considerable number of researches have been performed using different types of 2D field-effect transistors (FETs) in magnetic fields to study the electronic transport in the quantum limit \cite{wakabayashi78,yang18,long20,li14}. In the study done by Yang \textit{et al.} \cite{yang18}, the authors have observed quantized Hall plateaus and Shubnikov–de Haas oscillations for longitudinal conductivity against gate voltage in black phosphorus FET under high magnetic fields in low temperatures. Since the Fermi level of the system can be altered with applied gate voltage, this behavior can be easily mapped into our results given in Fig. 4. However, the specialty of our outcomes is we owned the capability of manipulating the broadening of the conductivity regions using an external dressing field. Although Yang \textit{et al.} \cite{yang18}, achieved this broadening manipulation by changing the temperature in a low range, in this study we present a general mathematical description to perform that using only a high-intensity electromagnetic field.

  The realization of the underlying mechanism of 2D FETs in the quantum realm promises its potential in next-generation nanoelectronic applications. In a particular application that uses the switching operation of the above-discussed FETs with quantum Hall effects, we can achieve high and low output conductivities by changing the input gate voltage. As a result of manipulating the broadening of conductivity regions, we can shrink the broadening of conductivity peaks around Landau levels using a high-intensity dressing field. This will enhance the sensitivity of FETs which provides the ability to observe narrow changes in gate voltage. Based on these available nanoelectronic devices and their feasible optimization, we believe that our mathematical description offers great potential to realize advanced nanoelectronic devices. Furthermore, this theoretical model will help to develop simulation tools that will design the quantum effects in magnetotransport properties of 2D nanostructures.
  }
\end{itemize}

\subsubsection*{Comment 2 -
\color{RoyalBlue} Moreover, this current research direction has a significant overlap with previous experimental and theoretical studies of quantum Hall systems that started with the observation of zero-resistance states in high mobility systems [Zudov et al, Phys. Rev. B 64, 201311 (2001), Mani et al, Phys. Rev. Lett. 92, 146801 (2004)] and gave rise to theoretical models for the phenomenon [Durst et al, Phys. Rev. Lett. 91, 086803 (2003), Dmitriev et al., Phys. Rev. B 71, 115316 (2005), Dmitriev et al, Phys. Rev. B 80, 165327 (2009)]. The present manuscript needs to connect to various known phenomena discussed earlier in the literature on quantum Hall systems.}

We thank the reviewer for pointing out that the nessisity of a comparison between the our theoritical model and previous discussions on transport properties of quantum Hall systems.
Among the mentioned studies we can identify that experimental researches \cite{zudov01,mani02,zudov03,mani04} are specificly aim on the unusual oscillations of the magnetoresistance induced by the microwave (millimeterwave) radiation in 2DEG quantum Hall systems. These oscillations are know as \textit{microwave-induced resistance oscillations} (MIROs). To describe these behaviors Durst \textit{et al.} \cite{durst03} introduced a simple theoretical model assuming that the experimentally observed oscillations are a consequence of photoexcited disorder-scattered electrons.
However, later a novel model was proposed in Ref. \cite{dmitriev03,dmitriev05,dmitriev09} considering the changes of the electron distribution function done by the microwave field. These more generalized models \cite{dmitriev03,dmitriev05,dmitriev09} have succesfully describe the behavior of MIROs in experimentally relevent emperatures which was missed in previous model \cite{durst03}.
Furthermore we can recognize that the underlying mechanism of all these models \cite{durst03,dmitriev03,dmitriev05,dmitriev09} is microwave photon absorbtion by electron in the considering system. However, in our theoritical model we consider higher frequencies than microwaves for the dressing field. Therefore we can clearly identify major dissimilarities between our system under the analysis and MIROs systems \cite{durst03,dmitriev03,dmitriev05,dmitriev09}. We listed these dissimilarities as follows for our comparison.
\begin{itemize}
  \item
  The mentioned experiments on magnetoresistance oscillations on 2DEG quantum Hall systems \cite{zudov01,mani02,zudov03,mani04} were performed in microwave frequency range ($30 -\SI{150}{\giga\hertz}$). This leads to build the theoritical models presented in Ref. \cite{durst03,dmitriev03} by assuming that these oscillations are caused by photoexcited electrons. Since these models consider on relevant frequency range (microwave radiation), it will allow to acknowledge the photon absorbtion by electrons. In cotrast to that, our consideration is only focus on the systems with high frequency range dressing fields which will not be associated to any photon absorbtion.
  \item
  The applied microwave radiation power on the MIROs experiments \cite{mani02,zudov03} are varies around $\SI{1}{\milli\watt\per\square\centi\metre}$ range. However, in our analysis we take the dressing field as a high intensity field, where we can not address the dressing field as a perturbation to our considering system. This leads us to use regonize new Floquet states together with conventinal Landau levels. As we mentioned in our results, we used high intensity dressing fields of magnitude around $\SI{100}{\watt\per\square\centi\metre}$ range in our numerical calculations.
  \item
  These fascinating MIROs are only observed under the weak magnetic fields ($B < \SI{0.2}{\tesla}$) in the experiments performed in Ref. \cite{zudov01,mani02,zudov03,mani04}. In this range of weak magnetic fields we can only observe MIROs and Shubnikov–de Haas (SdH) oscillations reveal only on high intensity magnetic fields. In comaparison to our analysis, we are  interested in the SdH oscillations and manipulation of their characteristics. Thesefore in our analysis we aim on the effects induced by high intensity magnetic fields ($B \sim \SI{1}{\tesla}$).
  \item
  In our work, we analysed the 2DEG quantum Hall system with high-intensity dressing field which does not contribute to the energy exchange between a high-frequency dressing field and electrons. Therefore we have assumed the applied electromagnetic radiation as a purely dressing field. There are two possible absorbtion mechanism in 2DEG quantum Hall system namely electron transitions bewtween distinct Landau levels and electron transitions between distinct states of a same broadened Landau level. To avoid these absorbtions we have tuned the dressing field to a high frequency in our system under the analysis. Furthermore due to high intensity dressing field, it will restructure the entire electronic states of the conventinal 2DEG quantum Hall systems. We have address these modifications throuth the Floquet theory. However, the MIROs models \cite{durst03,dmitriev03} are based on low frequecy case where the system is able to absorb low frequency photons from the field.
\end{itemize}

Examine above mentioned points, we can clearly identify that the high-frequency and low-frequency illumination of 2DEG quantum Hall systems leads to two particular natures of the magnetotransport properties. However, we identify that it is crusial to acknowledge these well-defined findings in MIROs and the their relatioship to the our considered system. We have added this clarification in the manuscript.

\begin{itemize}
  \item Section I - fifth paragraph (page 1):\\
  {\color{Maroon}
  In contrast, we can observe more exciting phenomena by simultaneously applying a dressing field to a quantum Hall system already under a non-oscillating magnetic field.
  Whilst there exist several leading theories for calculating conductivity tensor elements in quantum Hall systems \cite{ando74_1,ando82,endo09}, these studies  have not been utilized to describe the optical manipulation of charge transport. Recently, an experimental research on the effect of microwave illumination of 2DEG systems revealed microwave induced resistance oscillations (MIRO) under weak magnetic fields \cite{zudov01,mani02,zudov03,mani04}.
  This inspired investigations on theoretical description of MIROs and several semiclassical and quantum kinetic equation formalisms have been proposed to address the underlying mechanism of MIROs \cite{durst03,dmitriev03,dmitriev05,dmitriev09}. These analytical formalisms provide a good explanation of the experimental characteristics of MIROs. However, these experimental and theoretical works have been linked to the photon absorbtion from low-frequecy (microwave) fields. In contrast to that, high-frequency external illumination on 2DEG quantum Hall system need to be study as a purely dressing (nonabsorbable) field.
  The impacts induced by a purely dressing field on magnetotransport properties of 2DEG quantum Hall system need to be descriped by a nonabsorbtion mechanism and it has escaped the researchers’ attention before.
  Then, lately Dini \textit{et al.} \cite{dini16} have investigated the one-directional conductivity behavior of dressed quantum Hall systems for high-frequency case. However, they have not adopted the state-of-the-art model to describe the conductivity in a quantum Hall system. In their study, they used the conductivity models from Refs. \cite{ando74_1,ando82}, and as mentioned in Endo \textit{et al.} \cite{endo09}, those models predict a semi-elliptical broadening against Fermi level for each Landau level and provide less agreement with the empirical results.
  }
  \item Section VI - fifth paragraph (page 8):\\
  {\color{Maroon}
  By comparing the theoritical \cite{ando72,ando74_1,ando74_2,ando74_3,ando74_4,ando82,endo09} and experimental \cite{endo09,wakabayashi78,ochiai190,mancoff96,arapov02,grbic04,caviglia10} studies on magnetoresistance of 2DEG qunatum Hall systems under zero radiation with our results, we can identify that longitudinal conductivity oscillations in Fig.~[4] are repetition of the Shubnikov–de Haas(SdH) oscillations.
  As observed in the experimental work by Caviglia \textit{et al.} \cite{caviglia10}, SdH oscillations period is only depend on the perpendicular component of the applied magnetic field.
  Therefore, we can identify that many experiments done in low temperatures \cite{endo09,ochiai190,mancoff96,arapov02,grbic04,caviglia10} with different type of 2DEG has been oberved these SdH oscillations against applied magnetic field.
  The magnetic field's perpendicular component amplitude defines the cyclotron frequency ($\omega_0$) of the consdiering system.
  Therefore we can achieve the same oscillation behavior for longitudinal conductivity in 2DEG quantum Hall system by changing the gate voltage. Due to the change of the gate volatge applied to the system will modify the Fermi level of the system.
  Since the Landau level energy is only depend on the cyclotron frequency, this will generate the same SdH oscillations against gate volatage.
  This osillating behavior with applied gate volatge has been observed in experiment done by Wakabayashi \textit{et al.} \cite{wakabayashi78} in an inversion layer on a silicon surface at low temperature. With comparing these empirical observations with our results given in Fig[4], we can easly identify that our oscillations are also shows the same characteristic behavior with $\hbar\omega_0$ periodicity against Fermi level value. Furthermore, when we increase the gate volatge, the Fermi level of the system get increased. This will locate to the the Fermi level on higher order Landau levels. As illustrated in Fig.~[4], this will result a higher conductivity peak value at higher order Landau level. This same behaviour also observed in the experimental observations presented in Ref. \cite{wakabayashi78} as well. However, in our analysis we have analytically explained the controllability of these conductivity regions using a dressing field.
  Lately, several experimental \cite{zudov01,mani02,zudov03,mani04} and theoretical \cite{durst03,dmitriev03,dmitriev05,dmitriev09} studies have uncovered a number of remarkable magnetotransport properties induced by microwave radiation on 2DEG quantum Hall systems. However these experiments and theoritical models only examined the behavior of MIROs in 2DEG systems which are based on photon absorbtion from the applied low-frequency field.
  It is important to state that the difference between our SdH oscillations and MIROs \cite{zudov01,mani02,zudov03,mani04} by considering the frequency range of the applied electromagnetic field. In our case, the applied dressing field is in off-resonant range and that will only change the the broadening of Landau levels and avoid any contribution towards the photon absorbtions. This will clearly describe the crusial difference between low-frequency illumination and high-frequency illumination effects on 2DEG quantum systems. Therefore our approach can help to fill the gap in knowledge of the high-frequency dressing field effects on 2DEG quantum Hall systems.
  }
\end{itemize}

\subsubsection*{Comment 3 -
\color{RoyalBlue} The manuscript will also provide more impact if it demonstrates how the new results can help to improve the future development of nanoelectronic devices. After these questions are addressed, the manuscript will be suitable for publication in Physical Review B. Otherwise, it will fit better to a more mathematically oriented journal.
}

We thank the reviewer for pointing out the importance of these facts that will help to elevate the value of our work. As we mentioned in the commnet 1, consideirng the important of discussing physical significance of our results, we added a new section(Section VII) to address these facts.

\begin{itemize}
  \item Section VII (page 8):\\
  {\color{Maroon}
  \subsection*{Physical significance of outcomes}

  With the realization of 2DEGs in Si-MOSFETs (Metal Oxide Semiconductor Field Effect Transistors) \cite{fowler66}, Klitzing \textit{et al.} \cite{klitzing80} made the first transport measurements on such systems to reveal the quantum Hall effect. The empirical discovery of these unusual properties marked the beginning of a whole new realm in condensed matter physics that continues to produce phenomenal advancements in electronic systems. The quantum Hall effects in a 2DEG under a static magnetic field are described by plateaus quantized to integer values of the conductivity quantum ($\flatfrac{e^2}{\hbar}$) in the off-diagonal conductivity, with simultaneously peaks at inter-plateau transition for the diagonal conductivity \cite{endo09}. This is due to the applied magnetic field and it changes the energy spectrum of 2DEG in a dramatic way. The magnetic field causes the density of states in 2DEG to split up into a sequence of delta functions, separated by an energy $\hbar\omega_0$, with $\omega_0$ the cyclotron frequency which depends on the applied magnetic field.
  However, experimental results demonstrate that these Landau levels are broadened and the main source of these broadening in low temperatures is the disorders in materials \cite{ando85,dial07}. This behavior implies the oscillating behavior of the experimental measurement of longitudinal conductivity which is known as Shubnikov–de Haas (SdH) oscillations. \cite{endo09,wakabayashi78}.

  Our theoretical analysis on longitudinal conductivity behavior of dressed quantum Hall system developed by considering low-temperature limit with gaussian impurity broadening assumptions. As illustrated in Fig. 4, we were also able to demonstrate the same SdH oscillations as experimental results \cite{endo09,wakabayashi78} through our model. Under the undressed ($I=0$) condition, our results overlap with the conductivity measurement for quantum Hall systems. However, from our results given in Fig. 5, we demonstrate that we can manipulate the broadening of these conductivity peaks using an external dressing field. In low temperatures, the principal cause of broadening of these conductivity peaks is impurity scattering and using an external dressing field we can suppress the scattering which results in shrinkage of both the scattering-induced broadening and the longitudinal conductivity.

  Research on novel states of matter has driven the evolution of present-day nanoelectronic devices. In particular, controllable manipulation of material properties through a gate electric field has revolutionized the development of material science and technology \cite{ahn03,deng18}.
  The charge carrier concentration of a considering system is an imperative parameter that defines the conductivity properties of the system. We can manipulate that using the electrostatic field-effect mechanism and it is an ideal tool to control the conductivity in some specific systems.
  A 2DEG under static magnetic field with quantum Hall effects is an excellent example that the gate electric field can be used to manipulate the conductivity. A considerable number of researches have been performed using different types of 2D field-effect transistors (FETs) in magnetic fields to study the electronic transport in the quantum limit \cite{wakabayashi78,yang18,long20,li14}. In the study done by Yang \textit{et al.} \cite{yang18}, the authors have observed quantized Hall plateaus and Shubnikov–de Haas oscillations for longitudinal conductivity against gate voltage in black phosphorus FET under high magnetic fields in low temperatures. Since the Fermi level of the system can be altered with applied gate voltage, this behavior can be easily mapped into our results given in Fig. 4. However, the specialty of our outcomes is we owned the capability of manipulating the broadening of the conductivity regions using an external dressing field. Although Yang \textit{et al.} \cite{yang18}, achieved this broadening manipulation by changing the temperature in a low range, in this study we present a general mathematical description to perform that using only a high-intensity electromagnetic field.

  The realization of the underlying mechanism of 2D FETs in the quantum realm promises its potential in next-generation nanoelectronic applications. In a particular application that uses the switching operation of the above-discussed FETs with quantum Hall effects, we can achieve high and low output conductivities by changing the input gate voltage. As a result of manipulating the broadening of conductivity regions, we can shrink the broadening of conductivity peaks around Landau levels using a high-intensity dressing field. This will enhance the sensitivity of FETs which provides the ability to observe narrow changes in gate voltage. Based on these available nanoelectronic devices and their feasible optimization, we believe that our mathematical description offers great potential to realize advanced nanoelectronic devices. Furthermore, this theoretical model will help to develop simulation tools that will design the quantum effects in magnetotransport properties of 2D nanostructures.
  }
\end{itemize}

\subsubsection*{Comment 4 -
\color{RoyalBlue} The quantum Quantum Hall effect requires high mobility samples. In these samples, the structure of the disorder is usually complicated and combines both short-length potentials of impurities and long-length electrostatic inhomogeneities. The interplay of these components of disorder opens exciting questions about the transport properties of 2DEGs. What is the structure of disorder considered in the present manuscript and hidden in the notations for Vimp? What are the conditions for validity of eq. (15)?
}

We thank the reviewer for raising this important question. In our previous manuscript, we have presented the detailed derivation of Eq.~[15] with a discusssion on models of disorder under the Appendix C. However we will discuss on our selection of disorder model and approximations made to derive the Eq.~[15] here as well.

We modeled the effect caused by impurities in the considered system as a single perturbation potential. Analyzing the electric propeties for a specific impurity configuration is a rather formidable task and is not examined in this work since it is unlikely to have exactly the evaluated impurity configuration in an experiment. Therefore, in this study we consider on the statistically averaged propeties of 2DEG over impurity configurations. In addition, we consider only the our considering static disorder corresponds to the situation in which the electrons scatter elastically. First, we adapt the Edwards model \cite{akkermans10} to represent the randomly distributed impurities over the considering system and we approximate this into a Gaussian white noise.

Since we are presenting the the perturbation potential $V(\vb{r})$ by a group of randomly localized impurities, we take into account $N_{imp}$ number of identical single impurity potentials distributed at randomly but in fixed positions $\vb{r}_i$. Thus, we can describe the scattering potential $V(\vb{r})$ as the sum of uncorrelated single impurity potentials $\upsilon(\vb{r})$
\begin{equation} \label{eq:1}
  V(\vb{r}) =
  \sum_{i=1}^{N_{imp}}
  \upsilon (\vb{r}-\vb{r}_i).
\end{equation}
Furthermore, we model the perturbation $V(\vb{r})$ as a Gaussian random potential where one can choose the zero of energy such that the potential is zero on average. This model is characterized by the following two equations \cite{akkermans10}
\begin{subequations}
\begin{equation} \label{eq:2}
  \expval{\upsilon(\vb{r})}_{imp} =0,
\end{equation}
\begin{equation} \label{eq:3}
  \expval{\upsilon(\vb{r})\upsilon(\vb{r'})}_{imp} = \Upsilon(\vb{r}-\vb{r'}),
\end{equation}
\end{subequations}
where $\expval{\cdot}_{imp}$ represents the average over the impurity disorder and $\Upsilon(\vb{r}-\vb{r'})$ is any decaying function depends only on $\vb{r}-\vb{r'}$. In addition, this model assumes that $\upsilon (\vb{r}-\vb{r'})$ only depends on the magnitude of the position difference $|\vb{r}-\vb{r'}|$, and it decays with a characteristic length $r_c$. Since this study considers the case where the wavelength of radiation or scattering electron is much greater than $r_c$, it is a better approximation to make two-point correlation function to be
\begin{equation} \label{eq:4}
  \expval{\upsilon(\vb{r})\upsilon(\vb{r'})}_{imp} = \Upsilon_{imp}^2\delta(\vb{r}-\vb{r'}),
\end{equation}
where $\Upsilon_{imp}^2$ is a constant. A random potential $V(\vb{r})$ with this property is called white noise \cite{akkermans10}. Then we can approximately model the total scattering potential as
\begin{equation} \label{eq:5}
  V(\vb{r}) =
  \sum_{i=1}^{N_{imp}}
  \Upsilon_{imp} \delta(\vb{r}-\vb{r}_i).
\end{equation}

Then we calculate the Floquet-Fermi golden rule for a dressed quantum Hall system using this perturbative impurity potential. In the derivation we have defined the $V_{imp}$ value which is constant in moemntum space. It is a propety of Gaussian white noise impurity distribution \cite{akkermans10,wackerl20}. However, throughout the derivation we use only the first order contribution (Born approximation) of the impurity potential. All the detailed step for the derivation of the Floquet-Fermi golden rule for a dressed quantum Hall system are included in Appendix C.

Since previous studies on Floquet-Drude consuctivity \cite{wackerl20}, and magnetotransport propeties in dressed qunatum Hall systems \cite{dini16} have  used this perticular Gaussian white noise potential, we also selected this pertucular impurity model to describe our system. This open a path to compare our analytical outcomes with these previous models. As you have mentioned in the comment, consideration of other impurity disorder models and their impact on the magnetotransport properties of dressed qunatum Hall system would be an interesting future research possibility.

Since we have not mentioned the validity conditions we used to derive the Eq.~[15] in the main text of the previous manuscript, we have added validity conditions to the new manuscript.

\begin{itemize}
  \item Section IV - first paragraph (page 4):\\
  {\color{Maroon}
  The Floquet-Fermi golden rule was proposed in Ref. \cite{wackerl20} as an approach to analyze the transport properties of dressed quantum systems with impurities.
  However, this theory has not been applied for a dressed quantum Hall system in the previous studies. In this analysis, we use Floquet-Fermi golden rule to identify the effects induced by impurities on the magneto-transport properties.
  With the help of $t-t'$ formalism \cite{wackerl20,grifoni98,sambe75,peskin93,althorpe97} and applying Floquet states derived in Eq.~(14), we can derive an  expression for $(l,l')$-th element of the inverse scattering time matrix for the $N$-th Landau level as
  \begin{equation} \label{eq:15}
    \begin{aligned}
      \left(\frac{1}{\tau(\varepsilon,k_x)}\right)^{ll'}_N =
      \frac{ \pi \hbar\varrho^2}{e^2B^2} \delta(\varepsilon - \varepsilon_N)
      \int_{-\infty}^{\infty} &
      J_l \bm{\left(} \frac{b\hbar}{eB}({k}_x - k_1) \bm{\right)}
      J_{l'} \bm{\left(} \frac{b\hbar}{eB}({k}_x - k_1) \bm{\right)}
      \\
      & \times
      \left|
      \int_{-\infty}^{\infty}
      \chi_N \left( \frac{\hbar}{eB}k_2 \right)
      \chi_N \bm{\left(} \frac{\hbar}{eB}
      \left( k_1 - {k}_x - k_2 \right) \bm{\right)}
      dk_2 \right|^2 d k_1,
    \end{aligned}
  \end{equation}
  where $\varrho = \eta_{imp} L_x \left[\flatfrac{ V_{imp}}{\pi}\right]^{1/2}$, $\varepsilon$ is a given energy value, $J_l(\cdot)$ are Bessel functions of the first kind with $l$-th integer order, and $\varepsilon_N$ is the energy of the $N$-th Landau level.
  A more detailed derivation is given in Appendix C.
  We modeled the effect caused by impurities in the considered system as a single perturbation potential. Analyzing the electric propeties for a specific impurity configuration is a rather formidable task and is not examined in this work since it is unlikely to have exactly the evaluated impurity configuration in an experiment.
  Furthermore, we assumed that our perturbation potential is formed by a group of randomly distributed impurities under the Edwards impurity model \cite{akkermans10,wackerl20}.
  Thus, we represented the total scattering potential in the 2DEG as a sum of uncorrelated single impurity potentials $\upsilon(\vb{r})$. Then we  approximate the impurity potential into a Gaussian white noise \cite{akkermans10,wackerl20}.
  Here $\eta_{imp}$ is the number of impurities in a unit area, $V_{imp} = \expval{|V_{{k'}_x,k_x}|^2}_{imp}$ with $V_{{k'}_x,k_x} = \mel**{k'_x}{\upsilon(x) }{k_x}$, and $\braket{x}{k_x} = e^{-ik_x x}$.
  Moreover, in this analysis, $\expval{\cdot}_{imp}$ represents the average over the impurity disorder. In this derivation, we only considered the first order (the Born approximation) contribution from the impurity potential and
  }
\end{itemize}

\subsubsection*{Comment 5 -
\color{RoyalBlue} If continuous illumination is applied to an electron system, the system will heat indefinitely. The distribution function is stabilized when the electron-electron and electron-phonon scattering mechanisms are included. What are the conditions of validity for eqs. (33-34)? The deviations of the distribution function from its equilibrium result in effects that can significantly overcome the equilibrium contributions, see Dmitriev et al, Phys. Rev. B 80, 165327 (2009). However, to capture these effects, a treatment within Floquet-Drude formalism is insufficient and a complete quantum kinetic equation has to be analyzed. Can the authors argue why their results still present some interest to the community even if they potentially disregard more significant contributions?
}

This is a good, valid question. First, we discuss the validity of the Eq.~[33] and Eqq.~[34]. In our analysis we have derived the longitudinal conductivity of a dressed quantum Hall system without any partial distribution function specification until Eq.~[31]. Then we selected an special scenario where we assumed that the system is under low temperature limit. Since previous theoritical studies on magnetotransport propeties of 2DEG quantum Hall systems were derived under this assumption, we selected this conditions to provide a better comparison with their outcomes. In addition, as mentioned in previous experimental studies on SdH oscillations \cite{zudov03,mani02,arapov02} in 2DEG undressed quantum Hall systems with increasing tempurature, we can identify the decaying behaviour of SdH oscillations. Therefore, if we need to maintain a good oscillator behavior we need to work in low temperatures. Since in our analysis, we study the manipulation of SdH oscillations using a dressing field, we have extend our general derivation in given Eq.~[31] into a simple expression for low temperature case. On the other hand, under the off-resonant condition, absorption of photons is avoided, and we can choose the the Fermi-Dirac distribution as our partical distribution function in Eq.~[31] as given in Eq.~[33]. Under low temperature assumption, we can assume that the derivative of the Fermi-Dirac distribution is sharply peaked around the Fermi energy as given mentioned in the Eq.~[34]. Applying these two conditions, we can simplify our derivation into new expression that can be compared with previous studies on magnetotransport in dressed quantum Hall systems.

Yes, it is true that when we consider the experimental observations of a qunatum Hall system a strong dressing field can increase the temperature of the sample. However, as mentioned in previous study by Dini \textit{et al.} \cite{dini16} and Kibis \textit{et al.} \cite{kibis15}, we can maintain low temperature by applying narroaw pulses of a strong dressing field to observe the discussed phenomena.

Although we have extend our general derivation given in Eq.~[31] into special consideration for comparison, we can improve our theoritical model into different type of partical distribution functions. As discussed by Dmitriev \textit{et al.} \cite{dmitriev05}, a well-defined analytical discussion need to address the changes in distribution functions in our derived general longitudinal conductivity formula and this would be an interesting extension to our study.

\subsubsection*{Comment 6 -
\color{RoyalBlue} The authors presented the results for the conductivity tensor's xx- and yy-components. They seem to be identical, apart of a dimensional factor $(eB)^2$ in the denominator. The equality of these two components is expected for isotropic systems. Does the polarization of the electromagnetic field break the isotropy? Do the two components of the conductivity remain equal even for an arbitrary direction of a linearly polarized field? Is there a non-Hall contribution to the xy-component?
}

We thank the reviewer for pointing out the importance mistake we have made in the $yy$-component calculation. We have made an mistake in the current operator calculations and this has leaded to a unexpected normalization factor in the conductivity expresiion. Since we have only analyzed the $xx$-component this is not used to any further calculations and remaining analysis is not effected by this change. After this changes we can predict that the both $xx$-component and $yy$-component shows the same conductivity behaviours. We added these changes to the calculations done in Appendix D and the conductivity calculation under Section V.

\begin{itemize}
  \item Appendix D - (page 14):\\
  {\color{Maroon}
  \subsection*{\label{appendix_d}Current operators for a dressed quantum Hall system}

  In this section, we derive the current density operator for the $N$-th Landau level in a dressed quantum Hall system. We already found the exact solution for the time-dependent Schrödinger equation with the Hamiltonian give in Eq.~(1) and we identified them as the Floquet states in Eq.~(14). For the simplicity of notation, we can represent the Floquet modes derived in Eq.~(10) as quantum states using their corresponding quantum numbers as follows
  \begin{equation} \tag{D1}
    \ket{\phi_{n,m}} = \ket{n,k_x}.
  \end{equation}
  Using this complete set of quantum states \cite{wackerl20,holthaus15,grifoni98}, we can represent the single particle current operator's matrix element as
  \begin{equation} \tag{D2}
    \left(\vb{j} \right)_{nm,n'm'} = \mel{n,k_x}{\;\hat{\vb{j}}\;}{n',k'_x}.
  \end{equation}
  Next, we can identify the particle current operator for our system \cite{mahan00,bruus04} as
  \begin{equation} \tag{D3}
    \hat{\vb{j}} = \frac{1}{\widetilde{m}} \left\{\hat{\vb{p}} - e\left[\vb{A}_s + \vb{A}_d(t) \right]\right\},
  \end{equation}
  where $\widetilde{m}$ is the mass of the considering particle.

  First, we consider the $x$-directional particle current operator component, and we can identify that as
  \begin{equation} \tag{D4}
    \hat{j}_x = \frac{1}{\widetilde{m}} \left(-i\hbar\pdv{x} + eBy \right).
  \end{equation}
  Next, we calculate the matrix elements of $x$-directional current operator against our Floquet mode basis
  \begin{equation} \tag{D5}
    \left({j}_x \right)_{nm,n'm'} =
    \mel{n,k_x}{\;\frac{1}{\widetilde{m}} \left(-i\hbar\pdv{x} + eBy \right)}{n',k'_x},
  \end{equation}
  and we evaluate these using the Floquet modes derived in Eq.~(7) as follows
  \begin{equation} \tag{D6}
    \begin{aligned}
      \left({j}_x \right)_{nm,n'm'} = &
      \frac{1}{{\widetilde{m}}}
      \delta_{k_x,k'_x}
      \int
      \left(\hbar k'_x + eBy\right) \\
      & \times
       \chi_{n}\bm{\left(}y - y_0 - \zeta(t)\bm{\right)}
      \chi_{n'}\bm{\left(}y - y_0 - \zeta(t)\bm{\right)}
      dy.
    \end{aligned}
  \end{equation}
  Let $[y - y_0 - \zeta(t)] = \bar{y}$ and we can obtain
  \begin{equation} \tag{D7}
    \begin{aligned}
      \left({j}_x \right)_{nm,n'm'} =
      \frac{1}{{\widetilde{m}}}
      \delta_{k_x,k'_x}
      \int &
      \left[ \hbar k'_x + eB\bar{y} -\hbar k'_x + eB\zeta(t)\right]
      \chi_{n}(\bar{y})
      \chi_{n'}(\bar{y})
      d\bar{y}.
    \end{aligned}
  \end{equation}
  Using the following integral identities of the Floquet modes which are made up with Gauss-Hermite functions \cite{vedenyapin11,szego59}
  \begin{equation} \tag{D8}
    \int
    \chi_{n}({y})
    \chi_{n'}({y}) d{y} =
    \kappa\delta_{n',n},
  \end{equation}
  \begin{equation} \tag{D9}
    \int
    y \chi_{n}({y})\chi_{n'}({y}) d{y} =
    \left(\sqrt{\frac{n+1}{2}} \delta_{n',n+1} + \sqrt{\frac{n}{2}}
    \delta_{n',n-1} \right),
  \end{equation}
  we simplify the matrix elements of $x$-directional current operator to obtain
  \begin{equation} \tag{D10}
    \begin{aligned}
      &\left({j}_x \right)_{nm,n'm'} =
      \frac{eB}{{\widetilde{m}}}
      \delta_{k_x,k'_x}
      \left[
      \left(\sqrt{\frac{n+1}{2}} \delta_{n',n+1} + \sqrt{\frac{n}{2}}
      \delta_{n',n-1}\right)
      + \zeta(t) \delta_{n',n}
      \right].
    \end{aligned}
  \end{equation}
  Due to high complexity of extract solution, in this study we only consider the constant contribution. Therefore, we can identify the $0$-th component of the Fourier series as
  \begin{equation} \tag{D11}
    \begin{aligned}
        \left({j}_x \right)_{nm,n'm'} =&
        \frac{eB}{\widetilde{m}}
        \delta_{k_x,k'_x}
        \left(\sqrt{\frac{n+1}{2}} \delta_{n',n+1} + \sqrt{\frac{n}{2}}
        \delta_{n',n-1} \right).
    \end{aligned}
  \end{equation}
  For a electric current operator, we can apply the electron's charge and the  effective mass of the electron to the above derived equation. This leads to
  \begin{equation} \tag{D12}
    \begin{aligned}
        \Big({j}^x_{s=0}\Big)_{nm,n'm'}^{electron}  =&
        \frac{e^2B}{{m_e}}
        \delta_{k_x,k'_x}
        \left(\sqrt{\frac{n+1}{2}} \delta_{n',n+1} + \sqrt{\frac{n}{2}}
        \delta_{n',n-1} \right).
    \end{aligned}
  \end{equation}

  Moreover, we can identify the $y$-directional current operator component as
  \begin{equation} \tag{D13}
    \hat{j}_y = \frac{1}{\widetilde{m}} \left(-i\hbar\pdv{y} - \frac{eE}{\omega}\cos(\omega t) \right).
  \end{equation}
  Using this operator, we can represent the matrix elements of $y$-directional current operator in Floquet mode basis as
  \begin{equation} \tag{D14}
    \left({j}_y \right)_{nm,n'm'} =
    \mel{n,k_x}{\;\frac{-1}{\widetilde{m}} \left(i\hbar\pdv{y} + \frac{eE}{\omega}\cos(\omega t) \right)}{n',k'_x}.
  \end{equation}
  After following the same steps done for the $x$-directional current operator, and recursion relation of the first derivative of Gauss-Hermite functions
  \begin{equation} \tag{D15}
    \begin{aligned}
      \pdv{\chi_{n}({y})}{y} =
      \kappa\left[
      -  \sqrt{\frac{n+1}{2}} \chi_{n+1}({y})
      +  \sqrt{\frac{n}{2}} \chi_{n - 1}({y})
      \right],
    \end{aligned}
  \end{equation}
  we can identify the $0$-th component of matrix elements for $y$-directional electric current operator as
  \begin{equation} \tag{D16}
    \begin{aligned}
      \Big({j}^y_{s=0}\Big)_{nm,n'm'}^{electron} = &
      \frac{ie\hbar\kappa^2}{m_e}
      \delta_{k_x,k'_x}
      \left(
      \sqrt{\frac{n}{2}} \delta_{n',n-1}
      - \sqrt{\frac{n+1}{2}} \delta_{n',n+1}
      \right).
    \end{aligned}
  \end{equation}
  \begin{equation} \tag{D17}
    \begin{aligned}
      \Big({j}^y_{s=0}\Big)_{nm,n'm'}^{electron} = &
      \frac{ie^2 B}{m_e}
      \delta_{k_x,k'_x}
      \left(
      \sqrt{\frac{n}{2}} \delta_{n',n-1}
      - \sqrt{\frac{n+1}{2}} \delta_{n',n+1}
      \right).
    \end{aligned}
  \end{equation}
  }
  \item Section V - (page 7):\\
  {\color{Maroon}
  Following the same steps as above derivation, we can derive the longitudinal conductivity in the $y$-direction by applying the electric current operator for $y$-direction derived in Appendix D
  \begin{equation} \tag{36}
    \begin{aligned}
      {\sigma}^{yy} =
      \frac{e^2}{\pi\hbar A} &
      \sum_{n}
      \frac{(n+1)}{\gamma_{n}\gamma_{n+1}}
      \left[
        \frac{1}
        {
          1 + \left(\frac{X_F - n -1}{\gamma_{n+1}}\right)^2
        }
      \right]
      \left[
        \frac{1}
        {
          1 + \left(\frac{X_F - n}{\gamma_{n}}\right)^2
        }
      \right].
    \end{aligned}
  \end{equation}
  }
\end{itemize}

% For analysing our system, we mainly employ the Floquet-Drude conductivity formula introduced by Wackerl \textit{et al.} \cite{wackerl20}. The Floquet-Drude conductivity formula was only developed for the diogonal components of the conductivity tensor of the considering system. In this analysis we were only focused on SdH oscillations in longitudinal conductivity components we analysed the dressing field effect on two diogonal components of the conductivity tensor.
%
% As mentioned in Ref.\cite{wackerl20} and our work, applied strong dressing field tends to change the qunatum state of the consideirng system and creates novel states called Floquet states. The propeties of these states are depends on the characteristics of applied dressing field. Therefore, the polarization method also change the behavior of the Floquet states.
% These polarization dependent conductivity behaviors in 2DEG systems can be found in references \cite{wackerl20,morina15} and in their work authors have illustrated the conductivity behavior with circular and linear polarized dressing fields. In addition, they have mentioned the impact on the direction of the linearly polarization dressing field and this will break the isotropy behaviour of the longitudinal conductivities in the consideirng system.
%
% We also followed the same formalism of Floquet states for 2DEG quantum Hall systems. In our work also we were able to identify that the applied dressing field's polarization direction can be effect the magnitude of the longitudinal conductivity component. A detailed derivation of these two longitudinal conductivity component's current operators are given in Appendix D. If we applied a arbitary directional polarized dressing field, we can derive a expression for the longitudinal conductivity components by considering dressing field contribution for each direction ($x$ and $y$) with same steps in this analysis. This would change the amplitude of the considering dressing field ($E$) in each direction. However, now there will be a new conponent in parallel to the considering conductivity component and we need to solve the Schrödinger equation with the modified Hamiltonian in Eq.~[2] for the new system with these changes.
% However in our analysis, our main objective was to identify the manipulation capabilities of SdH oscillations using dressing field intensity. In addition, we only consider the longitudinal components in this analysis by extending the Floquet-Drude conductivity formula that derived only for the diogonal components in conductivity tensor. Therefore we are unable to predict the off-diogonal component behavior of the system through our analysis and we need a novel formalism to handle these type of behaviors of dressed 2DEG quanutm Hall systems.

\newpage
\subsection*{Response to the comments of Reviewer 2}

We would like to thank the reviewer for his/her insightful comments our work. To address your comments, we had to add new materials and reorganize our text. As a result we have been able to considerably improve our discussion on the underlying assumptions and corresponding physical significance of our analysis. We hence believe that both the presentation and scientific content of this manuscript is of a much higher quality than our previous submission. We hope our response, and the corresponding changes we incorporated to the manuscript would be sufficient to clarify the issues raised.


\subsubsection*{Comment 1 -
\color{RoyalBlue} The paper is very low on comparison with experiments, for which there is a large amount of data available for the zero radiation case. I would like to see detailed analysis of how Figs 4 and 5 compare with available
experimental data. Without this addition, physical relevance of such
detailed calculations is questionable.
}

We thank the reviewer for pointing out the importance of this fact. To overcome this shortage of discussion we made a detailed analysis on our results against the undressed 2DEG quantum Hall system behavior. We have inserted this discussion into the our manuscript under Section VI.

\begin{itemize}
  \item Section VI - fifth paragraph (page 8):\\
  {\color{Maroon}
  By comparing the theoritical \cite{ando72,ando74_1,ando74_2,ando74_3,ando74_4,ando82,endo09} and experimental \cite{endo09,wakabayashi78,ochiai190,mancoff96,arapov02,grbic04,caviglia10} studies on magnetoresistance of 2DEG qunatum Hall systems under zero radiation with our results, we can identify that longitudinal conductivity oscillations in Fig.~[4] are repetition of the Shubnikov–de Haas(SdH) oscillations.
  As observed in the experimental work by Caviglia \textit{et al.} \cite{caviglia10}, SdH oscillations period is only depend on the perpendicular component of the applied magnetic field.
  Therefore, we can identify that many experiments done in low temperatures \cite{endo09,ochiai190,mancoff96,arapov02,grbic04,caviglia10} with different type of 2DEG has been oberved these SdH oscillations against applied magnetic field.
  The magnetic field's perpendicular component amplitude defines the cyclotron frequency ($\omega_0$) of the consdiering system.
  Therefore we can achieve the same oscillation behavior for longitudinal conductivity in 2DEG quantum Hall system by changing the gate voltage. Due to the change of the gate volatge applied to the system will modify the Fermi level of the system.
  Since the Landau level energy is only depend on the cyclotron frequency, this will generate the same SdH oscillations against gate volatage.
  This osillating behavior with applied gate volatge has been observed in experiment done by Wakabayashi \textit{et al.} \cite{wakabayashi78} in an inversion layer on a silicon surface at low temperature. With comparing these empirical observations with our results given in Fig[4], we can easly identify that our oscillations are also shows the same characteristic behavior with $\hbar\omega_0$ periodicity against Fermi level value. Furthermore, when we increase the gate volatge, the Fermi level of the system get increased. This will locate to the the Fermi level on higher order Landau levels. As illustrated in Fig.~[4], this will result a higher conductivity peak value at higher order Landau level. This same behaviour also observed in the experimental observations presented in Ref. \cite{wakabayashi78} as well. However, in our analysis we have analytically explained the controllability of these conductivity regions using a dressing field.
  Lately, several experimental \cite{zudov01,mani02,zudov03,mani04} and theoretical \cite{durst03,dmitriev03,dmitriev05,dmitriev09} studies have uncovered a number of remarkable magnetotransport properties induced by microwave radiation on 2DEG quantum Hall systems. However these experiments and theoritical models only examined the behavior of MIROs in 2DEG systems which are based on photon absorbtion from the applied low-frequency field.
  It is important to state that the difference between our SdH oscillations and MIROs \cite{zudov01,mani02,zudov03,mani04} by considering the frequency range of the applied electromagnetic field. In our case, the applied dressing field is in off-resonant range and that will only change the the broadening of Landau levels and avoid any contribution towards the photon absorbtions. This will clearly describe the crusial difference between low-frequency illumination and high-frequency illumination effects on 2DEG quantum systems. Therefore our approach can help to fill the gap in knowledge of the high-frequency dressing field effects on 2DEG quantum Hall systems.
  }
\end{itemize}

\subsubsection*{Comment 2 -
\color{RoyalBlue} In the presence of disorder, Anderson localization and topologically protected edge modes are dominant considerations when calculating DC transport coefficients. How are these factors taken into account?
}

We agree with the reviewer that the discussion on the Anderson localization and topologically protected edge modes are was missing from the manuscript.
We would like to mentioned the reviewer that while deriving Eq.~[31], we have assumed that the applyed static magnetic field is placed in a lower intensity range where we can neglect a number of effects known to appear in a 2DEG quantum Hall system. Under this condition we can neglect the localization in the tails of Landau level peaks, the formation of the edge states, the electron-electron interactions and the spin splitting as well. As mentioned in the work done by Endo \textit{et al.} \cite{endo09}, under the given magnetic field range, the undressed magnetotransport properties of GaAs/AlGaAs 2DEG quantum Hall system provides excellent agrrement between the experimental observations and theory. The authors of the Ref.\cite{endo09} has neglect the effects mentioned above during the development the theory. We also follwed the same theoritical derivation with given assumtions but with a additionl dressing field. Our results for undressed system gives a better aggrement with the results from Ref.\cite{endo09}, while we provide a better description to manipulate the width of the Landau level conductivity peaks. We have added this clarification in the manuscript.

\begin{itemize}
  \item Section I - sixth paragraph (page 2):\\
  {\color{Maroon}
  It is important to note that, in this analysis we consider the applied magnetic field at a range where the Anderson localization, the formation of the edge states, the spin splitting and the electron–electron interactions can be neglected. The use of this assumption can also be found in study done by  Endo \textit{et al.} \cite{endo09}.
  }
  \item Section VIII - sixth paragraph (page 2):\\
  {\color{Maroon}
  We assumed the impurities in the material as a Gaussian random scattering potential and the applied magnetic field within a order where we can neglect effects of the the Anderson localization, the formation of the edge states, the spin splitting and the electron–electron interactions.
  }
\end{itemize}


\subsubsection*{Comment 3 -
\color{RoyalBlue} Can the authors reproduce the quantized Hall conductance of filled Landau levels in the presence of disorder? Please discuss.
}



\subsubsection*{Comment 4 -
\color{RoyalBlue} The way some previous works are cited is disappointing. Example from page 8: “Despite this behavior being identified in previous works, their results did not coincide with the more accurate description of conductivity components in undressed quantum Hall systems.” This is not informative. Where, specifically, did results from earlier works fall short of reality, which have been better addressed in this work?
}

\subsubsection*{Comment 5 -
\color{RoyalBlue} Adding some physical insight into the remarkable observation of radiation-induced narrowing of lineshapes (Figs 4, 5) will help elevate
this work.
}

We agree with the reviewer that a discussion on our theoritical results and their physical significance into modern nanoelectronic devices is a vital requirement. We have added a new Section VII to overcome the above mentioned deficiency of our manuscript. The total content of the section is given below,

\begin{itemize}
  \item Section VII (page 8):\\
  {\color{Maroon}
  \subsection*{Physical significance of the outcomes}

  With the realization of 2DEGs in Si-MOSFETs (Metal Oxide Semiconductor Field Effect Transistors) \cite{fowler66}, Klitzing \textit{et al.} \cite{klitzing80} made the first transport measurements on such systems to reveal the quantum Hall effect. The empirical discovery of these unusual properties marked the beginning of a whole new realm in condensed matter physics that continues to produce phenomenal advancements in electronic systems. The quantum Hall effects in a 2DEG under a static magnetic field are described by plateaus quantized to integer values of the conductivity quantum ($\flatfrac{e^2}{\hbar}$) in the off-diagonal conductivity, with simultaneously peaks at inter-plateau transition for the diagonal conductivity \cite{endo09}. This is due to the applied magnetic field and it changes the energy spectrum of 2DEG in a dramatic way. The magnetic field causes the density of states in 2DEG to split up into a sequence of delta functions, separated by an energy $\hbar\omega_0$, with $\omega_0$ the cyclotron frequency which depends on the applied magnetic field.
  However, experimental results demonstrate that these Landau levels are broadened and the main source of these broadening in low temperatures is the disorders in materials \cite{ando85,dial07}. This behavior implies the oscillating behavior of the experimental measurement of longitudinal conductivity which is known as Shubnikov–de Haas (SdH) oscillations. \cite{endo09,wakabayashi78}.

  Our theoretical analysis on longitudinal conductivity behavior of dressed quantum Hall system developed by considering low-temperature limit with gaussian impurity broadening assumptions. As illustrated in Fig. 4, we were also able to demonstrate the same SdH oscillations as experimental results \cite{endo09,wakabayashi78} through our model. Under the undressed ($I=0$) condition, our results overlap with the conductivity measurement for quantum Hall systems. However, from our results given in Fig. 5, we demonstrate that we can manipulate the broadening of these conductivity peaks using an external dressing field. In low temperatures, the principal cause of broadening of these conductivity peaks is impurity scattering and using an external dressing field we can suppress the scattering which results in shrinkage of both the scattering-induced broadening and the longitudinal conductivity.

  Research on novel states of matter has driven the evolution of present-day nanoelectronic devices. In particular, controllable manipulation of material properties through a gate electric field has revolutionized the development of material science and technology \cite{ahn03,deng18}.
  The charge carrier concentration of a considering system is an imperative parameter that defines the conductivity properties of the system. We can manipulate that using the electrostatic field-effect mechanism and it is an ideal tool to control the conductivity in some specific systems.
  A 2DEG under static magnetic field with quantum Hall effects is an excellent example that the gate electric field can be used to manipulate the conductivity. A considerable number of researches have been performed using different types of 2D field-effect transistors (FETs) in magnetic fields to study the electronic transport in the quantum limit \cite{wakabayashi78,yang18,long20,li14}. In the study done by Yang \textit{et al.} \cite{yang18}, the authors have observed quantized Hall plateaus and Shubnikov–de Haas oscillations for longitudinal conductivity against gate voltage in black phosphorus FET under high magnetic fields in low temperatures. Since the Fermi level of the system can be altered with applied gate voltage, this behavior can be easily mapped into our results given in Fig. 4. However, the specialty of our outcomes is we owned the capability of manipulating the broadening of the conductivity regions using an external dressing field. Although Yang \textit{et al.} \cite{yang18}, achieved this broadening manipulation by changing the temperature in a low range, in this study we present a general mathematical description to perform that using only a high-intensity electromagnetic field.

  The realization of the underlying mechanism of 2D FETs in the quantum realm promises its potential in next-generation nanoelectronic applications. In a particular application that uses the switching operation of the above-discussed FETs with quantum Hall effects, we can achieve high and low output conductivities by changing the input gate voltage. As a result of manipulating the broadening of conductivity regions, we can shrink the broadening of conductivity peaks around Landau levels using a high-intensity dressing field. This will enhance the sensitivity of FETs which provides the ability to observe narrow changes in gate voltage. Based on these available nanoelectronic devices and their feasible optimization, we believe that our mathematical description offers great potential to realize advanced nanoelectronic devices. Furthermore, this theoretical model will help to develop simulation tools that will design the quantum effects in magnetotransport properties of 2D nanostructures.
  }
\end{itemize}


\newpage
\bibliography{response}

\bigskip
\bigskip

Sincerely yours,

\def\s#1#2#3{\vbox{\hsize=4.5cm
		\kern2cm
		\hrule\kern1ex
		\hbox to \hsize{\strut\hfil #1 \hfil}
		\hbox to \hsize{\strut\hfil #2 \hfil}
		\hbox to \hsize{\strut\hfil #3 \hfil}}}

\hbox to \hsize{\s{Malin Premaratne}{(Corresponding Author)}{\href{malin.premaratne@monash.edu}{malin.premaratne@monash.edu}}}


\end{document}

% ****** End of file dressed_quantum_hall.tex ****** %
