\subsection{Position space representation}

First we define the time integral of the Lagrangian of the classical oscillator given in Eq.~(\ref{eq:5}), over a period $T=2\pi/\omega$ as
\begin{equation} \label{eq:b1}
  \Delta_{\varepsilon} = \frac{1}{T} \int_0^T L(\zeta,\dot{\zeta},t') dt'.
\end{equation}
Additionally, performing this integral, we can obtain a more simplified result
\begin{equation} \label{eq:b2}
  \Delta_{\varepsilon} = \frac{e^2E^2}{4m_e(\omega_0^2 - \omega^2)}.
\end{equation}
Next, we define another parameter
\begin{equation} \label{eq:b3}
  \xi =
  \int_0^t \; L(\zeta,\dot{\zeta},t') dt' -
  \Delta_{\varepsilon} t,
\end{equation}
and after simplifying, we can identify
\begin{equation} \label{eq:b4}
  \xi =
  \frac{e^2E^2\qty(3\omega^2 - \omega_0^2)}{8m_e\omega(\omega_0^2 - \omega^2)^2} \sin(2\omega t),
\end{equation}
which is a periodic function in time. Using these parameters, we can factorize the wave function given in Eq.~(\ref{eq:2}) as linearly time-dependent part and periodic time-dependent part as follows
\begin{equation} \label{eq:b5}
  \begin{aligned}
    \psi_{\alpha}&(x,y,t) \\
    & =
    \exp(\frac{i}{\hbar} \left(-\epsilon_nt + \Delta_{\varepsilon} t \right) )
    \frac{1}{\sqrt{L_x}} \chi_n\bm{\left(}y - y_0 - \zeta(t)\bm{\right)}
    \\
    & \quad\times
    \exp \left(
       \frac{i}{\hbar}
       \left\{
       p_x x +
       \frac{eE(y-y_0)}{\omega}\cos(\omega t)  \right. \right. \\
    & \left. \left. \qquad\qquad\qquad\qquad
    + m_e\dot{\zeta}(t) \left[ y-y_0 -\zeta(t) \right]
    + \xi \right\}
     \right).
  \end{aligned}
\end{equation}
This leads to separate the linear time-dependent phase component as the quasienergies
\begin{equation} \label{eq:b6}
  \varepsilon_{n} =
  \hbar \omega_0\qty(n + \frac{1}{2}) - \Delta_{\varepsilon},
\end{equation}
while rest of the components as time-periodic Floquet modes
\begin{equation} \label{eq:b7}
  \begin{aligned}
    \phi_{n,m}(x,y,t) =  &
    \frac{1}{\sqrt{L_x}} \chi_n\bm{\left(}y - y_0 - \zeta(t)\bm{\right)} \\
    & \times
    \exp \left(
       \frac{i}{\hbar}
       \left\{
       p_x x +
       \frac{eE(y-y_0)}{\omega}\cos(\omega t)  \right. \right. \\
    & \left. \left. \qquad\qquad\quad
    + m_e\dot{\zeta}(t) \left[ y-y_0 -\zeta(t) \right]
    + \xi \right\}
     \right).
  \end{aligned}
\end{equation}

\subsection{Momentum space representation}

We perform continuous Fourier transform over the considering confined space $A=L_xL_y$ on the Floquet modes given in Eq.~(\ref{eq:7}) to realize the Floquet modes in momentum space
\begin{equation} \label{eq:b8}
  \begin{aligned}
    \phi_{n,m}&(k_x,k_y,t) \\
    & =
    \exp(\frac{-i\gamma(t)}{\hbar}y_0)
    \exp(\frac{-i}{\hbar}\left[m_e \dot{\zeta}(t) \zeta(t) - \xi \right]) \\
    & \quad\times
    \int_{-L_y/2}^{L_y/2} \exp(-i \left[k_y - \gamma(t) \right]y)
      \chi_n \bm{\left(}y - \mu(t)\bm{\right)} dy \\
    & \quad\times
    \frac{1}{\sqrt{L_x}} \int_{-L_x/2}^{L_x/2}
     \exp(-ik_x x)\exp( \flatfrac{i p_x }{\hbar}x ) dx.
  \end{aligned}
\end{equation}
Here we used new two parameters
\begin{equation} \label{eq:b9}
  \mu(t) = \frac{eE\sin(\omega t)}{m_e(\omega_0^2 - \omega^2)} + y_0,
\end{equation}
and
\begin{equation} \label{eq:b10}
  \gamma(t) =
  \frac{eE\omega_0^2\cos(\omega t)}{\hbar\omega(\omega_0^2 - \omega^2)}.
\end{equation}
Subsequently, using the Fourier transform identity \cite{bruus04}
\begin{equation} \label{eq:b11}
  \int_{-L_x/2}^{L_x/2}
  \exp( -ik_x x + \flatfrac{i p_x x}{\hbar}) dx =
  L_x \delta_{k_x,\flatfrac{p_x}{\hbar}},
\end{equation}
we can derive
\begin{equation} \label{eq:b12}
  \begin{aligned}
    \phi_{n,m}(k_x,k_y,t)  =
    \Phi_{n,m}&(k_y,t)
    \delta_{k_x,\flatfrac{p_x}{\hbar}}
    \exp(\frac{-i}{\hbar} \gamma(t)y_0) \\
    & \times
    \exp(\frac{-i}{\hbar}
    \left[ m_e \dot{\zeta}(t) \zeta(t) - \xi \right]),
  \end{aligned}
\end{equation}
where we can define $\Phi_{n,m}(k_y,t)$ as
\begin{equation} \label{eq:b13}
  \begin{aligned}
    &\Phi_{n,m}(k_y,t) \\
    & \quad=
    \sqrt{L_x}
    \int_{-L_y/2}^{L_y/2}
    \chi_n \bm{\left(}y - \mu(t)\bm{\right)}
    \exp \bm{\left(}-i\left[k_y - \gamma(t) \right]y\bm{\right)} dy.
  \end{aligned}
\end{equation}
Substituting ${k'_y} = k_y -\gamma(t)$ with $y' = y -\mu(t)$, and assuming that the size of the considered 2DEG sample in $y$-direction is considerably large ($L_y \rightarrow \infty$), we can obtain
\begin{equation} \label{eq:b14}
  \Phi_{n,m}({k'_y} ,t) =
  {\sqrt{L_x}} e^{-i {k'_y}\mu}
  \int_{-\infty}^{\infty} \chi_{n}(y') \exp(-i{k'_y} y') dy'.
\end{equation}
Moreover, we can identify that the above integral represents the Fourier transform of $\chi_n$ functions. In addition, using the symmetric conditions of the Fourier transform for Gauss-Hermite functions $\theta_n(x)$ \cite{celeghini21}
\begin{equation} \label{eq:b15}
  \mathcal{FT}[\theta_n(\kappa x),x,k] = \frac{i^n}{|\kappa|}\theta_n(k/\kappa),
\end{equation}
we can simply the Eq.~(\ref{eq:b14}) as
\begin{equation} \label{eq:b16}
  \Phi_{n,m}({k'_y} ,t) =
  \sqrt{L_x} e^{-i {k'_y}\mu} \widetilde{\chi}_n(k'_y),
\end{equation}
with
\begin{equation} \label{eq:b17}
  \widetilde{\chi}_{n}(k) =
  \frac{i^n}{\sqrt{2^{n} n! \sqrt{\pi}}}
  \left(\frac{1}{\kappa} \right)^{1/2}
  e^{-\flatfrac{k^2}{(2 \kappa^2)}}
  \mathcal{H}_n (\flatfrac{k}{\kappa}).
\end{equation}
Finally, substitute Eq.~(\ref{eq:b16}) back into Eq.~(\ref{eq:b12}) and this leads to
\begin{equation} \label{eq:b18}
  \begin{aligned}
    \phi_{n,m}(k_x,k_y,t)  = &
    {\sqrt{L_x}} \widetilde{\chi}_n \bm{\left(} k_y - b\cos(\omega t)\bm{\right)} \\
    & \times
    \exp \left(
      i\xi -ik_y  \left[d\sin(\omega t) + \frac{\hbar k_x}{eB} \right]
    \right),
  \end{aligned}
\end{equation}
where
\begin{equation} \label{eq:b19}
  b =
  \frac{eE\omega_0^2}{\hbar\omega(\omega_0^2 - \omega^2)},
\end{equation}
and
\begin{equation} \label{eq:b20}
  d =
 \frac{eE}{m_e(\omega_0^2 - \omega^2)}.
\end{equation}
It is necessary to notice that $k_x$ is quantized with $k_x = \flatfrac{2\pi m}{L_x} ~,~ m \in \mathbb{Z}$.
