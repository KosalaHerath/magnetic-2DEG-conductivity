In this analysis, we reported a generalized mathematical model for trasport propeties in a 2DEG under a stationary magnetic field and a dressing field. With the uniform magnetic field charged particles can only settle in discrete energy values and this leads to the Landau quantization. We modelled the behaviour of electrons in Landau levels under the dressing field utilizing Floquet-Drude conductivity method by accounting imputies in the material as Gaussian random scattering potential. Finally we derived expressions for x-directional and y-directional longitudinal components of electric conductivity tensor for the considering system.

Our derived analytical expressions revealed that the electric conductivity of the dressed quantum Hall system can be controlled by the applied dressing field’s intensity. Using detailed numerical calculations with empirical system parameters, we further analysed the manipulation of conductivity components using the dressing field. We found that the graphical illustrations we gained are capable of produce the same behaviour as the experiment conductivity found in quantum Hall systems withiout a dressing field. Furthermore we identified that by regulating the intensity of radiation, the conductivity regions near Landau levels can be squeezed. Dispite this behavour has been identified previous works, their results are not concide with the more accurate description of conductivity components in undressed quantum Hall systems. However our generalized analysis of conductivity in dressed quantum Hall system provides well-suited description for  these special quantum Hall systems.

In summary, the primary purpose of this paper were to expand the current descriptions on trasport properties in dressed quantum Hall systems and our  detailed theoritical analysis shows that we were able to demostrate that recently introduced Floquet-Drude conductivty model can be adopted to extend the models that are used to describe the trasport characteristics in quantum Hall systems. Finally, we identify that our relizations of this paper can be used towards creating nanoscale quantum devices, understaning their mechanism and enhancing their performance. Due to owing the ability of controlling the conductivity regions, high intensity external illumination can be used as a trigger for two-dimentional quantum swtiching devices which are employed as buiding blocks of next generation nanoscale electronic devices.
