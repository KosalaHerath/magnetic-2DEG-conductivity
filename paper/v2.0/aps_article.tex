% ****** Start of file apssamp.tex ******
%
%   This file is part of the APS files in the REVTeX 4.2 distribution.
%   Version 4.2a of REVTeX, December 2014
%
%   Copyright (c) 2014 The American Physical Society.
%
%   See the REVTeX 4 README file for restrictions and more information.
%
% TeX'ing this file requires that you have AMS-LaTeX 2.0 installed
% as well as the rest of the prerequisites for REVTeX 4.2
%
% See the REVTeX 4 README file
% It also requires running BibTeX. The commands are as follows:
%
%  1)  latex apssamp.tex
%  2)  bibtex apssamp
%  3)  latex apssamp.tex
%  4)  latex apssamp.tex
%
\documentclass[%
 reprint,
%superscriptaddress,
%groupedaddress,
%unsortedaddress,
%runinaddress,
%frontmatterverbose,
%preprint,
%preprintnumbers,
%nofootinbib,
%nobibnotes,
%bibnotes,
 amsmath,amssymb,
 aps,
%pra,
prb,
%rmp,
%prstab,
%prstper,
%floatfix,
]{revtex4-2}

\usepackage{graphicx}% Include figure files
\usepackage{dcolumn}% Align table columns on decimal point
\usepackage{bm}% bold math
\usepackage{amsfonts}
\usepackage{amsmath}
\usepackage{physics}
\usepackage{amssymb}
\usepackage{mathtools}
\usepackage{siunitx}
\usepackage[colorlinks=true, allcolors=blue]{hyperref}% add hypertext capabilities
\bibliographystyle{apsrev4-2}
%% Language and font encodings
% \usepackage[english]{babel}
% \usepackage[utf8x]{inputenc}
% \usepackage[T1]{fontenc}

%\usepackage[mathlines]{lineno}% Enable numbering of text and display math
%\linenumbers\relax % Commence numbering lines

%\usepackage[showframe,%Uncomment any one of the following lines to test
%%scale=0.7, marginratio={1:1, 2:3}, ignoreall,% default settings
%%text={7in,10in},centering,
%%margin=1.5in,
%%total={6.5in,8.75in}, top=1.2in, left=0.9in, includefoot,
%%height=10in,a5paper,hmargin={3cm,0.8in},
%]{geometry}

\begin{document}

\preprint{APS/123-QED}

\title{A generalized model for the charge transport properties of \\dressed quantum Hall systems}% Force line breaks with \\
% \thanks{A footnote to the article title}%

\author{Kosala Herath,}
 % \altaffiliation[Also at ]{Physics Department, XYZ University.}%Lines break automatically or can be forced with \\
\author{Malin Premaratne}%
\affiliation{%
 Advanced Computing and Simulation Laboratory(A$\chi$L), Department of Electrical and Computer Systems Engineering,\\
 Monash University, Clayton, Victoria 3800, Australia
}%

% \collaboration{MUSO Collaboration}%\noaffiliation
%
% \author{Charlie Author}
%  \homepage{http://www.Second.institution.edu/~Charlie.Author}
% \affiliation{
%  Second institution and/or address\\
%  This line break forced% with \\
% }%
% \affiliation{
%  Third institution, the second for Charlie Author
% }%
% \author{Delta Author}
% \affiliation{%
%  Authors' institution and/or address\\
%  This line break forced with \textbackslash\textbackslash
% }%
%
% \collaboration{CLEO Collaboration}%\noaffiliation

\date{\today}% It is always \today, today,
             %  but any date may be explicitly specified

\begin{abstract}
A generalized mathematical model for the transport properties of systems exposed to a stationary magnetic and a strong electromagnetic field is presented. The new formulation, which applies to the two-dimensional dressed quantum Hall systems, is based on Landau quantization theory and Floquet-Drude conductivity method. We model our system as a two-dimensional electron gas (2DEG) that interacts with two external fields. To incorporate the strong light coupling with the 2DEG, we utilize the Floquet theory to analyze the effect non perturbatively. Moreover, the Floquet Fermi "golden rule" is adopted to explore the scattering effects for Floquet states in disordered quantum Hall systems. Based on our fully analytical expression and particular graphical representations, we demonstrate
that the characteristics of conductivities in two-dimensional quantum Hall systems can manipulate using a dressed field. The outcomes align with theoretical descriptions which are already well-suited with experimental results at the same time our theory provides a more generalized analysis on the properties of conductivity in quantum Hall systems. Thus, this model more realistically describes that how to use external strong radiation as a tool to utilize transport properties in various 2D nanostructures which serve as a basis for nano-optoelectronic devices.

% \begin{description}
% \item[Usage]
% Secondary publications and information retrieval purposes.
% \item[Structure]
% You may use the \texttt{description} environment to structure your abstract;
% use the optional argument of the \verb+\item+ command to give the category of each item.
% \end{description}
\end{abstract}

%\keywords{Suggested keywords}%Use showkeys class option if keyword
                              %display desired
\maketitle

\section{\label{sec_introduction} Introduction}
Interactions between light and matter have dragged research attention in the fields of optoelectronics, sensing, energy harvesting, quantum computing, bio-information, and in many branches of recent technologies. For many years, the foremost aims for examing the characteristics of dressed fermion systems were focused on the different types of atomic and molecular arrangements. These researches of extreme electron-light engagements introduced an astonishing scope of twentieth-century physics namely quantum optic physics.

On the other hand, in nanostructures that are applicable in electronic devices, the investigations with the help of quantum optic were centered on polaritonic and exciton influences on nanostructures and material characteristics of dressed electrons in two-dimensional(2D) materials and quantum wires. When considering the transport characteristics of dressed nanostructures, they are still expecting extensive analysis.

Therefore, transport properties of nanostructures exposed to a high intensity periodic electromagnetic fields have been explored theoretically in this study. The dressing field is analyzed non perturbatively using the Floquet theory whilst the probing field is examined perturbatively by applying the linear response method using the Kubo formula. The general Floquet-Drude conductivity has been derived in a fully closed analytical form in most recent research [1,2], introducing a novel type of Green’s functions namely four-times Green’s functions.
As a consequence, the established formalism introduces a novel approach to manipulate the transport characteristics of nanostructures by an intense dressing field. From an empirical sense, this study applies directly to various nanostructures illuminated by a high-intensity electromagnetic field.
In this research we have developed a robust mathematical model for dressed two-dimensional electron gas(2DEG) exposed to another stationary magnetic field and that will enable efficient manipulation of transport characteristics in nanoscale electronic devices.

When a stationary magentic field applied perpendicularly across the surface of 2DEG systems, the orbital motion of electrons becomes completely quantized and the energy spectrium becomes discrete by creating Landau levels. Such a singular system known as a quantum Hall system and in this study we explictly calculate the diagonal ($\sigma_{xx},\sigma_{yy}$) components of the conductivity tensor in the periodically driven quantum Hall systems.

Although there are already exist a number of adavanced theories devoted to the calculation of conductivity tensor elements in a quantum Hall systems [3-5], they have not been applied to the optically manipulation the magneto-electric properties of the quantum Hall systems. However K. Dini et al. [6] have recently investigated the one directional conductivity behaviour of dressed quantum Hall systems, they have not used the state of art model to describe the conductivity in a quantum Hall system. In their study they used the conductivity models from T. Ando et al. [3,4] and as mentioned in A. Endo et al. investigation [5] those models are far less accurate representation of the experimantally observed Landau levels because they present a semi-elliptical broadening.

In this study we develop a genralized mathematical model to describe transport properties of dressed quantum Hall systems using Floquet-Drude conductivity [1,2]. In addition, we demonstrate that our generalized model is agreed with the state of art conductivity model [5] for specalized quantum Hall system which has been considered without the external dressing field.  Therefore this theory describes that the dressing field can be used as a tool to utilize transport properties in various 2D nanostructures which serve as a basis for nano-optoelectonic devices.


\section{\label{sec_schrodinger_problem} Schrodinger problem for a dressed quantum Hall system}
% Section 02 - Schrodinger problem for landau levels in dressed 2DEG

Our system consist of a two-dimentional free electron gas (2DEG) confined in the $(x,y)$ plane of the three-dimentional coordinate space. In our analysis, the 2DEG is subjected to a stationary magnetic field $\vb{B} = (0,0,B)^{\text{T}}$ which is pointed towards the $z$ axis. In addition a linearly polorized strong light is applied perpendicular to the 2DEG plane and we specially tune the frequency of the field $\omega$ such that the optical field behaves as a purely dressing field (nonabsorbable). Without limiting the generality we can choose $y$-polorized electric field $\vb{E} = (0,E\sin(\omega t),0)^{\text{T}}$ for the dressing field configuration (Fig.~\ref{fig_1}).
\begin{figure}[b]
\includegraphics{figures/fig_1}
\caption{\label{fig_1} Two dimentional eletron gas (2DEG) confined in the $(x,y)$ plane while both stationary magnetic field $\vb{B}$ and strong dressing field with y-polorized electric field $\vb{E}$ are being applied perpendicular to the surface of 2DEG.}
\end{figure}
Here $B$ and $E$ represent the amplitude of the stationary magnetic field and oscillating electric field respectively.

Using Landau gauge for the stationary magnetic field, we can represent it using vector potential as $\vb{A}_{s} = (-By,0,0)^{\text{T}}$ and choosing Coulomb gauge, the vector potential of the dynamic dressing radiation can be presented as $\vb{A}_{d}(t) = (0,[E/\omega ]\cos(\omega t),0)^{\text{T}}$. These vector potentials are coupled to the momentum of 2DEG as kinetic momentum \cite{mahan81,bruus04} and this leads to the time-dependent Hamiltonian
\begin{equation} \label{eq_1}
  \hat{H}_e(t) = \frac{1}{2m_e}\Big[\hat{\vb{p}} - e\big(\vb{A}_{s}+\vb{A}_{d}(t)\big)\Big]^2,
\end{equation}
where $m_e$ is the effective electron mass and $e$ is the magnitude of the electron charge. $\hat{\vb{p}} = (\hat{p}_x,\hat{p}_y,0)^{\text{T}}$ represents the canonical momentum operator for 2DEG with electron momentums $p_{x,y}$.
The exact solutions for the time-dependent Schrödinger equation $i\hbar \dv{t}\psi = \hat{H}_e(t) \psi$ was already given by Refs. \cite{husmi53,ditt98,dini16} and we can present them as a set of wave functions defined by two quantum numbers $(n,m)$
\begin{equation} \label{eq_2}
  \begin{aligned}
    \psi_{n,m}&(x,y,t)  = \frac{1}{\sqrt{L_x}}
    \chi_n\left[y - y_0 - \zeta(t)\right]
    \text{exp}\bigg(
    \frac{i}{\hbar}\bigg[- \varepsilon_nt \\
    &
    + p_x x + \frac{eE(y - y_0)}{\omega}\cos(\omega t)+
    m_e\dot{\zeta}(t)\big[y - y_0 -\zeta(t)\big]\\
    & +
    \int_0^{t}dt'L(\zeta,\dot{\zeta},t')\bigg]\bigg),
  \end{aligned}
\end{equation}
where $n \in \mathbb{Z}^{+}_0$ and $m \in \mathbb{Z}$ ; see Appendix A. Here $L_{x,y}$ are dimention of the 2DEG surface, $\hbar$ is the reduced Planck constant, and $y_0 = -p_x/eB$ is the center of the cyclotron orbit along $y$ axis. $\chi_n$ are well known solutions(Gauss-Hermite functions) for Schrödinger equation of a stationary quantum harmonic oscillartor
\begin{equation} \label{eq_3}
  \chi_n(x) \equiv
   \frac{\sqrt{\kappa}}{\sqrt{2^{n}n!}}
  e^{-\kappa^2 x^2/2}
  \mathcal{H}_n \qty(\kappa x) \quad \text{with}
  \quad
  \kappa = \sqrt{\frac{m_e \omega_0}{\hbar}},
\end{equation}
with eigenvalues given by $\varepsilon_n = \hbar \omega_0 (n + 1/2)$ and $\omega_0 = eB/m_e$ is the cyclotron frequency. Each $n$ value defines the  energy($\varepsilon_n$) of the respective Landau level. The path shift of the driven classical oscillator $\zeta(t)$ is given by
\begin{equation} \label{eq_4}
  \zeta(t) = \frac{eE}{m_e(\omega_0^2 - \omega^2)}\sin(\omega t),
\end{equation}
while the Lagrangian of the classical oscialltor $L(\zeta,\dot{\zeta},t)$ can be idenfied as
\begin{equation} \label{eq_5}
  L(\zeta,\dot{\zeta},t) = \frac{1}{2} m_e\dot{\zeta}^2(t) - \frac{1}{2}m_e\omega_0^2 \zeta^2(t) + eE\zeta(t) \sin(\omega t).
\end{equation}
















x


\section{\label{sec_floquet_theory} Floquet theory perspective}
% Section 03 - Floquet theory perspective

Symmetry conditions often give useful insights into the behaviors of physical quantum systems.
For instance, the famous Bloch analysis of electrons in quantum systems introduces a mathematical explanation for quantum systems occupying a discrete translational symmetry in the configuration space. Similarly, Floquet theory gives a mathematical formalism that can be used for translational symmetry in time rather than in space \cite{floquet83,grifoni98,holthaus15}.
The Floquet-Drude conductivity theory was employed recently by Wackerl \textit{et al.} \cite{wackerl20} as a method to analyze the transport properties of quantum systems exposed to strong radiation.
In their work, they have presented more accurate results than the former theoretical descriptions for the conductivity of nanoscale systems in the  presence of a dressing field. Therefore, we apply the Floquet-Drude conductivity theory to analyze our 2DEG system which is subjected to both a stationary magnetic field and a dressing field.

First, we need to identify the \textit{quasienergies} and time periodic \textit{Floquet modes} \cite{grifoni98} for the wave functions given in Eq.~(\ref{eq:2}). By factorizing the wave function into a linearly time-dependent part and a periodic time-dependent part, we present the quasienergies with
\begin{equation} \label{eq:6}
  \varepsilon_{n} =
  \hbar \omega_0 \left(n + \flatfrac{1}{2}\right) - \Delta_{\varepsilon},
\end{equation}
which only depends on a single quantum number $n$. Furthermore, we can recognize the Floquet modes as
\begin{equation} \label{eq:7}
  \begin{aligned}
    \phi_{n,m}(x,y,t) =&
    \frac{1}{\sqrt{L_x}} \chi_{n}\bm{\left(}y - y_0 -\zeta(t)\bm{\right)} \\
    & \times
      \exp \bm{\left(}
      \frac{i}{\hbar}\left\{
      p_x x + \frac{eE}{\omega}(y - y_0)\cos(\omega t) \right.\right. \\
    & \left.\left. \qquad\qquad +
      m_e\dot{\zeta}(t)\left[y - y_0 -\zeta(t)\right] +
      \xi \right\}\bm{\right)},
  \end{aligned}
\end{equation}
with
\begin{equation} \label{eq:8}
  \Delta_{\varepsilon} = \frac{e^2E^2}{4m_e(\omega_0^2 - \omega^2)},
\end{equation}
and
\begin{equation} \label{eq:9}
  \xi = \frac{e^2E^2 \left(3\omega^2 - \omega_0^2 \right)}
  {8m_e\omega(\omega_0^2 - \omega^2)^2} \sin(2\omega t).
\end{equation}
For a detailed derivation, refer to Appendix \ref{appendix_b}.
It is important to note that these Floquet modes are time-periodic ($T=2\pi/\omega$) functions. At resonance $\omega = \omega_0$, the energy levels occupy a continuous spectrum and the quasienergy formalism is no longer valid \cite{popov70}. Therefore, in this work we choose a dressing field frequency obeying the condition $\omega \neq \omega_0$.

Performing the Fourier transform over the confined 2D space, we obtain the momentum space ($k_x,k_y$) representation of Floquet modes
\begin{equation} \label{eq:10}
  \begin{aligned}
    \phi_{n,m}\big(k_x,k_y,t\big)  =&
    \sqrt{L_x}
    \widetilde{\chi}_{n} \bm{\left(}k_y - b\cos(\omega t)\bm{\right) }\\
    & \times
    \exp \bm{\left(} i\xi -ik_y  \left[d\sin(\omega t) + y_0 \right] \bm{\right)},
  \end{aligned}
\end{equation}
where
\begin{equation} \label{eq:11}
  \widetilde{\chi}_{n}(k) =
  i^n \left(\frac{1}{ 2^{n} n! \sqrt{\pi} \kappa}\right)^{1/2}
  e^{-\flatfrac{k^2}{(2 \kappa^2)}}
  \mathcal{H}_{n} \left(\flatfrac{k}{\kappa}\right).
\end{equation}
Here we have introduced new parameters
\begin{equation} \label{eq:12}
  b =
  \frac{eE\omega_0^2}{\hbar\omega(\omega_0^2 - \omega^2)},
\end{equation}
and
\begin{equation} \label{eq:13}
  d =
 \frac{eE}{m_e(\omega_0^2 - \omega^2)}.
\end{equation}
Using Floquet theory, we can re-write the wave functions derived in Eq.~(\ref{eq:2}) as the \textit{Floquet states} in momentum space
\begin{equation} \label{eq:14}
  \psi_{n,m}(k_x,k_y,t) =
  \exp(-\flatfrac{i\varepsilon_{n}t}{\hbar}) \phi_{n,m} (k_x,k_y,t).
\end{equation}


\section{\label{sec_inverse_scattering_time}  Inverse Scattering Time Analysis}
% Section 04 - Inverse Scattering Time Analysis

The Floquet-Fermi golden rule was proposed in Ref. \cite{wackerl20} as an approach to analyze the transport properties of dressed quantum systems with impurities.
However, this theory has not been applied for a dressed quantum Hall system in the previous studies. In this analysis, we use Floquet-Fermi golden rule to identify the effects induced by impurities on the magneto-transport properties.
With the help of $t-t'$ formalism \cite{wackerl20,grifoni98,sambe75,peskin93,althorpe97} and applying Floquet states derived in Eq.~(\ref{eq:14}), we can derive an  expression for $(l,l')$-th element of the inverse scattering time matrix for the $N$-th Landau level as
\begin{widetext}
\begin{equation} \label{eq:15}
  \begin{aligned}
    \left(\frac{1}{\tau(\varepsilon,k_x)}\right)^{ll'}_N = &
    \frac{ \varrho^2}{eB} \delta(\varepsilon - \varepsilon_N) \\
    & \times
    \int_{-\infty}^{\infty}
    J_l \bm{\left(} \frac{b\hbar}{eB}({k}_x - k_1) \bm{\right)}
    J_{l'} \bm{\left(} \frac{b\hbar}{eB}({k}_x - k_1) \bm{\right)}
    \left|
    \int_{-\infty}^{\infty}
    \chi_N \left( \frac{\hbar}{eB}k_2 \right)
    \chi_N \bm{\left(} \frac{\hbar}{eB}
    \left( k_1 - {k}_x - k_2 \right) \bm{\right)}
    dk_2 \right|^2 d k_1,
  \end{aligned}
\end{equation}
\end{widetext}
where $\varrho = \eta_{imp} L_x \left[\flatfrac{ V_{imp}}{(eB)}\right]^{1/2}$, $\varepsilon$ is a given energy value, $J_l(\cdot)$ are Bessel functions of the first kind with $l$-th integer order, and $\varepsilon_N$ is the energy of the $N$-th Landau level.
A more detailed derivation is given in Appendix \ref{appendix_c}.
We modeled the effect caused by impurities in the considered system as a single perturbation potential.
Since random impurities in a disordered metal offer a better approximation for experimental conditions, we assumed that our perturbation potential is formed by a group of randomly distributed impurities.
Thus, the total scattering potential in the 2DEG has been represented as a sum of uncorrelated single impurity potentials $\upsilon(\vb{r})$. Here $\eta_{imp}$ is the number of impurities in a unit area, $V_{imp} = \expval{|V_{{k'}_x,k_x}|^2}_{imp}$ with $V_{{k'}_x,k_x} = \mel**{k'_x}{\upsilon(x) }{k_x}$, and $\braket{x}{k_x} = e^{-ik_x x}$.
Moreover, in this analysis, $\expval{\cdot}_{imp}$ represents the average over the impurity disorder.

Next, we analyze the contribution of the inverse scattering time matrix elements on the transport properties of our system.
Since the disorder in the system can not significantly alter the eigenenergy values of the undressed system \cite{wackerl20}, we can neglect the contribution of all off-diagonal elements in the inverse scattering time matrix. Subsequently we consider only the central diagonal element (${l=l'=0}$) of the inverse scattering time matrix which has the largest contribution to the transport characteristics. Along with this assumption, we introduce a new parameter as the scattering-induced broadening of the $N$-th Landau level \cite{dini16,endo09}
\begin{equation} \label{eq:16}
 \Gamma^{00}_{N}(\varepsilon,k_x) =
 \hbar \left(\frac{1}{\tau(\varepsilon,k_x)}\right)^{00}_N,
\end{equation}
which simplifies to
\begin{widetext}
  \begin{equation} \label{eq:17}
   \begin{aligned}
     &\Gamma^{00}_{N} (\varepsilon,k_x) =
     \frac { \varrho^2}{eB}
     \delta(\varepsilon - \varepsilon_{N})
     \int_{-\infty}^{\infty}
     J_0^2 \bm{\left(} \frac{b\hbar}{eB}({k}_x - k_1) \bm{\right)}
     \left|
     \int_{-\infty}^{\infty}
     \chi_N \left( \frac{\hbar}{eB}k_2 \right)
     \chi_N \bm{\left(} \frac{\hbar}{eB}
     \left( k_1 - {k}_x - k_2 \right) \bm{\right)}
     dk_2 \right|^2 d k_1.
   \end{aligned}
  \end{equation}
In addition, for a scattering scenario taking place within the same Landau level, we are able to present the delta distribution of the energy by the  interpretation \cite{dini16}
\begin{equation} \label{eq:18}
 \delta(\varepsilon - \varepsilon_{N}) \approx
 \frac{1}{\pi \Gamma^{00}_N (\varepsilon,k_x)}.
\end{equation}
Then we write the central element of inverse scattering time matrix in the more compact form
\begin{equation} \label{eq:19}
  \begin{aligned}
    &\Gamma^{00}_{N}(\varepsilon,k_x) =
     \varrho
      \left[
      \int_{-\infty}^{\infty}
      J_0^2 \bm{\left(} \lambda_1 (k_x - k_1) \bm{\right)}
      \left|
      \int_{-\infty}^{\infty}
      \widetilde{\chi}_{N}\left(\lambda_2 k_2 \right)
      \widetilde{\chi}_{N} \bm{\left(} \lambda_2 (k_1 - k_2 - k_x) \bm{\right)}
      dk_2 \right|^2
      dk_1
      \right]^{-\frac{1}{2}},
  \end{aligned}
\end{equation}
where $ \lambda_1 = \flatfrac{\hbar b}{(eB)}$ and  $\lambda_2 = \flatfrac{\hbar \kappa}{(eB)}$.
To analyze the effects of the dressing field on the scattering-induced broadening, we introduce the normalized $N$-th Landau level scattering-induced broadening as
\begin{equation} \label{eq:20}
    \Lambda_N(k_x) =
    \frac{\Gamma^{00}_N (\varepsilon,k_x)}
    {\Gamma^{00}_{N=0}(\varepsilon,k_x)\big|_{E=0}},
\end{equation}
which can be evaluated with
\begin{equation} \label{eq:21}
    \Lambda_N (k_x) =
    \left[
    \frac
    {
      \int_{-\infty}^{\infty}
      J_0^2 \bm{\left(} \lambda_1 (k_x - k_1) \bm{\right)}
      \left|
      \int_{-\infty}^{\infty}
      \widetilde{\chi}_{N}\left(\lambda_2 k_2 \right)
      \widetilde{\chi}_{N} \bm{\left(} \lambda_2 (k_1 - k_2 - k_x) \bm{\right)}
      dk_2 \right|^2
      dk_1
    }
    {
      \int_{-\infty}^{\infty}
      \left|
      \int_{-\infty}^{\infty}
      \widetilde{\chi}_0 \left(\lambda_2 k_2 \right)
      \widetilde{\chi}_0 \bm{\left(} \lambda_2 (k_1 - k_2 - k_x) \bm{\right)}
      dk_2 \right|^2
      dk_1
    }
    \right]^{1/2}.
\end{equation}
\end{widetext}

\begin{figure}[h!]
\includegraphics[scale=0.55]{figures/fig_2}
\caption{\label{fig_3} The dependence of normalized scattering-induced broadening $\Lambda_N$ for each Landau level ($N =0,1,2,3,4$) against $x$-directional momentum value $k_x$ in a GaAs-based quantum well under a nonoscillating magnetic field with $B = \SI{1.2}{\tesla}$, dressing field with a  frequency of $\omega =\SI{2e12}{\radian\per\second}$ and intensity $I =\SI{100}{\watt\per\square\centi\metre}$.
In this calculation, we have assumed that the natural  broadening of $0$-th Landau level $\Gamma_0$ is $\SI{0.24}{\milli\eV}$.}
\end{figure}
\begin{figure}[t!]
\includegraphics[scale=0.55]{figures/fig_3}
\caption{\label{fig_4} The dependence of normalized scattering-induced broadening $\Lambda_N$ for each Landau level ($N =0,1,2,3,4$) against dressing field intensity $I$, in a GaAs-based quantum well under a nonoscillating magnetic field with $B = \SI{1.2}{\tesla}$, dressing field with a frequency of $\omega =\SI{2e12}{\radian\per\second}$. In this calculation, we have assumed that the natural broadening of $0$-th Landau level $\Gamma_0$ is $\SI{0.24}{\milli\eV}$.}
\end{figure}

Normalized energy band broadening against $x$-directional momentum component ${k_x}$ for different Landau levels ($N = 0,1,2,3,4$) has been calculated for GaAs-based quantum well and the results are depicted in Fig.~(\ref{fig_3}) and Fig.~(\ref{fig_4}). To make a comparison, we have selected the experiment parameters to match with analysis in Ref.~\cite{endo09}.
In that study, the authors have assumed that the effective mass of the electron in GaAs-based quantum well system is $m_e \approx 0.07\widetilde{m}_e$ where $\widetilde{m}_e$ is the mass of the electron \cite{endo09,winkler03,wackerl20}. In addition, they used the broadening of the undressed $0$-th Landau level $\Gamma_0$ as $\SI{0.24}{\milli\eV}$. Therefore, in our calculations, we assumed that the natural least Landau level broadening also has this value: $\Gamma^{00}_{N=0}|_{E=0} = \SI{0.24}{\milli\eV}$.
Here, we observe that the normalized energy broadening value for each Landau level is independent of the $x$-directional momentum $k_x$ value and we are able to manipulate it by the dressing field. When the dressing field's intensity increases, the energy broadening is reduced, which leads to changes in the transport properties of the dressed quantum Hall system.
To analyze these adjustments in detail, we derive an analytical expression for the conductivity of a dressed quantum Hall system in the next section.


\section{\label{sec_floquet_drude_conductivity} Floquet-Drude Conductivity in a Dressed Quantum Hall System}
% Section 05 - Floquet-Drude Conductivity in Quantum Hall Systems

A general theory for the conductivity in dressed systems with the disorder averaging was introduced by M. Wackerl in Refs.~\cite{wackerl20,wackerlthesis20}. Within this theory, the general $x$-directional longitudinal DC limit conductivity can be express with
\begin{equation} \label{eq_20}
  \begin{aligned}
    {\sigma}^{xx} &=
    \frac{-1}{4\pi\hbar A}
    \int_{\Pi-\hbar\omega/2}^{\Pi+ \hbar\omega/2} d\varepsilon \bigg[
    \qty(
    -\frac{\partial f}{\partial \varepsilon})
    \\
    & \times
    \tr
    \qty[
    {j}^x_0
    \qty(
    \vb{G}^{r} (\varepsilon) - \vb{G}^{a} (\varepsilon)
    )
    {j}^x_0
    \qty(
    \vb{G}^{r} (\varepsilon) - \vb{G}^{a} (\varepsilon)
    )
    ]\bigg],
  \end{aligned}
\end{equation}
where ${j}^x_0$ and $\vb{G}^{r,a} (\varepsilon)$ are $x$-directional electric current operator matrix elements' $s=0$ the Fourier component (see Appendix \ref{appendix_d}) and white noise disorder averaged Floquet Green function matrix \cite{wackerl20,wackerlthesis20} respectively defined against the Floquet modes of the considering system. Here we have assumed that only $s=0$ Fourier component of the current operator is contributing to the conductivity. In addition, $A$ is the area of the considering two-dimentional system, partial distribution function is given by $f$ and $\Pi$ is a function that can be chosen such that
\begin{equation} \label{eq_21}
    \Pi- \frac{\hbar \omega}{2}
    \leq \varepsilon_N
    <
    \Pi + \frac{\hbar \omega}{2},
\end{equation}
where $ \varepsilon_N$ are quasienergies of all considering Floquet states.

Then Eq.~(\ref{eq_20}) can be expanded in off resonant regime ($\omega\tau_0 \gg 1$), where $\tau_0$ is the scattering time of the undriven system, using only central entry Fourier components ($l=l'=0$) of Floquet modes $\ket{\phi_{n,m}} \equiv \ket{n,k_x}$
\begin{widetext}
\begin{equation} \label{eq_22}
  \begin{aligned}
    {\sigma}^{xx} =
    \frac{-1}{4\pi\hbar A} &
    \int_{\Pi-\hbar\omega/2}^{\Pi+ \hbar\omega/2} d\varepsilon
    \qty(
    -\frac{\partial f}{\partial \varepsilon})
    \frac{1}{V_{k_x}} \sum_{k_x}
    \sum_{n}
    \mel{n,k_x}{
    {j}^x_0
    \qty(
    \vb{G}^{r} (\varepsilon) - \vb{G}^{a} (\varepsilon)
    )
    {j}^x_0
    \qty(
    \vb{G}^{r}_0 (\varepsilon) - \vb{G}^{a}_0 (\varepsilon)
    )
    }
    {n,k_x},
  \end{aligned}
\end{equation}
where $V_{k_x}$ is the volume of considering $x$-directional momentum space. Then this can evaluate by considering each matrix elements separately
\begin{equation} \label{eq_23}
  \begin{aligned}
    {\sigma}^{xx}  = &
    \frac{-1}{4\pi\hbar A}
    \int_{\Pi-\hbar\omega/2}^{\Pi+ \hbar\omega/2} d\varepsilon
    \qty(
    -\frac{\partial f}{\partial \varepsilon})
    \frac{1}{V_{k_x}^4} \sum_{k_x} \sum_{n}
    \sum_{{k_x}_1,{k_x}_2,{k_x}_3}
    \sum_{n_1,n_2,n_3}\\
    & \times
    \mel{n,k_x}{
    {j}^x_0}
    {n_1,{k_x}_1}
    \mel{{n_1,{k_x}_1}}{
    \qty(
    \vb{G}^{r} (\varepsilon) - \vb{G}^{a} (\varepsilon)
    )}
    {n_2,{k_x}_2}
    \mel{n_2,{k_x}_2}{
    {j}^x_0}
    {n_3,{k_x}_3}
    \mel{n_3,{k_x}_3}{
    \qty(
    \vb{G}^{r} (\varepsilon) - \vb{G}^{a} (\varepsilon)
    )
    }
    {n,k_x}.
  \end{aligned}
\end{equation}
Since we can diagonalize the impurity averaged Green's function using unitary transformation ($\vb{T}  \equiv \ket{n,k_x}$) mentioned in Refs.~\cite{wackerl20,wackerlthesis20,tsuji08}, and we evaluate the matrix element of the difference between retarded and advanced Green's functions
\begin{equation} \label{eq_24}
  \mel{{n_1,{k_x}_1}}{
  \vb{T}^{\dagger}
  \qty(
  \vb{G}^{r} (\varepsilon) - \vb{G}^{a} (\varepsilon)
  )\vb{T}}
  {n_2,{k_x}_2} =
  \qty[
  \frac{2i \text{Im}\qty(\vb{T}^{\dagger} \vb{\Sigma}^r \vb{T})
  \delta_{n_1,n_2}\delta_{{k_x}_1,{k_x}_2}}
  {
  \qty(
  \frac{1}{\hbar}\varepsilon -
  \frac{1}{\hbar}\varepsilon_{n_1}
  )^2
  + \qty[\text{Im}\qty(\vb{T}^{\dagger} \vb{\Sigma}^r \vb{T})]^2
  }],
\end{equation}
and
\begin{equation} \label{eq_25}
  \mel{{n_3,{k_x}_3}}{
  \vb{T}^{\dagger}
  \qty(
  \vb{G}^{r} (\varepsilon) - \vb{G}^{a} (\varepsilon)
  )\vb{T}}
  {n,{k_x}} =
  \qty[
  \frac{2i \text{Im}\qty(\vb{T}^{\dagger} \vb{\Sigma}^r \vb{T})
  \delta_{n_3,n}\delta_{{k_x}_3,{k_x}}}
  {
  \qty(
  \frac{1}{\hbar}\varepsilon -
  \frac{1}{\hbar}\varepsilon_{n}
  )^2
  + \qty[\text{Im}\qty(\vb{T}^{\dagger} \vb{\Sigma}^r \vb{T})]^2
  }],
\end{equation}
using the retarded self-energy matrix $\vb{\Sigma}^r$ which is the sum of all irreducible diagrams \cite{wackerl20,wackerlthesis20}. Using the matrix elements of electric current operator in Landau levels (see Appendix \ref{appendix_d}) and
results form Eq.~(\ref{eq_24}) and Eq.~(\ref{eq_25}) back into Eq.~(\ref{eq_23}) we obtain
\begin{equation} \label{eq_26}
  \begin{aligned}
    {\sigma}^{xx}  =
    \frac{-1}{4\pi\hbar A} &
    \int_{\Pi-\hbar\omega/2}^{\Pi+ \hbar\omega/2} d\varepsilon
    \qty(
    -\frac{\partial f}{\partial \varepsilon})
    \frac{1}{V_{k_x}} \sum_{k_x} \sum_{n}
    \sum_{n_1,n_2}
    \\
    & \times
    \frac{e^2B}{{m_e}}
    \qty(\sqrt{\frac{n+1}{2}} \delta_{n_1,n+1} + \sqrt{\frac{n}{2}}
    \delta_{n_1,n-1})
    \qty[
    \frac{2i \text{Im}\qty(\vb{T}^{\dagger} \vb{\Sigma}^r \vb{T})
    \delta_{n_1,n_2}}
    {
    \qty(
    \frac{1}{\hbar}\varepsilon -
    \frac{1}{\hbar}\varepsilon_{n_1}
    )^2
    + \qty[\text{Im}\qty(\vb{T}^{\dagger} \vb{\Sigma}^r \vb{T})]^2
    }] \\
    & \times
    \frac{e^2B}{{m_e}}
    \qty(\sqrt{\frac{n_2+1}{2}} \delta_{n,n_2+1} + \sqrt{\frac{n_2}{2}}
    \delta_{n,n_2-1})
    \qty[
    \frac{2i \text{Im}\qty(\vb{T}^{\dagger} \vb{\Sigma}^r \vb{T})
    }
    {
    \qty(
    \frac{1}{\hbar}\varepsilon -
    \frac{1}{\hbar}\varepsilon_{n}
    )^2
    + \qty[\text{Im}\qty(\vb{T}^{\dagger} \vb{\Sigma}^r \vb{T})]^2
    }],
  \end{aligned}
\end{equation}
and the only non-zero term would be
\begin{equation} \label{eq_27}
  \begin{aligned}
    {\sigma}^{xx} & =
    \frac{-1}{4\pi\hbar A}
    \frac{e^4B^2}{{{m_e}^2}}
    \int_{\Pi-\hbar\omega/2}^{\Pi+ \hbar\omega/2} d\varepsilon
    \qty(
    -\frac{\partial f}{\partial \varepsilon})
    \frac{1}{V_{k_x}} \sum_{k_x} \sum_{n}
    \qty(n+1)
    \\
    & \times
    \qty[
    \frac{2i \text{Im}\qty(\vb{T}^{\dagger} \vb{\Sigma}^r \vb{T})_{\varepsilon_{n+1}}
    }
    {
    \qty(
    \frac{1}{\hbar}\varepsilon -
    \frac{1}{\hbar}\varepsilon_{n+1}
    )^2
    + \qty[\text{Im}\qty(\vb{T}^{\dagger} \vb{\Sigma}^r \vb{T})_{\varepsilon_{n+1}}]^2
    }]
    \qty[
    \frac{2i \text{Im}\qty(\vb{T}^{\dagger} \vb{\Sigma}^r \vb{T})_{\varepsilon_{n}}
    }
    {
    \qty(
    \frac{1}{\hbar}\varepsilon -
    \frac{1}{\hbar}\varepsilon_{n}
    )^2
    + \qty[\text{Im}\qty(\vb{T}^{\dagger} \vb{\Sigma}^r \vb{T})_{\varepsilon_{n}}]^2
    }].
  \end{aligned}
\end{equation}
Since the inverse scattering time matrix being equal to the diagonalized contrast of the retarded and advanced self-energy and on the diagonal the contrast of the retarded and advanced Green's function can be represented with the imaginary component of the retarded self-energy \cite{wackerl20,wackerlthesis20} we can identify the following property
\begin{equation} \label{eq_28}
  \qty(\frac{1}{\tau(\varepsilon,k_x)})^{ll} =
  -2\text{Im}\qty[ \vb{T}^{\dagger} \vb{\Sigma}^r(\varepsilon,k_x) \vb{T}]^{ll},
\end{equation}
and using central element ($l=0$) of the inverse scattering time matrix we can modify the derived conductivity expression
\begin{equation} \label{eq_29}
    {\sigma}^{xx}   =
    \frac{1}{\pi\hbar A}
    \frac{e^4B^2}{{{m_e}^2}}
    \int_{\Pi-\hbar\omega/2}^{\Pi+ \hbar\omega/2} d\varepsilon
    \qty(
    -\frac{\partial f}{\partial \varepsilon})
    \frac{1}{V_{k_x}} \sum_{k_x} \sum_{n}
    \qty(n+1)
    \qty[
    \frac{\tilde{{\Gamma}}(\varepsilon_{n+1})
    }
    {
    \qty(
    \varepsilon_F - \varepsilon_{n+1}
    )^2
    + \tilde{{\Gamma}}^2(\varepsilon_{n+1})
    }]
    \qty[
    \frac{\tilde{{\Gamma}}(\varepsilon_{n})
    }
    {
    \qty(
    \varepsilon_F - \varepsilon_{n}
    )^2
    + \tilde{{\Gamma}}^2(\varepsilon_{n})
    }],
\end{equation}
\end{widetext}
with $\tilde{{\Gamma}}(\varepsilon_n,k_x) \equiv \qty({\hbar}/{2\tau(\varepsilon_n,k_x)})^{00}$. We already identified that the inverse scattering time matrix's central element is not $k_x$ dependent we can get the sum over all available momentum space in $x$ direction. However, by considering the condition that the center of the force of the oscillator $y_0$ must physically lie within the considering system $-L_y/2 < y_0 < L_y/2$, we can identify
\begin{equation} \label{eq_30}
 -\frac{m_e\omega_0 Ly}{2\hbar} \leq k_x \leq \frac{m_e\omega_0 Ly}{2\hbar}.
\end{equation}
Then we use the Fermi-Dirac distribution as our partial distribution function ($f$) for this system
\begin{equation} \label{eq_31}
  f(\varepsilon) = \frac{1}{\qty[\exp(\varepsilon - \varepsilon_F)/k_B T]+1},
\end{equation}
where $k_B$ is Boltzmann constant, $T$ is absolute temperature and $\varepsilon_F$ is Fermi energy of the system. Considering the above distribution with extremely low temperature conditions we can approximate
\begin{equation} \label{eq_32}
  - \pdv{f(\varepsilon)}{\varepsilon} \approx \delta(\varepsilon - \varepsilon_F),
\end{equation}
and by letting $\Pi = \varepsilon_F$, the expression for conductivity leads to
\begin{equation} \label{eq_33}
  \begin{aligned}
    {\sigma}^{xx}  =
    \frac{e^2}{\hbar}
    \frac{1}{\pi A} &
    \sum_{n}
    \frac{\qty(n+1)}{\gamma_{n}\gamma_{n+1}} \\
    &\times
    \qty[
      \frac{1}
      {
        1 + \qty(\frac{X_F - n -1}{\gamma_{n+1}})^2
      }
    ]
    \qty[
      \frac{1}
      {
        1 + \qty(\frac{X_F - n}{\gamma_{n}})^2
      }
    ],
  \end{aligned}
\end{equation}
where $X_F \equiv ({\varepsilon_F}/{\hbar \omega_0} - {1}/{2})$
and
$\gamma_n \equiv {\tilde{{\Gamma}}(\varepsilon_n)}/{\hbar \omega_0}$.
Same as above derivation we can derive the longitudinal conductivity in $y$-direction by using the current operator derived in Appendix \ref{appendix_d}
\begin{equation} \label{eq_34}
  \begin{aligned}
    {\sigma}^{yy} =
    \frac{e^2}{\hbar}
    \frac{1}{\pi A}&
    \frac{1}{e^2B^2}
    \sum_{n}
    \frac{\qty(n+1)}{\gamma_{n}\gamma_{n+1}} \\
    & \times
    \qty[
      \frac{1}
      {
        1 + \qty(\frac{X_F - n -1}{\gamma_{n+1}})^2
      }
    ]
    \qty[
      \frac{1}
      {
        1 + \qty(\frac{X_F - n}{\gamma_{n}})^2
      }
    ].
  \end{aligned}
\end{equation}


\section{\label{sec_manipulate_conductivity} Manipulate Conductivity in Quantum Hall Systems}
% Section 06 - Manipulate Conductivity in Quantum Hall Systems

To identify the characteristics of the longitudinal conductivity of the quantum Hall systems with external dressing field, first we can derive an expression for a normalized
longitudinal conductivity as a function of Fermi energy $X_F$ and intensity of the dressing field $I$. Here we have the normalized x-directional conductivity using the natural conductivity of the least Landau level
\begin{equation} \label{eq_35}
  \begin{aligned}
    \frac{\sigma^{xx}}{\sigma^{0}} = &
    \sum_{n}
    \frac{\qty(n+1)}{0.0037\Lambda_n \Lambda_{n+1}} \\
    & \times
    \qty[
      \frac{1}
      {
        1 + \qty(\frac{X_F - n -1}{0.06\Lambda_n})^2
      }
    ]
    \qty[
      \frac{1}
      {
        1 + \qty(\frac{X_F - n}{0.06\Lambda_{n+1}})^2
      }
    ],
  \end{aligned}
\end{equation}
where $\sigma^0 = (e^2/\pi \hbar A)$. We use this expression to illustrate the changes that can be done to the longitudinal conductivity in 2DEG using external dressing field. As given in Fig.~\ref{fig_5} and \ref{fig_6} we can manipulate the longitudinal conductivity $\sigma_{xx}$ using external dressing field's intensity and the Fermi level $X_F$ of the considering system. For a given dressing field intensity, the longitudinal conductivity vary against the Fermi level of the system by creating sharp peaks at each Landau level energy values. Since electrons are restricted to have only Landau level energies, the conductivity gets very low values when the Fermi level is not align with any of the Landau level energy values. In contrast, on each Landau level, the conductivity can achieve very high values compared to other areas and as illustrates on Fig.~\ref{fig_5} the peak value of longitudinal conductivity on each Landau level gets increase with the Landau level number.

\begin{figure}[t]
\includegraphics[scale=0.55]{figures/fig_5}
\caption{\label{fig_5} Normalized longitudinal conductivity $\sigma_{xx}$ against Fermi level $X_F$ with different intensities $I$ of the external dressing field in a GaAs-based quantum well under a nonoscillating magnetic field with $B = 1.2~\text{T}$, dressing field with frequency of $\omega =2\times10^{12}~\text{rad}\text{s}^{-1}$ and $I_0 =100~\text{W}/\text{cm}^{2}$. In this calculation we have assumed that the natural  broadening of $0$-th Landau level $\Gamma_0$ is $0.24\;\text{me}V$.}
\end{figure}
\begin{figure}[t]
\includegraphics[scale=0.55]{figures/fig_6}
\caption{\label{fig_6} $3$rd Landau level’s normalized longitudinal conductivity $\sigma_{xx}$ against Fermi level $X_F$ with different intensities $I$ of the external dressing field in a GaAs-based quantum well under a nonoscillating magnetic field with $B = 1.2~\text{T}$, dressing field with frequency of $\omega =2\times10^{12}~\text{rad}\text{s}^{-1}$ and $I_0 =100~\text{W}/\text{cm}^{2}$. In this calculation we have assumed that the natural  broadening of $0$-th Landau level $\Gamma_0$ is $0.24\;\text{me}V$}.
\end{figure}

Considering the effect of the external dressing field on longitudinal conductivity of 2DEG, we can identify that high intensities shrink the conductivity regions near Landau levels. However, the peak value of the conductivity at each Landau level has the same value as the undressed system. This demonstrate that we are able to tune the width of the regions of conductivity in these quantum Hall systems with the help of a dressing field.
These characteristics are aligned with results demonstrated by K. Dini \textit{et al.} \cite{dini16} and as they remarked since the Fermi level of the system can be change with the applied gate voltage of the material this can be used as a 2D switch for nanoscale optoelectronics applications. Controlling  the external dressing field we are able to fine-tune the switching mechanism for optimized performance.
Furthermore, we can distinguish that the shapes and behavior of the conductivity regions illustrated in Fig.~\ref{fig_5} and \ref{fig_6} are generally incompatible with the results reported in Ref.~\cite{dini16}. This is due to the selection of the conventional longitudinal conductivity theory of 2DEG from Ref.~\cite{ando74_1,ando82}. The semi-elliptical conductivity regions illustrated in Ref.~\cite{dini16,ando74_1,ando82}, have less consistence with the experimentally observed Landau levels representation \cite{endo09}.
In our study on the transport properties of quantum Hall systems, we developed the conductivity expression starting from Floquet-Drude conductivity \cite{wackerl20} and our results are much more align with the result mentioned in Ref. \cite{endo09}.
The description of conductivity of quantum Hall systems demonstrated in Ref.~\cite{endo09} has excellent agreement between the theory and experiment obtained in a GaAs/AlGaAs 2DES for the low magnetic field range. However, they have not considered the tunability that can be achieved with the external strong dressing field. In this analysis we account both magnetic and dressing field effects that can be applied into the transport properties of 2DEG, and we have presented a more generalized theory. As a concluding remark, in this study we were able to demonstrate that using Floquet-Drude conductivity method one can derive a more experimental fitting and generalized mathematical model that describes the transport properties of quantum Hall systems.


\section{\label{sec_conclusions} Conclusions}
In this paper, we introduced a generalized mathematical model for prediciting  charge transport properties in a 2DEG under a nonoscillating magnetic field and a high intensity light. Under the uniform magnetic field, the charged particles can only settle in discrete energy values which leads to the Landau quantization. We modeled the behavior of electrons in Landau levels under the dressing field utilizing the Floquet-Drude conductivity method. We assumed the impurities in the material as a Gaussian random scattering potential. Finally, we derived expressions for x-directional and y-directional longitudinal components of electric conductivity tensor for the considered system.

Our derived analytical expressions disclosed that the transport characteristics of the dressed quantum Hall system can be controlled by the applied dressing field’s intensity. Using detailed numerical calculations with empirical system parameters, we further analyzed the manipulation of conductivity components using the dressing field.
We found that the graphical illustrations that we obtained from these numerical calculatiuons are capable of producing the same behavior as experiments on quantum Hall systems in the absence of a dressing field.
Furthermore, we identified that by increasing the intensity of the radiation, the conductivity regions near the Landau levels can be squeezed. Despite this behavior being identified in previous works, their results did not coincide with the more accurate description of conductivity components in undressed quantum Hall systems. However, our generalized analysis on conductivity of dressed quantum Hall systems provide a well-suited description for these specific quantum Hall systems.

In summary, the primary purpose of this study was to broaden the modern descriptions on transport properties of dressed quantum Hall systems. Moreover, our detailed theoretical analysis showed that the recently introduced Floquet-Drude conductivity model can be adopted to extend the models that were used to describe the transport characteristics in quantum Hall systems. Owing the ability to control the conductivity regions, high intensity external illumination may be used as a trigger for two-dimensional quantum switching devices which are employed as the building blocks of next generation nanoelectronic devices. As a concluding remark, we believe that our findings of this paper can be used towards understanding the 2D opto-electronic nano transistors, enhancing their performance, and inventing novel appliances.


\begin{acknowledgments}
K.H wish to acknowledge the members of A$\chi$L at Monash University for their encouragement and support and specially T.N. Perera and R.T.Wijesekara for insightful discussions. The work of K.H is supported by the Monash University Institute of Graduate Research.

\end{acknowledgments}

\appendix

\section{\label{appendix_a} Wave function solutions for a dressed quantum Hall system}
The derivation of the solutions for the time-dependent Schrödinger equation with our system's Hamiltonian (Eq. \ref{eq_1}) is quite similar to that followed in Refs. \cite{husimi53,dini16}. We start with expanding the Hamiltonian for two-dimensional case
\begin{equation} \label{eq_a1}
  \hat{H}_e(t) = \frac{1}{2m_e}\left[
    \left[\hat{p}_x + eBy \right]^2 +
    \left[\hat{p}_y - \frac{eE}{\omega}\cos(\omega t)\right]^2
  \right].
\end{equation}
Since $\left[\hat{H}_e(t),\hat{p}_x \right] =0$, both of these operators share same eigenfunctions
${L_x}^{-1/2}\exp(\frac{ip_x x}{\hbar})$ where $p_x = 2\pi \hbar m/L_x~$ with $ m \in \mathbb{Z}$.
Thus, we re-arrange the Hamiltonian using the definition of canonical momentum in $y$-direction and this leads to
\begin{equation} \label{eq_a2}
    \hat{H}_e(t) = \frac{1}{2m_e}\left[
      \left[{p}_x + eBy \right]^2 +
      \left[-i\hbar \pdv{y}- \frac{eE}{\omega}\cos(\omega t)\right]^2
    \right].
\end{equation}
Subsequenty we define the \textit{center of the cyclotron orbit} on the $y$-axis $y_0 = {-p_x}/{eB}$ and the \textit{cyclotron frequency} $\omega_0 = {eB}/{m_e}$. This leads to a new arrangement of the Hamiltonian
\begin{equation} \label{eq_a3}
  \begin{aligned}
    \hat{H}_e(t) =
      \frac{m_e \omega_0^2}{2}\tilde{y}^2 +
      \frac{1}{2m_e}\bigg[
      -\hbar^2 \pdv[2]{\tilde{y}} & +
      \frac{2i\hbar eE}{\omega}\cos(\omega t) \pdv{\tilde{y}} \\
      & +
      \frac{e^2E^2}{\omega^2}\cos[2](\omega t)
      \bigg],
  \end{aligned}
\end{equation}
where we used a variable substitution $\tilde{y} = (y - y_0)$. Furthermore, we assume that the wave function solutions for the time-dependent Schrödinger equation of considered quantum system
\begin{equation} \label{eq_a4}
    i \hbar \dv{\psi}{t} = \hat{H}_e(t)\psi,
\end{equation}
can be presented by the following form
\begin{equation} \label{eq_a5}
    \psi_m(x,\tilde{y},t) = \frac{1}{\sqrt{L_x}} \exp\bigg(
      \frac{ip_x x}{\hbar} +
      \frac{ieE\tilde{y}}{\hbar \omega}\cos(\omega t)
    \bigg) \vartheta(\tilde{y},t),
\end{equation}
where $\vartheta(\tilde{y},t)$ is a function that satisfy the property
\begin{equation} \label{eq_a6}
    \bigg[
    \frac{m_e \omega_0^2}{2}\tilde{y}^2
    - {eE\tilde{y}}\sin(\omega t)
    -
    \frac{\hbar^2}{2m_e}
    \pdv[2]{\tilde{y}}
    - i \hbar \dv{t}
    \bigg]
    \vartheta(\tilde{y},t) = 0.
\end{equation}
If we turn off the dressing field ($E=0$), this equation leads to the Schrödinger equation with the simple harmonic oscillator Hamiltonian
\begin{equation} \label{eq_a7}
     i \hbar \dv{\vartheta(\tilde{y},t)}{t} =
    \bigg[
    \frac{\hat{p}_{\tilde{y}}^2}{2m_e} +
    \frac{1}{2}m_e \omega_0^2\tilde{y}^2
    \bigg]
    \vartheta(\tilde{y},t).
\end{equation}
Thus, we can identify $S(t) \equiv eE\sin(\omega t)$ term as an external force act on the harmonic oscillator, and we can solve this as a forced harmonic oscillator in $\tilde{y}$ axis
\begin{equation} \label{eq_a8}
  \begin{aligned}
    i \hbar \dv{\vartheta(\tilde{y},t)}{t} =
    \bigg[
    -
    \frac{\hbar^2}{2m_e}
    \pdv[2]{\tilde{y}} +
    \frac{1}{2}m_e \omega_0^2\tilde{y}^2
    - \tilde{y}S(t)]
    \bigg]
    \vartheta(\tilde{y},t).
  \end{aligned}
\end{equation}

This system is exactly solvable, and we can solve the equation using the methods explained by Husimi \cite{husimi53}. We introduce a time-dependent shifted coordinate $ y' = \tilde{y} - \zeta(t)$ and perform the following unitary transformation
\begin{equation} \label{eq_a9}
    \vartheta(y',t) = \exp(\frac{im_e\dot{\zeta}y'}{\hbar})\varphi(y',t).
\end{equation}
This leads to
\begin{equation} \label{eq_a10}
  \begin{aligned}
    i \hbar \pdv{\varphi(y',t)}{t}   &=
    \bigg[
        -  \frac{\hbar^2}{2m_e}\pdv[2]{{y'}}
        + \frac{1}{2} m_e \omega_0^2 y'^2 \\
        & +
        \Big[
            m_e\ddot{\zeta} + m_e\omega_0^2\zeta - S(t)
        \Big]y' \\
        &
        +
        \Big[
            - \frac{1}{2} m_e\dot{\zeta}^2 + \frac{1}{2}m_e\omega_0^2 \zeta^2 - \zeta S(t)
        \Big]
    \bigg]\varphi(y',t).
  \end{aligned}
\end{equation}
Subsequenty, we can restrict $\zeta(t)$ function such that
\begin{equation} \label{eq_a11}
  m_e\ddot{\zeta} + m_e\omega_0^2\zeta = S(t),
\end{equation}
and that simply our previous expression as
\begin{equation} \label{eq_a12}
  \begin{aligned}
    i \hbar \pdv{\varphi(y',t)}{t}   =
    \bigg[
        -  \frac{\hbar^2}{2m_e}\pdv[2]{{y'}} &
        + \frac{1}{2} m_e \omega_0^2 {y'}^2 \\
        &
        - L(\zeta,\dot{\zeta},t)
    \bigg]\varphi(y',t).
  \end{aligned}
\end{equation}
Here
\begin{equation} \label{eq_a13}
  L(\zeta,\dot{\zeta},t) = \frac{1}{2} m_e\dot{\zeta}^2 - \frac{1}{2}m_e\omega_0^2 \zeta^2 + \zeta S(t),
\end{equation}
is the Lagrangian of a classical driven oscillator. To proceed further, another unitary transform can be introduced as follows
\begin{equation} \label{eq_a14}
    \varphi(y',t) = \exp(\frac{i}{\hbar}\int_0^{t}dt'L(\zeta,\dot{\zeta},t')) \chi(y',t),
\end{equation}
and subtitling Eq.~(\ref{eq_a14}) back in Eq.~(\ref{eq_a12}) yields
\begin{equation} \label{eq_a15}
    i \hbar \pdv{t} \chi(y',t)  =
    \bigg[
        -  \frac{\hbar^2}{2m_e}\pdv[2]{{y'}}
        + \frac{1}{2} m_e \omega_0^2 {y'}^2
    \bigg] \chi(y',t).
\end{equation}
This is the well-known Schrödinger equation of the quantum harmonic oscillator.
This allows us to identify the well known eigenfunctions \cite{griffiths18,shankar94}
\begin{equation} \label{eq_a16}
  \chi_n(y) =
   \frac{\sqrt{\kappa}}{\sqrt{2^{n}n!}}
  e^{-\kappa^2 y^2/2}
  \mathcal{H}_n \qty(\kappa y),
\end{equation}
with eigenvalues
\begin{equation} \label{eq_a17}
  \epsilon_n = \hbar \omega_0 \bigg(n + \frac{1}{2}\bigg)
  ~\text{for}~
  n \in \mathbb{Z}^{+}_0.
\end{equation}
Here, $\kappa = \sqrt{{m_e \omega_0}/{\hbar}}$, and $\mathcal{H}_n$ are the Hermite polynomials.
Thus, we can identify the solutions for Eq.~(\ref{eq_a8}) as
\begin{equation} \label{eq_a18}
  \begin{aligned}
    \vartheta_n(\tilde{y},t) = \chi_n(\tilde{y} - \zeta(t))
     \text{exp}\bigg(\frac{i}{\hbar}\bigg[&- \epsilon_nt +
    m_e\dot{\zeta(t)}\big[\tilde{y}-\zeta(t)\big] \\
     & + \int_0^{t}dt'L(\zeta,\dot{\zeta},t')\bigg]\bigg).
  \end{aligned}
\end{equation}
Since $\chi_n(x)$ functions forms a complete set, any general solution $\vartheta_(\tilde{y},t)$ can be presented with the help of the solutions derived in Eq.~(\ref{eq_a18}).

Finally, we consider our scenario where we assumed that $S(t) = eE\sin(\omega t)$, and we derive the solution for Eq.~(\ref{eq_a11}) as
\begin{equation} \label{eq_a19}
  \zeta(t) = \frac{eE}{m_e(\omega_0^2 - \omega^2)}\sin(\omega t).
\end{equation}
Subtitling solutions in Eq.~(\ref{eq_a18}) back in Eq.~(\ref{eq_a5}), we obtain a set of wave functions with two different quantum number ($n$,$m$) that satisfy the time-dependent Schrödinger equation Eq.~(\ref{eq_a4}) as follows
\begin{equation} \label{eq_a20}
  \begin{aligned}
    \psi_{n,m}&(x,y,t) \\
    &=  \frac{1}{\sqrt{L_x}}
    \chi_n\left(y - y_0 - \zeta(t)\right)\\
    &\quad\times
    \text{exp}\bigg(
    \frac{i}{\hbar}\bigg[- \epsilon_nt
    + p_x x + \frac{eE[y - y_0]}{\omega}\cos(\omega t)\\
    & \quad\quad+
    m_e\dot{\zeta}(t)\big[y - y_0 -\zeta(t)\big] +
    \int_0^{t}dt'L(\zeta,\dot{\zeta},t')\bigg]\bigg).
  \end{aligned}
\end{equation}


\section{\label{appendix_b} Floquet modes and quasienergies}
\subsection{Position space representation}

First define the time integral of Laggrangian of the classical oscillator given in Eq.~(\ref{eq_5}), over a $T=2\pi/\omega$ period as
\begin{equation} \label{eq_b1}
  \Delta_{\varepsilon} \equiv \frac{1}{T} \int_0^T dt' \; L(\zeta,\dot{\zeta},t'),
\end{equation}
and after performing the integral using Eq.~(\ref{eq_4}), we can obtain more simplified result:
\begin{equation} \label{eq_b2}
  \Delta_{\varepsilon} = \frac{(eE)^2}{4m_e(\omega_0^2 - \omega^2)}.
\end{equation}
Next define another paramter
\begin{equation} \label{eq_b3}
  \xi \equiv
  \int_0^t dt' \; L(\zeta,\dot{\zeta},t') -
  \Delta_{\varepsilon} t,
\end{equation}
and after simplying, this leads to
\begin{equation} \label{eq_b4}
  \xi =
  \frac{(eE)^2\qty(3\omega^2 - \omega_0^2)}{8m_e\omega(\omega_0^2 - \omega^2)^2} \sin(2\omega t),
\end{equation}
which is a periodic function in time with $2\omega$ frequency. Now using these  parmaters we can factorize the wavefunction Eq.~(\ref{eq_2}) as linearly time dependend part and periodic time dependend part as follows
\begin{equation} \label{eq_b5}
  \begin{aligned}
    \psi_{\alpha}&(x,y,t)  =
    \exp(\frac{i}{\hbar}\qty[-\varepsilon_nt + \Delta_{\varepsilon} t ])
    \frac{1}{\sqrt{L_x}} \chi_n\big(y - y_0 - \zeta(t)\big)
    \\
    & \times
    \text{exp}\bigg(
     \frac{i}{\hbar}\bigg[
     p_x x +
     \frac{eEy}{\omega}\cos(\omega t) \\
     & +
     m_e\dot{\zeta(t)}\big[y-\zeta(t)\big]
     + \int_0^{t}dt'L(\zeta,\dot{\zeta},t') - \Delta_{\varepsilon} t  \bigg]
     \bigg),
  \end{aligned}
\end{equation}
and this leads to seperate linear time dependent phase component as the quasienergies
\begin{equation} \label{eq_b6}
  \varepsilon_{n} =
  \hbar \omega_0\qty(n + \frac{1}{2}) - \Delta_{\varepsilon},
\end{equation}
while rest of the components as time-periodic Floquet modes
\begin{equation} \label{eq_b7}
  \begin{aligned}
    \phi_{n,m}(x,y,t) \equiv &
    \frac{1}{\sqrt{L_x}} \chi_{n}\left[y - y_0 - \zeta(t)\right]
    \text{exp}\bigg(
     \frac{i}{\hbar}\bigg[
     p_x x \\
     & +
     \frac{eE(y - y_0)}{\omega}\cos(\omega t) \\
     & +
     m_e\dot{\zeta}(t)\big[y - y_0 -\zeta(t)\big]
     + \xi \bigg]\bigg).
  \end{aligned}
\end{equation}

\subsection{Momentum space representation}

To write the Floquet modes in momentum space, we perform continuous Fourier transform over the considering confined space $A=L_xL_y$ for Eq.~(\ref{eq_7}):
\begin{equation} \label{eq_b8}
  \begin{aligned}
    \phi_{n,m}&(k_x,k_y,t) \\
    & =
    \exp(
     \frac{-i\gamma(t)}{\hbar}
     y_0)
    \exp(\frac{-i}{\hbar}
    \qty[
    m_e \dot{\zeta}(t) \zeta(t) - \xi
    ])\\
    & \times
    \int_{-L_y/2}^{L_y/2} dy\; \exp(-i\qty[k_y - \gamma(t)]y)
    \chi_{n}\qty[y - \mu(t)] \\
     & \times
     \frac{1}{\sqrt{L_x}}
     \int_{-L_x/2}^{L_x/2} dx\;
     \exp(-ik_x x)
     \exp( \frac{i p_x }{\hbar}x ),
  \end{aligned}
\end{equation}
where
\begin{equation} \label{eq_b9}
  \mu(t) \equiv \frac{eE\sin(\omega t)}{m_e(\omega_0^2 - \omega^2)} + y_0
  \quad,\quad
  \gamma(t) \equiv
  \frac{eE\omega_0^2\cos(\omega t)}{\hbar\omega(\omega_0^2 - \omega^2)}.
\end{equation}
Next using the identity \cite{bruus04}
\begin{equation} \label{eq_b10}
  \int_{L_x} dx\;
  \exp( -ik_x x + \frac{i p_x }{\hbar}x ) =
  L_x \delta_{k_x,\frac{p_x}{\hbar}},
\end{equation}
we can derive
\begin{equation} \label{eq_b11}
  \begin{aligned}
    \phi_{n,m}&(k_x,k_y,t)  = 
    \Phi_{n,m}(k_y,t)
    \delta_{k_x,\frac{p_x}{\hbar}}\\
    & \times
    \exp(
     \frac{-i\gamma(t)}{\hbar}
     y_0)
    \exp(\frac{-i}{\hbar}
    \qty[
    m_e \dot{\zeta}(t) \zeta(t) - \xi
    ]).
  \end{aligned}
\end{equation}
Here we defined $\Phi_{n,m}(k_y,t)$ as
\begin{equation} \label{eq_b12}
  \begin{aligned}
    \Phi_{n,m}(k_y,t) \equiv
    \sqrt{L_x}
    \int_{-L_y/2}^{L_y/2} dy\; &
    \chi_{n}\qty[y - \mu(t)] \\
    & \times
    \exp(
      -i\qty[k_y - \gamma(t)]
      y).
  \end{aligned}
\end{equation}
Subtituting $  {k'_y} = k_y -\gamma(t)$ and $y' = y -\mu(t)$ and assuming that size of the 2DEG sample in $y$-direction is large ($L_y \rightarrow \infty$), we can obtain
\begin{equation} \label{eq_b13}
  \Phi_{n,m}({k'_y} ,t) =
  {\sqrt{L_x}} e^{-i {k'_y}\mu}
  \int_{-\infty}^{\infty} dy'\;
  \chi_{n}\qty(y')
  \exp(-i{k'_y} y').
\end{equation}
We can identify that the integral represnts the Fourier transform of $\{\chi_n\}$ functions and using the symmetric conditions \cite{celeghini21} for the Fourier transform of Gauss-Hermite functions $\theta_n(x)$:
\begin{equation} \label{eq_b14}
  \mathcal{FT}[\theta_n(\kappa x),x,k] = \frac{i^n}{|\kappa|}\theta_n(k/\kappa),
\end{equation}
Eq.~(\ref{eq_b13}) can be simplified as
\begin{equation} \label{eq_b15}
  \Phi_{n,m}({k'_y} ,t) =
    \sqrt{L_x}e^{-i {k'_y}\mu}
    \tilde{\chi}_{n}\qty({k'_y}),
\end{equation}
with
\begin{equation} \label{eq_b16}
  \tilde{\chi}_{n}\qty(k) =
  \frac{i^n}{\sqrt{2^{n} n! \sqrt{\pi}}}
  \qty(\frac{1}{\kappa})^{1/2}
  e^{-\frac{k^2}{2 \kappa^2}}
  \mathcal{H}_{\alpha} \qty(\frac{k}{\kappa}).
\end{equation}
Subtitute Eq.~(\ref{eq_b15}) back in Eq.~(\ref{eq_b11}) and this leads to
\begin{equation} \label{eq_b17}
  \begin{aligned}
    \phi_{n,m}(k_x,k_y,t)  = &
    {\sqrt{L_x}}
    \tilde{\chi}_{n}\qty(k_y - b\cos(\omega t)) \\
    & \times
    \text{exp}\left(
      i\xi
      -ik_y  \qty[d\sin(\omega t) + \frac{\hbar k_x}{eB}]
    \right),
  \end{aligned}
\end{equation}
where
\begin{equation} \label{eq_b18}
  b \equiv
  \frac{eE\omega_0^2}{\hbar\omega(\omega_0^2 - \omega^2)} \quad
  d \equiv
 \frac{eE}{m_e(\omega_0^2 - \omega^2)},
\end{equation}
and it is necessary to notice that $k_x$ is quantized with $k_x = 2\pi m/L_x ~,~ m \in \mathbb{Z}$.


\section{\label{appendix_c} Floquet-Fermi golden rule for a dressed quantum  Hall system}
The derivation of the Floquet Fermi golden rule for our quantum Hall system with the help of $t-t'$ formalism is given here in detail. The $t$-$t'$-Floquet states \cite{grifoni98,wackerl20}
\begin{equation} \label{eq_c1}
  \ket{\psi_{n,m}(t,t')} =
  \exp(-\frac{i}{\hbar}\varepsilon_{n} t)\ket{\phi_{n,m}(t')}.
\end{equation}
derived by seperating the aperiodic and periodic components of Eq.~(\ref{eq_12}), fullfill the $t$-$t'$-Schrödinger equation \cite{grifoni98,wackerl20}
\begin{equation} \label{eq_c2}
  i \hbar \pdv{t}\ket{\psi_{n,m}(t,t')} =
  H_F(t') \ket{\psi_{n,m}(t,t')},
\end{equation}
where \textit{Floquet Hamiltonian} defined as
\begin{equation} \label{eq_c3}
  H_F(t') \equiv
  H_e(t') - i\hbar \pdv{t'}.
\end{equation}
Next we can identify the the time evolution operator corresponding to the $t$-$t'$-Schrödinger equation
\begin{equation} \label{eq_c4}
  U_F(t,t_0;t') = \exp(-\frac{i}{\hbar}H_F(t')\qty[t-t_0]),
\end{equation}
and the advantage of $t$-$t'$ formalism lies on this time evolution operator which avoids any time odering operatos \cite{wackerl20}.

For our scenario, consider a time-independent total perturbation $V(\vb{r})$ which has been switched on at the reference time $t=t_0$, then the $t$-$t'$-Schrödinger equation becomes
\begin{equation} \label{eq_c5}
  i \hbar \pdv{t}\ket{\Psi_{n,m}(t,t')} =
  \qty[H_F(t') + V(\vb{r})]\ket{\Psi_{n,m}(t,t')},
\end{equation}
by introducing new wave function $\Psi_{n,m}$ for the system with the given total perturabation. If $t\leq t_0$, both solutions of the Schrödinger equations (Eq.~(\ref{eq_c2}) and Eq.~(\ref{eq_c5})) coincide
\begin{equation} \label{eq_c6}
  \ket{\psi_{n,m}(t,t')} =\ket{\Psi_{n,m}(t,t')} \quad
  \text{when} \quad
  t \leq t_0.
\end{equation}
Now move into the interaction picture representation \cite{bruus04,mahan00} of the $t$-$t'$-Floquet state
\begin{equation} \label{eq_c7}
  \ket{\Psi_{n,m}(t,t')}_I = U_0^{\dagger}(t,t_0;t')
  \ket{\Psi_{n,m}(t,t')},
\end{equation}
and due to time independency, the perturbation in the interaction picture has the same form as Schrödinger picture
\begin{equation} \label{eq_c8}
  V_I(\vb{r}) = U_0^{\dagger}(t,t_0;t')V(\vb{r})U_0(t,t_0;t') =
  V(\vb{r}).
\end{equation}
This leads to the $t$-$t'$-Schrödinger eqution in the interction picture
\begin{equation} \label{eq_c9}
  i \hbar \pdv{t}\ket{\Psi_{n,m}(t,t')}_I =
  V_I(\vb{r})\ket{\Psi_{n,m}(t,t')}_I,
\end{equation}
with the recursive solution \cite{bruus04,mahan00}
\begin{equation} \label{eq_c10}
  \begin{aligned}
  \ket{\Psi_{n,m}(t,t')}_I = &\ket{\Psi_{n,m}(t_0,t')}_I \\
  &+
  \frac{1}{i\hbar}
  \int_{t_0}^t dt_1 \;
  V_I(\vb{r}) \ket{\Psi_{n,m}(t_1,t')}_I.
  \end{aligned}
\end{equation}
Iterating the solution only upto the first order (Born approximation) we obtain
\begin{equation} \label{eq_c11}
  \begin{aligned}
    \ket{\Psi_{n,m}(t,t')}_I \approx &\ket{\psi_{n,m}(t_0,t')} \\
    &+
    \frac{1}{i\hbar}
    \int_{t_0}^t dt_1 \;
    V_I(\vb{r}) \ket{\psi_{n,m}(t_0,t')}.
  \end{aligned}
\end{equation}

In addition, since our Floquet states create a basis we can represent any solution using these Floquet states
\begin{equation} \label{eq_c12}
  \ket{\Psi_{\alpha}(t,t')} = \sum_{\beta} a_{\alpha,\beta}(t,t')
  \ket{\psi_{\beta}(t,t')}.
\end{equation}
where we used a single notation to represent two quantum numbers; $\alpha \equiv (n_{\alpha},m_{\alpha})$ and $\beta \equiv (n_{\beta},m_{\beta})$.
Then we can identify the \textit{scattering amplitude} as $a_{\alpha,\beta}(t,t') =
\braket{\psi_{\beta}(t,t')}{\Psi_{\alpha}(t,t')}$ and this can evaluate with
\begin{equation} \label{eq_c13}
  \begin{aligned}
  a_{\alpha,\beta}(t,t') = &
  \braket{\psi_{\beta}(t,t')}{\psi_{\alpha}(t,t')} \\
  &+
  \frac{1}{i\hbar}
  \int_{t_0}^t dt_1 \;
  \bra{\psi_{\beta}(t_1,t')}
  V(\vb{r}) \ket{\psi_{\alpha}(t_1,t')}.
  \end{aligned}
\end{equation}
Since the $t$-$t'$-Floquet states are orthonormal and assuming $t_0 = 0$ and $\alpha \neq \beta$ this leads to
\begin{equation} \label{eq_c14}
  a_{\alpha,\beta}(t,t') =
  -
  \frac{i}{\hbar}
  \int_{0}^t dt_1 \;
  \bra{\psi_{\beta}(t_1,t')}
  V(\vb{r}) \ket{\psi_{\alpha}(t_1,t')}.
\end{equation}

Now consider a scattering event from a $t$-$t'$-Floquet state $\ket{\psi_{\beta}(t,t')}$ into another $t$-$t'$-Floquet state $\ket{\Psi_{\alpha}(t,t')}$ with constant quansienergy $\varepsilon$ which has been
\begin{equation} \label{eq_c16}
  \begin{aligned}
  \psi_{\beta}(\vb{k'},t,t') &= \exp(-\frac{i}{\hbar}\varepsilon_{\beta} t)
  \phi_{\beta}(\vb{k'},t') \\
  &
  \longrightarrow
  \Psi_{\alpha}(\vb{k},t,t') = \exp(-\frac{i}{\hbar}\varepsilon t)
  \Phi_{\alpha}(\vb{k},t').
  \end{aligned}
\end{equation}
\begin{figure}[b]
  \includegraphics[scale=1.0]{figures/fig_2.pdf}
  \caption{Scattering from $\ket{\psi_{\beta}(t,t')}$ to constant energy state $\ket{\Psi_{\alpha}(t,t')}$ due to scattering potential created by impurities.}
  \label{fig_2}
\end{figure}
















x


\section{\label{appendix_d} Current operators for a dressed quantum Hall system}
In this section we are hoping to derive the current density operator for $N$-th Landau level. We already found the extact solution for our time depenedent Hamiltonian Eq.~(\ref{eq_1}) and we identified them as Floquet states with quesienergies Eq.~(\ref{eq_12}). The Floquet modes derived in Eq. Eq.~(\ref{eq_9}) can be represented as states using quantum number for the simplicity of notation as follows
\begin{equation} \label{eq_d1}
  \ket{\phi_{n,m}} \equiv \ket{n,k_x}
\end{equation}
Using above complete set of eigenstates of Floquet Hamiltonian Eq.~(\ref{eq_c3}) \cite{wackerl20,holthaus15,grifoni98} we can represent the single particle current operator's matrix element as
\begin{equation} \label{eq_d2}
  \qty(\vb{j})_{nm,n'm'} \equiv \mel{n,k_x}{\;\hat{\vb{j}}\;}{n',k'_x},
\end{equation}
and particle current operator for our system \cite{mahan00,bruus04} by
\begin{equation} \label{eq_d3}
  \hat{\vb{j}} = \frac{1}{m} \qty(\hat{\vb{p}} - e\qty[\vb{A}_s + \vb{A}_d(t)]).
\end{equation}
where $m$ is mass of the considering particle.

First consider the transverse conductivity in $x$-direction and we can identify that $x$-directional current operator as
\begin{equation} \label{eq_d4}
  \hat{j}_x = \frac{1}{m} \qty(-i\hbar\pdv{x} + eBy).
\end{equation}
Now calculate the matrix elements of $x$-directional current operator in Floquet mode basis
\begin{equation} \label{eq_d5}
  \qty({j}_x)_{nm,n'm'} =
  \mel{n,k_x}{\;\frac{1}{m} \qty(-i\hbar\pdv{x} + eBy)\;}{n',k'_x}.
\end{equation}
Then evaluate these using Floquet modes derived in Eq.~(\ref{eq_7}) and obtain
\begin{equation} \label{eq_d6}
  \begin{aligned}
    \qty({j}_x)_{nm,n'm'} = &
    \frac{1}{{m}}
    \delta_{k_x,k'_x}
    \int dy \;
    \Big[
    \qty(\hbar k'_x + eBy) \\
    & \times
     \chi_{n}\big(y - y_0 - \zeta(t)\big)
    \chi_{n'}\big(y - y_0 - \zeta(t)\big)
    \Big].
  \end{aligned}
\end{equation}
Then let $y - y_0 - \zeta(t) = \bar{y}$ and we can derive
\begin{equation} \label{eq_d7}
  \begin{aligned}
    \qty({j}_x)_{nm,n'm'} =
    \frac{1}{{m}}
    \delta_{k_x,k'_x}
    \int d\bar{y} \;
    \Big[ &
    \qty(\hbar k'_x + eB\bar{y} -\hbar k'_x + eB\zeta(t)) \\
    & \times
    \chi_{n}(\bar{y})
    \chi_{n'}(\bar{y})
    \Big].
  \end{aligned}
\end{equation}
Using following integral identities of Floquet modes which are made up with  Gauss-Hermite functions \cite{vedenyapin11,szego59}
\begin{subequations} \label{eq_d8}
  \begin{align}
    \int d{y} \;
    \chi_{n}({y})
    \chi_{n'}({y}) &=
    \delta_{n',n} \\
    \int dy \;
    y
    \chi_{n}({y})
    \chi_{n'}({y}) &=
    \qty(\sqrt{\frac{n+1}{2}} \delta_{n',n+1} + \sqrt{\frac{n}{2}}
    \delta_{n',n-1}),
  \end{align}
\end{subequations}
we simplfy the matrix elements of $x$-directional current operator to
\begin{equation} \label{eq_d9}
  \begin{aligned}
    \qty({j}_x)_{nm,n'm'} =
    \frac{1}{{m}}
    \delta_{k_x,k'_x}
    eB
    \bigg[
    \bigg(\sqrt{\frac{n+1}{2}} \delta_{n',n+1} & + \sqrt{\frac{n}{2}}
    \delta_{n',n-1}\bigg) \\
    &
    + \zeta(t) \delta_{n',n}
    \bigg]
  \end{aligned}
\end{equation}

Due to high complexity of extract solution, in this study we only consider the constant contribution. Therefore we evaluate the $s=0$ component of the Fourier series with
\begin{equation} \label{eq_d10}
    \qty({j}^x_{s=0})_{nm,n'm'} =
    \frac{eB}{{m}}
    \delta_{k_x,k'_x}
    \qty(\sqrt{\frac{n+1}{2}} \delta_{n',n+1} + \sqrt{\frac{n}{2}}
    \delta_{n',n-1}).
\end{equation}
For electric current operator we can apply the electron's charge and effective mass and this leads to
\begin{equation} \label{eq_d11}
    \Big({j}^x_{s=0}\Big)_{nm,n'm'}^{elctron} =
    \frac{e^2B}{{m_e}}
    \delta_{k_x,k'_x}
    \qty(\sqrt{\frac{n+1}{2}} \delta_{n',n+1} + \sqrt{\frac{n}{2}}
    \delta_{n',n-1}).
\end{equation}

Next we consider the transverse conductivity in $y$-direction and we can identify that $y$-directional current operator as
\begin{equation} \label{eq_d12}
  \hat{j}_y = \frac{1}{m} \qty(-i\hbar\pdv{y} - \frac{eE}{\omega}\cos(\omega t)).
\end{equation}
Then the matrix elements of $y$-directional current operator in Floquet mode basis are derived as
\begin{equation} \label{eq_d13}
  \qty({j}_y)_{nm,n'm'} =
  \mel{n,k_x}{\;\frac{-1}{m} \qty(i\hbar\pdv{y} + \frac{eE}{\omega}\cos(\omega t))\;}{n',k'_x}.
\end{equation}
After following the same steps done for $x$-directional current operator, we can derive the $s=0$ component of matrix elements for $y$-directional electric current operator
\begin{equation} \label{eq_d14}
    \Big({j}^y_{s=0}\Big)_{nm,n'm'}^{elctron} =
    \frac{ie\hbar}{{m_e}}
    \delta_{k_x,k'_x}
    \qty[
    \sqrt{\frac{n}{2}} \delta_{n',n-1}
    - \sqrt{\frac{n+1}{2}} \delta_{n',n+1}
    ].
\end{equation}


% The \nocite command causes all entries in a bibliography to be printed out
% whether or not they are actually referenced in the text. This is appropriate
% for the sample file to show the different styles of references, but authors
% most likely will not want to use it.
\nocite{*}

\bibliography{aps_article}% Produces the bibliography via BibTeX.

\end{document}















%
% ****** End of file apssamp.tex ******
