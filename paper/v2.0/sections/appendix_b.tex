\subsection{Position space representation}

First we define the time integral of the Lagrangian of the classical oscillator given in Eq.~(\ref{eq_5}), over a period $T=2\pi/\omega$ as
\begin{equation} \label{eq_b1}
  \Delta_{\varepsilon} = \frac{1}{T} \int_0^T dt' \; L(\zeta,\dot{\zeta},t').
\end{equation}
Additionally, performing this integral, we can obtain a more simplified result
\begin{equation} \label{eq_b2}
  \Delta_{\varepsilon} = \frac{(eE)^2}{4m_e(\omega_0^2 - \omega^2)}.
\end{equation}
Next, we define another parameter
\begin{equation} \label{eq_b3}
  \xi =
  \int_0^t dt' \; L(\zeta,\dot{\zeta},t') -
  \Delta_{\varepsilon} t,
\end{equation}
and after simplifying, we can identify
\begin{equation} \label{eq_b4}
  \xi =
  \frac{(eE)^2\qty(3\omega^2 - \omega_0^2)}{8m_e\omega(\omega_0^2 - \omega^2)^2} \sin(2\omega t),
\end{equation}
which is a periodic function in time. Using these parameters, we can factorize the wave function given in Eq.~(\ref{eq_2}) as linearly time-dependent part and periodic time-dependent part as follows
\begin{equation} \label{eq_b5}
  \begin{aligned}
    \psi_{\alpha}&(x,y,t)  =
    \exp(\frac{i}{\hbar}\qty[-\epsilon_nt + \Delta_{\varepsilon} t ])
    \frac{1}{\sqrt{L_x}} \chi_n\big(y - y_0 - \zeta(t)\big)
    \\
    & \times
    \text{exp}\bigg(
     \frac{i}{\hbar}\bigg[
     p_x x +
     \frac{eEy}{\omega}\cos(\omega t) \\
     & \quad+
     m_e\dot{\zeta(t)}\big[y-\zeta(t)\big]
     + \int_0^{t}dt'L(\zeta,\dot{\zeta},t') - \Delta_{\varepsilon} t  \bigg]
     \bigg).
  \end{aligned}
\end{equation}
This leads to separate the linear time-dependent phase component as the quasienergies
\begin{equation} \label{eq_b6}
  \varepsilon_{n} =
  \hbar \omega_0\qty(n + \frac{1}{2}) - \Delta_{\varepsilon},
\end{equation}
while rest of the components as time-periodic Floquet modes
\begin{equation} \label{eq_b7}
  \begin{aligned}
    \phi_{n,m}(x,y,t) =  &
    \frac{1}{\sqrt{L_x}} \chi_{n}\left(y - y_0 - \zeta(t)\right)
    \text{exp}\bigg(
     \frac{i}{\hbar}\bigg[
     p_x x \\
     & +
     \frac{eE[y - y_0]}{\omega}\cos(\omega t) \\
     & +
     m_e\dot{\zeta}(t)\big[y - y_0 -\zeta(t)\big]
     + \xi \bigg]\bigg).
  \end{aligned}
\end{equation}

\subsection{Momentum space representation}

We perform continuous Fourier transform over the considering confined space $A=L_xL_y$ on the Floquet modes given in Eq.~(\ref{eq_7}) to realize the Floquet modes in momentum space
\begin{equation} \label{eq_b8}
  \begin{aligned}
    \phi_{n,m}&(k_x,k_y,t) \\
    & =
    \exp(
     \frac{-i\gamma(t)}{\hbar}
     y_0)
    \exp(\frac{-i}{\hbar}
    \qty[
    m_e \dot{\zeta}(t) \zeta(t) - \xi
    ])\\
    & \quad\times
    \int_{-L_y/2}^{L_y/2} dy\; \exp(-i\qty[k_y - \gamma(t)]y)
    \chi_{n}\qty[y - \mu(t)] \\
     & \quad\times
     \frac{1}{\sqrt{L_x}}
     \int_{-L_x/2}^{L_x/2} dx\;
     \exp(-ik_x x)
     \exp( \frac{i p_x }{\hbar}x ).
  \end{aligned}
\end{equation}
Here we used new two parameters
\begin{equation} \label{eq_b9a}
  \mu(t) = \frac{eE\sin(\omega t)}{m_e(\omega_0^2 - \omega^2)} + y_0,
\end{equation}
and
\begin{equation} \label{eq_b9b}
  \gamma(t) =
  \frac{eE\omega_0^2\cos(\omega t)}{\hbar\omega(\omega_0^2 - \omega^2)}.
\end{equation}
Subsequenty, using the Fourier transform identity \cite{bruus04}
\begin{equation} \label{eq_b10}
  \int_{L_x} dx\;
  \exp( -ik_x x + \frac{i p_x }{\hbar}x ) =
  L_x \delta_{k_x,\frac{p_x}{\hbar}},
\end{equation}
we can derive
\begin{equation} \label{eq_b11}
  \begin{aligned}
    \phi_{n,m}&(k_x,k_y,t)  =
    \Phi_{n,m}(k_y,t)
    \delta_{k_x,\frac{p_x}{\hbar}}\\
    & \times
    \exp(
     \frac{-i\gamma(t)}{\hbar}
     y_0)
    \exp(\frac{-i}{\hbar}
    \qty[
    m_e \dot{\zeta}(t) \zeta(t) - \xi
    ]),
  \end{aligned}
\end{equation}
where we can define $\Phi_{n,m}(k_y,t)$ as
\begin{equation} \label{eq_b12}
  \begin{aligned}
    \Phi_{n,m}(k_y,t) =
    \sqrt{L_x}
    \int_{-L_y/2}^{L_y/2} dy\; &
    \chi_{n}\qty[y - \mu(t)] \\
    & \times
    \exp(
      -i\qty[k_y - \gamma(t)]
      y).
  \end{aligned}
\end{equation}
Substituting ${k'_y} = k_y -\gamma(t)$ with $y' = y -\mu(t)$, and assuming that the size of the considered 2DEG sample in $y$-direction is considerably large ($L_y \rightarrow \infty$), we can obtain
\begin{equation} \label{eq_b13}
  \Phi_{n,m}({k'_y} ,t) =
  {\sqrt{L_x}} e^{-i {k'_y}\mu}
  \int_{-\infty}^{\infty} dy'\;
  \chi_{n}\qty(y')
  \exp(-i{k'_y} y').
\end{equation}
Moreover, we can identify that the above integral represents the Fourier transform of $\{\chi_n\}$ functions. In addition, using the symmetric conditions of the Fourier transform for Gauss-Hermite functions $\theta_n(x)$ \cite{celeghini21}
\begin{equation} \label{eq_b14}
  \mathcal{FT}[\theta_n(\kappa x),x,k] = \frac{i^n}{|\kappa|}\theta_n(k/\kappa),
\end{equation}
we can simply the Eq.~(\ref{eq_b13}) as
\begin{equation} \label{eq_b15}
  \Phi_{n,m}({k'_y} ,t) =
    \sqrt{L_x}e^{-i {k'_y}\mu}
    \tilde{\chi}_{n}\qty({k'_y}),
\end{equation}
with
\begin{equation} \label{eq_b16}
  \tilde{\chi}_{n}\qty(k) =
  \frac{i^n}{\sqrt{2^{n} n! \sqrt{\pi}}}
  \qty(\frac{1}{\kappa})^{1/2}
  e^{-\frac{k^2}{2 \kappa^2}}
  \mathcal{H}_{\alpha} \qty(\frac{k}{\kappa}).
\end{equation}
Finally, substitute Eq.~(\ref{eq_b15}) back into Eq.~(\ref{eq_b11}) and this leads to
\begin{equation} \label{eq_b17}
  \begin{aligned}
    \phi_{n,m}(k_x,k_y,t)  = &
    {\sqrt{L_x}}
    \tilde{\chi}_{n}\qty(k_y - b\cos(\omega t)) \\
    & \times
    \text{exp}\left(
      i\xi
      -ik_y  \qty[d\sin(\omega t) + \frac{\hbar k_x}{eB}]
    \right),
  \end{aligned}
\end{equation}
where
\begin{equation} \label{eq_b18}
  b \equiv
  \frac{eE\omega_0^2}{\hbar\omega(\omega_0^2 - \omega^2)},
\end{equation}
and
\begin{equation} \label{eq_b19}
  d \equiv
 \frac{eE}{m_e(\omega_0^2 - \omega^2)}.
\end{equation}
It is necessary to notice that $k_x$ is quantized with $k_x = 2\pi m/L_x ~,~ m \in \mathbb{Z}$.
