% Section 03 - Floquet theory perspective

Symmetry conditions often give useful insights into the behaviors of physical quantum systems.
For instance, the famous Bloch analysis of electrons in quantum systems introduces a mathematical explanation for quantum systems occupying a discrete translational symmetry in the configuration space. Similarly, Floquet theory gives a mathematical formalism that can be used for translational symmetry in time rather than in space \cite{floquet83,grifoni98,holthaus15}.
The Floquet-Drude conductivity theory was employed recently by Wackerl \textit{et al.} \cite{wackerl20} as a method to analyze the transport properties of quantum systems exposed to strong radiation.
In their work, they have presented more accurate results than the former theoretical descriptions for the conductivity of nanoscale systems in the  presence of a dressing field. Therefore, we apply the Floquet-Drude conductivity theory to analyze our 2DEG system which is subjected to both a stationary magnetic field and a dressing field.

First, we need to identify the \textit{quasienergies} and time periodic \textit{Floquet modes} \cite{grifoni98} for the wave functions given in Eq.~(\ref{eq:2}). By factorizing the wave function into a linearly time-dependent part and a periodic time-dependent part, we present the quasienergies with
\begin{equation} \label{eq:6}
  \varepsilon_{n} =
  \hbar \omega_0 \left(n + \flatfrac{1}{2}\right) - \Delta_{\varepsilon},
\end{equation}
which only depends on a single quantum number $n$. Furthermore, we can recognize the Floquet modes as
\begin{equation} \label{eq:7}
  \begin{aligned}
    \phi_{n,m}(x,y,t) =&
    \frac{1}{\sqrt{L_x}} \chi_{n}\bm{\left(}y - y_0 -\zeta(t)\bm{\right)} \\
    & \times
      \exp \bm{\left(}
      \frac{i}{\hbar}\left\{
      p_x x + \frac{eE}{\omega}(y - y_0)\cos(\omega t) \right.\right. \\
    & \left.\left. \qquad\qquad +
      m_e\dot{\zeta}(t)\left[y - y_0 -\zeta(t)\right] +
      \xi \right\}\bm{\right)},
  \end{aligned}
\end{equation}
with
\begin{equation} \label{eq:8}
  \Delta_{\varepsilon} = \frac{e^2E^2}{4m_e(\omega_0^2 - \omega^2)},
\end{equation}
and
\begin{equation} \label{eq:9}
  \xi = \frac{e^2E^2 \left(3\omega^2 - \omega_0^2 \right)}
  {8m_e\omega(\omega_0^2 - \omega^2)^2} \sin(2\omega t).
\end{equation}
For a detailed derivation, refer to Appendix \ref{appendix_b}.
It is important to note that these Floquet modes are time-periodic ($T=2\pi/\omega$) functions. At resonance $\omega = \omega_0$, the energy levels occupy a continuous spectrum and the quasienergy formalism is no longer valid \cite{popov70}. Therefore, in this work we choose a dressing field frequency obeying the condition $\omega \neq \omega_0$.

Performing the Fourier transform over the confined 2D space, we obtain the momentum space ($k_x,k_y$) representation of Floquet modes
\begin{equation} \label{eq:10}
  \begin{aligned}
    \phi_{n,m}\big(k_x,k_y,t\big)  =&
    \sqrt{L_x}
    \widetilde{\chi}_{n} \bm{\left(}k_y - b\cos(\omega t)\bm{\right) }\\
    & \times
    \exp \bm{\left(} i\xi -ik_y  \left[d\sin(\omega t) + y_0 \right] \bm{\right)},
  \end{aligned}
\end{equation}
where
\begin{equation} \label{eq:11}
  \widetilde{\chi}_{n}(k) =
  i^n \left(\frac{1}{ 2^{n} n! \sqrt{\pi} \kappa}\right)^{1/2}
  e^{-\flatfrac{k^2}{(2 \kappa^2)}}
  \mathcal{H}_{n} \left(\flatfrac{k}{\kappa}\right).
\end{equation}
Here we have introduced new parameters
\begin{equation} \label{eq:12}
  b =
  \frac{eE\omega_0^2}{\hbar\omega(\omega_0^2 - \omega^2)},
\end{equation}
and
\begin{equation} \label{eq:13}
  d =
 \frac{eE}{m_e(\omega_0^2 - \omega^2)}.
\end{equation}
Using Floquet theory, we can re-write the wave functions derived in Eq.~(\ref{eq:2}) as the \textit{Floquet states} in momentum space
\begin{equation} \label{eq:14}
  \psi_{n,m}(k_x,k_y,t) =
  \exp(-\flatfrac{i\varepsilon_{n}t}{\hbar}) \phi_{n,m} (k_x,k_y,t).
\end{equation}
