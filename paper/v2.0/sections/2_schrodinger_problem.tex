% Section 02 - Schrodinger problem for dressed quantum Hall system

Our system consists of a 2DEG placed on the $xy$-plane of the three-dimensional coordinate space. In our analysis, the 2DEG is subjected to a nonoscillating magnetic field $\vb{B} = (0,0,B)^{T}$, which is pointed towards the $z$-axis. In addition, a linearly polarized strong light is applied perpendicular to the 2DEG surface. We specially select the frequency of the dressing field $\omega$ to be in the off-resonant regime such that the field behaves as a purely dressing field. Furthermore, without limiting the generality, we choose a $y$-polarized electric field $\vb{E} = (0,E\sin(\omega t),0)^{T}$ for the linearly polarized dressing field as given in Fig.~\ref{fig:1}.
\begin{figure}[b]
\includegraphics[scale=0.9]{figures/fig_1.pdf}
\caption{\label{fig:1} Our 2DEG system only confined in the $xy$-plane while both of the stationary magnetic field $\vb{B}$ and the dressing field are applied perpendicular to the plane of 2DEG. The dressing field is linearly polarized with a $y$-polarized electric field $\vb{E}$.}
\end{figure}
Here $B$ and $E$ represent the amplitudes of the stationary magnetic field and oscillating electric field, respectively.

Using Landau gauge for the stationary magnetic field, we can represent it as a vector potential $\vb{A}_{s} = (-By,0,0)^{T}$. Furthermore, we model the dynamic dressing field in the Coulomb gauge as $\vb{A}_{d}(t) = (0,[E\cos(\omega t)]/\omega,0)^{T}$. These vector potentials are coupled to the momentum of 2DEG as kinetic momentum \cite{mahan00,bruus04}. Thus, we can represent our system with a time-dependent Hamiltonian
\begin{equation} \label{eq:1}
  \hat{H}_e(t) =
  \frac{1}{2m_e}
  \left\{ \hat{\vb{p}} - e\right[ \vb{A}_{s}+\vb{A}_{d}(t) \left] \right\}^2,
\end{equation}
where $m_e$ is the effective electron mass, $e$ is the magnitude of the electron charge, and $\hat{\vb{p}} = (\hat{p}_x,\hat{p}_y,0)^{T}$ represents the canonical momentum operator for 2DEG with electron momentum $(p_{x},p_{y},0)^{T}$.
The exact solutions for the time-dependent Schrödinger equation $i\hbar \dv*{\psi}{t} = \hat{H}_e(t)\psi$ were already derived in Refs. \cite{husimi53,ditt98,dini16}. Here we present them as a set of wave functions defined by two quantum numbers $(n,m)$
\begin{equation} \label{eq:2}
  \begin{aligned}
    &\psi_{n,m}(x,y,t)  \\
      & \quad = \frac{1}{\sqrt{L_x}}
        \chi_n \bm{\left(} y - y_0 - \zeta(t) \bm{\right)}\\
      & \qquad \times
        \exp \bm{\left(}
        \frac{i}{\hbar}\left\{- \epsilon_n t
        + p_x x + \frac{eE}{\omega}(y - y_0)\cos(\omega t) \right.\right.\\
      & \left.\left. \qquad\qquad +
        m_e\dot{\zeta}(t)\left[y - y_0 -\zeta(t)\right] +
        \int_0^{t}L(\zeta,\dot{\zeta},t')\,dt' \right\}\bm{\right)},
  \end{aligned}
\end{equation}
where $n \in \mathbb{Z}^+_0$ and $m \in \mathbb{Z}$. Here $L_{x}$ and $L_{y}$ are dimensions of the 2DEG surface, and $\hbar$ is the reduced Planck constant. The center of the cyclotron orbit on the $y$-axis is given by $y_0 = \flatfrac{-p_x}{(eB)}$ with $p_x = \flatfrac{2\pi\hbar m}{L_x}$.
Moreover, $\chi_n$ are well-known eigenstate solutions for the Schrödinger equation of the stationary quantum harmonic oscillator
\begin{equation} \label{eq:3}
  \chi_n(y) =
  \left( \frac{\kappa}{2^{n}n! \sqrt{\pi}}\right)^{1/2}
  e^{-\kappa^2 y^2/2}
  \mathcal{H}_n \qty(\kappa y),
\end{equation}
with eigenvalues $\epsilon_n = \hbar \omega_0 [n + (1/2)]$ where $\kappa = \sqrt{{m_e \omega_0}/{\hbar}}$, $\mathcal{H}_n(\cdot)$ is the $n$-th Hermite polynomial, and the cyclotron frequency $\omega_0 = eB/m_e$.
The path shift of the driven classical oscillator $\zeta(t)$ can be given by
\begin{equation} \label{eq:4}
  \zeta(t) = \frac{eE}{m_e(\omega_0^2 - \omega^2)}\sin(\omega t),
\end{equation}
and we introduce $\dot{\zeta}(t) = \dv*{\zeta(t)}{t}$ for the sake of notational ease. We can identify the Lagrangian of the driven classical oscillator $L(\zeta,\dot{\zeta},t)$ as
\begin{equation} \label{eq:5}
  L(\zeta,\dot{\zeta},t) = \frac{1}{2} m_e\dot{\zeta}^2(t) - \frac{1}{2}m_e\omega_0^2 \zeta^2(t) + eE\zeta(t) \sin(\omega t).
\end{equation}
For details of the full derivation, refer to Appendix \ref{appendix_a}.
The exponential phase shifts in Eq.~(\ref{eq:2}) represent the influence of the stationary magnetic field and dressing field on the electron behavior of our system. Therefore, we can observe that the magneto-transport characteristics of 2DEG can be renormalized by a nonoscillating magnetic field along with a dressing field.
