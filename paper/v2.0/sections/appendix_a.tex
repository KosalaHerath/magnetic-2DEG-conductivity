The derivation of the solutions for the time-dependent Schrödinger equation with our system's Hamiltonian (Eq. \ref{eq:1}) is quite similar to that followed in Refs. \cite{husimi53,dini16}. We start with expanding the Hamiltonian for two-dimensional scenario
\begin{equation} \label{eq:a1}
  \hat{H}_e(t) = \frac{1}{2m_e}
  \left\{
    \left(\hat{p}_x + eBy \right)^2 +
    \left[
      \hat{p}_y - \frac{eE}{\omega}\cos(\omega t)
    \right]^2
  \right\}.
\end{equation}
Since $\left[\hat{H}_e(t),\hat{p}_x \right] =0$, both of these operators share same eigenfunctions
${L_x}^{-1/2}\exp(\flatfrac{ip_x x}{\hbar})$ where $p_x = \flatfrac{2\pi \hbar m}{L_x}$ with $ m \in \mathbb{Z}$.
Thus, we re-arrange the Hamiltonian using the definition of canonical momentum in $y$-direction and this leads to
\begin{equation} \label{eq:a2}
    \hat{H}_e(t) = \frac{1}{2m_e}
    \left\{
      \left({p}_x + eBy \right)^2 +
      \left[
        i\hbar \pdv{y}- \frac{eE}{\omega}\cos(\omega t)
      \right]^2
    \right\}.
\end{equation}
Subsequently we define the \textit{center of the cyclotron orbit} on the $y$-axis as $y_0 = \flatfrac{-p_x}{(eB)}$, and the \textit{cyclotron frequency} as  $\omega_0 = {eB}/{m_e}$. This leads to a new arrangement of the Hamiltonian
\begin{equation} \label{eq:a3}
  \begin{aligned}
    \hat{H}_e(t) =
      \frac{m_e \omega_0^2}{2}\tilde{y}^2 +
      \frac{1}{2m_e} &
      \left[
        -\hbar^2 \pdv[2]{\tilde{y}}  +
        \frac{2i\hbar eE}{\omega} \cos(\omega t) \pdv{\tilde{y}}
        \right.\\
        & \left. \qquad\qquad\quad +
        \frac{e^2E^2}{\omega^2}\cos[2](\omega t)
        \right],
  \end{aligned}
\end{equation}
where we used a variable substitution $\tilde{y} = (y - y_0)$. Furthermore, we assume that the wave function solutions for the time-dependent Schrödinger equation of considered quantum system
\begin{equation} \label{eq:a4}
    i \hbar \dv{\psi}{t} = \hat{H}_e(t)\psi,
\end{equation}
can be presented by the following form
\begin{equation} \label{eq:a5}
    \psi_m(x,\tilde{y},t) = \frac{1}{\sqrt{L_x}}
    \exp\left(
      \frac{ip_x x}{\hbar} +
      \frac{ieE\tilde{y}}{\hbar \omega}\cos(\omega t)
    \right) \vartheta(\tilde{y},t),
\end{equation}
where $\vartheta(\tilde{y},t)$ is a function that satisfy the property
\begin{equation} \label{eq:a6}
    \left[
    \frac{m_e \omega_0^2}{2} \tilde{y}^2
    - {eE\tilde{y}}\sin(\omega t)
    - \frac{\hbar^2}{2m_e} \pdv[2]{\tilde{y}}
    - i \hbar \dv{t}
    \right]
    \vartheta(\tilde{y},t) = 0.
\end{equation}
If we turn off the dressing field ($E=0$), this equation leads to the Schrödinger equation with the simple harmonic oscillator Hamiltonian
\begin{equation} \label{eq:a7}
     i \hbar \dv{\vartheta(\tilde{y},t)}{t} =
    \left(
    \frac{\hat{p}_{\tilde{y}}^2}{2m_e} +
    \frac{1}{2}m_e \omega_0^2\tilde{y}^2
    \right)
    \vartheta(\tilde{y},t).
\end{equation}
Thus, we can identify $S(t) = eE\sin(\omega t)$ term as an external force that act on the harmonic oscillator, and we can solve Eq.~(\ref{eq:a6}) as a forced harmonic oscillator on $\tilde{y}$ axis
\begin{equation} \label{eq:a8}
  \begin{aligned}
    i \hbar \dv{\vartheta(\tilde{y},t)}{t} =
    \bigg[
    -
    \frac{\hbar^2}{2m_e}
    \pdv[2]{\tilde{y}} +
    \frac{1}{2}m_e \omega_0^2\tilde{y}^2
    - \tilde{y}S(t)]
    \bigg]
    \vartheta(\tilde{y},t).
  \end{aligned}
\end{equation}
This system is exactly solvable, and we can solve the equation using the methods explained by Husimi \cite{husimi53}. We introduce a time-dependent shifted coordinate $ y' = \tilde{y} - \zeta(t)$ and perform the following unitary transformation
\begin{equation} \label{eq:a9}
    \vartheta(y',t) = \exp(\frac{im_e\dot{\zeta}y'}{\hbar})\varphi(y',t).
\end{equation}
This leads to
\begin{equation} \label{eq:a10}
  \begin{aligned}
    i \hbar \pdv{\varphi(y',t)}{t}   &=
    \left\{
        -  \frac{\hbar^2}{2m_e}\pdv[2]{{y'}}
        + \frac{1}{2} m_e \omega_0^2 y'^2
        \right. \\
        & \left. \qquad
        - \frac{1}{2} m_e\dot{\zeta}^2 + \frac{1}{2}m_e\omega_0^2 \zeta^2 - \zeta S(t)
        \right. \\
        & \left. \qquad +
        \left[
            m_e\ddot{\zeta} + m_e\omega_0^2\zeta - S(t)
        \right]y'
    \right\} \varphi(y',t),
  \end{aligned}
\end{equation}
where we introduce $\dot{\zeta} = \dv*{\zeta}{t}$ and $\ddot{\zeta} = \dv*[2]{\zeta}{t}$ for the sake of notational convenience. Subsequently, we can restrict $\zeta(t)$ function such that
\begin{equation} \label{eq:a11}
  m_e\ddot{\zeta} + m_e\omega_0^2\zeta = S(t),
\end{equation}
and that simply our previous expression as
\begin{equation} \label{eq:a12}
  \begin{aligned}
    i \hbar \pdv{\varphi(y',t)}{t} =&
    \left[
        -  \frac{\hbar^2}{2m_e}\pdv[2]{{y'}}
        + \frac{1}{2} m_e \omega_0^2 {y'}^2 \right. \\
        & \left. \qquad\qquad\qquad
        - L(\zeta,\dot{\zeta},t)
    \right]\varphi(y',t).
  \end{aligned}
\end{equation}
Here
\begin{equation} \label{eq:a13}
  L(\zeta,\dot{\zeta},t) = \frac{1}{2} m_e\dot{\zeta}^2 - \frac{1}{2}m_e\omega_0^2 \zeta^2 + \zeta S(t),
\end{equation}
is the Lagrangian of a classical driven oscillator. To proceed further, another unitary transform can be introduced as follows
\begin{equation} \label{eq:a14}
    \varphi(y',t) = \exp(\frac{i}{\hbar}\int_0^{t}dt'L(\zeta,\dot{\zeta},t')) \chi(y',t),
\end{equation}
and subtitling Eq.~(\ref{eq:a14}) back in Eq.~(\ref{eq:a12}), we can obtain
\begin{equation} \label{eq:a15}
    i \hbar \pdv{t} \chi(y',t)  =
    \left(
        -  \frac{\hbar^2}{2m_e}\pdv[2]{{y'}}
        + \frac{1}{2} m_e \omega_0^2 {y'}^2
    \right) \chi(y',t).
\end{equation}
This is the well-known Schrödinger equation of the quantum harmonic oscillator.
This allows us to identify the well known eigenfunction solutions \cite{griffiths18,shankar94}
\begin{equation} \label{eq:a16}
  \chi_n(y) =
  \sqrt{\frac{\kappa}{2^{n}n!\sqrt{\pi}}}
  e^{-\kappa^2 y^2/2}
  \mathcal{H}_n \qty(\kappa y),
\end{equation}
with eigenvalues
\begin{equation} \label{eq:a17}
  \epsilon_n = \hbar \omega_0 \bigg(n + \frac{1}{2}\bigg)
  ~\text{for}~
  n \in \mathbb{Z}^{+}_0.
\end{equation}
Here, $\kappa = \sqrt{{m_e \omega_0}/{\hbar}}$, and $\mathcal{H}_n$ are the Hermite polynomials.
Thus, we can identify the solutions for Eq.~(\ref{eq:a8}) as
\begin{equation} \label{eq:a18}
  \begin{aligned}
    \vartheta_n(\tilde{y},t) = \chi_n\bm{(}\tilde{y} - \zeta(t)\bm{)}
     \text{exp}&
     \left(
     \frac{i}{\hbar}
     \left\{
     - \epsilon_nt +
     m_e \left[\tilde{y}-\zeta(t)\right] \dot{\zeta(t)} \right. \right. \\
     & \left. \left. \qquad
     + \int_0^{t}L(\zeta,\dot{\zeta},t') dt'
     \right\}
     \right).
  \end{aligned}
\end{equation}
Since $\chi_n(x)$ functions forms a complete set, any general solution for  $\vartheta_(\tilde{y},t)$ can be presented using the solutions derived in Eq.~(\ref{eq:a18}).

Finally, we consider our scenario where we can assume that $S(t) = eE\sin(\omega t)$, and we can derive the solution for Eq.~(\ref{eq:a11}) as
\begin{equation} \label{eq:a19}
  \zeta(t) = \frac{eE}{m_e(\omega_0^2 - \omega^2)}\sin(\omega t).
\end{equation}
Substituting solutions given in Eq.~(\ref{eq:a18}) back in Eq.~(\ref{eq:a5}), we obtain a set of wave functions with two different quantum number ($n$,$m$) that satisfy the time-dependent Schrödinger equation Eq.~(\ref{eq:a4}) as follows
\begin{equation} \label{eq:a20}
  \begin{aligned}
    &\psi_{n,m}(x,y,t) \\
    & \quad=  \frac{1}{\sqrt{L_x}}
    \chi_n\left(y - y_0 - \zeta(t)\right)\\
    &\qquad\times
    \text{exp}\left(
    \frac{i}{\hbar}
    \left\{- \epsilon_nt
    + p_x x
    + \frac{eE(y - y_0)}{\omega}\cos(\omega t) \right. \right. \\
    & \left. \left. \qquad\qquad +
    m_e\left[y - y_0 -\zeta(t)\right] \dot{\zeta}(t)+
    \int_0^{t}L(\zeta,\dot{\zeta},t') dt'
    \right\}
    \right).
  \end{aligned}
\end{equation}
