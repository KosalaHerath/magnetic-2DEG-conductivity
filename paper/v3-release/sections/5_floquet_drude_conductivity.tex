% Section 05 - Floquet-Drude Conductivity in Quantum Hall Systems

A general theory for the conductivity of a dressed system with disorder was reported by Wackerl \textit{et al.} \cite{wackerl20,wackerlthesis20}. This theory, the general $x$-directional longitudinal DC-limit conductivity has been characterized as
\begin{equation} \label{eq:22}
  \begin{aligned}
    \sigma^{xx} = &
    \frac{-1}{4\pi\hbar A}
    \int_{\Pi-\hbar\omega/2}^{\Pi+ \hbar\omega/2}
    \left(
    -\pdv*{f}{\varepsilon }\right) \\
    &  \quad\times
    \tr
    \left\{
    {j}^x_0
    \left[
    \vb{G}^{r} (\varepsilon) - \vb{G}^{a} (\varepsilon)
    \right]
    {j}^x_0
    \left[
    \vb{G}^{r} (\varepsilon) - \vb{G}^{a} (\varepsilon)
    \right]
    \right\} d\varepsilon,
  \end{aligned}
\end{equation}
where $j^x_0$ is the $x$-directional electric current operator matrix elements' $0$-th Fourier component. Here, $\vb{G}^{r}(\varepsilon)$ and $\vb{G}^{a} (\varepsilon)$ are the retarded and advanced white noise disorder averaged Floquet Green function matrices \cite{wackerl20,wackerlthesis20}, respectively. These matrices are defined against the Floquet modes of the considering system. Here we have assumed that only $0$-th Fourier component of the current operator is contributing to the conductivity. In addition, $A$ is the area of the considered two-dimensional system, $f$ is the partial distribution function, and $\Pi$ is a function that can be chosen such that
\begin{equation} \label{eq:23}
    \Pi- \flatfrac{\hbar \omega}{2}
    \leq \varepsilon_N
    <
    \Pi + \flatfrac{\hbar \omega}{2}.
\end{equation}
Here $ \varepsilon_N$ are quasienergies of all relevant Floquet states, and $tr\{\cdot\}$ is the trace of the considering operator.

Next, we restrict our analysis into the off-resonant regime $\omega\tau_0 \gg 1$), where $\tau_0$ is the scattering time of the undriven system. Thus, we can expand the $x$-directional longitudinal conductivity given in
Eq.~(\ref{eq:22}) using only the central entry Fourier components ($l=l'=0$) of Floquet modes $\ket{\phi_{n,m}} = \ket{n,k_x}$ as
\begin{widetext}
\begin{equation} \label{eq:24}
  \begin{aligned}
    \sigma^{xx} =
    \frac{-1}{4\pi\hbar A}
    \int_{\Pi-\hbar\omega/2}^{\Pi+ \hbar\omega/2}
    \left(
      -\frac{\partial f}{\partial \varepsilon}
    \right)
    \frac{1}{V_{k_x}} \sum_{k_x}
    \sum_{n}
    \mel{n,k_x}
    {
      {j}^x_0
      \left[
        \vb{G}^{r} (\varepsilon) - \vb{G}^{a} (\varepsilon)
      \right]
      {j}^x_0
      \left[
        \vb{G}^{r} (\varepsilon) - \vb{G}^{a} (\varepsilon)
      \right]
    }
    {n,k_x}
    d\varepsilon,
  \end{aligned}
\end{equation}
where $V_{k_x}$ is the volume of considering $x$-directional momentum space. Next, we evaluate the above expression as follows
\begin{equation} \label{eq:25}
  \begin{aligned}
    \sigma^{xx}  = &
    \frac{-1}{4\pi\hbar A}
    \int_{\Pi-\hbar\omega/2}^{\Pi+ \hbar\omega/2}
    \left(
      -\frac{\partial f}{\partial \varepsilon}
    \right)
    \frac{1}{V_{k_x}^4}
    \sum_{k_x} \sum_{n}
    \sum_{{k_x}_1,{k_x}_2,{k_x}_3}
    \sum_{n_1,n_2,n_3} \\
    & \quad\times \mel{n,k_x}{
    {j}^x_0}
    {n_1,{k_x}_1}
    \mel{{n_1,{k_x}_1}}{
    \left[
      \vb{G}^{r} (\varepsilon) - \vb{G}^{a} (\varepsilon)
    \right]
    }
    {n_2,{k_x}_2}
    \mel{n_2,{k_x}_2}{
    {j}^x_0}
    {n_3,{k_x}_3}
    \mel{n_3,{k_x}_3}{
    \left[
      \vb{G}^{r} (\varepsilon) - \vb{G}^{a} (\varepsilon)
    \right]
    }
    {n,k_x} d\varepsilon.
  \end{aligned}
\end{equation}
We can diagonalize the impurity averaged Green's functions using a unitary transformation ($\vb{T}  = \ket{n,k_x}$) as mentioned in Refs.~\cite{wackerl20,wackerlthesis20,tsuji08}. Thus, we evaluate the matrix elements of the difference between retarded and advanced Green's functions as
\begin{equation} \label{eq:26}
  \mel{{n_1,{k_x}_1}}
  {
    \vb{T}^{\dagger}
    \left[
    \vb{G}^{r} (\varepsilon) - \vb{G}^{a} (\varepsilon)
    \right]\vb{T}
  }
  {n_2,{k_x}_2} =
  \frac
  {
    2i \Im(\vb{T}^{\dagger} \vb{\Sigma}^r \vb{T})
    \delta_{n_1,n_2}\delta_{{k_x}_1,{k_x}_2}
  }
  {
    \left(
    \flatfrac{\varepsilon}{\hbar} -
    \flatfrac{\varepsilon_{n_1}}{\hbar}
    \right)^2
    + \left[\Im(\vb{T}^{\dagger} \vb{\Sigma}^r \vb{T})\right]^2
  },
\end{equation}
and
\begin{equation} \label{eq:27}
  \mel{{n_3,{k_x}_3}}{
  \vb{T}^{\dagger}
  \left[
  \vb{G}^{r} (\varepsilon) - \vb{G}^{a} (\varepsilon)
  \right]\vb{T}}
  {n,{k_x}} =
  \frac{
    2i \Im(\vb{T}^{\dagger} \vb{\Sigma}^r \vb{T})
    \delta_{n_3,n}\delta_{{k_x}_3,{k_x}}
  }
  {
    \left(
    \flatfrac{\varepsilon }{\hbar}-
    \flatfrac{\varepsilon_{n}}{\hbar}
    \right)^2
    + \left[\Im(\vb{T}^{\dagger} \vb{\Sigma}^r \vb{T})\right]^2
  }.
\end{equation}
Here we introduced the retarded self-energy matrix $\vb{\Sigma}^r$ which is the sum of all irreducible diagrams \cite{wackerl20,wackerlthesis20}. Applying the matrix elements of the electric current operator in Landau levels and
expressions from Eq.~(\ref{eq:26}) and Eq.~(\ref{eq:27}) back into Eq.~(\ref{eq:25}) we obtain
\begin{equation} \label{eq:28}
  \begin{aligned}
    \sigma^{xx}  =
    \frac{-1}{4\pi\hbar A}
    \int_{\Pi-\hbar\omega/2}^{\Pi+ \hbar\omega/2}&
    \left(
    -\frac{\partial f}{\partial \varepsilon}
    \right)
    \frac{1}{V_{k_x}}
    \sum_{k_x} \sum_{n} \sum_{n_1,n_2}
    \\
    & \times
    \frac{e^2B}{{m_e}}
    \left(
    \sqrt{\frac{n+1}{2}} \delta_{n_1,n+1} + \sqrt{\frac{n}{2}}\delta_{n_1,n-1}
    \right)
    \left\{
    \frac
    {
      2i \Im(\vb{T}^{\dagger} \vb{\Sigma}^r \vb{T})
      \delta_{n_1,n_2}
    }
    {
      \left(
      \flatfrac{\varepsilon}{\hbar} -
      \flatfrac{\varepsilon_{n_1}}{\hbar}
      \right)^2
      + \left[\Im(\vb{T}^{\dagger} \vb{\Sigma}^r \vb{T})\right]^2
    }
    \right\} \\
    & \times
    \frac{e^2B}{{m_e}}
    \left(
    \sqrt{\frac{n_2+1}{2}} \delta_{n,n_2+1} + \sqrt{\frac{n_2}{2}}
    \delta_{n,n_2-1}
    \right)
    \left\{
    \frac{
      2i \Im(\vb{T}^{\dagger} \vb{\Sigma}^r \vb{T})
    }
    {
      \left(
      \flatfrac{\varepsilon }{\hbar}-
      \flatfrac{\varepsilon_{n}}{\hbar}
      \right)^2
      + \left[\Im(\vb{T}^{\dagger} \vb{\Sigma}^r \vb{T})\right]^2
    }
    \right\}  d\varepsilon,
  \end{aligned}
\end{equation}
For the full derivation of electric current operators in a quantum Hall system, refer to Appendix \ref{appendix_d}.
After the expansion, we can identify the the only non-zero term as follows
\begin{equation} \label{eq:29}
  \begin{aligned}
    \sigma^{xx} =
    \frac{-1}{4\pi\hbar A}
    \frac{e^4B^2}{{{m_e}^2}} &
    \int_{\Pi-\hbar\omega/2}^{\Pi+ \hbar\omega/2}
    \left(
    -\frac{\partial f}{\partial \varepsilon}\right)
    \frac{1}{V_{k_x}} \sum_{k_x} \sum_{n}
    (n+1)
    \\
    & \times
    \left\{
    \frac
    {
      2i \Im\bm{(}\vb{T}^{\dagger} \vb{\Sigma}^r(\varepsilon_{n+1},k_x) \vb{T}\bm{)}
    }
    {
      \left(
      \flatfrac{\varepsilon}{\hbar} -
      \flatfrac{\varepsilon_{n+1}}{\hbar}
      \right)^2
      + \left[\Im\bm{(}\vb{T}^{\dagger} \vb{\Sigma}^r(\varepsilon_{n+1},k_x) \vb{T}\bm{)}\right]^2
    }
    \right\}
    \left\{
    \frac{
      2i \Im\bm{(}\vb{T}^{\dagger} \vb{\Sigma}^r(\varepsilon_n,k_x) \vb{T}\bm{)}
    }
    {
      \left(
      \flatfrac{\varepsilon }{\hbar}-
      \flatfrac{\varepsilon_{n}}{\hbar}
      \right)^2
      + \left[\Im\bm{(}\vb{T}^{\dagger} \vb{\Sigma}^r(\varepsilon_n,k_x) \vb{T}\bm{)}\right]^2
    }
    \right\}
    d\varepsilon.
  \end{aligned}
\end{equation}
The inverse scattering time matrix is equal to the diagonalized contrast of the retarded and advanced self-energy \cite{wackerl20,wackerlthesis20}. In addition, on the diagonal the contrast of the retarded and advanced Green's function can be represented with the imaginary component of the retarded self-energy \cite{wackerl20,wackerlthesis20}. Subsequently, we can identify the following property
\begin{equation} \label{eq:30}
  \left(\frac{1}{\tau(\varepsilon,k_x)}\right)^{ll} =
  -2\text{Im} \left[ \vb{T}^{\dagger} \vb{\Sigma}^r(\varepsilon,k_x) \vb{T}\right]^{ll}.
\end{equation}
Afterwards, considering only the central element ($l=0$) of the inverse scattering time matrix, we can restructure the derived conductivity expression in Eq.~(\ref{eq:29}) as follows
\begin{equation} \label{eq:31}
    \sigma^{xx}   =
    \frac{1}{\pi\hbar A}
    \frac{e^4B^2}{{{m_e}^2}}
    \int_{\Pi-\hbar\omega/2}^{\Pi+ \hbar\omega/2}
    \left(
      -\frac{\partial f}{\partial \varepsilon}
    \right)
    \frac{1}{V_{k_x}} \sum_{k_x} \sum_{n} (n+1)
    \left[
    \frac{\widetilde{{\Gamma}}(\varepsilon_{n+1})
    }
    {
    \left(
    \varepsilon_F - \varepsilon_{n+1}
    \right)^2
    + \widetilde{{\Gamma}}^2(\varepsilon_{n+1})
    }
    \right]
    \left[
    \frac{\widetilde{{\Gamma}}(\varepsilon_{n})
    }
    {
    \left(
    \varepsilon_F - \varepsilon_{n}
    \right)^2
    + \widetilde{{\Gamma}}^2(\varepsilon_{n})
    }
    \right]
    d\varepsilon,
\end{equation}
\end{widetext}
with $\widetilde{{\Gamma}}(\varepsilon_n,k_x) = \left[\flatfrac{\hbar}{2\tau(\varepsilon_n,k_x)}\right]^{00}$. Since we already identified that the inverse scattering time matrix's central element is independent of $k_x$ value, we can drop the $k_x$-dependent in the $\widetilde{{\Gamma}}(\varepsilon_n,k_x)$ terms. Subsequently, we can get the sum over available momentum in $x$-direction.
However, by considering the condition that the center of the cyclotron orbit $y_0$ must physically lie within the considered system, we can identify that
\begin{equation} \label{eq:32}
 -\flatfrac{m_e\omega_0 Ly}{(2\hbar)} \leq k_x \leq \flatfrac{m_e\omega_0 Ly}{(2\hbar)}.
\end{equation}
We use the Fermi-Dirac distribution as our partial distribution function ($f$) for our system
\begin{equation} \label{eq:33}
  f(\varepsilon) = \frac{1}{\exp[(\varepsilon - \varepsilon_F)/k_B T]+1},
\end{equation}
where $k_B$ is the Boltzmann constant, $T$ is the absolute temperature and $\varepsilon_F$ is the Fermi energy of the system. Considering the above distribution for extremely low-temperature conditions, we can use the following approximation
\begin{equation} \label{eq:34}
  - \pdv{f(\varepsilon)}{\varepsilon} \approx \delta(\varepsilon - \varepsilon_F).
\end{equation}
Moreover, let $\Pi = \varepsilon_F$ and the derived expression in Eq.~(\ref{eq:31}) leads to
\begin{equation} \label{eq:35}
  \begin{aligned}
    \sigma^{xx}  =
    \frac{e^2}{\pi\hbar A}
    \sum_{n} &
    \frac{(n+1)}{\gamma_{n}\gamma_{n+1}} \\
    &\times
    \left[
      \frac{1}
      {
        1 + \left(\frac{X_F - n -1}{\gamma_{n+1}}\right)^2
      }
    \right]
    \left[
      \frac{1}
      {
        1 + \left(\frac{X_F - n}{\gamma_{n}}\right)^2
      }
    \right],
  \end{aligned}
\end{equation}
where $X_F = \left[\flatfrac{\varepsilon_F}{(\hbar \omega_0)} - \flatfrac{1}{2}\right]$
and
$\gamma_n = \flatfrac{\widetilde{{\Gamma}}(\varepsilon_n)}{(\hbar \omega_0)}$.
Following the same steps as above derivation, we can derive the longitudinal conductivity in the $y$-direction by applying the electric current operator for $y$-direction derived in Appendix \ref{appendix_d}
\begin{equation} \label{eq:36}
  \begin{aligned}
    {\sigma}^{yy} =
    \frac{e^2}{\pi\hbar A} &
    \frac{1}{e^2B^2}
    \sum_{n}
    \frac{(n+1)}{\gamma_{n}\gamma_{n+1}} \\
    & \times
    \left[
      \frac{1}
      {
        1 + \left(\frac{X_F - n -1}{\gamma_{n+1}}\right)^2
      }
    \right]
    \left[
      \frac{1}
      {
        1 + \left(\frac{X_F - n}{\gamma_{n}}\right)^2
      }
    \right].
  \end{aligned}
\end{equation}
