In this paper, we introduced a generalized mathematical model for predicting  charge transport properties in a 2DEG under a nonoscillating magnetic field and a high-intensity light. Under the uniform magnetic field, the charged particles can only settle in discrete energy values, which leads to the Landau quantization. We modeled the behavior of electrons in Landau levels under the dressing field utilizing the Floquet-Drude conductivity method. We assumed the impurities in the material as a Gaussian random scattering potential. Finally, we derived expressions for x-directional and y-directional longitudinal components of electric conductivity tensor for the considered system.

Our derived analytical expressions disclosed that we are able to control the transport characteristics of the dressed quantum Hall system by the applied dressing field’s intensity. Using detailed numerical calculations with empirical system parameters, we further analyzed the manipulation of conductivity components using the dressing field.
We found that the graphical illustrations that we obtained from these numerical calculations are capable of producing the same behavior as experiments on quantum Hall systems in the absence of a dressing field.
Furthermore, we identified that by increasing the intensity of the radiation, we can squeeze the conductivity regions near the Landau levels. Despite this behavior being identified in previous works, their results did not coincide with the more accurate description of conductivity components in undressed quantum Hall systems. However, our generalized analysis on the conductivity of dressed quantum Hall systems provide a well-suited description for these specific quantum Hall systems.

In summary, the primary purpose of this study was to broaden the modern descriptions on transport properties of dressed quantum Hall systems. Moreover, our detailed theoretical analysis showed that we can adapt the recently introduced Floquet-Drude conductivity model to extend the models that were used to describe the transport characteristics in quantum Hall systems. Owing to the ability to control the conductivity regions, high intensity external illumination may be used as a trigger for two-dimensional quantum switching devices, which are employed as the building blocks of next-generation nanoelectronic devices. As a concluding remark, we believe that our findings of this paper can be used towards understanding the 2D optoelectronic nanotransistors, enhancing their performance and inventing novel appliances.
