A generalized mathematical model for predicting the transport properties of a quantum system exposed to a stationary magnetic field and a high intensity electromagnetic field is presented. The new formulation, which applies to two-dimensional(2D) dressed quantum Hall systems, is based on Landau quantization theory and Floquet-Drude conductivity approach.
We model our system as a two-dimensional electron gas (2DEG) that interacts with two external fields. To analyze the strong light coupling with the 2DEG, we employ the Floquet theory as a nonperturbative procedure. Moreover, the Floquet Fermi golden rule is adopted to explore the impurity scattering effects against Floquet states in disordered quantum Hall systems.
We derive fully analytical expressions to describe longitudinal components in conductivity tensor in dressed quantum Hall systems and apply the developed model to a numerical evaluation with empirical parameters. Then, we demonstrate that the conductivity characteristics in quantum Hall systems can be manipulated using a strong external light.
Our results align with well-established and experimentally
verified theoretical descriptions for undressed systems while providing a more generalized analysis on the characteristics of conductivity in quantum Hall systems.
Thus, our model can be used to more accurately interpret the employment of external strong radiation as a tool to engineer transport properties in various 2D nanostructures.
