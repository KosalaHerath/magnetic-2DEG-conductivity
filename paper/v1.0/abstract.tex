A generalized mathematical model for the transport properties of quantum systems exposed to a stationary magnetic field and a high intensity electromagnetic field is presented. The new formulation, which applies to the two-dimensional dressed quantum Hall systems, is based on Landau quantization theory and Floquet-Drude conductivity approach. We model our system as a two-dimensional electron gas (2DEG) that interacts with two external fields. To analyze the strong light coupling with the 2DEG, we utilize the Floquet theory as a nonperturbative procedure. Moreover, the Floquet Fermi golden rule is adopted to explore the impurity scattering effects against Floquet states in disordered quantum Hall systems. Based on our fully analytical expression and particular graphical representations, we demonstrate that the characteristics of conductivity in two-dimensional quantum Hall systems can be manipulated using an external strong light. The outcomes align with the theoretical descriptions for undressed systems which are already well-suited with experimental results and at the same time our theory provides a more generalized analysis on the characteristics of conductivity of quantum Hall systems. Thus, this model more realistically describes that how to employ external strong radiation as a tool to utilize transport properties in various 2D nanostructures which serve as a basis for nanoscale quantum devices.
