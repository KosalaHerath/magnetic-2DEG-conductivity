The derivation of the Floquet Fermi golden rule for our quantum Hall system with the help of $t-t'$ formalism is given here in detail. The $t$-$t'$-Floquet states \cite{grifoni98,wackerl20}
\begin{equation} \label{eq_c1}
  \ket{\psi_{n,m}(t,t')} =
  \exp(-\frac{i}{\hbar}\varepsilon_{n} t)\ket{\phi_{n,m}(t')}.
\end{equation}
derived by seperating the aperiodic and periodic components of Eq.~(\ref{eq_12}), fullfill the $t$-$t'$-Schrödinger equation \cite{grifoni98,wackerl20}
\begin{equation} \label{eq_c2}
  i \hbar \pdv{t}\ket{\psi_{n,m}(t,t')} =
  H_F(t') \ket{\psi_{n,m}(t,t')},
\end{equation}
where \textit{Floquet Hamiltonian} defined as
\begin{equation} \label{eq_c3}
  H_F(t') \equiv
  H_e(t') - i\hbar \pdv{t'}.
\end{equation}
Next we can identify the the time evolution operator \cite{wackerl20} corresponding to the $t$-$t'$-Schrödinger equation given in Eq.~(\ref{eq_c2})
\begin{equation} \label{eq_c4}
  U_0(t,t_0;t') = \exp(-\frac{i}{\hbar}H_F(t')\qty[t-t_0])
\end{equation}
and the advantage of $t$-$t'$ formalism lies on this time evolution operator which avoids any time odering operatos.
Consider a time-independent total perturbation $V(\vb{r})$ switched on at the reference time $t=t_0$, then Schrödinger equation becomes
























x
