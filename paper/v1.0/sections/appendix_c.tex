The derivation of the Floquet Fermi golden rule for our quantum Hall system with the help of $t-t'$ formalism is given here in detail. The $t$-$t'$-Floquet states \cite{grifoni98,wackerl20}
\begin{equation} \label{eq_c1}
  \ket{\psi_{n,m}(t,t')} =
  \exp(-\frac{i}{\hbar}\varepsilon_{n} t)\ket{\phi_{n,m}(t')}.
\end{equation}
derived by seperating the aperiodic and periodic components of Eq.~(\ref{eq_12}), fullfill the $t$-$t'$-Schrödinger equation \cite{grifoni98,wackerl20}
\begin{equation} \label{eq_c2}
  i \hbar \pdv{t}\ket{\psi_{n,m}(t,t')} =
  H_F(t') \ket{\psi_{n,m}(t,t')},
\end{equation}
where \textit{Floquet Hamiltonian} defined as
\begin{equation} \label{eq_c3}
  H_F(t') \equiv
  H_e(t') - i\hbar \pdv{t'}.
\end{equation}
Next we can identify the the time evolution operator corresponding to the $t$-$t'$-Schrödinger equation
\begin{equation} \label{eq_c4}
  U_F(t,t_0;t') = \exp(-\frac{i}{\hbar}H_F(t')\qty[t-t_0]),
\end{equation}
and the advantage of $t$-$t'$ formalism lies on this time evolution operator which avoids any time odering operatos \cite{wackerl20}.

For our scenario, consider a time-independent total perturbation $V(\vb{r})$ which has been switched on at the reference time $t=t_0$, then the $t$-$t'$-Schrödinger equation becomes
\begin{equation} \label{eq_c5}
  i \hbar \pdv{t}\ket{\Psi_{n,m}(t,t')} =
  \qty[H_F(t') + V(\vb{r})]\ket{\Psi_{n,m}(t,t')},
\end{equation}
by introducing new wave function $\Psi_{n,m}$ for the system with the given total perturabation. If $t\leq t_0$, both solutions of the Schrödinger equations (Eq.~(\ref{eq_c2}) and Eq.~(\ref{eq_c5})) coincide
\begin{equation} \label{eq_c6}
  \ket{\psi_{n,m}(t,t')} =\ket{\Psi_{n,m}(t,t')} \quad
  \text{when} \quad
  t \leq t_0.
\end{equation}
Now move into the interaction picture representation \cite{bruus04,mahan00} of the $t$-$t'$-Floquet state
\begin{equation} \label{eq_c7}
  \ket{\Psi_{n,m}(t,t')}_I = U_0^{\dagger}(t,t_0;t')
  \ket{\Psi_{n,m}(t,t')},
\end{equation}
and due to time independency, the perturbation in the interaction picture has the same form as Schrödinger picture
\begin{equation} \label{eq_c8}
  V_I(\vb{r}) = U_0^{\dagger}(t,t_0;t')V(\vb{r})U_0(t,t_0;t') =
  V(\vb{r}).
\end{equation}
This leads to the $t$-$t'$-Schrödinger eqution in the interction picture
\begin{equation} \label{eq_c9}
  i \hbar \pdv{t}\ket{\Psi_{n,m}(t,t')}_I =
  V_I(\vb{r})\ket{\Psi_{n,m}(t,t')}_I,
\end{equation}
with the recursive solution \cite{bruus04,mahan00}
\begin{equation} \label{eq_c10}
  \begin{aligned}
  \ket{\Psi_{n,m}(t,t')}_I = &\ket{\Psi_{n,m}(t_0,t')}_I \\
  &+
  \frac{1}{i\hbar}
  \int_{t_0}^t dt_1 \;
  V_I(\vb{r}) \ket{\Psi_{n,m}(t_1,t')}_I.
  \end{aligned}
\end{equation}
Iterating the solution only upto the first order (Born approximation) we obtain
\begin{equation} \label{eq_c11}
  \begin{aligned}
    \ket{\Psi_{n,m}(t,t')}_I \approx &\ket{\psi_{n,m}(t_0,t')} \\
    &+
    \frac{1}{i\hbar}
    \int_{t_0}^t dt_1 \;
    V_I(\vb{r}) \ket{\psi_{n,m}(t_0,t')}.
  \end{aligned}
\end{equation}
In addition, since our Floquet states create a basis for composite space we can represent any solution using our Floquet states
\begin{equation} \label{eq_c12}
  \ket{\Psi_{n,m}(t,t')} = \sum_{\beta} a_{n,m\beta}(t,t')
  \ket{\psi_{\beta}(t,t')}.
\end{equation}
Therefore we can derive a equation for this \textit{scattering amplitude} as
\begin{equation} \label{eq_c13}
  a_{n,m\beta}(t,t') =
  \braket{\psi_{\beta}(t,t')}{\Psi_{n,m}(t,t')} =
  -
  \frac{i}{\hbar}
  \int_{0}^t dt_1 \;
  \bra{\psi_{\beta}(t_1,t')}
  V(\vb{r}) \ket{\psi_{n,m}(t_1,t')}.
\end{equation}



















x
