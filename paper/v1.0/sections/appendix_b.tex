\subsection{Position space representation}

First define the time integral of Laggrangian of the classical oscillator given in Eq.~(\ref{eq_5}), over a $T=2\pi/\omega$ period as
\begin{equation} \label{eq_b1}
  \Delta_{\varepsilon} \equiv \frac{1}{T} \int_0^T dt' \; L(\zeta,\dot{\zeta},t'),
\end{equation}
and after performing the integral using Eq.~(\ref{eq_4}), we can obtain more simplified result:
\begin{equation} \label{eq_b2}
  \Delta_{\varepsilon} = \frac{(eE)^2}{4m_e(\omega_0^2 - \omega^2)}.
\end{equation}
Next define another paramter
\begin{equation} \label{eq_b3}
  \xi \equiv
  \int_0^t dt' \; L(\zeta,\dot{\zeta},t') -
  \Delta_{\varepsilon} t,
\end{equation}
and after simplying, this leads to
\begin{equation} \label{eq_b4}
  \xi =
  \frac{(eE)^2\qty(3\omega^2 - \omega_0^2)}{8m_e\omega(\omega_0^2 - \omega^2)^2} \sin(2\omega t),
\end{equation}
which is a periodic function in time with $2\omega$ frequency. Now using these  parmaters we can factorize the wavefunction Eq.~(\ref{eq_2}) as linearly time dependend part and periodic time dependend part as follows
\begin{equation} \label{eq_b5}
  \begin{aligned}
    \psi_{\alpha}(x,y,t)  = &
    \exp(\frac{i}{\hbar}\qty[-\varepsilon_nt + \Delta_{\varepsilon} t ])
    \frac{1}{\sqrt{L_x}} \chi_n\big(y - y_0 - \zeta(t)\big)
    \\
    & \times
    \text{exp}\bigg(
     \frac{i}{\hbar}\bigg[
     p_x x +
     \frac{eEy}{\omega}\cos(\omega t) \\
     & +
     m_e\dot{\zeta(t)}\big[y-\zeta(t)\big]
     + \int_0^{t}dt'L(\zeta,\dot{\zeta},t') - \Delta_{\varepsilon} t  \bigg]
     \bigg),
  \end{aligned}
\end{equation}
and this leads to seperate linear time dependent phase component as the quasienergies
\begin{equation} \label{eq_b6}
  \varepsilon_{n} =
  \hbar \omega_0\qty(n + \frac{1}{2}) - \Delta_{\varepsilon}
\end{equation}
while rest of the components as time-periodic Floquet modes
\begin{equation} \label{eq_b7}
  \begin{aligned}
    \phi_{n,m}(x,y,t) \equiv &
    \frac{1}{\sqrt{L_x}} \chi_{n}\left[y - y_0 - \zeta(t)\right]
    \text{exp}\bigg(
     \frac{i}{\hbar}\bigg[
     p_x x \\
     & +
     \frac{eE(y - y_0)}{\omega}\cos(\omega t) \\
     & +
     m_e\dot{\zeta}(t)\big[y - y_0 -\zeta(t)\big]
     + \xi \bigg]\bigg).
  \end{aligned}
\end{equation}

\subsection{Momentum space representation}

To write the Floquet modes in momentum space, we perform continuous Fourier transform over the considering confined space $A=L_xL_y$ for Eq.~(\ref{eq_7})
\begin{equation} \label{eq_b8}
  \begin{aligned}
    \phi_{n,m}(k_x,k_y,t)  &=
    \int_{-L_y/2}^{L_y/2} dy\; \exp(-i\qty[k_y - \gamma(t)]y)
    \chi_{n}\qty[y - \mu(t)] \\
     & \times
     \frac{1}{\sqrt{L_x}}
     \int_{-L_x/2}^{L_x/2} dx\;
     \exp(-ik_x x)
     \exp( \frac{i p_x }{\hbar}x ) \\
     &
     \times
     \exp(
      \frac{-i\gamma(t)}{\hbar}
      y_0)
     \exp(\frac{-i}{\hbar}
     \qty[
     m_e \dot{\zeta}(t) \zeta(t) - \xi
     ]),
  \end{aligned}
\end{equation}
where
\begin{equation} \label{eq_b9}
  \mu(t) \equiv \frac{eE\sin(\omega t)}{m_e(\omega_0^2 - \omega^2)} + y_0
  \quad,\quad
  \gamma(t) \equiv
  \frac{eE\omega_0^2\cos(\omega t)}{\hbar\omega(\omega_0^2 - \omega^2)}.
\end{equation}
Next using the identity \cite{bruus04}
\begin{equation} \label{eq_b10}
  \int_{L_x} dx\;
  \exp( -ik_x x + \frac{i p_x }{\hbar}x ) =
  L_x \delta_{k_x,\frac{p_x}{\hbar}},
\end{equation}
we can derive
\begin{equation} \label{eq_b11}
  \begin{aligned}
    \phi_{n,m}(k_x,k_y,t)  =
    \exp(
     \frac{-i\gamma(t)}{\hbar}
     y_0) &
    \exp(\frac{-i}{\hbar}
    \qty[
    m_e \dot{\zeta}(t) \zeta(t) - \xi
    ]) \\
    & \times
    \Phi_{n,m}(k_y,t)
    \delta_{k_x,\frac{p_x}{\hbar}}.
  \end{aligned}
\end{equation}
Here we defined $\Phi_{n,m}(k_y,t)$ as
\begin{equation} \label{eq_b12}
  \begin{aligned}
    \Phi_{n,m}(k_y,t) \equiv
    \sqrt{L_x}
    \int_{-L_y/2}^{L_y/2} dy\; &
    \chi_{n}\qty[y - \mu(t)] \\
    & \times
    \exp(
      -i\qty[k_y - \gamma(t)]
      y).
  \end{aligned}
\end{equation}
Subtituting $  {k'_y} = k_y -\gamma(t)$ and $y' = y -\mu(t)$ and assuming that size of the 2DEG sample in $y$-direction is large ($L_y \rightarrow \infty$), we can obtain
\begin{equation} \label{eq_b13}
  \Phi_{n,m}({k'_y} ,t) =
  {\sqrt{L_x}} e^{-i {k'_y}\mu}
  \int_{-\infty}^{\infty} dy'\;
  \chi_{n}\qty(y')
  \exp(-i{k'_y} y').
\end{equation}
We can identify that the integral represnts the Fourier transform of $\{\chi_n\}$ functions and using the symmetric conditions \cite{celeghini21} for the Fourier transform of Gauss-Hermite functions $\theta_n(x)$:
\begin{equation} \label{eq_b14}
  \mathcal{FT}[\theta_n(\kappa x),x,k] = \frac{i^n}{|\kappa|}\theta_n(k/\kappa),
\end{equation}
Eq.~(\ref{eq_b13}) can be simplified as
\begin{equation} \label{eq_b15}
  \Phi_{n,m}({k'_y} ,t) =
    \sqrt{L_x}e^{-i {k'_y}\mu}
    \tilde{\chi}_{n}\qty({k'_y}),
\end{equation}
with
\begin{equation} \label{eq_b16}
  \tilde{\chi}_{n}\qty(k) =
  \frac{i^n}{\sqrt{2^{n} n! \sqrt{\pi}}}
  \qty(\frac{1}{\kappa})^{1/2}
  e^{-\frac{k^2}{2 \kappa^2}}
  \mathcal{H}_{\alpha} \qty(\frac{k}{\kappa}).
\end{equation}
Subtitute Eq.~(\ref{eq_b15}) back in Eq.~(\ref{eq_b11}) and this leads to
\begin{equation} \label{eq_b17}
  \begin{aligned}
    \phi_{n,m}(k_x,k_y,t)  = &
    {\sqrt{L_x}}
    \tilde{\chi}_{n}\qty(k_y - b\cos(\omega t)) \\
    & \times
    \text{exp}\left(
      i\xi
      -ik_y  \qty[d\sin(\omega t) + \frac{\hbar k_x}{eB}]
    \right),
  \end{aligned}
\end{equation}
where
\begin{equation} \label{eq_b18}
  b \equiv
  \frac{eE\omega_0^2}{\hbar\omega(\omega_0^2 - \omega^2)} \quad
  d \equiv
 \frac{eE}{m_e(\omega_0^2 - \omega^2)},
\end{equation}
and it is necessary to notice that $k_x$ is quantized with $k_x = 2\pi m/L_x ~,~ m \in \mathbb{Z}$.
