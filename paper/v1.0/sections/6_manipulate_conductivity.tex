% Section 06 - Manipulate Conductivity in Quantum Hall Systems

To identify the characteristics of the longitudinal conductivity of the quantum Hall systems with external dressing field, first we can derive an expression for a normalized
longitudinal conductivity as a function of Fermi energy $X_F$ and intensity of the dressing field $I$. Here we have the normalized x-directional conductivity using the natural conductivity of the least Landau level
\begin{equation} \label{eq_35}
  \begin{aligned}
    \frac{\sigma^{xx}}{\sigma^{0}} = &
    \sum_{n}
    \frac{\qty(n+1)}{0.0037\Lambda_n \Lambda_{n+1}} \\
    & \times
    \qty[
      \frac{1}
      {
        1 + \qty(\frac{X_F - n -1}{0.06\Lambda_n})^2
      }
    ]
    \qty[
      \frac{1}
      {
        1 + \qty(\frac{X_F - n}{0.06\Lambda_{n+1}})^2
      }
    ],
  \end{aligned}
\end{equation}
where $\sigma^0 = (e^2/\pi \hbar A)$. We use this expression to illustrate the changes that can be done to the longitudinal conductivity in 2DEG using external dressing field. As given in Fig.~\ref{fig_5} and \ref{fig_6} we can manipulate the longitudinal conductivity $\sigma_{xx}$ using external dressing field's intensity and the Fermi level $X_F$ of the considering system. For a given dressing field intensity, the longitudinal conductivity vary against the Fermi level of the system by creating sharp peaks at each Landau level energy values. Since electrons are restricted to have only Landau level energies, the conductivity gets very low values when the Fermi level is not align with any of the Landau level energy values. In contrast, on each Landau level, the conductivity can achieve very high values compared to other areas and as illustrates on Fig.~\ref{fig_5} the peak value of longitudinal conductivity on each Landau level gets increase with the Landau level number.

\begin{figure}[t]
\includegraphics[scale=0.55]{figures/fig_5}
\caption{\label{fig_5} Normalized longitudinal conductivity $\sigma_{xx}$ against Fermi level $X_F$ with different intensities $I$ of the external dressing field in a GaAs-based quantum well under a nonoscillating magnetic field with $B = 1.2~\text{T}$, dressing field with frequency of $\omega =2\times10^{12}~\text{rad}\text{s}^{-1}$ and $I_0 =100~\text{W}/\text{cm}^{2}$. In this calculation we have assumed that the natural  broadening of $0$-th Landau level $\Gamma_0$ is $0.24\;\text{me}V$.}
\end{figure}
\begin{figure}[t]
\includegraphics[scale=0.55]{figures/fig_6}
\caption{\label{fig_6} $3$rd Landau level’s normalized longitudinal conductivity $\sigma_{xx}$ against Fermi level $X_F$ with different intensities $I$ of the external dressing field in a GaAs-based quantum well under a nonoscillating magnetic field with $B = 1.2~\text{T}$, dressing field with frequency of $\omega =2\times10^{12}~\text{rad}\text{s}^{-1}$ and $I_0 =100~\text{W}/\text{cm}^{2}$. In this calculation we have assumed that the natural  broadening of $0$-th Landau level $\Gamma_0$ is $0.24\;\text{me}V$}.
\end{figure}

Considering the effect of the external dressing field on longitudinal conductivity of 2DEG, we can identify that high intensities shrink the conductivity regions near Landau levels. However, the peak value of the conductivity at each Landau level has the same value as the undressed system. This demonstrate that we are able to tune the width of the regions of conductivity in these quantum Hall systems with the help of a dressing field.
These characteristics are aligned with results demonstrated by K. Dini \textit{et al.} \cite{dini16} and as they remarked since the Fermi level of the system can be change with the applied gate voltage of the material this can be used as a 2D switch for nanoscale optoelectronics applications. Controlling  the external dressing field we are able to fine-tune the switching mechanism for optimized performance.
Furthermore, we can distinguish that the shapes and behavior of the conductivity regions illustrated in Fig.~\ref{fig_5} and \ref{fig_6} are generally incompatible with the results reported in Ref.~\cite{dini16}. This is due to the selection of the conventional longitudinal conductivity theory of 2DEG from Ref.~\cite{ando74_1,ando82}. The semi-elliptical conductivity regions illustrated in Ref.~\cite{dini16,ando74_1,ando82}, have less consistence with the experimentally observed Landau levels representation \cite{endo09}.
In our study on the transport properties of quantum Hall systems, we developed the conductivity expression starting from Floquet-Drude conductivity \cite{wackerl20} and our results are much more align with the result mentioned in Ref. \cite{endo09}.
The description of conductivity of quantum Hall systems demonstrated in Ref.~\cite{endo09} has excellent agreement between the theory and experiment obtained in a GaAs/AlGaAs 2DES for the low magnetic field range. However, they have not considered the tunability that can be achieved with the external strong dressing field. In this analysis we account both magnetic and dressing field effects that can be applied into the transport properties of 2DEG, and we have presented a more generalized theory. As a concluding remark, in this study we were able to demonstrate that using Floquet-Drude conductivity method one can derive a more experimental fitting and generalized mathematical model that describes the transport properties of quantum Hall systems.
