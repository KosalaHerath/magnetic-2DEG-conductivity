% Section 05 - Floquet-Drude Conductivity in Quantum Hall Systems

A general theory for the conductivity of a dressed system with the disorder averaging was reported by Wackerl \textit{et al.} \cite{wackerl20,wackerlthesis20}. This theory, the general $x$-directional longitudinal DC-limit conductivity has been characterized as
\begin{equation} \label{eq_20}
  \begin{aligned}
    {\sigma}^{xx} &=
    \frac{-1}{4\pi\hbar A}
    \int_{\Pi-\hbar\omega/2}^{\Pi+ \hbar\omega/2} d\varepsilon \bigg[
    \qty(
    -\frac{\partial f}{\partial \varepsilon})
    \\
    & \times
    \tr
    \qty[
    {j}^x_0
    \qty(
    \vb{G}^{r} (\varepsilon) - \vb{G}^{a} (\varepsilon)
    )
    {j}^x_0
    \qty(
    \vb{G}^{r} (\varepsilon) - \vb{G}^{a} (\varepsilon)
    )
    ]\bigg],
  \end{aligned}
\end{equation}
where ${j}^x_0$ and $\vb{G}^{r,a} (\varepsilon)$ are the $x$-directional electric current operator matrix elements' $0$-th Fourier component and the white noise disorder averaged Floquet Green function matrix \cite{wackerl20,wackerlthesis20} respectively defined against the Floquet modes of the considering system. Here we have assumed that only $0$-th Fourier component of the current operator is contributing to the conductivity. In addition, $A$ is the area of the considered two-dimensional system, $f$ is the partial distribution function, and $\Pi$ is a function that can be chosen such that
\begin{equation} \label{eq_21}
    \Pi- \frac{\hbar \omega}{2}
    \leq \varepsilon_N
    <
    \Pi + \frac{\hbar \omega}{2}.
\end{equation}
Here $ \varepsilon_N$ are quasienergies of all relevant Floquet states, and $tr[\cdot]$ is the trace of the considering operator.

Next, we restrict our analysis into off-resonant regime $\omega\tau_0 \gg 1$), where $\tau_0$ is the scattering time of the undriven system. Thus, the $x$-directional longitudinal conductivity given in
Eq.~(\ref{eq_20}) can be expanded using only the central entry Fourier components ($l=l'=0$) of Floquet modes $\ket{\phi_{n,m}} \equiv \ket{n,k_x}$ as
\begin{widetext}
\begin{equation} \label{eq_22}
  \begin{aligned}
    {\sigma}^{xx} =
    \frac{-1}{4\pi\hbar A} &
    \int_{\Pi-\hbar\omega/2}^{\Pi+ \hbar\omega/2} d\varepsilon
    \qty(
    -\frac{\partial f}{\partial \varepsilon})
    \frac{1}{V_{k_x}} \sum_{k_x}
    \sum_{n}
    \mel{n,k_x}{
    {j}^x_0
    \qty(
    \vb{G}^{r} (\varepsilon) - \vb{G}^{a} (\varepsilon)
    )
    {j}^x_0
    \qty(
    \vb{G}^{r}_0 (\varepsilon) - \vb{G}^{a}_0 (\varepsilon)
    )
    }
    {n,k_x},
  \end{aligned}
\end{equation}
where $V_{k_x}$ is the volume of considering $x$-directional momentum space. Next, we evaluate the above expression as follows
\begin{equation} \label{eq_23}
  \begin{aligned}
    {\sigma}^{xx}  = &
    \frac{-1}{4\pi\hbar A}
    \int_{\Pi-\hbar\omega/2}^{\Pi+ \hbar\omega/2} d\varepsilon
    \qty(
    -\frac{\partial f}{\partial \varepsilon})
    \frac{1}{V_{k_x}^4} \sum_{k_x} \sum_{n}
    \sum_{{k_x}_1,{k_x}_2,{k_x}_3}
    \sum_{n_1,n_2,n_3}\\
    & \times
    \mel{n,k_x}{
    {j}^x_0}
    {n_1,{k_x}_1}
    \mel{{n_1,{k_x}_1}}{
    \qty(
    \vb{G}^{r} (\varepsilon) - \vb{G}^{a} (\varepsilon)
    )}
    {n_2,{k_x}_2}
    \mel{n_2,{k_x}_2}{
    {j}^x_0}
    {n_3,{k_x}_3}
    \mel{n_3,{k_x}_3}{
    \qty(
    \vb{G}^{r} (\varepsilon) - \vb{G}^{a} (\varepsilon)
    )
    }
    {n,k_x}.
  \end{aligned}
\end{equation}
We can diagonalize the impurity averaged Green's functions using a unitary transformation ($\vb{T}  \equiv \ket{n,k_x}$) as mentioned in Refs.~\cite{wackerl20,wackerlthesis20,tsuji08}. Thus, we evaluate the matrix elements of the difference between retarded and advanced Green's functions as
\begin{equation} \label{eq_24}
  \mel{{n_1,{k_x}_1}}{
  \vb{T}^{\dagger}
  \qty(
  \vb{G}^{r} (\varepsilon) - \vb{G}^{a} (\varepsilon)
  )\vb{T}}
  {n_2,{k_x}_2} =
  \qty[
  \frac{2i \text{Im}\qty(\vb{T}^{\dagger} \vb{\Sigma}^r \vb{T})
  \delta_{n_1,n_2}\delta_{{k_x}_1,{k_x}_2}}
  {
  \qty(
  \frac{1}{\hbar}\varepsilon -
  \frac{1}{\hbar}\varepsilon_{n_1}
  )^2
  + \qty[\text{Im}\qty(\vb{T}^{\dagger} \vb{\Sigma}^r \vb{T})]^2
  }],
\end{equation}
and
\begin{equation} \label{eq_25}
  \mel{{n_3,{k_x}_3}}{
  \vb{T}^{\dagger}
  \qty(
  \vb{G}^{r} (\varepsilon) - \vb{G}^{a} (\varepsilon)
  )\vb{T}}
  {n,{k_x}} =
  \qty[
  \frac{2i \text{Im}\qty(\vb{T}^{\dagger} \vb{\Sigma}^r \vb{T})
  \delta_{n_3,n}\delta_{{k_x}_3,{k_x}}}
  {
  \qty(
  \frac{1}{\hbar}\varepsilon -
  \frac{1}{\hbar}\varepsilon_{n}
  )^2
  + \qty[\text{Im}\qty(\vb{T}^{\dagger} \vb{\Sigma}^r \vb{T})]^2
  }].
\end{equation}
Here we introduced the retarded self-energy matrix $\vb{\Sigma}^r$ which is the sum of all irreducible diagrams \cite{wackerl20,wackerlthesis20}. Applying the matrix elements of the electric current operator in Landau levels and
expressions from Eq.~(\ref{eq_24}) and Eq.~(\ref{eq_25}) back into Eq.~(\ref{eq_23}) we obtain
\begin{equation} \label{eq_26}
  \begin{aligned}
    {\sigma}^{xx}  =
    \frac{-1}{4\pi\hbar A} &
    \int_{\Pi-\hbar\omega/2}^{\Pi+ \hbar\omega/2} d\varepsilon
    \qty(
    -\frac{\partial f}{\partial \varepsilon})
    \frac{1}{V_{k_x}} \sum_{k_x} \sum_{n}
    \sum_{n_1,n_2}
    \\
    & \times
    \frac{e^2B}{{m_e}}
    \qty(\sqrt{\frac{n+1}{2}} \delta_{n_1,n+1} + \sqrt{\frac{n}{2}}
    \delta_{n_1,n-1})
    \qty[
    \frac{2i \text{Im}\qty(\vb{T}^{\dagger} \vb{\Sigma}^r \vb{T})
    \delta_{n_1,n_2}}
    {
    \qty(
    \frac{1}{\hbar}\varepsilon -
    \frac{1}{\hbar}\varepsilon_{n_1}
    )^2
    + \qty[\text{Im}\qty(\vb{T}^{\dagger} \vb{\Sigma}^r \vb{T})]^2
    }] \\
    & \times
    \frac{e^2B}{{m_e}}
    \qty(\sqrt{\frac{n_2+1}{2}} \delta_{n,n_2+1} + \sqrt{\frac{n_2}{2}}
    \delta_{n,n_2-1})
    \qty[
    \frac{2i \text{Im}\qty(\vb{T}^{\dagger} \vb{\Sigma}^r \vb{T})
    }
    {
    \qty(
    \frac{1}{\hbar}\varepsilon -
    \frac{1}{\hbar}\varepsilon_{n}
    )^2
    + \qty[\text{Im}\qty(\vb{T}^{\dagger} \vb{\Sigma}^r \vb{T})]^2
    }],
  \end{aligned}
\end{equation}
For the full derivation of electric current operators in quantum Hall system refer to Appendix \ref{appendix_d}.
After the expansion, the only non-zero term would be
\begin{equation} \label{eq_27}
  \begin{aligned}
    {\sigma}^{xx} =
    \frac{-1}{4\pi\hbar A}
    \frac{e^4B^2}{{{m_e}^2}} &
    \int_{\Pi-\hbar\omega/2}^{\Pi+ \hbar\omega/2} d\varepsilon
    \qty(
    -\frac{\partial f}{\partial \varepsilon})
    \frac{1}{V_{k_x}} \sum_{k_x} \sum_{n}
    \qty(n+1)
    \\
    & \times
    \qty[
    \frac{2i \text{Im}\qty(\vb{T}^{\dagger} \vb{\Sigma}^r \vb{T})_{\varepsilon_{n+1}}
    }
    {
    \qty(
    \frac{1}{\hbar}\varepsilon -
    \frac{1}{\hbar}\varepsilon_{n+1}
    )^2
    + \qty[\text{Im}\qty(\vb{T}^{\dagger} \vb{\Sigma}^r \vb{T})_{\varepsilon_{n+1}}]^2
    }]
    \qty[
    \frac{2i \text{Im}\qty(\vb{T}^{\dagger} \vb{\Sigma}^r \vb{T})_{\varepsilon_{n}}
    }
    {
    \qty(
    \frac{1}{\hbar}\varepsilon -
    \frac{1}{\hbar}\varepsilon_{n}
    )^2
    + \qty[\text{Im}\qty(\vb{T}^{\dagger} \vb{\Sigma}^r \vb{T})_{\varepsilon_{n}}]^2
    }].
  \end{aligned}
\end{equation}
The inverse scattering time matrix is equal to the diagonalized contrast of the retarded and advanced self-energy \cite{wackerl20,wackerlthesis20}. In addition, on the diagonal the contrast of the retarded and advanced Green's function can be represented with the imaginary component of the retarded self-energy \cite{wackerl20,wackerlthesis20}. Subsequenty, we can identify the following property
\begin{equation} \label{eq_28}
  \qty(\frac{1}{\tau(\varepsilon,k_x)})^{ll} =
  -2\text{Im}\qty[ \vb{T}^{\dagger} \vb{\Sigma}^r(\varepsilon,k_x) \vb{T}]^{ll}.
\end{equation}
Aterwards, considering only the central element ($l=0$) of the inverse scattering time matrix, we can restructure the derived conductivity expression in \ref{eq_27} as follows
\begin{equation} \label{eq_29}
    {\sigma}^{xx}   =
    \frac{1}{\pi\hbar A}
    \frac{e^4B^2}{{{m_e}^2}}
    \int_{\Pi-\hbar\omega/2}^{\Pi+ \hbar\omega/2} d\varepsilon
    \qty(
    -\frac{\partial f}{\partial \varepsilon})
    \frac{1}{V_{k_x}} \sum_{k_x} \sum_{n}
    \qty(n+1)
    \qty[
    \frac{\tilde{{\Gamma}}(\varepsilon_{n+1})
    }
    {
    \qty(
    \varepsilon_F - \varepsilon_{n+1}
    )^2
    + \tilde{{\Gamma}}^2(\varepsilon_{n+1})
    }]
    \qty[
    \frac{\tilde{{\Gamma}}(\varepsilon_{n})
    }
    {
    \qty(
    \varepsilon_F - \varepsilon_{n}
    )^2
    + \tilde{{\Gamma}}^2(\varepsilon_{n})
    }],
\end{equation}
\end{widetext}
with $\tilde{{\Gamma}}(\varepsilon_n,k_x) \equiv \qty({\hbar}/{2\tau(\varepsilon_n,k_x)})^{00}$. Since we already identified that the inverse scattering time matrix's central element is independent of $k_x$ value, we can drop the $k_x$-dependent in the $\tilde{{\Gamma}}(\varepsilon_n,k_x)$ terms and get the sum over all available momentum space in $x$ direction.
However, by considering the condition that the center of the cyclotron orbit $y_0$ must physically lie within the considered system, we can identify that
\begin{equation} \label{eq_30}
 -\frac{m_e\omega_0 Ly}{2\hbar} \leq k_x \leq \frac{m_e\omega_0 Ly}{2\hbar}.
\end{equation}
We use the Fermi-Dirac distribution as our partial distribution function ($f$) for our system
\begin{equation} \label{eq_31}
  f(\varepsilon) = \frac{1}{\qty[\exp(\varepsilon - \varepsilon_F)/k_B T]+1},
\end{equation}
where $k_B$ is theBoltzmann constant, $T$ is the absolute temperature and $\varepsilon_F$ is the Fermi energy of the system. Considering the above distribution for extremely low temperature conditions, we can use the following approximation
\begin{equation} \label{eq_32}
  - \pdv{f(\varepsilon)}{\varepsilon} \approx \delta(\varepsilon - \varepsilon_F).
\end{equation}
Moreover, let $\Pi = \varepsilon_F$ and the derived expression in \ref{eq_29} leads to
\begin{equation} \label{eq_33}
  \begin{aligned}
    {\sigma}^{xx}  =
    \frac{e^2}{\hbar}
    \frac{1}{\pi A} &
    \sum_{n}
    \frac{\qty(n+1)}{\gamma_{n}\gamma_{n+1}} \\
    &\times
    \qty[
      \frac{1}
      {
        1 + \qty(\frac{X_F - n -1}{\gamma_{n+1}})^2
      }
    ]
    \qty[
      \frac{1}
      {
        1 + \qty(\frac{X_F - n}{\gamma_{n}})^2
      }
    ],
  \end{aligned}
\end{equation}
where $X_F \equiv ({\varepsilon_F}/{\hbar \omega_0} - {1}/{2})$
and
$\gamma_n \equiv {\tilde{{\Gamma}}(\varepsilon_n)}/{\hbar \omega_0}$.
Following the same steps as above derivation, we can derive the longitudinal conductivity in $y$-direction by applying the electric current operator for $y$-direction derived in Appendix \ref{appendix_d}
\begin{equation} \label{eq_34}
  \begin{aligned}
    {\sigma}^{yy} =
    \frac{e^2}{\hbar}
    \frac{1}{\pi A}&
    \frac{1}{e^2B^2}
    \sum_{n}
    \frac{\qty(n+1)}{\gamma_{n}\gamma_{n+1}} \\
    & \times
    \qty[
      \frac{1}
      {
        1 + \qty(\frac{X_F - n -1}{\gamma_{n+1}})^2
      }
    ]
    \qty[
      \frac{1}
      {
        1 + \qty(\frac{X_F - n}{\gamma_{n}})^2
      }
    ].
  \end{aligned}
\end{equation}
