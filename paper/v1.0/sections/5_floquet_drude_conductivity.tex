% Section 05 - Floquet-Drude Conductivity in Quantum Hall Systems

A general theory for the conductivity in dressed systems with the disorder averaging was introduced by M. Wackerl in Ref.~\cite{wackerl20,wackerlthesis20}. Within this theory, the general $x$-directional longitudinal DC limit conductivity can be express with
\begin{equation} \label{eq_20}
  \begin{aligned}
    {\sigma}^{xx} &=
    \frac{-1}{4\pi\hbar A}
    \int_{\lambda-\hbar\omega/2}^{\lambda+ \hbar\omega/2} d\varepsilon \bigg[
    \qty(
    -\frac{\partial f}{\partial \varepsilon})
    \\
    & \times
    \tr
    \qty[
    {j}^x_0
    \qty(
    \vb{G}^{r} (\varepsilon) - \vb{G}^{a} (\varepsilon)
    )
    {j}^x_0
    \qty(
    \vb{G}^{r} (\varepsilon) - \vb{G}^{a} (\varepsilon)
    )
    ]\bigg],
  \end{aligned}
\end{equation}
where ${j}^x_0$ and $\vb{G}^{r,a} (\varepsilon)$ are $x$-directional electric current operator matrix elements' $s=0$ the Fourier component (see Appendix D) and white noise disorder averaged Floquet Green function matrix \cite{wackerl20,wackerlthesis20} respectively defined against the Floquet modes of the considering system. Here we have assumed that only $s=0$ the Fourier component of the current operator is contributing to the conductivity. In addition, $A$ is the area of the considering two-dimentional system, partial distribution function is given by $f$ and $\lambda$ is a function that can be chosen such that
\begin{equation} \label{eq_21}
    \lambda - \frac{\hbar \omega}{2}
    \leq \varepsilon_N
    <
    \lambda + \frac{\hbar \omega}{2},
\end{equation}
where $ \varepsilon_N$ are quasienergies of all considering Floquet states.

Then Eq.~(\ref{eq_20}) can be expanded in off resonant regime ($\omega\tau_0 \gg 1$), where $\tau_0$ is the scattering time of the undriven system, using only central entry Fourier components ($l=l'=0$) of Floquet modes $\ket{\phi_{n,m}} \equiv \ket{n,k_x}$
\begin{widetext}
\begin{equation} \label{eq_22}
  \begin{aligned}
    {\sigma}^{xx} =
    \frac{-1}{4\pi\hbar A} &
    \int_{\lambda-\hbar\omega/2}^{\lambda+ \hbar\omega/2} d\varepsilon
    \qty(
    -\frac{\partial f}{\partial \varepsilon})
    \frac{1}{V_{k_x}} \sum_{k_x}
    \sum_{n}
    \mel{n,k_x}{
    {j}^x_0
    \qty(
    \vb{G}^{r} (\varepsilon) - \vb{G}^{a} (\varepsilon)
    )
    {j}^x_0
    \qty(
    \vb{G}^{r}_0 (\varepsilon) - \vb{G}^{a}_0 (\varepsilon)
    )
    }
    {n,k_x},
  \end{aligned}
\end{equation}
where $V_{k_x}$ is the volume of considering $x$-directional momentum space. Then this can evaluate by considering each matrix elements separately
\begin{equation} \label{eq_23}
  \begin{aligned}
    {\sigma}^{xx}  = &
    \frac{-1}{4\pi\hbar A}
    \int_{\lambda-\hbar\omega/2}^{\lambda+ \hbar\omega/2} d\varepsilon
    \qty(
    -\frac{\partial f}{\partial \varepsilon})
    \frac{1}{V_{k_x}^4} \sum_{k_x} \sum_{n}
    \sum_{{k_x}_1,{k_x}_2,{k_x}_3}
    \sum_{n_1,n_2,n_3}\\
    & \times
    \mel{n,k_x}{
    {j}^x_0}
    {n_1,{k_x}_1}
    \mel{{n_1,{k_x}_1}}{
    \qty(
    \vb{G}^{r} (\varepsilon) - \vb{G}^{a} (\varepsilon)
    )}
    {n_2,{k_x}_2}
    \mel{n_2,{k_x}_2}{
    {j}^x_0}
    {n_3,{k_x}_3}
    \mel{n_3,{k_x}_3}{
    \qty(
    \vb{G}^{r} (\varepsilon) - \vb{G}^{a} (\varepsilon)
    )
    }
    {n,k_x}.
  \end{aligned}
\end{equation}
Since we can diagonalize the impurity averaged Green's function using unitary transformation ($\vb{T}  \equiv \ket{n,k_x}$) mentioned in Ref.~\cite{wackerl20,wackerlthesis20,tsuji08}, and we evaluate the matrix element of the difference between retarded and advanced Green's functions
\begin{equation} \label{eq_24}
  \mel{{n_1,{k_x}_1}}{
  \vb{T}^{\dagger}
  \qty(
  \vb{G}^{r} (\varepsilon) - \vb{G}^{a} (\varepsilon)
  )\vb{T}}
  {n_2,{k_x}_2} =
  \qty[
  \frac{2i \text{Im}\qty(\vb{T}^{\dagger} \vb{\Sigma}^r \vb{T})
  \delta_{n_1,n_2}\delta_{{k_x}_1,{k_x}_2}}
  {
  \qty(
  \frac{1}{\hbar}\varepsilon -
  \frac{1}{\hbar}\varepsilon_{n_1}
  )^2
  + \qty[\text{Im}\qty(\vb{T}^{\dagger} \vb{\Sigma}^r \vb{T})]^2
  }],
\end{equation}
and
\begin{equation} \label{eq_25}
  \mel{{n_3,{k_x}_3}}{
  \vb{T}^{\dagger}
  \qty(
  \vb{G}^{r} (\varepsilon) - \vb{G}^{a} (\varepsilon)
  )\vb{T}}
  {n,{k_x}} =
  \qty[
  \frac{2i \text{Im}\qty(\vb{T}^{\dagger} \vb{\Sigma}^r \vb{T})
  \delta_{n_3,n}\delta_{{k_x}_3,{k_x}}}
  {
  \qty(
  \frac{1}{\hbar}\varepsilon -
  \frac{1}{\hbar}\varepsilon_{n}
  )^2
  + \qty[\text{Im}\qty(\vb{T}^{\dagger} \vb{\Sigma}^r \vb{T})]^2
  }],
\end{equation}
using the retarded self-energy matrix $\vb{\Sigma}^r$ which is the sum of all irreducible diagrams \cite{wackerl20,wackerlthesis20}. Using the matrix elements of electric current operator in Landau levels (see Appendix D) and
results form Eq.~(\ref{eq_24}) and Eq.~(\ref{eq_25}) back into Eq.~(\ref{eq_23}) we obtain
\begin{equation} \label{eq_26}
  \begin{aligned}
    {\sigma}^{xx}  =
    \frac{-1}{4\pi\hbar A} &
    \int_{\lambda-\hbar\omega/2}^{\lambda+ \hbar\omega/2} d\varepsilon
    \qty(
    -\frac{\partial f}{\partial \varepsilon})
    \frac{1}{V_{k_x}} \sum_{k_x} \sum_{n}
    \sum_{n_1,n_2}
    \\
    & \times
    \frac{e^2B}{{m_e}}
    \qty(\sqrt{\frac{n+1}{2}} \delta_{n_1,n+1} + \sqrt{\frac{n}{2}}
    \delta_{n_1,n-1})
    \qty[
    \frac{2i \text{Im}\qty(\vb{T}^{\dagger} \vb{\Sigma}^r \vb{T})
    \delta_{n_1,n_2}}
    {
    \qty(
    \frac{1}{\hbar}\varepsilon -
    \frac{1}{\hbar}\varepsilon_{n_1}
    )^2
    + \qty[\text{Im}\qty(\vb{T}^{\dagger} \vb{\Sigma}^r \vb{T})]^2
    }] \\
    & \times
    \frac{e^2B}{{m_e}}
    \qty(\sqrt{\frac{n_2+1}{2}} \delta_{n,n_2+1} + \sqrt{\frac{n_2}{2}}
    \delta_{n,n_2-1})
    \qty[
    \frac{2i \text{Im}\qty(\vb{T}^{\dagger} \vb{\Sigma}^r \vb{T})
    }
    {
    \qty(
    \frac{1}{\hbar}\varepsilon -
    \frac{1}{\hbar}\varepsilon_{n}
    )^2
    + \qty[\text{Im}\qty(\vb{T}^{\dagger} \vb{\Sigma}^r \vb{T})]^2
    }],
  \end{aligned}
\end{equation}
and the only non-zero term would be
\begin{equation} \label{eq_27}
  \begin{aligned}
    \frac{-1}{4\pi\hbar A}
    {\sigma}^{xx} & =
    \frac{e^4B^2}{{{m_e}^2}}
    \int_{\lambda-\hbar\omega/2}^{\lambda+ \hbar\omega/2} d\varepsilon
    \qty(
    -\frac{\partial f}{\partial \varepsilon})
    \frac{1}{V_{k_x}} \sum_{k_x} \sum_{n}
    \qty(n+1)
    \\
    & \times
    \qty[
    \frac{2i \text{Im}\qty(\vb{T}^{\dagger} \vb{\Sigma}^r \vb{T})_{\varepsilon_{n+1}}
    }
    {
    \qty(
    \frac{1}{\hbar}\varepsilon -
    \frac{1}{\hbar}\varepsilon_{n+1}
    )^2
    + \qty[\text{Im}\qty(\vb{T}^{\dagger} \vb{\Sigma}^r \vb{T})_{\varepsilon_{n+1}}]^2
    }]
    \qty[
    \frac{2i \text{Im}\qty(\vb{T}^{\dagger} \vb{\Sigma}^r \vb{T})_{\varepsilon_{n}}
    }
    {
    \qty(
    \frac{1}{\hbar}\varepsilon -
    \frac{1}{\hbar}\varepsilon_{n}
    )^2
    + \qty[\text{Im}\qty(\vb{T}^{\dagger} \vb{\Sigma}^r \vb{T})_{\varepsilon_{n}}]^2
    }].
  \end{aligned}
\end{equation}
Since the inverse scattering time matrix being equal to the diagonalized contrast of the retarded and advanced self-energy and on the diagonal the contrast of the retarded and advanced Green's function can be represented with the imaginary component of the retarded self-energy \cite{wackerl20,wackerlthesis20} we can identify the following property
\begin{equation} \label{eq_28}
  \qty(\frac{1}{\tau(\varepsilon,k_x)})^{ll} =
  -2\text{Im}\qty[ \vb{T}^{\dagger} \vb{\Sigma}^r(\varepsilon,k_x) \vb{T}]^{ll},
\end{equation}
and using central element ($l=0$) of the inverse scattering time matrix we can modify the derived conductivity expression
\begin{equation} \label{eq_29}
    {\sigma}^{xx}   =
    \frac{1}{\pi\hbar A}
    \frac{e^4B^2}{{{m_e}^2}}
    \int_{\lambda-\hbar\omega/2}^{\lambda+ \hbar\omega/2} d\varepsilon
    \qty(
    -\frac{\partial f}{\partial \varepsilon})
    \frac{1}{V_{k_x}} \sum_{k_x} \sum_{n}
    \qty(n+1)
    \qty[
    \frac{\Gamma(\varepsilon_{n+1})
    }
    {
    \qty(
    \varepsilon_F - \varepsilon_{n+1}
    )^2
    + \Gamma^2(\varepsilon_{n+1})
    }]
    \qty[
    \frac{\Gamma(\varepsilon_{n})
    }
    {
    \qty(
    \varepsilon_F - \varepsilon_{n}
    )^2
    + \Gamma^2(\varepsilon_{n})
    }],
\end{equation}
\end{widetext}
with $\Gamma(\varepsilon_n,k_x) \equiv \qty({\hbar}/{2\tau(\varepsilon_n,k_x)})^{00}$. We already identified that the inverse scattering time matrix's central element is not $k_x$ dependent we can get the sum over all available momentum space in $x$ direction. However, by considering the condition that the center of the force of the oscillator $y_0$ must physically lie within the considering system $-L_y/2 < y_0 < L_y/2$, we can identify
\begin{equation} \label{eq_30}
 -\frac{m_e\omega_0 Ly}{2\hbar} \leq k_x \leq \frac{m_e\omega_0 Ly}{2\hbar}.
\end{equation}
Then using Fermi-Dirac distribution as our partial distribution function ($f$) for this system
\begin{equation} \label{eq_31}
  f(\varepsilon) = \frac{1}{\qty[\exp(\varepsilon - \varepsilon_F)/k_B T]+1},
\end{equation}
where $k_B$ is Boltzmann constant, $T$ is absolute temperature and $\varepsilon_F$ is Fermi energy of the system. Considering the above distribution with extremely low temperature conditions we can approximate
\begin{equation} \label{eq_32}
  - \pdv{f(\varepsilon)}{\varepsilon} \approx \delta(\varepsilon - \varepsilon_F),
\end{equation}
and by letting $\lambda = \varepsilon_F$, the expression for conductivity leads to
\begin{equation} \label{eq_33}
  \begin{aligned}
    {\sigma}^{xx}  =
    \frac{e^2}{\hbar}
    \frac{1}{\pi A} &
    \sum_{n}
    \frac{\qty(n+1)}{\gamma_{n}\gamma_{n+1}} \\
    &\times
    \qty[
      \frac{1}
      {
        1 + \qty(\frac{X_F - n -1}{\gamma_{n+1}})^2
      }
    ]
    \qty[
      \frac{1}
      {
        1 + \qty(\frac{X_F - n}{\gamma_{n}})^2
      }
    ],
  \end{aligned}
\end{equation}
where $X_F \equiv ({\varepsilon_F}/{\hbar \omega_0} - {1}/{2})$
and
$\gamma_n \equiv {\Gamma(\varepsilon_n)}/{\hbar \omega_0}$.
Same as above derivation we can derive the transverse conductivity in $y$-direction by using the current operator derived in Appendix D
\begin{equation} \label{eq_34}
  \begin{aligned}
    {\sigma}^{yy} =
    \frac{e^2}{\hbar}
    \frac{1}{\pi A}&
    \frac{1}{e^2B^2}
    \sum_{n}
    \frac{\qty(n+1)}{\gamma_{n}\gamma_{n+1}} \\
    & \times
    \qty[
      \frac{1}
      {
        1 + \qty(\frac{X_F - n -1}{\gamma_{n+1}})^2
      }
    ]
    \qty[
      \frac{1}
      {
        1 + \qty(\frac{X_F - n}{\gamma_{n}})^2
      }
    ].
  \end{aligned}
\end{equation}
