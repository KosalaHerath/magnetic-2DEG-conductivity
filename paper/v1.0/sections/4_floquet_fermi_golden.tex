In Ref. \cite{wackerl20} Floquet Fermi golden rule has been introduced as a method to analyse transport propeties in dressed quantum systems. However, this theory has not been applied for a dressed quantum Hall system and to identify magnetotransport properties in our system we use Floquet Fermi golden rule.
With the help of $t-t'$ formalism \cite{wackerl20,grifoni98,sambe75,peskin93,althorpe97} and using Floquet states derived in Eq.~(\ref{eq_12}) we can derive an  expression for the inverse scattering time matrix ($(l,l')$th element) for the $N$th Landau level, per a given energy $\varepsilon$ and momentum $k_x$ value for our considered quantum Hall system as
\begin{widetext}
\begin{equation} \label{eq_13}
  \begin{aligned}
    \qty(\frac{1}{\tau(\varepsilon,k_x)})^{ll'}_{N} =
    \frac { \eta_{imp}^2 L_x^2 \hbar V_{imp}}{\qty(eB)^2}
    \delta(\varepsilon - \varepsilon_{N})
    \int_{-\infty}^{\infty} d k_1
    &
    J_l\qty(\frac{g\hbar}{eB}[{k}_x - k_1])
    J_{l'}\qty(\frac{g\hbar}{eB}[{k}_x - k_1]) \\
    & \times
    \qty|
    \int_{-\infty}^{\infty} dk_2 \;
    {\chi}_{N}\qty(\frac{\hbar}{eB}k_2)
    {\chi}_{N}\qty(\frac{\hbar}{eB} \qty[k_1 - {k}_x - k_2])|^2,
  \end{aligned}
\end{equation}
\end{widetext}
where $J_l(\cdot)$ are Bessel functions of the first kind with $l$th integer order and $\varepsilon_N$ is the energy of $N$th Landau level; see Appendix C. In this study, the perturbation potential is assumed to be formed by an ensemble of randomly distributed impurities, since random impurities in a disorded metal is a better approximation for experimental results. The total  scattering potential in 2DEG is then given by the sum over uncorrelated single impurity potentials $\upsilon(\vb{r})$. Here $\eta_{imp}$ is the number of impurities in a unit area and $V_{imp} = \expval{|V_{{k'}_x,k_x}|^2}$ with $V_{{k'}_x,k_x} = \mel**{k'_x}{\upsilon(x) }{k_x}$ where $\braket{x}{k_x} = e^{-ik_x x}/\sqrt{L_x}$.















x
