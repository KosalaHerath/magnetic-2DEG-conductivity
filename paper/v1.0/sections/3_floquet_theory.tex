The general interpretations of physical systems are mostly derived using symmetry conditions in quantum theory. Famous Bloch analysis of electrons in quantum systems introduces a mathematical explanation of quantum systems occupying a discrete translational symmetry in configuration space. Floquet theory gives a mathematical formalism that can be used for translational symmetry in time rather than in space \cite{floquet83,grifoni98,holthaus15}. Examine the transport properties of systems exposed to strong radiation using the Floquet-Drude conductivity method introduced recently by M. Wackerl \cite{wackerl20}. In their analysis they have presented more accurate results than previously existed theoretical descriptions for the conductivity in presence of a strong dressing field. Therefore, we are hoping to apply the Floquet-Drude conductivity method to analyse our 2DEG system which is subjected to both a staionary magnetic field and a strong dressing field.

First we need to identify the \textit{quasienergies} and periodic \textit{Floquet modes} for derived wavefunctions in Eq.~(\ref{eq_2}). By factorizing the wavefunction into a linearly time dependent part and a peridic time dependend part, the quasienergies can be present as
\begin{equation} \label{eq_6}
  \varepsilon_{n} =
  \hbar \omega_0\qty(n + \frac{1}{2}) - \Delta_{\varepsilon},
\end{equation}
which is only depend on single quantum number ($n$) and Floquet modes are given by
\begin{equation} \label{eq_7}
  \begin{aligned}
    \phi_{n,m}(x,y,t) = &
    \frac{1}{\sqrt{L_x}} \chi_{n}\left[y - y_0 - \zeta(t)\right]
    \text{exp}\bigg(
     \frac{i}{\hbar}\bigg[
     p_x x \\
     & +
     \frac{eE(y - y_0)}{\omega}\cos(\omega t) \\
     & +
     m_e\dot{\zeta}(t)\big[y - y_0 -\zeta(t)\big]
     + \xi \bigg]\bigg),
  \end{aligned}
\end{equation}
with
\begin{equation} \label{eq_8}
  \Delta_{\varepsilon} = \frac{(eE)^2}{4m_e(\omega_0^2 - \omega^2)} ~ \text{and} ~
  \xi = \frac{(eE)^2\qty(3\omega^2 - \omega_0^2)}{8m_e\omega(\omega_0^2 - \omega^2)^2} \sin(2\omega t).
\end{equation}
It is important to notice that these Floquet modes are time-periodic ($T=2\pi/\omega$) functions.
Now using Floquet theory, the wave functions derived in Eq.~(\ref{eq_2}) can be written in position space as
\begin{equation} \label{eq_9}
  \psi_{n,m}(x,y,t) =
  \exp(-\frac{i}{\hbar}\varepsilon_{n}t)   \phi_{n,m} (x,y,t).
\end{equation}
Then performing Fourier trasform over the confined two-dimentional space $A=L_xL_y$, we obtain the momentum space($k_x,k_y$) representation of Floquet modes
\begin{equation} \label{eq_10}
  \begin{aligned}
    \phi_{n,m}(k_x,k_y,t)  = &
    {\sqrt{L_x}}
    \tilde{\chi}_{n}\qty(k_y - b\cos(\omega t)) \\
    & \times
    \text{exp}\left(
      i\xi
      -ik_y  \qty[d\sin(\omega t) + y_0]
    \right),
  \end{aligned}
\end{equation}
where
\begin{equation} \label{eq_11}
  \tilde{\chi}_{n}\qty(k) =
  \frac{i^n}{\sqrt{2^{n} n! \sqrt{\pi}}}
  \qty(\frac{1}{\kappa})^{1/2}
  e^{-\frac{k^2}{2 \kappa^2}}
  \mathcal{H}_{n} \qty(\frac{k}{\kappa})
\end{equation}
and
\begin{equation} \label{eq_12}
  b \equiv
  \frac{eE\omega_0^2}{\hbar\omega(\omega_0^2 - \omega^2)} \quad
  d \equiv
 \frac{eE}{m_e(\omega_0^2 - \omega^2)}.
\end{equation}
For details of the formalism the authors refer to Appendix B.
