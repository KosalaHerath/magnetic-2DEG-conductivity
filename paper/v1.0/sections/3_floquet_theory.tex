% Section 03 - Floquet theory perspective

The general interpretations of physical systems are mostly derived using symmetry conditions in quantum theory. Famous Bloch analysis of electrons in quantum systems introduces a mathematical explanation of quantum systems occupying a discrete translational symmetry in configuration space. Floquet theory gives a mathematical formalism that can be used for translational symmetry in time rather than in space \cite{floquet83,grifoni98,holthaus15}. Examine the transport properties of quantum systems exposed to strong radiation using the Floquet-Drude conductivity method introduced recently by Wackerl \textit{et al.} \cite{wackerl20}. In their analysis they have presented more accurate results than previously existed theoretical descriptions for the conductivity of nanoscale systems in presence of a dressing field. Therefore, we are hoping to apply the Floquet-Drude conductivity method to analyze our 2DEG system which is subjected to both a stationary magnetic field and a dressing field.

First we require to recognize the \textit{quasienergies} and time periodic \textit{Floquet modes} \cite{grifoni98} for the previously derived wave functions in Eq.~(\ref{eq_2}). By factorizing the wave function into a linearly time dependent part and a periodic time dependent part, the quasienergies can be present as
\begin{equation} \label{eq_6}
  \varepsilon_{n} =
  \hbar \omega_0\qty(n + \frac{1}{2}) - \Delta_{\varepsilon},
\end{equation}
which is only depended on a single quantum number ($n$) and Floquet modes are given by
\begin{equation} \label{eq_7}
  \begin{aligned}
    \phi_{n,m}(x,y,t) = &
    \frac{1}{\sqrt{L_x}} \chi_{n}\big(y - y_0 - \zeta(t)\big)\\
    & \times
    \text{exp}\bigg(
     \frac{i}{\hbar}\bigg[
     p_x x +
     \frac{eE[y - y_0]}{\omega}\cos(\omega t) \\
     & +
     m_e\dot{\zeta}(t)\big[y - y_0 -\zeta(t)\big]
     + \xi \bigg]\bigg),
  \end{aligned}
\end{equation}
with
\begin{equation} \label{eq_8}
  \Delta_{\varepsilon} = \frac{(eE)^2}{4m_e(\omega_0^2 - \omega^2)} ~ \text{and} ~
  \xi = \frac{(eE)^2\qty(3\omega^2 - \omega_0^2)}{8m_e\omega(\omega_0^2 - \omega^2)^2} \sin(2\omega t).
\end{equation}
It is important to notice that these Floquet modes are time-periodic ($T=2\pi/\omega$) functions. At the resonance $\omega = \omega_0$, the energy levels occupies a continuous form and the discrete quasienergies are no longer valid \cite{popov70}. Therefore, in this work we choose the dressing field frequency with the condition $\omega \neq \omega_0$.

Then performing Fourier transform over the confined two-dimensional space, we obtain the momentum space($k_x,k_y$) representation of Floquet modes
\begin{equation} \label{eq_9}
  \begin{aligned}
    \phi_{n,m}\big(k_x,k_y,t\big)  = &
    {\sqrt{L_x}}
    \tilde{\chi}_{n}\qty(k_y - b\cos(\omega t)) \\
    & \times
    \text{exp}\left(
      i\xi
      -ik_y  \qty[d\sin(\omega t) + y_0]
    \right),
  \end{aligned}
\end{equation}
where
\begin{equation} \label{eq_10}
  \tilde{\chi}_{n}\qty(k) =
  \frac{i^n}{\sqrt{2^{n} n! \sqrt{\pi}}}
  \qty(\frac{1}{\kappa})^{1/2}
  e^{-\frac{k^2}{2 \kappa^2}}
  \mathcal{H}_{n} \qty(\frac{k}{\kappa}).
\end{equation}
Here we used new parameters
\begin{equation} \label{eq_11}
  b \equiv
  \frac{eE\omega_0^2}{\hbar\omega(\omega_0^2 - \omega^2)} \quad \text{and} \quad
  d \equiv
 \frac{eE}{m_e(\omega_0^2 - \omega^2)}.
\end{equation}
For details of the full derivation please refer to Appendix \ref{appendix_b}. Now using Floquet theory, the wave functions derived in Eq.~(\ref{eq_2}) can be written in momentum space as Floquet states
\begin{equation} \label{eq_12}
  \psi_{n,m}(k_x,k_y,t) =
  \exp(-\frac{i}{\hbar}\varepsilon_{n}t)   \phi_{n,m} (k_x,k_y,t).
\end{equation}
