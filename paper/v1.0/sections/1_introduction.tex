Interactions between light and matter have dragged research attention in the fields of optoelectronics, sensing, energy harvesting, quantum computing, bio-information, and in many branches of recent technologies. For many years, the foremost aims for examing the characteristics of dressed fermion systems were focused on the different types of atomic and molecular arrangements. These researches of extreme electron-light engagements introduced an astonishing scope of twentieth-century physics namely quantum optic physics.

On the other hand, in nanostructures that are applicable in electronic devices, the investigations with the help of quantum optic were centered on polaritonic and exciton influences on nanostructures and material characteristics of dressed electrons in two-dimensional(2D) materials and quantum wires. When considering the transport characteristics of dressed nanostructures, they are still expecting extensive analysis.

Therefore, transport properties of nanostructures exposed to a high intensity periodic electromagnetic fields have been explored theoretically in this study. The dressing field is analyzed non perturbatively using the Floquet theory whilst the probing field is examined perturbatively by applying the linear response method using the Kubo formula. The general Floquet-Drude conductivity has been derived in a fully closed analytical form in most recent research [1,2], introducing a novel type of Green’s functions namely four-times Green’s functions.
As a consequence, the established formalism introduces a novel approach to manipulate the transport characteristics of nanostructures by an intense dressing field. From an empirical sense, this study applies directly to various nanostructures illuminated by a high-intensity electromagnetic field.
In this research we have developed a robust mathematical model for dressed two-dimensional electron gas(2DEG) exposed to another stationary magnetic field and that will enable efficient manipulation of transport characteristics in nanoscale electronic devices.

When a stationary magentic field applied perpendicularly across the surface of 2DEG systems, the orbital motion of electrons becomes completely quantized and the energy spectrium becomes discrete by creating Landau levels. Such a singular system known as a quantum Hall system and in this study we explictly calculate the diagonal ($\sigma_{xx},\sigma_{yy}$) components of the conductivity tensor in the periodically driven quantum Hall systems.

Although there are already exist a number of adavanced theories devoted to the calculation of conductivity tensor elements in a quantum Hall systems [3-5], they have not been applied to the optically manipulation the magneto-electric properties of the quantum Hall systems. However K. Dini et al. [6] have recently investigated the one directional conductivity behaviour of dressed quantum Hall systems, they have not used the state of art model to describe the conductivity in a quantum Hall system. In their study they used the conductivity models from T. Ando et al. [3,4] and as mentioned in A. Endo et al. investigation [5] those models are far less accurate representation of the experimantally observed Landau levels because they present a semi-elliptical broadening.

In this study we develop a genralized mathematical model to describe transport properties of dressed quantum Hall systems using Floquet-Drude conductivity [1,2]. In addition, we demonstrate that our generalized model is agreed with the state of art conductivity model [5] for specalized quantum Hall system which has been considered without the external dressing field.  Therefore this theory describes that the dressing field can be used as a tool to utilize transport properties in various 2D nanostructures which serve as a basis for nano-optoelectonic devices.
