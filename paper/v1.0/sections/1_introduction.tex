% Section 01 - Introduction

Manipulating light-matter interactions in the quantum regime paved the path for an astonishing number of useful technologies in the last century. Quantum optics, which study these interactions, have drawn research attention to the disciplines of optoelectronics \cite{liu16,wijesekara20,tao21}, sensing \cite{rodrigo2015,pirandola18,hapuarachchi2018}, energy harvesting \cite{yuan16,sun18},
quantum computing \cite{huh15,slussarenko19,andersen21}, bio-information \cite{marais18,bian20}, and many other specialities of recent technologies \cite{rivera20}.
The studies on quantum optics of nanostructures were generally centered on metamaterials \cite{shalaev07,si14}, quantum plasmonic effects \cite{hapuarachchi19,perera20}, lasers and amplifiers \cite{zhang05,chow13}, and quantum cavity physics \cite{tsang10,devi20}.
However, in recent years, one of the foremost aim of examining nanostructures under external radiation was understanding their electron transport characteristics \cite{kitagawa11,zhou11,kibis14,pervishko15,morina15,dehghani15,dini16,wackerl20}.

Better understanding the fundamental mechanisms of charge transport can allow us to invent novel nanoelectronic devices and optimize their performance \cite{premaratne21}.
Most recent studies on the subject have considered the driving field as a perturbation field \cite{pervishko15,morina15}. However, this assumption breaks down for systems under high-intensity illuminations \cite{grifoni98,wackerl20}.
Modeling an electromagnetic field under a perturbative formalism involves expanding the interaction terms in powers of the field intensity. At high intensities, the higher order terms influence the physics more strongly and the basis of the perturbative treatment begins to break down.
In these instances, a more accurate treatment needs to adopt. Thus, we treat the interacting fermion system and the radiation as one combined quantum system, namely dressed system \cite{morina15,cohen98,scully01}. Here the applied high-intensity  electromagnetic field identify as the dressing field.

Theoretical analyses on the transport properties of dressed fermion systems were recently reported in Refs. \cite{kibis14,morina15,wackerl20}.
Furthermore, in Ref. \cite{wackerl20} a general expression for conductivity in a dressed system has been derived in a fully closed analytical form. In their study, a novel type of Green’s functions, namely four-times Green’s functions were used to derive the Floquet-Drude conductivity formula. This opened the path to explore and exploit the charge transport attributes of nanostructures under an intense dressing field.

Quantum Hall effect \cite{girvin90} observed in two-dimensional fermion systems at low temperatures under strong stationary magnetic fields manifest remarkable magneto-transport behaviors. Transport properties of these systems have recently attracted both theoretical \cite{ando72,ando74_1,ando74_2,ando74_3,ando74_4,ando82,endo09} and experimental \cite{allerman95,tieke97,pan05} interest.
Endo \textit{et al.} \cite{endo09} presented the calculations of longitudinal and transverse conductivity tensor components and their relationship in a quantum Hall system. These theoretical calculations  align better with experimental observations compared to previous studies.

In contrast, more interesting phenomena can be observed by simultaneously applying a dressing field to a quantum Hall system already under a non-oscillating magnetic field.
Whilst there exist several leading theories for calculating conductivity tensor elements in quantum Hall systems \cite{ando74_1,ando82,endo09}, they have not been utilized to describe the optical manipulation of charge transport.
Recently, Dini \textit{et al.} \cite{dini16} have investigated the one-directional conductivity behavior of dressed quantum Hall systems. However, they have not adopted the state-of-the-art model to describe the conductivity in a quantum Hall system. In their study, they used the conductivity models from Refs. \cite{ando74_1,ando82}, and as mentioned in Endo \textit{et al.} \cite{endo09}, those models predict a semi-elliptical broadening agaist Fermi level for each Landau levels and provide less agreement with the empirical results.

In the present analysis, we present a robust mathematical model for a dressed two-dimensional electron gas (2DEG) subject to another nonoscillating magnetic field.
A stationary magnetic field is applied perpendicularly across the surface of the 2DEG system. This causes the orbital motion of the electrons to be quantized, and a discrete energy spectrum with Landau splitting is observed \cite{landau30}.
In this study, we explicitly calculate the longitudinal components ($\sigma^{xx},\sigma^{yy}$) of the conductivity tensor in a periodically driven quantum Hall system by developing a generalized analytical description using the Floquet-Drude conductivity \cite{wackerl20}.
Finally, we demonstrate that our generalized model reproduces the results of the state-of-the-art conductivity model in Ref. \cite{endo09}, which was developed for the more specific case of quantum Hall systems without the external dressing field.
Moreover, we find that the optical field can be used as a mechanism to regulate transport behavior in numerous two-dimensional nanostructures which can serve as a basis for many useful nanoelectronic devices. We believe that our theoretical analysis and visual depictions of numerical results will lead to a better understanding of manipulating charge transport. Moreover, this will inspire advanced  developments in nanoscale quantum devices.

The paper is organized as follows. In Sec.  \ref{sec_schrodinger_problem}, we introduce our dressed quantum Hall system and the exact wave function solutions for the given configuration. Sec. \ref{sec_floquet_theory}, provides the Floquet theory interpretation of these wave functions.
We introduce Floquet Fermi golden rule for a quantum Hall system in Sec. \ref{sec_inverse_scattering_time}, and use it in Sec. \ref{sec_floquet_drude_conductivity} to derive analytical expressions for longitudinal components of conductivity.
The derived theoritical model is further analyzed numerically using empirical system parameters and compared with previous studies in Sec. \ref{sec_manipulate_conductivity}.
In Sec. \ref{sec_conclusions}, we summarize our analysis and present our conlusions.
