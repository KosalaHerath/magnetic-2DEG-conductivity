% Section 01 - Introduction

Manipulating physical systems as a consequence of light engagements introduced an astonishing scope of twentieth-century physics namely quantum optics \cite{grynberg10}.
Quantum optics have dragged research attention in the disciplines of optoelectronics \cite{liu16,wijesekara20,tao21}, sensing \cite{rodrigo2015,pirandola18,hapuarachchi2018}, energy harvesting \cite{yuan16,sun18},
quantum computing \cite{huh15,slussarenko19,andersen21}, bio-information \cite{marais18,bian20}, and abundant affiliates of recent technologies \cite{rivera20}.
The studies on quantum optics of nanostructures were centered on metamaterials \cite{shalaev07,si14}, quantum plasmonic effects \cite{hapuarachchi19,perera20}, lasers and amplifiers \cite{zhang05,chow13}, quantum cavity physics \cite{tsang10,devi20}.
However, over the past few years, the foremost aims for examining nanostructures under external radiation were focused on understanding the transport characteristics \cite{kitagawa11,zhou11,kibis14,pervishko15,morina15,dehghani15,dini16,wackerl20}. 
The significance of describing fundamental mechanisms taking place in charge transport is that they will pave the way to invent new nanoelectronic devices and optimize their performance \cite{premaratne21}.
Among these works, most studies have considered the driving field as a perturbation field \cite{pervishko15,morina15} that is not applicable for systems under high-intensity illumination \cite{grifoni98,wackerl20}. When the external electromagnetic field account with a perturbation method and it gets high intensive, the number of higher-order interaction terms need to be considered gets increase. Then a more accurate and powerful concept need to be adopted and we need to treat the interacting fermion system and the radiation as a one unique quantum system (dressed system) \cite{morina15,cohen98,scully01}. On account of the transport characteristics of dressed nanostructures, they are still expecting extensive analysis.
Theoretical analyses of the transport properties of dressed fermions systems were recently reported in Ref. \cite{kibis14,morina15,wackerl20} where a general expression for Floquet-Drude conductivity has been derived in a fully closed analytical form in Ref. \cite{wackerl20}, introducing a novel type of Green’s functions namely four-times Green’s functions. As a consequence, the established formalism introduces a novel approach to manipulate the transport characteristics of nanostructures by an intense dressing field.

On the other hand, quantum Hall effect \cite{girvin90} observed for two-dimentional fermions systems at low temperatures under strong stationary magnetic field manifest remarkable magneto-transport behaviors and transport properties of these systems attracted many researches attention in experimentally \cite{allerman95,tieke97,pan05} and theoretically \cite{ando72,ando74_1,ando74_2,ando74_3,ando74_4,ando82,endo09}.
Akira Endo et al. \cite{endo09} previously presented calculations of longitudinal and transverse conductivity tensor components and their relationship in a quantum Hall system and these theoretical calculations are well-suited with the experimental observations while other former studies [46] yields a semi-elliptical broadening for Landau levels and gives less fitting interpretations for the empirical results \cite{endo09}.

In contrast, more fascinating transport phenomena can be achieved by creating a quantum Hall system using a nonoscillating magnetic field and simultaneously applying a dressing field.
Whilst there exist a number of leading theories devoted to the calculation of conductivity tensor elements in quantum Hall systems \cite{ando74_1,ando82,endo09}, they have not been utilized to describe the optical manipulation of the transport properties of the quantum Hall systems.
Furthermore, K. Dini et al. \cite{dini16} have recently investigated the one-directional conductivity behavior of dressed quantum Hall systems, and they have not adopted the state-of-the-art model to describe the conductivity in a quantum Hall system. In their study, they used the conductivity models from T. Ando et al. \cite{ando74_1,ando82} and as mentioned in A. Endo et al. work \cite{endo09} those models yields a semi-elliptical broadening for Landau levels and gives less fitting illustrations for the experimental results.

In the present analysis we report a robust mathematical model for dressed two-dimensional electron gas(2DEG) unveiled to another nonoscillating magnetic field.
A stationary magnetic field is applied perpendicularly across the surface of 2DEG system and the orbital motion of electrons becomes completely quantized, and the energy spectrum becomes discrete by creating Landau levels \cite{landau30}.
Such a singular system is known as a two-dimentional quantum Hall system and in this study, we explicitly calculate the longitudinal  ($\sigma^{xx},\sigma^{yy}$) components of the conductivity tensor in the periodically driven quantum Hall systems by developing a generalized mathematical model to describe transport properties of dressed quantum Hall systems using Floquet-Drude conductivity \cite{wackerl20}.
Finally, we demonstrate that our generalized model is agreed with state-of-the-art conductivity model \cite{endo09} already developed for specialized quantum Hall systems that have been considered without the external dressing field. 
Furthermore, the mechanism of optical manipulation can be used as a mechanism to utilize transport behavior in numerous two-dimentional nanostructures which serve as a basis for nanoelectronic devices and, we believe that our theoretical analysis and visual depictions of results can be used to understand the basic structure and optimize the performance of optical manipulation of transport properties in a two-dimentional quantum Hall system.

The paper is organized as follows. In Sec.  \ref{sec_schrodinger_problem}, we introduce our dressed quantum Hall system and the exact wave function solutions for the given configuration. Sec. \ref{sec_floquet_theory}, provides the Floquet theory interpretation of these wave functions.
We introduce Floquet Fermi golden rule for quantum Hall system in Sec. \ref{sec_inverse_scattering_time}, and we use it in Sec. \ref{sec_floquet_drude_conductivity} to derive analytical expressions for longitudinal components of conductivity.
The derived theoritical model is further analyzed numerically using empherical system parameters and compared with previous studies in Sec. \ref{sec_manipulate_conductivity}, followed by concluding remarks in Sec. \ref{sec_conclusions}.
