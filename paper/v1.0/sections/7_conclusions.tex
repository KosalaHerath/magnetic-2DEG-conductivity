In this analysis, we introduced a generalized mathematical model for charge transport properties in a 2DEG under a nonoscillating magnetic field and a high intensity light. Under the uniform magnetic field, the charged particles can only settle in discrete energy values and this leads to the Landau quantization. We modeled the behavior of electrons in Landau levels under the dressing field utilizing the Floquet-Drude conductivity method by considering impurities in the material as Gaussian random scattering potential. Finally, we derived expressions for x-directional and y-directional longitudinal components of electric conductivity tensor for the considered system.

Our derived analytical expressions disclosed that the transport characteristics of the dressed quantum Hall system can be controlled by the applied dressing field’s intensity. Using detailed numerical calculations with empirical system parameters, we further analyzed the manipulation of conductivity components using the dressing field. We found that the graphical illustrations we gained are capable of produce the same behavior as the experiment conductivity found in quantum Hall systems without a dressing field. Furthermore, we identified that by regulating the intensity of the radiation, the conductivity regions near the Landau levels can be squeezed. Despite, this behavior has been identified in previous works, their results are not coinciding with the more accurate description of conductivity components in undressed quantum Hall systems. However, our generalized analysis of conductivity in dressed quantum Hall systems provide a well-suited description for these special quantum Hall systems.

In summary, the primary purpose of this study was to broaden the modern descriptions of transport properties in dressed quantum Hall systems. Moreover, our detailed theoretical analysis showed that the recently introduced Floquet-Drude conductivity model can be adopted to extend the models that were used to describe the transport characteristics in quantum Hall systems. Due to owing the ability to control the conductivity regions, high intensity external illumination can be used as a trigger for two-dimensional quantum switching devices which are employed as the building blocks of next generation nanoelectronic devices. Finally, we identified that our findings of this paper can be used towards understanding nanoscale quantum devices, enhancing their performance, and inventing novel appliances.
