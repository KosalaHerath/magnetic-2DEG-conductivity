% Section 04 - Inverse Scattering Time Analysis

In Ref. \cite{wackerl20} Floquet Fermi golden rule has been introduced as a method to analyze transport properties in dressed disordered quantum systems. However, this theory has not been applied for a dressed quantum Hall system and to identify magneto-transport properties in our system we use Floquet Fermi golden rule.
With the help of $t-t'$ formalism \cite{wackerl20,grifoni98,sambe75,peskin93,althorpe97} and using Floquet states derived in Eq.~(\ref{eq_12}) we can derive an  expression for the inverse scattering time matrix ($(l,l')$th element) for the $N$-th Landau level, per a given energy $\varepsilon$ and momentum $k_x$ value for our considered quantum Hall system as
\begin{equation} \label{eq_13}
  \begin{aligned}
    \bigg(&\frac{1}{\tau(\varepsilon,k_x)}\bigg)^{ll'}_{N} \\
    & =
    \frac { \varrho^2}{eB}
    \delta(\varepsilon - \varepsilon_{N}) \\
    & \times
    \int_{-\infty}^{\infty} d k_1 \Bigg[
    J_l\qty(\frac{b\hbar}{eB}[{k}_x - k_1])
    J_{l'}\qty(\frac{b\hbar}{eB}[{k}_x - k_1]) \\
    & \times
    \qty|
    \int_{-\infty}^{\infty} dk_2 \;
    {\chi}_{N}\qty(\frac{\hbar}{eB}k_2)
    {\chi}_{N}\qty(\frac{\hbar}{eB} \qty[k_1 - {k}_x - k_2])|^2\Bigg],
  \end{aligned}
\end{equation}
where $\varrho \equiv \eta_{imp} L_x [ { V_{imp}}/{eB}]^{1/2}$,
$J_l(\cdot)$ are Bessel functions of the first kind with $l$-th integer order and $\varepsilon_N$ is the energy of $N$-th Landau level; see Appendix C. In this derivation, we assumed that the perturbation potential is generated by a group of randomly distributed impurities, since random impurities in a disordered metal is a better approximation for experimental results. Thus, the total scattering potential in 2DEG has been presented as the sum of uncorrelated single impurity potentials $\upsilon(\vb{r})$. Here $\eta_{imp}$ is the number of impurities in a unit area and $V_{imp} \equiv \expval{|V_{{k'}_x,k_x}|^2}$ with $V_{{k'}_x,k_x} = \mel**{k'_x}{\upsilon(x) }{k_x}$ where $\braket{x}{k_x} = e^{-ik_x x}/\sqrt{L_x}$.

Next we are going to analyze the contribution of the inverse scattering time matrix elements to the transport properties in 2DEG.
Since the disorder in the system can not be alter the eigenenergy values of the undressed system \cite{wackerl20}, we can neglect the all off-diagonal elements of the self-energy, and then we can consider only the central diagonal element (${l=l'=0}$) of the inverse scattering time matrix which has the largest contribution to the transport characteristics.
% \begin{equation} \label{eq_14}
%   \begin{aligned}
%     \bigg(&\frac{1}{\tau(\varepsilon,k_x)}\bigg)^{00}_{N} \\
%     & =
%     \frac { \varrho^2}{eB}
%     \delta(\varepsilon - \varepsilon_{N}) \\
%     & \times
%     \int_{-\infty}^{\infty} d k_1 \;
%     J_0^2\qty(\frac{b\hbar}{eB}[{k}_x -  k_1])
%     \\
%     & \times
%     \qty|
%     \int_{-\infty}^{\infty} d k_2 \;
%     {\chi}_{N}\qty(\frac{\hbar}{eB}k_2)
%     {\chi}_{N}\qty(\frac{\hbar}{eB} \qty[k_1 - {k}_x - k_2])|^2.
%   \end{aligned}
% \end{equation}
Introduce a new parameter as scattering-induced broadening of the $N$-th Landau level as \cite{dini16,endo09}
\begin{equation} \label{eq_14}
 \Gamma^{00}_{N}(\varepsilon,k_x) = \hbar \qty(\frac{1}{\tau(\varepsilon,k_x)})^{00}_{N},
\end{equation}
and this leads to
\begin{equation} \label{eq_15}
 \begin{aligned}
   \Gamma^{00}_{N}& (\varepsilon,k_x) \\
   & =
   \frac { \varrho^2}{eB}
   \delta(\varepsilon - \varepsilon_{N}) \\
   & \times
   \int_{-\infty}^{\infty} d k_1 \Bigg[
   J_0^2\qty(\frac{b\hbar}{eB}[{k}_x - k_1])
   \\
   & \times
   \qty|
   \int_{-\infty}^{\infty} dk_2 \;
   {\chi}_{N}\qty(\frac{\hbar}{eB}k_2)
   {\chi}_{N}\qty(\frac{\hbar}{eB} \qty[k_1 - {k}_x - k_2])|^2\Bigg].
 \end{aligned}
\end{equation}

\begin{figure}[t]
\includegraphics[scale=0.68]{figures/fig_3}
\caption{\label{fig_3} The dependence of normalized scattering-induced broadening $\Lambda_N$ for each Landau level ($N =0,1,2,3,4$) against $x$-directional momentum value $k_x$ in a GaAs-based quantum well under a nonoscillating magnetic field with $B = 1.2~\text{T}$, dressing field with frequency of $\omega =2\times10^{12}~\text{rad}\text{s}^{-1}$ and intensity $I =100~\text{W}/\text{cm}^{2}$.
In this calculation we have assumed that the natural  broadening of $0$-th Landau level $\Gamma_0$ is $0.24\;\text{me}V$.}
\end{figure}

\begin{figure}[b]
\includegraphics[scale=0.68]{figures/fig_4}
\caption{\label{fig_4} The dependence of normalized scattering-induced broadening $\Lambda_N$ for each Landau level ($N =0,1,2,3,4$) against dressing field intensity $I$, in a GaAs-based quantum well under a nonoscillating magnetic field with $B = 1.2~\text{T}$, dressing field with frequency of $\omega =2\times10^{12}~\text{rad}\text{s}^{-1}$. In this calculation we have assumed that the natural broadening of $0$-th Landau level $\Gamma_0$ is $0.24\;\text{me}V$.}
\end{figure}

In addition, for a scenario of scattering take place inside the same Landau level, we are able to present the delta distribution of the energy using the following interpretation \cite{dini16}
\begin{equation} \label{eq_16}
 \delta(\varepsilon - \varepsilon_{N}) \approx
 \frac{1}{\pi \Gamma^{00}_{N}(\varepsilon,k_x)},
\end{equation}
and we write the central element of inverse scattering time matrix in more compact form
\begin{equation} \label{eq_17}
  \begin{aligned}
    \Gamma^{00}_{N}(\varepsilon,& k_x) =
    \varrho
    \bigg[
    \int_{-\infty}^{\infty} d {k}_1 \;
    J_0^2\qty(\lambda_1[{k}_x - {k}_1]) \\
    & \times
    \qty|
    \int_{-\infty}^{\infty} d{k}_2 \;
    \tilde{\chi}_{N}\qty(\lambda_2 k_2)
    \tilde{\chi}_{N}\qty(\lambda_2 \qty[{k}_1 - {k}_2 - {k}_x])|^2
    \bigg]^{-\frac{1}{2}},
  \end{aligned}
\end{equation}
where $ \lambda_1 \equiv \hbar b/eB$ and  $\lambda_2 \equiv \hbar \kappa/eB$.
To analyze the contribution of dressing field on the scattering-induced broadening, normalized $N$-th Landau level scattering-induced broadening can be defined as
\begin{equation} \label{eq_18}
    \Lambda_N(k_x) \equiv
    \frac{\Gamma^{00}_{N}(\varepsilon,k_x)}{\Gamma^{00}_{N=0}(\varepsilon,k_x)\big|_{E=0}},
\end{equation}
and this leads to
\begin{widetext}
\begin{equation} \label{eq_19}
    \Lambda_N (k_x) =
    \qty[
    \frac
    {\int_{-\infty}^{\infty} d {k}_1 \;
    J_0^2\qty(\lambda_1[{k}_x - {k}_1])
    \qty|
    \int_{-\infty}^{\infty} d{k}_2 \;
    \tilde{\chi}_{N}\qty(\lambda_2 k_2)
    \tilde{\chi}_{N}\qty(\lambda_2 \qty[{k}_1 - {k}_2 - {k}_x])|^2}
    {\int_{-\infty}^{\infty} d {k}_1 \;
    \qty|
    \int_{-\infty}^{\infty} d{k}_2 \;
    \tilde{\chi}_{0}\qty(\lambda_2 k_2)
    \tilde{\chi}_{0}\qty(\lambda_2 \qty[{k}_1 - {k}_2 - {k}_x])|^2}
    ]^{1/2}.
\end{equation}
\end{widetext}

Normalized energy band broadening against ${k_x}$ for different Landau levels ($N = 0,1,2,3,4$) has been calculated for GaAs-based quantum well and results are given in Fig.~(\ref{fig_3}) and Fig.~(\ref{fig_4}). To make comparison we have selected the experiment parameters to match with analysis done in Ref.~\cite{endo09}.
In their study, they have assumed that effective mass of the electron in GaAs-based quantum well system is $m_e \approx 0.07\tilde{m}_e$ where $\tilde{m}_e$ is mass of the electron \cite{endo09,winkler03,wackerl20} and the broadening of the natural(without a dressing field) $0$-th Landau level $\Gamma_0$ is $0.24\;\text{me}V$. Therefore, in our calculations, we assumed that the natural least Landau level broadening also has this value: $\Gamma^{00}_{N=0}|_{E=0} = 0.24 \;\text{meV}$.
Here we can identify that the each Landau level normalized energy broadening value is independent of the x-directional momentum $k_x$ value and we are able to change it using applied electromagnetic field. When the applied field's intensity increase the energy broadening gets reduced which make changes in transport properties and in the next section, we are going to derive a analytical expression for the conductivity in dressed quantum Hall systems.
