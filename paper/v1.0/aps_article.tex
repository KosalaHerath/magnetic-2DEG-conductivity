% ****** Start of file apssamp.tex ******
%
%   This file is part of the APS files in the REVTeX 4.2 distribution.
%   Version 4.2a of REVTeX, December 2014
%
%   Copyright (c) 2014 The American Physical Society.
%
%   See the REVTeX 4 README file for restrictions and more information.
%
% TeX'ing this file requires that you have AMS-LaTeX 2.0 installed
% as well as the rest of the prerequisites for REVTeX 4.2
%
% See the REVTeX 4 README file
% It also requires running BibTeX. The commands are as follows:
%
%  1)  latex apssamp.tex
%  2)  bibtex apssamp
%  3)  latex apssamp.tex
%  4)  latex apssamp.tex
%
\documentclass[%
 reprint,
%superscriptaddress,
%groupedaddress,
%unsortedaddress,
%runinaddress,
%frontmatterverbose,
%preprint,
%preprintnumbers,
%nofootinbib,
%nobibnotes,
%bibnotes,
 amsmath,amssymb,
 aps,
%pra,
prb,
%rmp,
%prstab,
%prstper,
%floatfix,
]{revtex4-2}

\usepackage{graphicx}% Include figure files
\usepackage{dcolumn}% Align table columns on decimal point
\usepackage{bm}% bold math
%\usepackage{hyperref}% add hypertext capabilities
%\usepackage[mathlines]{lineno}% Enable numbering of text and display math
%\linenumbers\relax % Commence numbering lines

%\usepackage[showframe,%Uncomment any one of the following lines to test
%%scale=0.7, marginratio={1:1, 2:3}, ignoreall,% default settings
%%text={7in,10in},centering,
%%margin=1.5in,
%%total={6.5in,8.75in}, top=1.2in, left=0.9in, includefoot,
%%height=10in,a5paper,hmargin={3cm,0.8in},
%]{geometry}

\begin{document}

\preprint{APS/123-QED}

\title{Floquet-Drude Conductivity in Dressed Quantum Hall Systems}% Force line breaks with \\
% \thanks{A footnote to the article title}%

\author{Kosala Herath,}
 % \altaffiliation[Also at ]{Physics Department, XYZ University.}%Lines break automatically or can be forced with \\
\author{Malin Premaratne}%
\affiliation{%
 Advanced Computing and Simulation Laboratory(A$\chi$L), Department of Electrical and Computer Systems Engineering,\\
 Monash University, Clayton, Victoria 3800, Australia
}%

% \collaboration{MUSO Collaboration}%\noaffiliation
%
% \author{Charlie Author}
%  \homepage{http://www.Second.institution.edu/~Charlie.Author}
% \affiliation{
%  Second institution and/or address\\
%  This line break forced% with \\
% }%
% \affiliation{
%  Third institution, the second for Charlie Author
% }%
% \author{Delta Author}
% \affiliation{%
%  Authors' institution and/or address\\
%  This line break forced with \textbackslash\textbackslash
% }%
%
% \collaboration{CLEO Collaboration}%\noaffiliation

\date{\today}% It is always \today, today,
             %  but any date may be explicitly specified

\begin{abstract}
A generalized mathematical model for the transport properties of systems exposed to a stationary magnetic and a strong electromagnetic field is presented. The new formulation, which applies to the two-dimensional dressed quantum Hall systems, is based on Landau quantization theory and Floquet-Drude conductivity method. We model our system as a two-dimensional electron gas (2DEG) that interacts with two external fields. To incorporate the strong light coupling with the 2DEG, we utilize the Floquet theory to analyze the effect non perturbatively. Moreover, the Floquet Fermi "golden rule" is adopted to explore the scattering effects for Floquet states in disordered quantum Hall systems. Based on our fully analytical expression and particular graphical representations, we demonstrate
that the characteristics of conductivities in two-dimensional quantum Hall systems can manipulate using a dressed field. The outcomes align with theoretical descriptions which are already well-suited with experimental results at the same time our theory provides a more generalized analysis on the properties of conductivity in quantum Hall systems. Thus, this model more realistically describes that how to use external strong radiation as a tool to utilize transport properties in various 2D nanostructures which serve as a basis for nano-optoelectronic devices.

% \begin{description}
% \item[Usage]
% Secondary publications and information retrieval purposes.
% \item[Structure]
% You may use the \texttt{description} environment to structure your abstract;
% use the optional argument of the \verb+\item+ command to give the category of each item.
% \end{description}
\end{abstract}

%\keywords{Suggested keywords}%Use showkeys class option if keyword
                              %display desired
\maketitle

%\tableofcontents

\section{\label{sec:introduction} Introduction}
Interactions between light and matter have dragged research attention in the fields of optoelectronics, sensing, energy harvesting, quantum computing, bio-information, and in many branches of recent technologies. For many years, the foremost aims for examing the characteristics of dressed fermion systems were focused on the different types of atomic and molecular arrangements. These researches of extreme electron-light engagements introduced an astonishing scope of twentieth-century physics namely quantum optic physics.

On the other hand, in nanostructures that are applicable in electronic devices, the investigations with the help of quantum optic were centered on polaritonic and exciton influences on nanostructures and material characteristics of dressed electrons in two-dimensional(2D) materials and quantum wires. When considering the transport characteristics of dressed nanostructures, they are still expecting extensive analysis.

Therefore, transport properties of nanostructures exposed to a high intensity periodic electromagnetic fields have been explored theoretically in this study. The dressing field is analyzed non perturbatively using the Floquet theory whilst the probing field is examined perturbatively by applying the linear response method using the Kubo formula. The general Floquet-Drude conductivity has been derived in a fully closed analytical form in most recent research [1,2], introducing a novel type of Green’s functions namely four-times Green’s functions.
As a consequence, the established formalism introduces a novel approach to manipulate the transport characteristics of nanostructures by an intense dressing field. From an empirical sense, this study applies directly to various nanostructures illuminated by a high-intensity electromagnetic field.
In this research we have developed a robust mathematical model for dressed two-dimensional electron gas(2DEG) exposed to another stationary magnetic field and that will enable efficient manipulation of transport characteristics in nanoscale electronic devices.

When a stationary magentic field applied perpendicularly across the surface of 2DEG systems, the orbital motion of electrons becomes completely quantized and the energy spectrium becomes discrete by creating Landau levels. Such a singular system known as a quantum Hall system and in this study we explictly calculate the diagonal ($\sigma_{xx},\sigma_{yy}$) components of the conductivity tensor in the periodically driven quantum Hall systems.

Although there are already exist a number of adavanced theories devoted to the calculation of conductivity tensor elements in a quantum Hall systems [3-5], they have not been applied to the optically manipulation the magneto-electric properties of the quantum Hall systems. However K. Dini et al. [6] have recently investigated the one directional conductivity behaviour of dressed quantum Hall systems, they have not used the state of art model to describe the conductivity in a quantum Hall system. In their study they used the conductivity models from T. Ando et al. [3,4] and as mentioned in A. Endo et al. investigation [5] those models are far less accurate representation of the experimantally observed Landau levels because they present a semi-elliptical broadening.

In this study we develop a genralized mathematical model to describe transport properties of dressed quantum Hall systems using Floquet-Drude conductivity [1,2]. In addition, we demonstrate that our generalized model is agreed with the state of art conductivity model [5] for specalized quantum Hall system which has been considered without the external dressing field.  Therefore this theory describes that the dressing field can be used as a tool to utilize transport properties in various 2D nanostructures which serve as a basis for nano-optoelectonic devices.


\section{Schrödinger problem for Landau levels in dressed 2DEG
}
% Section 02 - Schrodinger problem for landau levels in dressed 2DEG

Our system consist of a two-dimentional free electron gas (2DEG) confined in the $(x,y)$ plane of the three-dimentional coordinate space. In our analysis, the 2DEG is subjected to a stationary magnetic field $\vb{B} = (0,0,B)^{\text{T}}$ which is pointed towards the $z$ axis. In addition a linearly polorized strong light is applied perpendicular to the 2DEG plane and we specially tune the frequency of the field $\omega$ such that the optical field behaves as a purely dressing field (nonabsorbable). Without limiting the generality we can choose $y$-polorized electric field $\vb{E} = (0,E\sin(\omega t),0)^{\text{T}}$ for the dressing field configuration (Fig.~\ref{fig_1}).
\begin{figure}[b]
\includegraphics{figures/fig_1}
\caption{\label{fig_1} Two dimentional eletron gas (2DEG) confined in the $(x,y)$ plane while both stationary magnetic field $\vb{B}$ and strong dressing field with y-polorized electric field $\vb{E}$ are being applied perpendicular to the surface of 2DEG.}
\end{figure}
Here $B$ and $E$ represent the amplitude of the stationary magnetic field and oscillating electric field respectively.

Using Landau gauge for the stationary magnetic field, we can represent it using vector potential as $\vb{A}_{s} = (-By,0,0)^{\text{T}}$ and choosing Coulomb gauge, the vector potential of the dynamic dressing radiation can be presented as $\vb{A}_{d}(t) = (0,[E/\omega ]\cos(\omega t),0)^{\text{T}}$. These vector potentials are coupled to the momentum of 2DEG as kinetic momentum \cite{mahan81,bruus04} and this leads to the time-dependent Hamiltonian
\begin{equation} \label{eq_1}
  \hat{H}_e(t) = \frac{1}{2m_e}\Big[\hat{\vb{p}} - e\big(\vb{A}_{s}+\vb{A}_{d}(t)\big)\Big]^2,
\end{equation}
where $m_e$ is the effective electron mass and $e$ is the magnitude of the electron charge. $\hat{\vb{p}} = (\hat{p}_x,\hat{p}_y,0)^{\text{T}}$ represents the canonical momentum operator for 2DEG with electron momentums $p_{x,y}$.
The exact solutions for the time-dependent Schrödinger equation $i\hbar \dv{t}\psi = \hat{H}_e(t) \psi$ was already given by Refs. \cite{husmi53,ditt98,dini16} and we can present them as a set of wave functions defined by two quantum numbers $(n,m)$
\begin{equation} \label{eq_2}
  \begin{aligned}
    \psi_{n,m}&(x,y,t)  = \frac{1}{\sqrt{L_x}}
    \chi_n\left[y - y_0 - \zeta(t)\right]
    \text{exp}\bigg(
    \frac{i}{\hbar}\bigg[- \varepsilon_nt \\
    &
    + p_x x + \frac{eE(y - y_0)}{\omega}\cos(\omega t)+
    m_e\dot{\zeta}(t)\big[y - y_0 -\zeta(t)\big]\\
    & +
    \int_0^{t}dt'L(\zeta,\dot{\zeta},t')\bigg]\bigg),
  \end{aligned}
\end{equation}
where $n \in \mathbb{Z}^{+}_0$ and $m \in \mathbb{Z}$ ; see Appendix A. Here $L_{x,y}$ are dimention of the 2DEG surface, $\hbar$ is the reduced Planck constant, and $y_0 = -p_x/eB$ is the center of the cyclotron orbit along $y$ axis. $\chi_n$ are well known solutions(Gauss-Hermite functions) for Schrödinger equation of a stationary quantum harmonic oscillartor
\begin{equation} \label{eq_3}
  \chi_n(x) \equiv
   \frac{\sqrt{\kappa}}{\sqrt{2^{n}n!}}
  e^{-\kappa^2 x^2/2}
  \mathcal{H}_n \qty(\kappa x) \quad \text{with}
  \quad
  \kappa = \sqrt{\frac{m_e \omega_0}{\hbar}},
\end{equation}
with eigenvalues given by $\varepsilon_n = \hbar \omega_0 (n + 1/2)$ and $\omega_0 = eB/m_e$ is the cyclotron frequency. Each $n$ value defines the  energy($\varepsilon_n$) of the respective Landau level. The path shift of the driven classical oscillator $\zeta(t)$ is given by
\begin{equation} \label{eq_4}
  \zeta(t) = \frac{eE}{m_e(\omega_0^2 - \omega^2)}\sin(\omega t),
\end{equation}
while the Lagrangian of the classical oscialltor $L(\zeta,\dot{\zeta},t)$ can be idenfied as
\begin{equation} \label{eq_5}
  L(\zeta,\dot{\zeta},t) = \frac{1}{2} m_e\dot{\zeta}^2(t) - \frac{1}{2}m_e\omega_0^2 \zeta^2(t) + eE\zeta(t) \sin(\omega t).
\end{equation}
















x


\section{Floquet theory}
% Section 03 - Floquet theory perspective

Symmetry conditions often give useful insights into the behaviors of physical quantum systems.
For instance, the famous Bloch analysis of electrons in quantum systems introduces a mathematical explanation for quantum systems occupying a discrete translational symmetry in the configuration space. Similarly, Floquet theory gives a mathematical formalism that can be used for translational symmetry in time rather than in space \cite{floquet83,grifoni98,holthaus15}.
The Floquet-Drude conductivity theory was employed recently by Wackerl \textit{et al.} \cite{wackerl20} as a method to analyze the transport properties of quantum systems exposed to strong radiation.
In their work, they have presented more accurate results than the former theoretical descriptions for the conductivity of nanoscale systems in the  presence of a dressing field. Therefore, we apply the Floquet-Drude conductivity theory to analyze our 2DEG system which is subjected to both a stationary magnetic field and a dressing field.

First, we need to identify the \textit{quasienergies} and time periodic \textit{Floquet modes} \cite{grifoni98} for the wave functions given in Eq.~(\ref{eq:2}). By factorizing the wave function into a linearly time-dependent part and a periodic time-dependent part, we present the quasienergies with
\begin{equation} \label{eq:6}
  \varepsilon_{n} =
  \hbar \omega_0 \left(n + \flatfrac{1}{2}\right) - \Delta_{\varepsilon},
\end{equation}
which only depends on a single quantum number $n$. Furthermore, we can recognize the Floquet modes as
\begin{equation} \label{eq:7}
  \begin{aligned}
    \phi_{n,m}(x,y,t) =&
    \frac{1}{\sqrt{L_x}} \chi_{n}\bm{\left(}y - y_0 -\zeta(t)\bm{\right)} \\
    & \times
      \exp \bm{\left(}
      \frac{i}{\hbar}\left\{
      p_x x + \frac{eE}{\omega}(y - y_0)\cos(\omega t) \right.\right. \\
    & \left.\left. \qquad\qquad +
      m_e\dot{\zeta}(t)\left[y - y_0 -\zeta(t)\right] +
      \xi \right\}\bm{\right)},
  \end{aligned}
\end{equation}
with
\begin{equation} \label{eq:8}
  \Delta_{\varepsilon} = \frac{e^2E^2}{4m_e(\omega_0^2 - \omega^2)},
\end{equation}
and
\begin{equation} \label{eq:9}
  \xi = \frac{e^2E^2 \left(3\omega^2 - \omega_0^2 \right)}
  {8m_e\omega(\omega_0^2 - \omega^2)^2} \sin(2\omega t).
\end{equation}
For a detailed derivation, refer to Appendix \ref{appendix_b}.
It is important to note that these Floquet modes are time-periodic ($T=2\pi/\omega$) functions. At resonance $\omega = \omega_0$, the energy levels occupy a continuous spectrum and the quasienergy formalism is no longer valid \cite{popov70}. Therefore, in this work we choose a dressing field frequency obeying the condition $\omega \neq \omega_0$.

Performing the Fourier transform over the confined 2D space, we obtain the momentum space ($k_x,k_y$) representation of Floquet modes
\begin{equation} \label{eq:10}
  \begin{aligned}
    \phi_{n,m}\big(k_x,k_y,t\big)  =&
    \sqrt{L_x}
    \widetilde{\chi}_{n} \bm{\left(}k_y - b\cos(\omega t)\bm{\right) }\\
    & \times
    \exp \bm{\left(} i\xi -ik_y  \left[d\sin(\omega t) + y_0 \right] \bm{\right)},
  \end{aligned}
\end{equation}
where
\begin{equation} \label{eq:11}
  \widetilde{\chi}_{n}(k) =
  i^n \left(\frac{1}{ 2^{n} n! \sqrt{\pi} \kappa}\right)^{1/2}
  e^{-\flatfrac{k^2}{(2 \kappa^2)}}
  \mathcal{H}_{n} \left(\flatfrac{k}{\kappa}\right).
\end{equation}
Here we have introduced new parameters
\begin{equation} \label{eq:12}
  b =
  \frac{eE\omega_0^2}{\hbar\omega(\omega_0^2 - \omega^2)},
\end{equation}
and
\begin{equation} \label{eq:13}
  d =
 \frac{eE}{m_e(\omega_0^2 - \omega^2)}.
\end{equation}
Using Floquet theory, we can re-write the wave functions derived in Eq.~(\ref{eq:2}) as the \textit{Floquet states} in momentum space
\begin{equation} \label{eq:14}
  \psi_{n,m}(k_x,k_y,t) =
  \exp(-\flatfrac{i\varepsilon_{n}t}{\hbar}) \phi_{n,m} (k_x,k_y,t).
\end{equation}


\section{Floquet Fermi Golden Rule}
In this section we are going to derive the Floquet Fermi goldern rule for previously derived quantum Floquet states using $t-t'$ formalism \cite{bruus04}.


\section{Inverse Scattering Time Analysis}
In Ref. \cite{wackerl20} Floquet Fermi golden rule has been introduced as a method to analyse transport propeties in dressed quantum systems. However, this theory has not been applied for a dressed quantum Hall system and to identify magnetotransport properties in our system we use Floquet Fermi golden rule.
With the help of $t-t'$ formalism \cite{wackerl20,grifoni98,sambe75,peskin93,althorpe97} and using Floquet states derived in Eq.~(\ref{eq_12}) we can derive an  expression for the inverse scattering time matrix ($(l,l')$th element) for the $N$th Landau level, per a given energy $\varepsilon$ and momentum $k_x$ value for our considered quantum Hall system as
\begin{widetext}
\begin{equation} \label{eq_13}
  \begin{aligned}
    \qty(\frac{1}{\tau(\varepsilon,k_x)})^{ll'}_{N} =
    \frac { \eta_{imp}^2 L_x^2 \hbar V_{imp}}{\qty(eB)^2}
    \delta(\varepsilon - \varepsilon_{N})
    \int_{-\infty}^{\infty} d k_1
    &
    J_l\qty(\frac{b\hbar}{eB}[{k}_x - k_1])
    J_{l'}\qty(\frac{b\hbar}{eB}[{k}_x - k_1]) \\
    & \times
    \qty|
    \int_{-\infty}^{\infty} dk_2 \;
    {\chi}_{N}\qty(\frac{\hbar}{eB}k_2)
    {\chi}_{N}\qty(\frac{\hbar}{eB} \qty[k_1 - {k}_x - k_2])|^2,
  \end{aligned}
\end{equation}
\end{widetext}
where $J_l(\cdot)$ are Bessel functions of the first kind with $l$th integer order and $\varepsilon_N$ is the energy of $N$th Landau level; see Appendix C. In this study, the perturbation potential is assumed to be formed by an ensemble of randomly distributed impurities, since random impurities in a disorded metal is a better approximation for experimental results. The total  scattering potential in 2DEG is then given by the sum over uncorrelated single impurity potentials $\upsilon(\vb{r})$. Here $\eta_{imp}$ is the number of impurities in a unit area and $V_{imp} = \expval{|V_{{k'}_x,k_x}|^2}$ with $V_{{k'}_x,k_x} = \mel**{k'_x}{\upsilon(x) }{k_x}$ where $\braket{x}{k_x} = e^{-ik_x x}/\sqrt{L_x}$.

Next we are going to analyse the contribution of the inverse scattering time matrix elements to the transport properies in 2DEG. The disorder in the system is not supposed to change the eigenenergies of the bare system \cite{wackerl20}, hence all off-diogonal elements of the self-energy were neglected and then we can consider only the central diagonal element (${l=l'=0}$) of the inverse scattering time matrix which has the largest contribution to the transport propeties
\begin{widetext}
  \begin{equation} \label{eq_14}
    \begin{aligned}
      \qty(\frac{1}{\tau(\varepsilon,k_x)})^{00}_{N} =
      \frac { \eta_{imp}^2 L_x^2 \hbar V_{imp}}{\qty(eB)^2}
      \delta(\varepsilon - \varepsilon_{N}) &
      \int_{-\infty}^{\infty} d k_1 \;
      J_0^2\qty(\frac{b\hbar}{eB}[{k}_x -  k_1])
      \\
      & \times
      \qty|
      \int_{-\infty}^{\infty} d k_2 \;
      {\chi}_{N}\qty(\frac{\hbar}{eB}k_2)
      {\chi}_{N}\qty(\frac{\hbar}{eB} \qty[k_1 - {k}_x - k_2])|^2.
    \end{aligned}
  \end{equation}
Introduce a new parameter with physical meaning of scattering-induced broadening of the Landau level as \cite{dini16,endo09}
\begin{equation} \label{eq_15}
 \Gamma^{00}_{N}(\varepsilon,k_x) \equiv \hbar \qty(\frac{1}{\tau(\varepsilon,k_x)})^{00}_{N},
\end{equation}
and this leads to
\begin{equation} \label{eq_16}
 \begin{aligned}
   \Gamma^{00}_{N}(\varepsilon,k_x)  =
   \frac { \eta_{imp}^2 L_x^2 \hbar V_{imp}}{\qty(eB)^2}
   \delta(\varepsilon - \varepsilon_{N}) &
   \int_{-\infty}^{\infty} d {k'}_x \;
   J_0^2\qty(\frac{b\hbar}{eB}[{k}_x - {k'}_x])
   \\
   & \times
   \qty|
   \int_{-\infty}^{\infty} dk_2 \;
   {\chi}_{N}\qty(\frac{\hbar}{eB}k_2)
   {\chi}_{N}\qty(\frac{\hbar}{eB} \qty[{k'}_x - {k}_x - k_2])|^2.
 \end{aligned}
\end{equation}
\end{widetext}
In addition, for the case of scattering within a same Landau level, one can present the delta distribution of the energy using the same physical interpretation \cite{dini16} as follows
\begin{equation} \label{eq_17}
 \delta(\varepsilon - \varepsilon_{N}) \approx
 \frac{1}{\pi \Gamma^{00}_{N}(\varepsilon,k_x)},
\end{equation}
and we write the central element of inverse scattering time matrix in more compact form
\begin{equation} \label{eq_18}
  \begin{aligned}
    \Gamma^{00}_{N}(\varepsilon,k_x)  &=
    \varrho
    \bigg[
    \int_{-\infty}^{\infty} d {k}_1 \;
    J_0^2\qty(\lambda_1[{k}_x - {k}_1]) \\
    & \times
    \qty|
    \int_{-\infty}^{\infty} d{k}_2 \;
    \tilde{\chi}_{N}\qty(\lambda_2 k_2)
    \tilde{\chi}_{N}\qty(\lambda_2 \qty[{k}_1 - {k}_2 - {k}_x])|^2
    \bigg]^{-\frac{1}{2}},
  \end{aligned}
\end{equation}
where $\varrho \equiv \eta_{imp} L_x [ { V_{imp}}/{eB}]^{1/2}$, $ \lambda_1 \equiv \hbar b/eB$ and  $\lambda_2 \equiv \hbar \kappa/eB$.
To analyze the contribution of dressing field on the scatering-induced broadening, normalized $N$-th Landau level scatering-induced broadening can be defined as
\begin{equation} \label{eq_19}
    \Lambda_N(k_x) \equiv
    \frac{\Gamma^{00}_{N}(\varepsilon,k_x)}{\Gamma^{00}_{N=0}(\varepsilon,k_x)\big|_{E=0}},
\end{equation}
and this leads to
\begin{widetext}
\begin{equation} \label{eq_20}
    \Lambda_N (k_x) =
    \qty[
    \frac
    {\int_{-\infty}^{\infty} d {k}_1 \;
    J_0^2\qty(\lambda_1[{k}_x - {k}_1])
    \qty|
    \int_{-\infty}^{\infty} d{k}_2 \;
    \tilde{\chi}_{N}\qty(\lambda_2 k_2)
    \tilde{\chi}_{N}\qty(\lambda_2 \qty[{k}_1 - {k}_2 - {k}_x])|^2}
    {\int_{-\infty}^{\infty} d {k}_1 \;
    \qty|
    \int_{-\infty}^{\infty} d{k}_2 \;
    \tilde{\chi}_{0}\qty(\lambda_2 k_2)
    \tilde{\chi}_{0}\qty(\lambda_2 \qty[{k}_1 - {k}_2 - {k}_x])|^2}
    ]^{1/2}.
\end{equation}
\end{widetext}

Normalized energy band broadening against ${k_x}$ for different Landau levels ($N = 0,1,2,3,4$) has been calculated for GaAs-based quantum well and results are given in Fig.~(\ref{fig_3}) and Fig.~(\ref{fig_4}). To make comparison we have selected the experiment parameters to match with analysis done in Ref.~\cite{endo09}. In their study, they have assumed that the broadening of the natural(without a dressing field) $0$-th Landau level $\Gamma_0$ is $0.24\;\text{me}V$. Therefore in our calculations, we assumed that the natural least Landau level broadening also has this value: $\Gamma^{00}_{N=0}|_{E=0} = 0.24 \;\text{meV}$.
Here we can identify that we can change the each Landau level normalized energy broadening value using apllied electromagnetic field. When the applied field's intensity increase the energy broadening gets reduced which make changes in conductivity of 2DEG.
\begin{figure}[t]
\includegraphics[scale=0.55]{figures/fig_3}
\caption{\label{fig_3} The dependence of normalized scattering-induced broadening $\Lambda_N$ for each Landau level ($N =0,1,2,3,4$) against $x$-directional momentum value $k_x$ in a GaAs-based quantum well at a magnetic field $B = 1.2~\text{T}$, dressing field with frequency $\omega =2\times10^{12}~\text{rad}\text{s}^{-1}$ and intensity $I =100~\text{W}/\text{cm}^{2}$. In this calculation we have assumed that the natural  broadening of $0$-th Landau level $\Gamma_0$ is $0.24\;\text{me}V$.}
\end{figure}
\begin{figure}[t]
\includegraphics[scale=0.45]{figures/fig_4}
\caption{\label{fig_4} The dependence of normalized scattering-induced broadening $\Lambda_N$ for each Landau level ($N =0,1,2,3,4$) against dressing field intensity $I$, in a GaAs-based quantum well at a magnetic field $B = 1.2~\text{T}$, dressing field with frequency $\omega =2\times10^{12}~\text{rad}\text{s}^{-1}$. In this calculation we have assumed that the natural broadening of $0$-th Landau level $\Gamma_0$ is $0.24\;\text{me}V$.}
\end{figure}


\section{Current Operator in Landau Levels}
Considerable research effort in recent years has been de- voted to synthesizing materials whose thermal conductivity.


\section{Floquet-Drude Conductivity in Quantum Hall Systems}
A general theory for the conductivity in dressed systems with the disorder averaging was introduced by M. Wackerl in Ref.~\cite{wackerl20,wackerlthesis20}. Within this theory, the general $x$-directional longitudinal DC limit conductivity can be express with
\begin{equation} \label{eq_21}
  \begin{aligned}
    {\sigma}^{xx} &=
    \frac{-1}{4\pi\hbar A}
    \int_{\lambda-\hbar\omega/2}^{\lambda+ \hbar\omega/2} d\varepsilon \bigg[
    \qty(
    -\frac{\partial f}{\partial \varepsilon})
    \\
    & \times
    \tr
    \qty[
    {j}^x_0
    \qty(
    \vb{G}^{r} (\varepsilon) - \vb{G}^{a} (\varepsilon)
    )
    {j}^x_0
    \qty(
    \vb{G}^{r} (\varepsilon) - \vb{G}^{a} (\varepsilon)
    )
    ]\bigg].
  \end{aligned}
\end{equation}
where ${j}^x_0$ and $\vb{G}^{r,a} (\varepsilon)$ are $x$-directional electric current operator matrix elements' $s=0$ th Fourier component (see Appendix D) and white noise disorder averaged Green function matrix \cite{wackerl20,wackerlthesis20} respectvely defined against the Floquet modes of the considering system. Here we have assumed that only $s=0$ th Fourier component of the current operator is contributing to the conductivity. In addition $A$ is the area of the considering two-dimentionl system and $\lambda$ is a function that can be choose such that
\begin{equation} \label{eq_22}
    \lambda - \frac{\hbar \omega}{2}
    \leq \varepsilon_N
    <
    \lambda + \frac{\hbar \omega}{2}
\end{equation}
where $ \varepsilon_N$ are quasienergies of all considering Floquet states.

Then Eq.~(\ref{eq_21}) can be expanded in off resonant regime ($\omega\tau_0 \gg 1$), where $\tau_0$ is the scattering time of the undriven system, using only central entry Fourier components ($l=l'=0$) of Floquet modes $\ket{\phi_{n,m}} \equiv \ket{n,k_x}$
\begin{widetext}
\begin{equation} \label{7.2}
  \begin{aligned}
    \lim_{\omega \to 0}
    \text{Re}[{\sigma}^{xx}(0,\omega)] & =
    \frac{-1}{4\pi\hbar A}
    \int_{\lambda-\hbar\Omega/2}^{\lambda+ \hbar\Omega/2} d\varepsilon
    \qty(
    -\frac{\partial f}{\partial \varepsilon})
    \\
    & \times
    \frac{1}{V_{k_x}} \sum_{k_x}
    \sum_{n}
    \mel{n,k_x}{
    {j}^x_0
    \qty(
    \vb{G}^{r} (\varepsilon) - \vb{G}^{a} (\varepsilon)
    )
    {j}^x_0
    \qty(
    \vb{G}^{r}_0 (\varepsilon) - \vb{G}^{a}_0 (\varepsilon)
    )
    }
    {n,k_x}
  \end{aligned}
\end{equation}
where $V_{k_x}$ is the volume of considering $x$-directional momentum space. 
\end{widetext}























 x


\section{Manipulate Conductivity in Quantum Hall System}
To identify the behaviour of the transverse conductivity of the quantum Hall systems with external dressing field we can derive a expression for a normalized
transverse conductivity as a function of fermi energy $X_F$ and intensity of the dressing field $I$. Here we have normalized the conductivity using the natural conductivity of least Landau level.
\begin{equation} \label{eq_36}
  \begin{aligned}
    \sigma^{xx}(X_F,I) = &
    \sum_{n}
    \frac{\qty(n+1)}{0.0037\Lambda_n \Lambda_{n+1}} \\
    & \times
    \qty[
      \frac{1}
      {
        1 + \qty(\frac{X_F - n -1}{0.06\Lambda_n})^2
      }
    ]
    \qty[
      \frac{1}
      {
        1 + \qty(\frac{X_F - n}{0.06\Lambda_{n+1}})^2
      }
    ]
  \end{aligned}
\end{equation}
Using above expression we can analyse the changes can be done to the transverse conductivity in 2DEG using external dressing field. As given in Fig.~\ref{fig:8.1} and \ref{fig:8.2} we can manipulate the transverse conductivity $\sigma_{xx}$ using external dressing field's intensity and the Fermi level $X_F$ of the considering system. When the applied field's intensity increase the broadening of energy bands of Landau levels get reduced and simulaniously the transverse conductivity also get low value for all the regions except the peak point of the energy band. Using this manipulation we can make more narrow or wide the conductivity regions. Since Fermi level of the system can be change with the applied gate voltage of the material this can be used as a 2D switch for optoelectonic applications \cite{dini16}. Controlling  the external dressing field we are able to fine-tune the switching mechanism for optimized performance.
\begin{figure}[t]
\includegraphics[scale=0.55]{figures/fig_5}
\caption{\label{fig_1} Normalized transverse conductivity $\sigma_{xx}$ against Fermi level $X_F$ with different intensities $I$ of the external dressing field in a GaAs-based quantum well at a magnetic field $B = 1.2~\text{T}$, dressing field with frequency $\omega =2\times10^{12}~\text{rad}\text{s}^{-1}$. In this calculation we have assumed that the natural  broadening of $0$-th Landau level $\Gamma_0$ is $0.24\;\text{me}V$.}
\end{figure}
\begin{figure}[t]
\includegraphics[scale=0.55]{figures/fig_6}
\caption{\label{fig_1} $3$rd Landau level’s normalized transverse conductivity $\sigma_{xx}$ against Fermi level $X_F$ with different intensities $I$ of the external dressing field in a GaAs-based quantum well at a magnetic field $B = 1.2~\text{T}$, dressing field with frequency $\omega =2\times10^{12}~\text{rad}\text{s}^{-1}$ and intensity $I =100~\text{W}/\text{cm}^{2}$. In this calculation we have assumed that the natural  broadening of $0$-th Landau level $\Gamma_0$ is $0.24\;\text{me}V$}.
\end{figure}


\section{Conclusions}
\input{sections/9_conclusions.tex}

\begin{acknowledgments}
We wish to acknowledge the support of the author community in using
REV\TeX{}, offering suggestions and encouragement, testing new versions,
\dots.
\end{acknowledgments}

\appendix

\section{Appendixes}

\section{A little more on appendixes}

% The \nocite command causes all entries in a bibliography to be printed out
% whether or not they are actually referenced in the text. This is appropriate
% for the sample file to show the different styles of references, but authors
% most likely will not want to use it.
\nocite{*}

\bibliography{aps_article}% Produces the bibliography via BibTeX.

\end{document}
%
% ****** End of file apssamp.tex ******
