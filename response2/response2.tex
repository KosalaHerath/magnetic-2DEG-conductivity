% ****** Start of file response2.tex ****** %
\documentclass{article}

%===============================================================================
% Import packages
%===============================================================================

\usepackage[top=0.8in,bottom=1in,left=1in,right=1in]{geometry}
% Physics package
\usepackage{physics}
% Paragraph skip package
\usepackage[parfill]{parskip}
% AMS packages
\usepackage{amsmath}
\usepackage{amssymb}
\usepackage{mathtools}
% Color package
\usepackage[dvipsnames]{xcolor}
% Acronym package
\usepackage[acronym]{glossaries}
% SI units package
\usepackage{siunitx}
% Hypertext package
\usepackage[hidelinks]{hyperref}
% Bold math
\usepackage{bm}
% Other packages
\usepackage{graphicx}
\usepackage{textcomp}
\usepackage{float}
\usepackage{cite}
\usepackage{multicol}
\usepackage{multirow}
% \usepackage{kpfonts}

% Bibliography style
\usepackage[numbers,sort&compress]{natbib}
\bibliographystyle{apsrev4-1}

\begin{document}

% Define acronyms
\newacronym{tls}{TLS}{two-level system}

%===============================================================================
% Add date
%===============================================================================
\today\\

Ashot Melikyan,\\
Associate Editor,\\
Physical Review B.\\

Dear Professor Melikyan,

Thank you very much for your efforts in managing the review process for our manuscript. We are grateful for the comments of the reviewers and we believe the manuscript we submit now has been significantly improved by the constructive criticism we received. Please find herewith, the reviewer comments and our responses with the changes highlighted.

Please note that in the following sections, the statements in {\color{RoyalBlue} \textbf{blue}} are the comments of the reviewers. Our responses are shown in black letters, and the modifications we have done to the manuscript are given in {\color{Red} red}.

\subsection*{Response to the comments of Reviewer 1}

We would like to thank the reviewer for his/her insightful comments on the theoretical foundations of our model. Accordingly, we have been able to significantly refine our discussion on the underlying assumptions and corresponding physical conditions of our system. We hence believe that our response and the corresponding changes we incorporated into the manuscript would be sufficient to clarify the issues raised.

\subsubsection*{Comment 1 -
\color{RoyalBlue} I still find that the justification of approximation in Eq. (34) is sketchy and likely misses some essential effects related to the impact of the radiation on the distribution function in microwave fields.}

We agree with the reviewer that the issue was not addressed comprehensively. Therefore, we made an extensive discussion on the underlying validity of the approximation in Eq.~[34]. First, we discuss the complications related to the derivation of full analytical expression for a realistic periodically driven system. Later, we clarify the actual conditions and requirements to approximate a stationary particle distribution function for a periodically driven system.

The accessibility of coherent driving fields such as lasers unlock many significant potentials to manipulate quantum systems. This leads to the Floquet engineering of solid-state, atomic, and photonic systems with the basis of Floquet states. In general, the powerful thermodynamic rules related to the static systems in thermal equilibrium, cannot be applied to the non-equilibrium regime where Floquet states are defined \cite{seetharam2015}. Therefore, to achieve the viability of the Floquet engineering in fermion systems, it is crucial to understand the conditions for a steady-state particle distribution. Nevertheless, it is one of the main challenges in the field. Among previous studies on this topic, a major point of investigation is time scales. As an example, if the switch-on of the driving field is not far from the past, the distribution function of a quantum system will highly depend on the switching-on protocol of the periodic driving \cite{dehghani2014}. Although we can consider the intermediate timescales, there would be many physical processes causing the heating of the quantum system. A variety of approaches for stabilizing the driven quantum system under this time scales have been investigated using bath engineering methods \cite{seetharam2015,weidinger2017,seetharam2019,rudner2020}. When the applied driving field gets more powerful, the heating caused by the field increases. Thus, the cooling system needs to be much stronger and this leads to a non-perturbative inclusion of the bath in the system. In addition, in the previous work of Genske \textit{et al.} \cite{genske2015}, the dynamics of the Floquet quasi-particles and their scattering processes have been addressed semi-classically using the Boltzmann approach.
However, a complete microscopic description of the heating of a driven system is a rather challenging task due to the multiple processes that need to be investigated.
It is important to notice that, the behavior of Floquet states under the long timescale conditions are even more challenging. In the absence of interaction to an external environment, a many-body driven quantum system will absorb energy from the driving field and increase the local entropy density \cite{rudner2020}. This leads towards a featureless state at long time scales. All these conclude that understanding heating processes in any timescale and addressing them to describe a realistic driven system needs the treatment of very rich physics. However, under proper parameter regimes discussed below, we can achieve a nearly quasi-steady state in isolated quantum systems on short to intermediate timescales \cite{lindner2017,bukov2015,eckardt2015,kuwahara2016,abanin2017,mori2018,rudner2020}.

Since there exists a novel regime of closed Floquet systems with a quasi-steady-state that persists for an considerably long time \cite{lindner2017}, we assume that the distribution is time-independent throughout our paper. The main reason behind the heating of a driven quantum system is the absorption of photons from the applied field. Thus, to suppress the heating rating we require to control the absorption processes. We can achieve this by fine-tuning the driving field. Before discussing further, first, we need to introduce a few energy scales mentioned in the Rudner \textit{et al.} work \cite{rudner2020}, such as the photon energy of the driving field $\hbar\omega$, the single-particle bandwidth of the low-lying bands $W$, the local interaction energy scale $U$, and the size of a gap in the spectrum $\Delta$.  Here, $\omega$ is the frequency of the driving field.
As mentioned in Rudner \textit{et al.} work \cite{rudner2020}, there are two promising regimes where the heating rate can be suppressed in a closed Floquet system. The first regime is the limit where the applied field's photon energy is larger than the bandwidth of the low-lying bands ($\hbar\omega \gg W$). In this case, the absorption process only can take place through high-order processes that involve many-particle rearrangements known as many-body resonances  \cite{bukov2016,lindner2017,rudner2020}. Therefore, these absorptions are heavily suppressed and this leads the system to an equilibrium-like
state. Now, the system becomes nearly time-independent where the external driving only renormalize the system's parameters rather than manipulating the distribution \cite{wackerlthesis20}.
This regime has been enabled many kinds of research on driven quantum systems \cite{kitagawa2011,lopez2015,pervishko2015,bukov2015,yudin2016}.
The next possible regime can be identified in gapped systems. When the high-lying energy bands are significantly separated from a set of low-lying bands ($\hbar\omega \ll \Delta$), many photons must be absorbed for a single excitation across the energy gap $\Delta$. This results in significant suppression of energy absorption processes and heating rates \cite{rudner2020}. However, if the local interaction strength $U$ is a dominating energy scale, the interactions make a huge impact on particle states in the system. This makes the study more complicated. Thus, to neglect the interactions, the interaction strength $U$ should be smaller than both the bandwidth $W$ and the photon energy $\hbar\omega$. Connecting the above-discussed energy scales and their unique requirements, we can identify a most promising regime for a driven system with steady particle distribution with an increasing energy scale hierarchy: $U \ll W \ll \hbar\omega \ll \Delta$ \cite{rudner2020,wackerlthesis20}. This will allow us to consider a quantum system with a time-independent particle distribution function under a driven field in our work.

Next, it is important to note that, the studies \cite{wackerl20,wackerlthesis20,dini16,endo09} which inspired our work have  used the steady-state particle distribution function assumption with very low-temperature conditions $k_B T \ll \varepsilon_F$. Here, $k_B$ is the Boltzmann constant, $T$ is the absolute temperature and $\varepsilon_F$ is the Fermi energy level.
Our main aim was to use a new approach namely Floquet-Drude conductivity \cite{wackerl20} to build our theoretical analysis on the dressed quantum Hall system.
Moreover, we expected to compare our theoretical description of the charge transport properties of dressed quantum Hall system with the previous discussions \cite{dini16,endo09}. Thus, we used the same steady-state particle distribution assumptions and conditions throughout our work.
This allowed us to identify the differences of our numerical results with previous outcomes from Refs. \cite{dini16,endo09} and separate the actual effects of the dressing field on quantum Hall systems.

Indeed, the consideration of impacts on a particle distribution function of a quantum system under microwave fields is essential. Nevertheless, as discussed above we can treat the driving field as a pure dressing field with the proper energy and time scale conditions. In our study, we are not restricted to microwave field radiation. Since we can employ the above-mentioned conditions in our study, we assume that the driving field only renormalizes the Floquet system's parameters rather than changing the particle distribution.

We agree with the reviewer that a more generalized analysis of this topic is required. However, the complexity related to the photon absorption processes is beyond the scope of the present study, and we plan to address them in our future work. We have now clearly mentioned these conditions related to the stable particle distribution function in the revised manuscript. We included the following text in the revised manuscript.

\begin{itemize}
  \item Section V - second paragraph (page 7):\\
  {\color{Red}
  To further evaluate the conductivity expression, we must specify the distribution function. Generic interacting Floquet systems absorb photons from the radiation and tend to heat up \cite{seetharam2019,rudner2020}.
  However, a full description of a common dressed system requires a proper treatment of photon interactions. The complexity attached to these interactions is very high. There are various strategies \cite{lindner2017,bukov2015,eckardt2015,kuwahara2016,abanin2017,mori2018,rudner2020} for overcoming this challenge of heating and for achieving non-equilibrium steady-state distribution. Working in regimes where heating rates are strongly suppressed enables steady particle distribution function in the driven isolated quantum systems.
  As mentioned in Rudner \textit{et al.} \cite{rudner2020}, placing proper conditions on energy and time scales, the distribution function of a Floquet system can be assumed to be steady. By properly selecting the frequency of the radiation, we can ensure that the inter-band electron transitions are controlled. This leads to significantly suppressed heating rates.
  Let's consider our dressed quantum Hall system under off-resonant conditions \cite{rudner2020,wackerlthesis20}, where photon absorption does not happen. Thus, we can assume that the driving field only renormalizes the system's parameters rather than changing the particle distribution function.
  Additionally, we can select the Fermi-Dirac distribution as our partial distribution function ($f$) for our Floquet system
  \begin{equation} \tag{33}
    f(\varepsilon) = \frac{1}{\exp[(\varepsilon - \varepsilon_F)/k_B T]+1},
  \end{equation}
  where $k_B$ is the Boltzmann constant, $T$ is the absolute temperature, and $\varepsilon_F$ is the Fermi energy of the system. To compare our results with the outcomes already known in the literature \cite{dini16,endo09}, we enforce the low-temperature conditions to the Fermi-Dirac distribution.
  At low-temperatures conditions, i.e., $k_BT \ll \varepsilon_F$, the derivative of this distribution is sharply peaked around the Fermi energy, and can be approximated by a delta function \cite{endo09}
  \begin{equation} \tag{34}
    - \pdv{f(\varepsilon)}{\varepsilon} \approx \delta(\varepsilon - \varepsilon_F).
  \end{equation}
  }
\end{itemize}

\subsubsection*{Comment 2 -
\color{RoyalBlue} Moreover, the authors are misinterpreting the results of earlier theoretical works. In particular, Ref. [31] demonstrates both the effect of photo excitation of electrons and the dressing of electronic states.}

We thank the reviewer for mentioning this deficiency. We believe this comment from the reviewer comes mainly from our statement

“This inspired investigations on the theoretical description of MIROs, and several semi-classical and quantum kinetic equation formalisms have been proposed to address the underlying mechanism of MIROs [54-57]. These formalisms provide a proper explanation for the experimental observations of MIROs. However, these experimental and theoretical works have been linked to photon absorption from low-frequency (microwave) electromagnetic fields.
In contrast to that, high-frequency external illumination on a 2DEG quantum Hall system needs to be studied as a pure dressing (non-absorbable) field.
The influence induced by a pure dressing field on
magneto-transport properties of 2DEG quantum Hall system needs to be described by a non-absorption mechanism, and it has escaped the researchers’ attention before.”

in Section I (page 1 and page 2) in the manuscript. It is important to note that the ``Ref.~[31]'' in our previous response to reviewer comments represents the following study:

\textit{I. A. Dmitriev, M. Khodas, A. D. Mirlin, D. G. Polyakov, and M. G. Vavilov, Phys. Rev. B 80, 165327 (2009).}

In our previous manuscript, we have represented it as ``Ref.~[57]''. In the present response, we refer to it as Ref.~\cite{dmitriev09}.
On re-reading the previous manuscript, we see that the language we had used here is ambiguous and admit that the statement gives a meaning that is quite different from what we were trying to convey.

What we meant to say here is that work presented in Dmitriev \textit{et al.} \cite{dmitriev09} only relates to the effects induced from a low-frequency (microwave) radiation, specially the microwave-induced resistance oscillations (MIROs). The MIROs can be observed in 2DEG subjected to both magnetic field and low-frequency (microwave) radiation. As mentioned by the reviewer, Ref. \cite{dmitriev09} have appropriately described the magneto-transport properties using the dressed states of electrons as well as the photon interactions. On the other hand, if the driving field is in a high-frequency regime, we can neglect the photon interactions as mentioned in the response to the first comment. In addition, we can identify that the high-frequency and low-frequency radiation lead to two different behaviors of the magneto-transport characteristics of dressed quantum Hall systems \cite{dini16}. With the low-frequency illumination, particular effects can be observed known as ``zero resistance states" where we can identify a increased longitudinal conductivity. However, we can predict a depletion of both the broadening of Landau levels and the longitudinal conductivity under high-frequency radiation.
Our analysis is focused on a high-frequency regime where we can significantly suppress the photon interactions in the Floquet system. Therefore, we meant to mention this difference between our study and the Ref. \cite{dmitriev09}. We never meant to say that Dmitriev \textit{et al.} \cite{dmitriev09} only consider the effects of photon excitation.

To incorporate your comment into the manuscript, we removed the above-mentioned statement and replaced it with wordings which hopefully would convey our meaning better.

\begin{itemize}
  \item Section I - second paragraph (page 5):\\
  {\color{Red}
  This inspired investigations on the theoretical description of MIROs, and several semi-classical and quantum kinetic equation formalisms have been proposed to address the underlying mechanism of MIROs \cite{durst03,dmitriev03,dmitriev05,dmitriev09}. These formalisms provide a proper explanation for the experimental observations of MIROs. However, these experimental and theoretical works have been linked to the effects of low-frequency (microwave) electromagnetic fields that increase the longitudinal conductivity.
  In contrast to that, high-frequency external illumination on a 2DEG quantum Hall system leads to different behavior that reduces the longitudinal conductivity \cite{dini16}.
  Since high-frequency radiation only contributes to the suppression of electron scattering, the driving field can be treated as a pure dressing (non-absorbable) field \cite{dini16}.
  Thus, the influence induced by a high-frequency pure dressing field on
  magneto-transport properties of 2DEG quantum Hall system need to be described by a non-absorption mechanism, and it has escaped the researchers’ attention before.
  }
\end{itemize}

In addition, we have made the following minor change in language and presentation to match the rest of the manuscript better to the changes done to address the reviewer’s comment.

\begin{itemize}
  \item Section VI - sixth paragraph (page 9):\\
  {\color{Red}
  However, these experiments and theoretical models only analyzed the behavior of MIROs in 2DEG systems which are based on low-frequency fields.
  }
  \item Section VI - sixth paragraph (page 9):\\
  {\color{Red} In our case, the applied dressing field is in the high-frequency regime where we can neglect the photon interactions. The only influence of the electromagnetic field is the suppression of electron scattering.}
\end{itemize}


\subsubsection*{Comment 3 -
\color{RoyalBlue} This paper also discusses the importance of the disorder structure and that the large parameter r\_c often characterizes the high mobility electron gas. The latter is considered short in the present manuscript. However, since the authors clarified specific assumptions in their theoretical model, I believe the paper can be published in its current form. Analyzing the system beyond the approximations made here can be done later.}

We admit that including an analytical description for the effect of long-range impurity potentials is valuable. However, in the studies \cite{dini16,endo09} that influenced our work, the transport properties of quantum Hall systems and dressed quantum Hall systems have been investigated with the assumption of short-range impurity potentials.
The main focus of the present study is to utilize the novel Floquet-Drude conductivity method \cite{wackerl20,wackerlthesis20} to describe dressed quantum Hall systems. Since it is required to compare our novel theoretical description of charge transport properties with the previous charge transport studies \cite{dini16,endo09} of quantum Hall systems, we adopted the same assumptions and conditions. In addition, Wackel \textit{et al.} \cite{wackerl20} have derived the Floquet-Drude conductivity theory with the short-range impurity potential assumption.  Thus, we need to improve the Floquet-Drude conductivity theory before using it on the dressed quantum Hall system with long-range impurity conditions. The analysis of the implications from long-range impurity potentials needs a proper treatment of high-order interactions.
The discussion of these interactions through the Floquet theory is a different topic. Thus, it is not discussed in this work. However, following the reviewer’s suggestion, we hope to discuss these effects in our future study.


\bibliography{response2}
\medskip
Sincerely yours,

\def\s#1#2#3{\vbox{\hsize=4.5cm
		\kern2cm
		\hrule\kern1ex
		\hbox to \hsize{\strut\hfil #1 \hfil}
		\hbox to \hsize{\strut\hfil #2 \hfil}
		\hbox to \hsize{\strut\hfil #3 \hfil}}}

\hbox to \hsize{\s{Malin Premaratne}{(Corresponding Author)}{\href{malin.premaratne@monash.edu}{malin.premaratne@monash.edu}}}


\end{document}

% ****** End of file dressed_quantum_hall.tex ****** %
