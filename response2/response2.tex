% ****** Start of file response2.tex ****** %
\documentclass{article}

%===============================================================================
% Import packages
%===============================================================================

\usepackage[top=0.8in,bottom=1in,left=1in,right=1in]{geometry}
% Physics package
\usepackage{physics}
% Paragraph skip package
\usepackage[parfill]{parskip}
% AMS packages
\usepackage{amsmath}
\usepackage{amssymb}
\usepackage{mathtools}
% Color package
\usepackage[dvipsnames]{xcolor}
% Acronym package
\usepackage[acronym]{glossaries}
% SI units package
\usepackage{siunitx}
% Hypertext package
\usepackage[hidelinks]{hyperref}
% Bold math
\usepackage{bm}
% Other packages
\usepackage{graphicx}
\usepackage{textcomp}
\usepackage{float}
\usepackage{cite}
\usepackage{multicol}
\usepackage{multirow}
% \usepackage{kpfonts}

% Bibliography style
\usepackage[numbers,sort&compress]{natbib}
\bibliographystyle{apsrev4-1}

\begin{document}

% Define acronyms
\newacronym{tls}{TLS}{two-level system}

%===============================================================================
% Add date
%===============================================================================
\today\\

Ashot Melikyan,\\
Associate Editor,\\
Physical Review B.\\

Dear Professor Melikyan,

Thank you very much for your efforts in managing the review process for our manuscript. We are grateful for the comments of the reviewers and we believe the manuscript we submit now has been significantly improved by the constructive criticism we received. Please find herewith, the reviewer comments and our responses with the changes highlighted.

Please note that in the following sections, the statements in {\color{RoyalBlue} \textbf{blue}} are the comments of the reviewers. Our responses are shown in black letters, and the modifications we have done to the manuscript are given in {\color{Red} red}.

\subsection*{Response to the comments of Reviewer 1}

We would like to thank the reviewer for his/her insightful comments on the theoretical foundations of our model. Accordingly, we have been able to significantly refine our discussion on the underlying assumptions and corresponding physical conditions of our system. We hence believe that our response, and the corresponding changes we incorporated to the manuscript would be sufficient to clarify the issues raised.

\subsubsection*{Comment 1 -
\color{RoyalBlue} I still find that the justification of approximation in Eq. (34) is sketchy and likely misses some essential effects related to the impact of the radiation on the distribution function in microwave fields.}

We agree with the reviewer that the issue was not addressed properly. Therefore, we have made an extensive discussion on the underlying validity of approximation in Eq.~[34]. Fisrt, we discuss about the complications we have to face when we are trying to derive a full description of a realistic periodically driven system. Later, we clarify the actual conditions and requirements to approximate a stationary particle distribution function for driven system.

The accessibility of coherent driving fields such as laser opens many interesting capabilities for manupulating quantum systems. This leads to Floquet engineeting of solid-state, atomic, and photonic systems with the basic of Floquet states. However, the powerful thermodynamic rules that related to the static systems in thermal equilibrium in general cannot be apply into the nonequilibrium regime where Floquet states are defined \cite{seetharam2015}. Therefore, in order to achieve the viability of Floquet engineering in fermion systems, it is crucial to undestand the conditions for a steady state particle distribution and it is one of the major challenge in the field. Among previous studies on this topic, a main point of inestigation is time-scales. As an example, if the switch-on of the driving field is not far from the past, the distribution function of a qunatum system will highly depend on the switching-on protocal of the periodic driving \cite{dehghani2014}. Although we can consider the intermediate timescales, there would be many physical processes causing the heating of the quantum system. A variety of approches for stabilizing the quantum system under this condition has been investigated using bath engineering methods \cite{seetharam2015,weidinger2017,seetharam2019,rudner2020}. When the applied driving field get stronger the heating caused by the field gets increase. Thus, the cooloing system needs to be much stronger and this leads to a non-perturbative inclusion of the bath to the system. In addition, in the previous work of Genske \textit{et al.} \cite{genske2015}, the dynamics of the Floquet quasiparticles and their scattering processes have addressed semi-classically using the Boltzmann approch. However, a fully microscopic description of heating of a driven system is a rather challenging task due to the multiple processes that need to be investigate.
It is important to notice that, the behavior of Floquet states under the long timescale conditions are even more challenging. In the absence of interaction to an external environment, many-body driven quantum system will absorb energy from the driving field and increase the local entropy density \cite{rudner2020}. This leads towards to a featureless state at long times. All these concludes that undestanding of heating processes in any timescale and adress them to describe relastic driven system needs the treatment of very rich physics. However, under proper parameter regimes to be discussed below, we can achieve a nearly quasi-steady state in isolated quantum systems on short to intermediate timescales \cite{lindner2017,bukov2015,eckardt2015,kuwahara2016,abanin2017,mori2018,rudner2020}.

Since there exist a novel regime of closed Floquet systems with quas-isteady state that persists for an exponentially long time \cite{lindner2017}, we assumed that the distribution is to be a time-independent through out our paper. The main reason behind the heating of a driven quantum system is absorbtion of photons from the applied field. Thus, to suppress the heating rating we require to control the absorption processes. We can achieve this by fine-tuning the driving field. Before discuss further, first we need to introduce few energy scales mentioned in the Rudner \textit{et al.} work \cite{rudner2020}, such as photon energy of the driving field $\hbar\omega$, the single-particle bandwidth of the low-lying bands $W$, the local interaction energy scale $U$, and the size of a gap in the spectrum $\Delta$.  Here, $\omega$ is the frequency of the driving field.
As mentioned in Rudner \textit{et al.} work \cite{rudner2020}, there are two promising regimes where the heating rate can be suppressed in a closed Floquet system. First regimes is the limit where the applied field's photon energy is larger than the bandwidth of the low-lying bands ($\hbar\omega \gg W$). In this case, absorption process only can take place through high-order processes that involve many-particle rearrangements known as many-body resonances \cite{bukov2016,lindner2017,rudner2020}. Therefore, these absorptions are heavily suppressed and this leads the system to a equilibrium-like
state. Now, the system becomes nearly time-independent where the external driving only renormalize the system's parameters rather than manipulating the distribution \cite{wackerlthesis20}.
This regime has been enabled many researches on driven quantum systems \cite{kitagawa2011,lopez2015,pervishko2015,bukov2015,yudin2016}.
The next possible regime can be identified in gapped systems. When the high-lying energy bands are significantly seperated from a set of low-lying bands ($\hbar\omega \ll \Delta$), many photons must be absorbed for a single excitation across the energy gap $\Delta$. This results a significant suppression of energy absorbtion processes and heating rates \cite{rudner2020}. However, if the local interaction strength $U$ is a dominating energy scale, the interactions make hige impact on particle states in the system. This makes the study more complicated. Thus, to neglect the interactions, the interaction strength should be smaller than the both the bandwidth and the photon energy. Connecting the above discussed energy scales and their unique requirements, we can identify a most promising regime for a driven system with steady particle distribution with an increasing energ scale hierarchy: $U \ll W \ll \hbar\omega \ll \Delta$ \cite{wackerlthesis20}. This will allow us to consider a quantum system with time-independent paricle distribution function under a driven field in our work.

Next, it is important to note that, the studies \cite{wackerl20,wackerlthesis20,dini16,endo09} inspire our work have been used this steady state particle distribution function assumption with very low temperature conditions $k_B T \ll \varepsilon_F$. Here, $k_B$ is the Boltzmann constant, $T$ is the absolute temperature and $\varepsilon_F$ is the Fermi energy level.
Our main aim was to use a new approch namely Floquet-Drude conductivity \cite{wackerl20} to build our theoritical analysis on the dressed quantum Hall system.
Since, we expected to compare our theoritical description on the charge transport properties of dressed quantum Hall system with the previous discussions \cite{dini16,endo09}, we used the same assumtions and conditions through out our work as well.
This allowed us to identify the differences of our numerical results with previous outcomes from Refs. \cite{dini16,endo09} and seperate the actural effects of the dressing field on quantum Hall systems.

It is true that the consideration of impacts on a particle distribution function of a quantum system under microwave fields is essential. Nevertheless, as discussed above we can treat the driving field as a pure dressing field with the proper energy and time scale conditions. In our study we are not rescticted to microwave field radiation. Since we can  employ the above mentioned conditions in our study, we assume that the driving field only renormalize the systems's parameters rather than changing the particle distribution.

We agree with the reviwer that a more generalized analysis on this topic is required. However, the complexity related to the photon absorbtion processes is beyond the scope of the present study, and we plan to address them in our future work. We have now clearly mentioned these conditions related to steady particle distribution function in the revised manuscript. We included the following text in the revised manuscript.

\begin{itemize}
  \item Section V - second paragraph (page 7):\\
  {\color{Red}
  To further evaluate the conductivity expression, we must specify the distribution function. Generic interacting Floquet systems absorb photons from the radiation and tend to heat up \cite{seetharam2019,rudner2020}.
  However, a full description of a common dressed system requires a proper treatment of photon interactions. The complexity attach to these interactions are very high. There are various strategies \cite{lindner2017,bukov2015,eckardt2015,kuwahara2016,abanin2017,mori2018,rudner2020} for overcoming this challenge of heating and for achieving nonequilibrium steady state distribution. Working in regimes where heating rates are stongly suppressed enables steady particle distribution function in driven isolated quantum systems.
  As mentioned in Rudner \textit{et al.} \cite{rudner2020}, placing proper conditions on energy and time scales, the distribution function of a Floquet system can be assumed to be steady. By properly selecting the frequency of the radiation, we can ensure that the interband electron transitions are controlled. This leads to significantly suppressed heating rates.
  Let's consider our dressed quantum Hall system under off-resonant conditions \cite{rudner2020,wackerlthesis20}, where photon absorption does not happen. Thus, we can assume that the driving field only renormalize the systems's parameters rather than changing the particle distribution function.
  Additionaly, we can select the Fermi-Dirac distribution as our partial distribution function ($f$) for our Floquet system
  \begin{equation} \tag{33}
    f(\varepsilon) = \frac{1}{\exp[(\varepsilon - \varepsilon_F)/k_B T]+1},
  \end{equation}
  where $k_B$ is the Boltzmann constant, $T$ is the absolute temperature, and $\varepsilon_F$ is the Fermi energy of the system. With the aim of compare our results with the outcomes already known in literature \cite{dini16,endo09}, we enfore the low-temperature conditions to the Fermi-Dirac distribution.
  At low-temperatures conditions, i.e., $k_BT \ll \varepsilon_F$, the derivative of this distribution is sharply peaked around the Fermi energy, and can be approximated by a delta function \cite{endo09}
  \begin{equation} \tag{34}
    - \pdv{f(\varepsilon)}{\varepsilon} \approx \delta(\varepsilon - \varepsilon_F).
  \end{equation}
  }
\end{itemize}

\subsubsection*{Comment 2 -
\color{RoyalBlue} Moreover, the authors are misinterpreting the results of earlier theoretical works. In particular, Ref. [31] demonstrates both the effect of photo excitations of electrons and the dressing of electronic states.}

We thank the reviewer for mention this deficiency. We believe this comment from the reviewer comes mainly from our statement

{\color{CadetBlue} “This inspired investigations on the theoretical description of MIROs, and several semiclassical and quantum kinetic equation formalisms have been proposed to address the underlying mechanism of MIROs [54-57]. These formalisms provide a proper explanation for the experimental observations of MIROs. However, these experimental and theoretical works have been linked to photon absorption from low-frequency (microwave) electromagnetic fields.
In contrast to that, high-frequency external illumination on a 2DEG quantum Hall system needs to be studied as a pure dressing (nonabsorbable) field.
The influence induced by a pure dressing field on
magneto-transport properties of 2DEG quantum Hall system needs to be described by a non-absorption mechanism, and it has escaped the researchers’ attention before.”}

in the Section I (page 1 and page 2) in the manuscript. It is important to note that the Ref. [31] in our previous response to reviewer comments represents the the
following study:

\textit{I. A. Dmitriev, M. Khodas, A. D. Mirlin, D. G. Polyakov, and M. G. Vavilov, Phys. Rev. B 80, 165327 (2009).}

In our previous manuscript we have represented it as Ref. [57]. In the present response we refer it as \cite{dmitriev09}.
On a re-reading the previous manuscript, we see that the language we had used here is ambiguous and admit that the statement gives a meaning that is quite different to what we were trying to convey.

What we meant to say here is that, work presented in Dmitriev \textit{et al.} \cite{dmitriev09} related to the microwave induced resistance oscillations (MIROs) which can be observed in 2DEG subjected to both magnetic field and low-frequency (microwave) radiation. As mentioned by the reviewer, Ref. \cite{dmitriev09} have properly described the magnto-trasnport properties using the dressed states of electrons as well as the photon interactions. On the other hand, if the driving field is in high-frequency regime, we can neglect the photon interactions as mentioned in the response to the first comment. In addition, we can identify that the high-frequency and low-frequency radiation lead to two different behaviors of the magnto-trasnport characteristics of dressed quantum Hall systems \cite{dini16}. With the low-frequency illumination, particular effects can be observed known as ``zero resistance states" where we can observe increased longitudinal conductivity. However, under high-frequency radiation decrease both the broadening of Landau levels and the longitudinal conductivity.
Our analysis is focused on high-frequency regime where we can significantly suppress the photon interactions in the Floquet system. Therefore, we meant to mention this difference between the our study and the Ref. \cite{dmitriev09}. We never meant to say that Dmitriev \textit{et al.} \cite{dmitriev09} only consider the effects of photon excitations.

To incorporate your comment to the manuscript, we removed the above mentioned statement and replaced it with wording which hopefully would convey our meaning better.

\begin{itemize}
  \item Section I - second paragraph (page 5):\\
  {\color{Red}
  This inspired investigations on the theoretical description of MIROs, and several semiclassical and quantum kinetic equation formalisms have been proposed to address the underlying mechanism of MIROs \cite{durst03,dmitriev03,dmitriev05,dmitriev09}. These formalisms provide a proper explanation for the experimental observations of MIROs. However, these experimental and theoretical works have been linked to effects of low-frequency (microwave) electromagnetic fields which results an increase of longitudinal conductivity.
  In contrast to that, high-frequency external illumination on a 2DEG quantum Hall system leads to different behavior that reduce the longitudinal conductivity \cite{dini16}.
  Since high-frequency radiation only contribute to the suppression of electron scattering, the driving field can be treated as a pure dressing (nonabsorbable) field \cite{dini16}.
  Thus, the influence induced by a high-frequency pure dressing field on
  magneto-transport properties of 2DEG quantum Hall system needs to be described by a non-absorption mechanism, and it has escaped the researchers’ attention before.
  }
\end{itemize}

In addition, we have made the following minor change in language and presentation to match the rest of the manuscript better to the changes done to address the reviewer’s comments.

\begin{itemize}
  \item Section VI - sixth paragraph (page 9):\\
  {\color{Red}
  However, these experiments and theoretical models only analyzed the behavior of MIROs in 2DEG systems which are based on low-frequency fields.
  }
  \item Section VI - sixth paragraph (page 9):\\
  {\color{Red} In our case, the applied dressing field is in the high-frequency regime where we can neglect the photon interactions. The only influence of the electromagnetic field is the suppression of electron scattering.}
\end{itemize}




































\bibliography{response2}
\medskip
Sincerely yours,

\def\s#1#2#3{\vbox{\hsize=4.5cm
		\kern2cm
		\hrule\kern1ex
		\hbox to \hsize{\strut\hfil #1 \hfil}
		\hbox to \hsize{\strut\hfil #2 \hfil}
		\hbox to \hsize{\strut\hfil #3 \hfil}}}

\hbox to \hsize{\s{Malin Premaratne}{(Corresponding Author)}{\href{malin.premaratne@monash.edu}{malin.premaratne@monash.edu}}}


\end{document}

% ****** End of file dressed_quantum_hall.tex ****** %
