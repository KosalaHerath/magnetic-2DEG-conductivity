% ****** Start of file response2.tex ****** %
\documentclass{article}

%===============================================================================
% Import packages
%===============================================================================

\usepackage[top=0.8in,bottom=1in,left=1in,right=1in]{geometry}
% Physics package
\usepackage{physics}
% Paragraph skip package
\usepackage[parfill]{parskip}
% AMS packages
\usepackage{amsmath}
\usepackage{amssymb}
\usepackage{mathtools}
% Color package
\usepackage[dvipsnames]{xcolor}
% Acronym package
\usepackage[acronym]{glossaries}
% SI units package
\usepackage{siunitx}
% Hypertext package
\usepackage[hidelinks]{hyperref}
% Bold math
\usepackage{bm}
% Other packages
\usepackage{graphicx}
\usepackage{textcomp}
\usepackage{float}
\usepackage{cite}
\usepackage{multicol}
\usepackage{multirow}
% \usepackage{kpfonts}

% Bibliography style
\usepackage[numbers,sort&compress]{natbib}
\bibliographystyle{apsrev4-1}

\begin{document}

% Define acronyms
\newacronym{tls}{TLS}{two-level system}

%===============================================================================
% Add date
%===============================================================================
\today\\

Ashot Melikyan,\\
Associate Editor,\\
Physical Review B.\\

Dear Professor Melikyan,

Thank you very much for your efforts in managing the review process for our manuscript. We are grateful for the comments of the reviewers and we believe the manuscript we submit now has been significantly improved by the constructive criticism we received. Please find herewith, the reviewer comments and our responses with the changes highlighted.

Please note that in the following sections, the statements in {\color{RoyalBlue} \textbf{blue}} are the comments of the reviewers. Our responses are shown in black letters, and the modifications we have done to the manuscript are given in {\color{Red} red}.

\subsection*{General changes to the manuscript}


We have made some minor changes in language and presentation to improve the clarity and organization of the manuscript.
\begin{itemize}
    \item Section I - seventh paragraph (page 2): \\
        {\color{Red} In Sec. VII, we discuss the physical significance of our theoretical results and their possible use in future nanoelectronic devices.
        Finally, we summarize our findings and present our conclusions in Sec. VIII.}
\end{itemize}

\subsection*{Response to the comments of Reviewer 1}

We would like to thank the reviewer for his/her insightful comments on the theoretical foundations of our model. Accordingly, we have been able to significantly refine our discussion on the underlying assumptions and corresponding physical conditions of our system. We hence believe that our response, and the corresponding changes we incorporated to the manuscript would be sufficient to clarify the issues raised.

\subsubsection*{Comment 1 -
\color{RoyalBlue} I still find that the justification of approximation in Eq. (34) is sketchy and likely misses some essential effects related to the impact of the radiation on the distribution function in microwave fields.}

We agree with the reviewer that the issue was not addressed properly. Therefore, we have made an extensive discussion on the underlying validity of approximation in Eq.~[34]. Fisrt, we discuss about the complications we have to face when we are trying to derive a full description of a realistic periodically driven system. Thus, we assume the particle distribution function is stationary under external driving field. Later, we clarify the actual conditions and requirements for the a stationary particle distribution function for driven system.

The accessibility of coherent driving fields such as laser opens many interesting potential for manupulating quantum systems. This leads to Floquet engineeting of solid-state, atomic, and photonic systems with the bais of Floquet states. However, the powerful thermodynamic rules that related to the static systems in thermal equilibrium in general cannot be apply into the nonequilibrium regime where Floquet states are defined \cite{seetharam2015}. Therefore, in order to achieve the viability of Floquet engineering in fermion systems, it is crucial to undestand the conditions for a steady state partical distribution and it is one of the major challenge in the field. Among previous studies on this topic, a main point of inestigation is time-scales. As an example, if the switch-on of the driving field is not far from the past, the distribution function of a qunatum system will highly depend on the switching-on protocal of the periodic driving \cite{dehghani2014}. Although we can consider the intermediate timescales, there would be many physical processes causing the heating of the quantum system. A variety of approches for stabilizing the quantum system under this condition has been investigated using bath engineering methods \cite{seetharam2015,weidinger2017,seetharam2019,rudner2020}. When the applied driving field get stronger the heating caused by the field gets increase. Thus, the cooloing system needs to be much stronger and this leads to a non-perturbative inclusion of the bath to the system. In addition, in the previous work of Genske \textit{et al.} \cite{genske2015}, the dynamics of the Floquet quasiparticles and their scattering processes have addressed semi-classically using the Boltzmann approch. However, a fully microscopic description of heating of a driven system is a rather challenging task due to the multiple processes that need to be investigate.
It is important to notice that, the behavior of Floquet states under the long timescale conditions are even more challenging. In the absence of interaction to an external environment, many-body driven quantum system will absorb energy from the driving field and increase the local entropy density \cite{rudner2020}. This leads towards to a featureless state at long times.



































\bibliography{response2}
\medskip
Sincerely yours,

\def\s#1#2#3{\vbox{\hsize=4.5cm
		\kern2cm
		\hrule\kern1ex
		\hbox to \hsize{\strut\hfil #1 \hfil}
		\hbox to \hsize{\strut\hfil #2 \hfil}
		\hbox to \hsize{\strut\hfil #3 \hfil}}}

\hbox to \hsize{\s{Malin Premaratne}{(Corresponding Author)}{\href{malin.premaratne@monash.edu}{malin.premaratne@monash.edu}}}


\end{document}

% ****** End of file dressed_quantum_hall.tex ****** %
