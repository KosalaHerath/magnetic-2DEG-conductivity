\section{Scattering theory}

Since in a real metal there would be many scatters that can be behave as obstacles for electron that have free wave functions. Therefore we need to calculate them to analyse the real behaviour of the electrons.

\noindent
Then the wave function of the electron in a real matel $\Psi(\vb{r},t)$ should satisfy the following time-dependet Schrodinger equation
\begin{equation} \label{2.1}
  i \hbar \pdv{\Psi(\vb{r},t)}{t} = [H_e(t) + U(\vb{r})] \Psi(\vb{r},t)
\end{equation}
where $U(\vb{r})$ is the total scattering potential. We have represented the all scatters using this potential. Since the solutions \eqref{1.52} are create a complete orthonormal basis we can represent this wave function using those as follows
\begin{equation} \label{2.2}
  \Psi(\vb{r},t) = \sum_j a_j(t) \ket{\psi_j(t)}
\end{equation}
where the difference inidces j corresponding to the different sets of all quantum numbers $p_x$ and $n$
\begin{equation} \label{2.3}
  j \rightarrow (m,n) \quad \text{where} \quad
  m,n =0,1,2,...
\end{equation}
with $m$ is defined for quantized momentum in $x$ direction
\begin{equation} \label{2.4}
  p_x = m \frac{2\pi \hbar}{L_x}
\end{equation}

\noindent
Now we can use the conventional pertubation theory to calculate scattering process of electron at a state $\ket{\psi_j}$ to a state $\ket{\psi_j'}$. For that assume an electron be in the $j$ state at the time $t=0$ and corresponding $a_j'(0) = \delta_{j,j'}$.

\noindent
First subtitute a general electron state $\Psi(\vb{r},t)$ at time $t$ as the incoming electron to the Schrodinger equation given in Eq. \eqref{2.1}
\begin{equation} \label{2.5}
  i \hbar \pdv{t} \sum_j a_j(t) \ket{\psi_j(t)}= [H_e(t) + U(\vb{r})] \sum_j a_j(t) \ket{\psi_j(t)}
\end{equation}
\begin{equation} \label{2.6}
  i \hbar\sum_j   \dot{a_j}(t) \ket{\psi_j(t)} + a_j(t)\pdv{t}\ket{\psi_j(t)}= [H_e(t) + U(\vb{r})] \sum_j a_j(t) \ket{\psi_j(t)}
\end{equation}
since all the ${\ket{\psi(t)}}$ staistfy the Schrodinger equation \eqref{1.18}
\begin{equation} \label{2.7}
  i \hbar\sum_j   \dot{a_j}(t) \ket{\psi_j(t)} = \sum_j U(\vb{r}) a_j(t) \ket{\psi_j(t)}.
\end{equation}
Then take inner product with state with the state $\ket{\psi_{j'}(t)}$
\begin{equation} \label{2.8}
  i \hbar\sum_j   \dot{a_j}(t) \braket{\psi_{j'}(t)}{\psi_j(t)} = \sum_j
  a_j(t) \bra{\psi_{j'}(t)} U(\vb{r}) \ket{\psi_j(t)}
\end{equation}
But using the \textit{Born approximation} we can assume that this incoming wave have the initial state of the electron at $t=0$ and therefore this equation will modified to
\begin{equation} \label{2.9}
  i \hbar\sum_j   \dot{a_j}(t) \braket{\psi_{j'}(t)}{\psi_j(t)} =
  \bra{\psi_{j'}(t)} U(\vb{r}) \ket{\psi_j(t)}
\end{equation}
due to orthonormality this becomes
\begin{equation} \label{2.10}
  i \hbar \dot{a_{j'}}(t) =
  \bra{\psi_{j'}(t)} U(\vb{r}) \ket{\psi_j(t)}
\end{equation}
and finally this leads to first order pertubation theory for Sscattering as follows
\begin{equation} \label{2.11}
   \dot{a_{j'}}(t) =
  -\frac{i}{\hbar}\bra{\psi_{j'}(t)} U(\vb{r}) \ket{\psi_j(t)}
\end{equation}
where
\begin{equation} \label{2.12}
   a_{j'}(t) =
  -\frac{i}{\hbar}
  \int_0^t dt' \int_S d\vb{r} \;
  \psi_{j'}^{*} (\vb{r},t') U(\vb{r}) {\psi_j(\vb{r},t')}
\end{equation}
where the integration should be performed over the 2DEG area $S=L_xL_y$. Then we can calculate this using the eqution we derived in \eqref{1.52} as follows
\begin{equation} \label{2.13}
  \begin{aligned}
    a_{j'}(t) & =
   -\frac{i}{\hbar}
   \int_0^t dt' \int_S d\vb{r} \;
   \bigg[
   \frac{1}{\sqrt{L_x}} \chi_{n'}^*\big(y - {y'}_0 -\zeta(t)\big) \\
   & \times
     \exp(
    \frac{i}{\hbar}\bigg[E_{n'}t' - m'\frac{2\pi \hbar x}{L_x} -
   \frac{eE(y-{y'}_0)}{\omega}\cos(\omega t')-
   m_e\dot{\zeta}(t)\big[y - {y'}_0 -\zeta(t')\big]
    - \int_0^{t'}dt'L(\zeta,\dot{\zeta},t")\bigg]) \\
    & \times
    U(\vb{r}) \\
    & \times
    \frac{1}{\sqrt{L_x}} \chi_{n}\big(y - y_0 -\zeta(t')\big) \\
    & \times
      \exp(
     \frac{i}{\hbar}\bigg[ - E_{n}t' + m\frac{2\pi \hbar x}{L_x} -
    \frac{eE(y-y_0)}{\omega}\cos(\omega t') -
    m_e\dot{\zeta}(t')\big[y - y_0 -\zeta(t')\big]
     - \int_0^{t'}d\tilde{t}L(\zeta,\dot{\zeta},\tilde{t})\bigg])
    \bigg]
  \end{aligned}
\end{equation}
then this will be simplified to
\begin{equation} \label{2.14}
  \begin{aligned}
    a_{j'}(t) & =
   -\frac{i}{\hbar}
   \int_0^t dt' \int_S d\vb{r} \;
   \bigg[
   \frac{1}{\sqrt{L_x}} \chi_{n'}^*\big(y - {y'}_0 -\zeta(t')\big)
   U(\vb{r})
   \frac{1}{\sqrt{L_x}} \chi_{n}\big(y - y_0 -\zeta(t')\big)  \\
   & \times
     \exp(
    \frac{i}{\hbar}\bigg[ E_{n'}t' - m'\frac{2\pi \hbar x}{L_x} -
   \frac{eE(y-{y'}_0)}{\omega}\cos(\omega t') -
   m_e\dot{\zeta}(t')\big[y - {y'}_0 -\zeta(t')\big]
    - \int_0^{t'}d\tilde{t}L(\zeta,\dot{\zeta},\tilde{t})\bigg]) \\
    & \times
      \exp(
     \frac{i}{\hbar}\bigg[ - E_{n}t' + m\frac{2\pi \hbar x}{L_x} +
    \frac{eE(y-y_0)}{\omega}\cos(\omega t') +
    m_e\dot{\zeta}(t')\big[y - y_0 -\zeta(t')\big]
     + \int_0^{t'}d\tilde{t}L(\zeta,\dot{\zeta},\tilde{t})\bigg])
    \bigg]
  \end{aligned}
\end{equation}
\begin{equation} \label{2.15}
  \begin{aligned}
    a_{j'}(t) & =
   -\frac{i}{\hbar}
   \int_0^t dt' \int_S d\vb{r} \;
   \bigg[
   \frac{1}{\sqrt{L_x}} \chi_{n'}^*\big(y - {y'}_0 -\zeta(t')\big)
   U(\vb{r})
   \frac{1}{\sqrt{L_x}} \chi_{n}\big(y - y_0 -\zeta(t')\big)
   \exp(\frac{2\pi i(m-m') \hbar x}{L_x})
   \\
   & \times
     \exp(
    \frac{i}{\hbar}\bigg[ E_{n'}t' +
   \frac{eE{y'}_0}{\omega}\cos(\omega t') +
   m_e\dot{\zeta}(t'){y'}_0
    \bigg])
      \exp(
     \frac{i}{\hbar}\bigg[ - E_{n}t' -
    \frac{eEy_0}{\omega}\cos(\omega t') -
    m_e\dot{\zeta}(t)y_0
    \bigg])
    \bigg].
  \end{aligned}
\end{equation}
The time dependence of the $chi_n(y)$ can neglect since it is integrate over all the values of the $y$ and we can write this as
\begin{equation} \label{2.16}
  \begin{aligned}
    a_{j'}(t) & =
   -\frac{i}{\hbar}
   \int_S d\vb{r} \;
   \frac{1}{\sqrt{L_x}} \chi_{n'}^*\big(y - {y'}_0 -\zeta(t')\big)
   U(\vb{r})
   \frac{1}{\sqrt{L_x}} \chi_{n}\big(y - y_0 -\zeta(t')\big)
   \exp(\frac{2\pi i(m-m') \hbar x}{L_x})
   \\ & \times
   \int_0^t dt' \;
   \bigg[
     \exp(
    \frac{i}{\hbar}\bigg[ (E_{n'} -E_{n}) t' +
   \frac{eE({y'}_0 - y_0)\omega_0^2}{\omega(\omega_0^2-\omega^2)}\cos(\omega t')
    \bigg])
    \bigg].
  \end{aligned}
\end{equation}
Using Jacobi-Anger expansion
\begin{equation} \label{2.17}
  e^{iz\cos(\theta)} = \sum_{l=-\infty}^{\infty} i^l J_j(z)e^{in\theta}
\end{equation}
above eqution can be modified as
\begin{equation} \label{2.18}
  \begin{aligned}
    a_{j'}(t)  =
   -\frac{i}{\hbar}
   U_{j'j}
   \int_0^t dt' \;
   \bigg[
   \sum_{l=-\infty}^{\infty} i^l J_l\bigg[\frac{eE({y'}_0 - y_0)\omega_0^2}{\hbar\omega(\omega_0^2-\omega^2)}\bigg]
     \exp(
    \frac{i}{\hbar} (E_{n'} -E_{n} + l\hbar\omega) t')
    \bigg]
  \end{aligned}
\end{equation}
where
\begin{equation} \label{2.19}
  U{j'j} \equiv \mel{\Phi_{j'}(\vb{r})}{U(\vb{r})}{\Phi_j(\vb{r})}
\end{equation}
with bare electron eigen states (without dressing field)
\begin{equation} \label{2.20}
  \Phi_{j}(\vb{r}) = \frac{1}{\sqrt{L_x}}\exp(\frac{2\pi im \hbar x}{L_x}) \chi_{n}(y).
\end{equation}
Considering time evalution from negative values we can write the same expression as follows
\begin{equation} \label{2.21}
  \begin{aligned}
    a_{j'}(t)  =
   -\frac{i}{\hbar}
   U_{j'j}
   \int_{-t/2}^{t/2} dt' \;
   \bigg[
   \sum_{l=-\infty}^{\infty} i^l J_l\bigg[\frac{eE({y'}_0 - y_0)\omega_0^2}{\hbar\omega(\omega_0^2-\omega^2)}\bigg]
     \exp(
    \frac{i}{\hbar} (E_{n'} -E_{n} + l\hbar\omega) t')
    \bigg].
  \end{aligned}
\end{equation}
To calculate scattering probability we can use this scattering amplitude's squre value
\begin{equation} \label{2.22}
  \begin{aligned}
    |a_{j'}(t)|^2  =
   \frac{|U_{j'j}|^2}{\hbar^2} &
   \int_{-t/2}^{t/2} dt' \;
   \bigg[
   \sum_{l=-\infty}^{\infty} {-i}^l J_l\bigg[\frac{eE({y'}_0 - y_0)\omega_0^2}{\hbar\omega(\omega_0^2-\omega^2)}\bigg]
     \exp(
    \frac{-i}{\hbar} (E_{n'} -E_{n} + l\hbar\omega) t')
    \bigg] \\
    & \times
    \int_{-t/2}^{t/2} dt^{''} \;
    \bigg[
    \sum_{k=-\infty}^{\infty} i^k J_k\bigg[\frac{eE({y'}_0 - y_0)\omega_0^2}{\hbar\omega(\omega_0^2-\omega^2)}\bigg]
      \exp(
     \frac{i}{\hbar} (E_{n'} -E_{n} + k\hbar\omega) t^{''})
     \bigg]
  \end{aligned}
\end{equation}
Considering long time $t\rightarrow \infty$ we can make the integral into a delta function as follows
\begin{equation} \label{2.23}
  \begin{aligned}
    |a_{j'}(t)|^2  =
   4\pi^2|U_{j'j}|^2 &
   \bigg[
   \sum_{l=-\infty}^{\infty} {-i}^l J_l\bigg[\frac{eE({y'}_0 - y_0)\omega_0^2}{\hbar\omega(\omega_0^2-\omega^2)}\bigg]
     \delta(-E_{n'} +E_{n} - l\hbar\omega)
     \bigg] \\
    & \times
    \bigg[
    \sum_{k=-\infty}^{\infty} i^k J_k\bigg[\frac{eE({y'}_0 - y_0)\omega_0^2}{\hbar\omega(\omega_0^2-\omega^2)}\bigg]
      \delta(E_{n'} - E_{n} + k\hbar\omega)
     \bigg]
  \end{aligned}
\end{equation}
and this implies $l=k$ and this leads to
\begin{equation} \label{2.24}
  \begin{aligned}
    |a_{j'}(t)|^2  =
   4\pi^2|U_{j'j}|^2
   \bigg[
   \sum_{l=-\infty}^{\infty} J_l^2\bigg[\frac{eE({y'}_0 - y_0)\omega_0^2}{\hbar\omega(\omega_0^2-\omega^2)}\bigg]
     \delta^2(E_{n'} - E_{n} + l\hbar\omega).
  \end{aligned}
\end{equation}
Then using the famous the square $\delta$ function transormation method
\begin{equation} \label{2.25}
  \begin{aligned}
     \delta^2(\epsilon ) = \delta(\epsilon )\delta^2(0)
     \lim_{t\rightarrow\infty} \int_{-t/2}^{t/2} e^{i0\times t' /\hbar} dt' =
     \frac{\delta(\epsilon) t}{2\pi \hbar}
  \end{aligned}
\end{equation}
we can calculatre the probability of electron scattering between states $j$ and $j'$ per unit time as
\begin{equation} \label{2.26}
    \mathcal{W}_{j'j} \equiv \dv{|a_{j'}(t)|^2}{t} =
    |U_{j'j}|^2
     \sum_{l=-\infty}^{\infty} J_l^2\bigg[\frac{eE({y'}_0 - y_0)\omega_0^2}{\hbar\omega(\omega_0^2-\omega^2)}\bigg]
    \times
    \frac{2\pi}{\hbar} \delta(E_{n'} - E_{n} + l\hbar\omega)
\end{equation}

\noindent
To avoid thee energy echange betwen a high-frequency field and electrons, the field should be purely dressing. We can achieve that by using the field with off-resonant and high frequency. Therefore, the only effect of the dressing field on 2DEG is the renormalization of the probability of elastic electron scattering within the same Landau level $(E_{n'}=E_n)$, which described by the term with $l=0$ the Eq. \eqref{2.26} leads to
\begin{equation} \label{2.27}
    \mathcal{W}_{j'j} = J_0^2\bigg[\frac{eE({y'}_0 - y_0)\omega_0^2}{\hbar\omega(\omega_0^2-\omega^2)}\bigg]
    \mathcal{W}_{j'j}^{(0)}
\end{equation}
where
\begin{equation} \label{2.28}
    \mathcal{W}_{j'j}^{(0)} = \frac{2\pi}{\hbar} |U_{j'j}|^2
    \delta(E_{n'} - E_{n})
\end{equation}
is the probability of scattering of a \textit{bare electron}. It is important to notice that the Bessel function factor depend on both the dressing field and stationaty magnetic field. This factor is responcible for all the effcts discussed in thisarticle.

\noindent
One can define the lifetime of the dressed electron at the Landau level $\tau$ is renormalized by the Bessel function as below
\begin{equation} \label{2.29}
    \frac{1}{\tau} \equiv \sum_{j'} \mathcal{W}_{j'j} =
    \sum_{j'}
    J_0^2\bigg[\frac{eE({y'}_0 - y_0)\omega_0^2}{\hbar\omega(\omega_0^2-\omega^2)}\bigg]
    \mathcal{W}_{j'j}^{(0)}
\end{equation}
where we have consider all posibilities that electron can jump to the state $j'$. Then rewrite the delat function as follows
\begin{equation} \label{2.30}
    \delta(\epsilon) =
    \frac{1}{\pi} \lim_{\Gamma \rightarrow 0 } \frac{\Gamma}{\Gamma^2 + \epsilon^2}
\end{equation}
where in this study we can assume that the paramater $\Gamma \equiv \hbar/\tau$ as scattering induced broading of the Landau level. But for the elestic scatteing within the same Landau level, we can write the $\delta$ function as
\begin{equation} \label{2.31}
    \delta(E_{n'} - E_{n}) \approx
    \frac{1}{\pi \Gamma}.
\end{equation}
Therefore Eq. \eqref{2.29} will change to
\begin{equation} \label{2.32}
    \frac{1}{\tau} =
    \sum_{j'}
    J_0^2\bigg[\frac{eE({y'}_0 - y_0)\omega_0^2}{\hbar\omega(\omega_0^2-\omega^2)}\bigg]
    \times
    \frac{2\pi}{\hbar} |U_{j'j}|^2 \times \frac{1}{\pi \Gamma}
\end{equation}
\begin{equation} \label{2.33}
    \frac{1}{\tau} =
    \sum_{j'}
    J_0^2\bigg[\frac{eE({y'}_0 - y_0)\omega_0^2}{\hbar\omega(\omega_0^2-\omega^2)}\bigg]
    \times
    \frac{2}{\hbar} |U_{j'j}|^2 \times \frac{\tau}{\hbar}
\end{equation}
and finally this can be modified to
\begin{equation} \label{2.34}
    \frac{1}{\tau} =
    \bigg[
    \frac{2}{\hbar^2}
    \sum_{j'}
    J_0^2\bigg[\frac{eE({y'}_0 - y_0)\omega_0^2}{\hbar\omega(\omega_0^2-\omega^2)}\bigg]
    |U_{j'j}|^2
    \bigg]^{1/2}
\end{equation}
where the summation is performed over electron states $j'$ within the same Landau level.

\noindent
Now lets specify more on the scattering potential where we can model them as randomly distributed delta fucntions as follows
\begin{equation} \label{2.35}
    U(\vb{r}) \equiv \sum_{i=1}^{N_s} U_0 \delta(\vb{r} - \vb{r}_i)
\end{equation}
where $N_s$ is the total number of scatters in the considering matel. Now we can calculate $|U_{j'j}|^2$ as follows
\begin{equation} \label{2.36}
  \begin{aligned}
    |U_{j'j}|^2 = &
    \sum_{i =1}^{N_s}
    \frac{1}{L_x^2}
    \int \int  dx_1 dy_1
      \exp(\frac{-{p'}_x x_1}{\hbar})
      \chi_n^* (y_1 - {y'}_0)
      U_0 \delta(x_1-x_i)\delta(y_1-y_i)
      \exp(\frac{{p}_x x_1}{\hbar})
      \chi_n (y_1 - {y}_0) \\
    & \times
    \int \int  dx_2 dy_2
      \exp(\frac{{p'}_x x_2}{\hbar})
      \chi_n (y_2 - {y'}_0)
      U_0 \delta(x_2 - x_i)\delta(y_2 - y_i)
      \exp(\frac{-{p}_x x_2}{\hbar})
      \chi_n^* (y_2 - {y}_0)
  \end{aligned}
\end{equation}
and considering only non-zero values for $x_1$ and $x_2$ integrals we can re-write this as
\begin{equation} \label{2.37}
  \begin{aligned}
    |U_{j'j}|^2 = &
    \sum_{i =1}^{N_s}
    \frac{U_0^2}{L_x^2}
    \int dy_1
      \exp(\frac{-{p'}_x x_i}{\hbar})
      \chi_n^* (y_1 - {y'}_0)
      \delta(y_1-y_i)
      \exp(\frac{{p}_x x_i}{\hbar})
      \chi_n (y_1 - {y}_0) \\
    & \times
    \int dy_2
      \exp(\frac{{p'}_x x_i}{\hbar})
      \chi_n (y_2 - {y'}_0)
      \delta(y_2 - y_i)
      \exp(\frac{-{p}_x x_i}{\hbar})
      \chi_n^* (y_2 - {y}_0)
  \end{aligned}
\end{equation}
and this will be simplified to
\begin{equation} \label{2.38}
  \begin{aligned}
    |U_{j'j}|^2 = &
    \sum_{i =1}^{N_s}
    \frac{U_0^2}{L_x^2}
    \int dy_1
      \chi_n^* (y_1 - {y'}_0)
      \delta(y_1-y_i)
      \chi_n (y_1 - {y}_0) \\
    & \times
    \int dy_2
      \chi_n (y_2 - {y'}_0)
      \delta(y_2 - y_i)
      \chi_n^* (y_2 - {y}_0).
  \end{aligned}
\end{equation}
Again considering only non-zero values for $y_1$ and $y_2$ integrals we can re-write this as
\begin{equation} \label{2.39}
  \begin{aligned}
    |U_{j'j}|^2 = &
    \sum_{i =1}^{N_s}
    \frac{U_0^2}{L_x^2}
      \chi_n^* (y_i - {y'}_0)
      \chi_n (y_i - {y}_0)
      \chi_n (y_i - {y'}_0)
      \chi_n^* (y_i - {y}_0).
  \end{aligned}
\end{equation}
\begin{equation} \label{2.40}
  \begin{aligned}
    |U_{j'j}|^2 =
    \frac{U_0^2}{L_x^2}
    \sum_{i =1}^{N_s}
      \chi_n^2 (y_i - {y'}_0)
      \chi_n^2 (y_i - {y}_0).
  \end{aligned}
\end{equation}
Now subtituting this derivation into the Eq. \eqref{2.34} we will get
\begin{equation} \label{2.41}
    \frac{1}{\tau} =
    \bigg[
    \frac{2U_0^2}{\hbar^2 L_x^2}
    \sum_{{y'}_0}
    J_0^2\bigg[\frac{eE({y'}_0 - y_0)\omega_0^2}{\hbar\omega(\omega_0^2-\omega^2)}\bigg]
    \sum_{i =1}^{N_s}
      \chi_n^2 (y_i - {y'}_0)
      \chi_n^2 (y_i - {y}_0)
    \bigg]^{1/2}
\end{equation}
where $j'$ reduced to ${p'}_x$ (since $n'=n$)and we can represent it by ${y'}_0$. Then this will modified to
\begin{equation} \label{2.42}
    \frac{1}{\tau} =
    \bigg[
    \frac{2U_0^2}{\hbar^2 L_x^2}
    \sum_{{y'}_0}
    \sum_{i =1}^{N_s}
    J_0^2\bigg[\frac{eE({y'}_0 - y_0)\omega_0^2}{\hbar\omega(\omega_0^2-\omega^2)}\bigg]
      \chi_n^2 (y_i - {y'}_0)
      \chi_n^2 (y_i - {y}_0)
    \bigg]^{1/2}.
\end{equation}
Now considering large size of sample and a macroscopically large $N_s$ scatters we can promate the summation to integrations as follows
\begin{equation} \label{2.43}
    \frac{1}{\tau} =
    \bigg[
    \frac{2U_0^2}{\hbar^2 L_x^2}
    \frac{eB L_x}{2\pi\hbar}\int d{y'}_0
    \frac{N_s} {L_x}\int dy_i
    J_0^2\bigg[\frac{eE({y'}_0 - y_0)\omega_0^2}{\hbar\omega(\omega_0^2-\omega^2)}\bigg]
      \chi_n^2 (y_i - {y'}_0)
      \chi_n^2 (y_i - {y}_0)
    \bigg]^{1/2}.
\end{equation}
Assuming $L_x = L_y$ we can define the area of the 2D material as
\begin{equation} \label{2.44}
    S \equiv L_xL_x = L_xL_y
\end{equation}
and then we can re-write the above as
\begin{equation} \label{2.45}
    \frac{1}{\tau} =
    \bigg[
    \frac{eBN_sU_0^2}{\pi\hbar^3 S}
    \int d{y'}_0
    \int dy_i
    J_0^2\bigg[\frac{eE({y'}_0 - y_0)\omega_0^2}{\hbar\omega(\omega_0^2-\omega^2)}\bigg]
      \chi_n^2 (y_i - {y'}_0)
      \chi_n^2 (y_i - {y}_0)
    \bigg]^{1/2}.
\end{equation}
Define the \textit{density of scatters} per unit area of 2DEG
\begin{equation} \label{2.46}
    n_s \equiv \frac{N_s}{S}
\end{equation}
and the \textit{magnetic length} as
\begin{equation} \label{2.47}
    l_0 \equiv \sqrt{\frac{\hbar}{eB}}.
\end{equation}
Now our Eq. \eqref{2.45} leads to
\begin{equation} \label{2.48}
    \frac{1}{\tau} =
    \sqrt{
    \frac{n_sU_0^2}{\pi l_0^2 \hbar^2}}
    \bigg[
    \int d{y'}_0
    \int dy_i
    J_0^2\bigg[\frac{eE({y'}_0 - y_0)\omega_0^2}{\hbar\omega(\omega_0^2-\omega^2)}\bigg]
      \chi_n^2 (y_i - {y'}_0)
      \chi_n^2 (y_i - {y}_0)
    \bigg]^{1/2}
\end{equation}
and now define new dummy variables as follows (since $y_0$ is a paramter)
\begin{equation} \label{2.49}
    ({y'}_0 - y_0) \rightarrow  y  \quad \text{and} \quad
    ({y}_i- {y'}_0) \rightarrow y'
\end{equation}
and finally we will get the eqution for the dressed electron lifetime at the $n$th Landau level as
\begin{equation} \label{2.50}
    \frac{1}{\tau} =
    \sqrt{
    \frac{n_sU_0^2}{\pi l_0^2 \hbar^2}}
    \bigg[
    \int \int  dy dy'\;
    J_0^2\bigg[\frac{eEy\omega_0^2}{\hbar\omega(\omega_0^2-\omega^2)}\bigg]
      \chi_n^2 (y')
      \chi_n^2 (y+y')
    \bigg]^{1/2}
\end{equation}
\hfill$\blacksquare$
