\section{Inverse Scattering Time Analysis}

We hace derived the inverse scattering time matrix element from previous section as follows
\begin{equation} \label{5.1}
  \begin{aligned}
    \qty(\frac{1}{\tau(\varepsilon,k_x)})^{ll'}_{N} =
    \frac { N_{imp}^2 A^2 \hbar V_{imp}}{16\pi^4 \qty(eB)^2}
    \delta(\varepsilon - \varepsilon_{N})
    \int_{-\infty}^{\infty} d {k'}_x
    &
    J_l\qty(\frac{g\hbar}{eB}[{k}_x - {k'}_x])
    J_{l'}\qty(\frac{g\hbar}{eB}[{k}_x - {k'}_x]) \\
    & \times
    \qty|
    \int_{-\infty}^{\infty} d\bar{k} \;
    {\chi}_{N}\qty(\frac{\hbar}{eB}\bar{k})
    {\chi}_{N}\qty(\frac{\hbar}{eB} \qty[{k'}_x - {k}_x - \bar{k}])|^2.
  \end{aligned}
\end{equation}

\noindent
The disorder in the system is not supposed to change the eigenenergies of the bare system, hence all off0diogonal elements of the self-energy were neglected. Thesefore we can consider only the diagonal elements of the inverse scattering time matrix
\begin{equation} \label{5.2}
  \begin{aligned}
    \qty(\frac{1}{\tau(\varepsilon,k_x)})^{ll}_{N} =
    \frac { N_{imp}^2 A^2 \hbar V_{imp}}{16\pi^4 \qty(eB)^2}
    \delta(\varepsilon - \varepsilon_{N}) &
    \int_{-\infty}^{\infty} d {k'}_x \;
    J_l^2\qty(\frac{g\hbar}{eB}[{k}_x - {k'}_x])
    \\
    & \times
    \qty|
    \int_{-\infty}^{\infty} d\bar{k} \;
    {\chi}_{N}\qty(\frac{\hbar}{eB}\bar{k})
    {\chi}_{N}\qty(\frac{\hbar}{eB} \qty[{k'}_x - {k}_x - \bar{k}])|^2.
  \end{aligned}
\end{equation}
% and introduce new variable as follows
% \begin{equation} \label{5.3}
%     y_1 = \frac{\hbar}{eB}[{k'}_x - k_x] \longrightarrow
%     d{k'}_x = \frac{eB}{\hbar}dy_1
% \end{equation}
% and
% \begin{equation} \label{5.4}
%     y_2 = \frac{\hbar}{eB}\bar{k} \longrightarrow
%     d\bar{k} = \frac{eB}{\hbar}dy_2
% \end{equation}
% then above equation modified to
% \begin{equation} \label{5.5}
%   \begin{aligned}
%     \qty(\frac{1}{\tau(\varepsilon,k_x)})^{ll}_{N} =
%     \frac { N_{imp}^2 A^2 \hbar V_{imp}}{16\pi^4 \qty(eB)^2}
%     \qty(\frac{eB}{\hbar})^3
%     \delta(\varepsilon - \varepsilon_{N}) &
%     \int_{-\infty}^{\infty} dy_1\;
%     J_l^2\qty(\frac{g\hbar}{eB}[{k}_x - {k'}_x])
%     \\
%     & \times
%     \qty|
%     \int_{-\infty}^{\infty} d\bar{k} \;
%     {\chi}_{n_{\beta}}\qty(\frac{\hbar}{eB}\bar{k})
%   \end{aligned}
% \end{equation}

\noindent
Lets consider how this expression change when we have turn off the dressing field ($E = 0$). Thesefore the inverse scattering time becomes valid for only $l=0$
\begin{equation} \label{5.3}
  \begin{aligned}
    \qty(\frac{1}{\tau(\varepsilon,k_x)})^{00}_{N}\Bigg|_{E=0}  =
    \frac { N_{imp}^2 A^2 \hbar V_{imp}}{16\pi^4 \qty(eB)^2}
    \delta(\varepsilon - \varepsilon_{N})
    \int_{-\infty}^{\infty} d {k'}_x \;
    \qty|
    \int_{-\infty}^{\infty} d\bar{k} \;
    {\chi}_{N}\qty(\frac{\hbar}{eB}\bar{k})
    {\chi}_{N}\qty(\frac{\hbar}{eB} \qty[{k'}_x - {k}_x - \bar{k}])|^2.
  \end{aligned}
\end{equation}

\noindent
Thesefore we can analyze the behaviour of the inverse scattering time with
$l=0$ central element of the matrix.
\begin{equation} \label{5.4}
    \Lambda_{00} \equiv
    \frac{\qty(1/\tau)^{00}_{N}}{\qty(1/\tau)^{00}_{N}\big|_{E=0}}
\end{equation}
and this will be
\begin{equation} \label{5.5}
    \Lambda_{00}(k_x) =
    \frac
    {\int_{-\infty}^{\infty} d {k'}_x \;
    J_l^2\qty(g\gamma[{k}_x - {k'}_x])
    \qty|
    \int_{-\infty}^{\infty} d\bar{k} \;
    {\chi}_{N}\qty(\gamma\bar{k})
    {\chi}_{N}\qty(\gamma\qty[{k'}_x - {k}_x - \bar{k}])|^2}
    {\int_{-\infty}^{\infty} d {k'}_x \;
    \qty|
    \int_{-\infty}^{\infty} d\bar{k} \;
    {\chi}_{N}\qty(\gamma\bar{k})
    {\chi}_{N}\qty(\gamma \qty[{k'}_x - {k}_x - \bar{k}])|^2}
\end{equation}
where
\begin{equation} \label{5.6}
    g = \frac{eE\omega_0^2}{\hbar\omega(\omega_0^2 - \omega^2)} \quad\quad
    \gamma = \frac{\hbar}{eB}
\end{equation}
and
\begin{equation} \label{5.7}
  \chi_N(y) = \frac{\sqrt{\kappa}}{\sqrt{2^{N}N!\sqrt{\pi}}}
  \exp(-\frac{\kappa^2 y^2}{2})
  \mathcal{H}_N\qty(\kappa y) \quad \text{with} \quad
  \kappa \equiv \sqrt{\frac{m_e \omega_0}{\hbar}}.
\end{equation}

\noindent
Lets calculate these constants for GaAs-based quantum well with following given system external paramters and physical constants.
\begin{table}[ht!]
\begin{center}
\begin{tabular}{ |l|c|l| }
 \hline
 \textbf{External paramter name} & \textbf{Symbol} & \textbf{Value in SI-units} \\ [0.5ex] \hline\hline
 Average intensity  & $I$ & $200\;\text{W}/\text{cm}^{2} = 2\times10^6\;\text{W}/\text{m}^{2}$ \\ \hline
 Magnetic field & $B$ & $1.2\;\text{T}$ \\ \hline
 Driving frequency & $\omega$ & $2\times10^{12} \;\text{rad}\text{s}^{-1}$ \\ \hline
 Effective mass &  $m_e$ & $0.071\times m = 6.467 \times10^{-32} \; \text{kg}$ \\ \hline
\end{tabular}
\caption {\label{tab:5.1}System external paramter values}
\end{center}
\end{table}

\begin{table}[ht!]
\begin{center}
\begin{tabular}{ |l|c|l| }
 \hline
 \textbf{Physical constant name} & \textbf{Symbol} & \textbf{Value in SI-units} \\ [0.5ex] \hline\hline
 Electron charge  & e & $1.602\times10^{-19} \;\text{C}$ \\ \hline
 Electron mass & $m$ & $9.109\times10^{-31}\; \text{kg}$ \\ \hline
 Reduced Planck's constant &  $\hbar$ & $1.054\times10^{-34} \;\text{kg}\text{m}^2\text{s}^{-1}$ \\ \hline
 Speed of light & $c$ & $2.998\times10^8\; \text{ms}^{-1}$ \\ \hline
 Vacuum permittivity & $\varepsilon_0$ & $8.854\times10^{-12}\; \text{C}^{2}\text{s}^2\text{kg}^{-1}\text{m}^{-3}$ \\ \hline
\end{tabular}
\caption {\label{tab:5.2}Physical constant values in SI-units}
\end{center}
\end{table}

\noindent
Therfore we can calculate following values
\begin{equation} \label{5.8}
  \omega_0 = \frac{eB}{m_e} = 2.97265 \times 10^{12}\; \text{s}^{-1}
\end{equation}
\begin{equation} \label{5.9}
  \gamma = \frac{\hbar}{eB} = 5.4851 \times 10^{-16}\; \text{m}^{2}
\end{equation}
\begin{equation} \label{5.10}
  g = \frac{eE\omega_0^2}{\hbar\omega(\omega_0^2 - \omega^2)}
  = 5.38756 \times 10^{7}\; \text{m}^{-1}
\end{equation}
\begin{equation} \label{5.11}
  \kappa = \sqrt{\frac{m_e \omega_0}{\hbar}}
  = 4.2698 \times 10^{7}\; \text{m}^{-1}
\end{equation}
Since
\begin{equation} \label{5.12}
  g\gamma
  = 2.95513 \times 10^{-8}\; \text{m} \quad \text{and} \quad
  \kappa\gamma = 2.34202 \times 10^{-8}\; \text{m}
\end{equation}
we can choose our integral dummy variables ${k'}_x$, $\bar{k}$ and momentum variable $k_x$ are in one range as follows
\begin{equation} \label{5.13}
  {k'}_x,\bar{k},k_x \approx 10^{-8}\; \text{m}^{-1}
\end{equation}


















xx
