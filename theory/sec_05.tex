\section{Inverse Scattering Time Analysis}

We hace derived the inverse scattering time matrix element from previous section as follows
\begin{equation} \label{5.1}
  \begin{aligned}
    \qty(\frac{1}{\tau(\varepsilon,k_x)})^{ll'}_{N} =
    \frac { N_{imp}^2 A^2 \hbar V_{imp}}{16\pi^4 \qty(eB)^2}
    \delta(\varepsilon - \varepsilon_{N})
    \int_{-\infty}^{\infty} d {k'}_x
    &
    J_l\qty(\frac{g\hbar}{eB}[{k}_x - {k'}_x])
    J_{l'}\qty(\frac{g\hbar}{eB}[{k}_x - {k'}_x]) \\
    & \times
    \qty|
    \int_{-\infty}^{\infty} d\bar{k} \;
    {\chi}_{N}\qty(\frac{\hbar}{eB}\bar{k})
    {\chi}_{N}\qty(\frac{\hbar}{eB} \qty[{k'}_x - {k}_x - \bar{k}])|^2.
  \end{aligned}
\end{equation}

\noindent
The disorder in the system is not supposed to change the eigenenergies of the bare system, hence all off-diogonal elements of the self-energy were neglected. Thesefore we can consider only the central diagonal element (${l=l'=0}$) of the inverse scattering time matrix which has the largest contribution
\begin{equation} \label{5.2}
  \begin{aligned}
    \qty(\frac{1}{\tau(\varepsilon,k_x)})^{00}_{N} =
    \frac { N_{imp}^2 A^2 \hbar V_{imp}}{16\pi^4 \qty(eB)^2}
    \delta(\varepsilon - \varepsilon_{N}) &
    \int_{-\infty}^{\infty} d {k'}_x \;
    J_l^2\qty(\frac{g\hbar}{eB}[{k}_x - {k'}_x])
    \\
    & \times
    \qty|
    \int_{-\infty}^{\infty} d\bar{k} \;
    {\chi}_{N}\qty(\frac{\hbar}{eB}\bar{k})
    {\chi}_{N}\qty(\frac{\hbar}{eB} \qty[{k'}_x - {k}_x - \bar{k}])|^2.
  \end{aligned}
\end{equation}

% and introduce new variable as follows
% \begin{equation} \label{5.3}
%     y_1 = \frac{\hbar}{eB}[{k'}_x - k_x] \longrightarrow
%     d{k'}_x = \frac{eB}{\hbar}dy_1
% \end{equation}
% and
% \begin{equation} \label{5.4}
%     y_2 = \frac{\hbar}{eB}\bar{k} \longrightarrow
%     d\bar{k} = \frac{eB}{\hbar}dy_2
% \end{equation}
% then above equation modified to
% \begin{equation} \label{5.5}
%   \begin{aligned}
%     \qty(\frac{1}{\tau(\varepsilon,k_x)})^{ll}_{N} =
%     \frac { N_{imp}^2 A^2 \hbar V_{imp}}{16\pi^4 \qty(eB)^2}
%     \qty(\frac{eB}{\hbar})^3
%     \delta(\varepsilon - \varepsilon_{N}) &
%     \int_{-\infty}^{\infty} dy_1\;
%     J_l^2\qty(\frac{g\hbar}{eB}[{k}_x - {k'}_x])
%     \\
%     & \times
%     \qty|
%     \int_{-\infty}^{\infty} d\bar{k} \;
%     {\chi}_{n_{\beta}}\qty(\frac{\hbar}{eB}\bar{k})
%   \end{aligned}
% \end{equation}

% \noindent
% Lets consider how this expression change when we have turn off the dressing field ($E = 0$). Thesefore the inverse scattering time becomes valid for only $l=0$
% \begin{equation} \label{5.3}
%   \begin{aligned}
%     \qty(\frac{1}{\tau(\varepsilon,k_x)})^{00}_{N}\Bigg|_{E=0}  =
%     \frac { N_{imp}^2 A^2 \hbar V_{imp}}{16\pi^4 \qty(eB)^2}
%     \delta(\varepsilon - \varepsilon_{N})
%     \int_{-\infty}^{\infty} d {k'}_x \;
%     \qty|
%     \int_{-\infty}^{\infty} d\bar{k} \;
%     {\chi}_{N}\qty(\frac{\hbar}{eB}\bar{k})
%     {\chi}_{N}\qty(\frac{\hbar}{eB} \qty[{k'}_x - {k}_x - \bar{k}])|^2.
%   \end{aligned}
% \end{equation}

\noindent
Now we can introduce a new parameter with physical meaning of scatterin-induced broading of the Landau level as follows
\begin{equation} \label{5.3}
  \Gamma^{00}_{N}(\varepsilon,k_x) \equiv \hbar \qty(\frac{1}{\tau(\varepsilon,k_x)})^{00}_{N}
\end{equation}
and this modify our previous expressing as
\begin{equation} \label{5.4}
  \begin{aligned}
    \Gamma^{00}_{N}(\varepsilon,k_x)  =
    \frac { N_{imp}^2 A^2 \hbar^2 V_{imp}}{16\pi^4 \qty(eB)^2}
    \delta(\varepsilon - \varepsilon_{N}) &
    \int_{-\infty}^{\infty} d {k'}_x \;
    J_0^2\qty(\frac{g\hbar}{eB}[{k}_x - {k'}_x])
    \\
    & \times
    \qty|
    \int_{-\infty}^{\infty} d\bar{k} \;
    {\chi}_{N}\qty(\frac{\hbar}{eB}\bar{k})
    {\chi}_{N}\qty(\frac{\hbar}{eB} \qty[{k'}_x - {k}_x - \bar{k}])|^2.
  \end{aligned}
\end{equation}
In addition, for the case of elastic scattering within the same Landau level, one can present the delta distribution of the energy using the same physical interpretation as follows
\begin{equation} \label{5.5}
  \delta(\varepsilon - \varepsilon_{N}) \approx
  \frac{1}{\pi \Gamma^{00}_{N}(\varepsilon,k_x)}
\end{equation}
and this leads to
\begin{equation} \label{5.6}
  \begin{aligned}
    \qty[\Gamma^{00}_{N}(\varepsilon,k_x)]^2  =
    \frac { N_{imp}^2 A^2 \hbar^2 V_{imp}}{16\pi^5 \qty(eB)^2}
    \int_{-\infty}^{\infty} d {k'}_x \;
    J_0^2\qty(\frac{g\hbar}{eB}[{k}_x - {k'}_x])
    \qty|
    \int_{-\infty}^{\infty} d\bar{k} \;
    {\chi}_{N}\qty(\frac{\hbar}{eB}\bar{k})
    {\chi}_{N}\qty(\frac{\hbar}{eB} \qty[{k'}_x - {k}_x - \bar{k}])|^2.
  \end{aligned}
\end{equation}
and
\begin{equation} \label{5.7}
  \begin{aligned}
    \Gamma^{00}_{N}(\varepsilon,k_x)  =
    \qty[
    \frac { N_{imp}^2 A^2 \hbar^2 V_{imp}}{16\pi^5 \qty(eB)^2}
    \int_{-\infty}^{\infty} d {k'}_x \;
    J_0^2\qty(\frac{g\hbar}{eB}[{k}_x - {k'}_x])
    \qty|
    \int_{-\infty}^{\infty} d\bar{k} \;
    {\chi}_{N}\qty(\frac{\hbar}{eB}\bar{k})
    {\chi}_{N}\qty(\frac{\hbar}{eB} \qty[{k'}_x - {k}_x - \bar{k}])|^2
    ]^{-1/2}.
  \end{aligned}
\end{equation}

\noindent
This can be write in more compact form as follows
\begin{equation} \label{5.8}
  \begin{aligned}
    \Gamma^{00}_{N}(\varepsilon,k_x)  =
    \qty[
    \frac { N_{imp}^2 A^2 \hbar^2 V_{imp}}{16\pi^5 \qty(eB)^2}
    \int_{-\infty}^{\infty} d {k'}_x \;
    J_0^2\qty(g\sigma[{k}_x - {k'}_x])
    \qty|
    \int_{-\infty}^{\infty} d\bar{k} \;
    {\chi}_{N}\qty(\sigma\bar{k})
    {\chi}_{N}\qty(\sigma\qty[{k'}_x - {k}_x - \bar{k}])|^2
    ]^{-1/2}
  \end{aligned}
\end{equation}
where
\begin{equation} \label{5.9}
    g = \frac{eE\omega_0^2}{\hbar\omega(\omega_0^2 - \omega^2)} \quad\quad
    \sigma = \frac{\hbar}{eB}
\end{equation}
and
\begin{equation} \label{5.10}
  \chi_N(x) = \frac{\sqrt{\kappa}}{\sqrt{2^{N}N!\sqrt{\pi}}}
  \exp(-\frac{\kappa^2 x^2}{2})
  \mathcal{H}_N\qty(\kappa x) \quad \text{with} \quad
  \kappa \equiv \sqrt{\frac{m_e \omega_0}{\hbar}}.
\end{equation}

\noindent
Using above definition we can identify Gauss-Hermite functions ($\tilde{\chi}_N$) and this can re-write as
\begin{equation} \label{5.11}
  \begin{aligned}
    \Gamma^{00}_{N}(\varepsilon,k_x)  =
    \qty[
    \frac { N_{imp}^2 A^2 \hbar^2 V_{imp} \kappa^4}{16\pi^5 \qty(eB)^2}
    \int_{-\infty}^{\infty} d {k'}_x \;
    J_0^2\qty(g\sigma[{k}_x - {k'}_x])
    \qty|
    \int_{-\infty}^{\infty} d\bar{k} \;
    \tilde{\chi}_{N}\qty(\sigma\kappa\bar{k})
    \tilde{\chi}_{N}\qty(\sigma\kappa\qty[{k'}_x - {k}_x - \bar{k}])|^2
    ]^{-1/2}
  \end{aligned}
\end{equation}
where
\begin{equation} \label{5.12}
  \tilde{\chi}_N(x) = \frac{1}{\sqrt{2^{N}N!\sqrt{\pi}}}
  \exp(-\frac{x^2}{2})
  \mathcal{H}_N\qty(x)
\end{equation}
and this will be simplified to
\begin{equation} \label{5.13}
  \begin{aligned}
    \Gamma^{00}_{N}(\varepsilon,k_x)  =
    \eta
    \qty[
    \int_{-\infty}^{\infty} d {k}_1 \;
    J_0^2\qty(\lambda_1[{k}_x - {k}_1])
    \qty|
    \int_{-\infty}^{\infty} d{k}_2 \;
    \tilde{\chi}_{N}\qty(\lambda_2 k_2)
    \tilde{\chi}_{N}\qty(\lambda_2 \qty[{k}_1 - {k}_2 - {k}_x])|^2
    ]^{-1/2}
  \end{aligned}
\end{equation}
where
\begin{equation} \label{5.14}
    \eta = \qty[\frac { N_{imp}^2 A^2 V_{imp}}{16\pi^5}]^{1/2} \quad,\quad
    \lambda_1 = g\sigma \quad,\quad
    \lambda_2 = \sigma \kappa.
\end{equation}
\noindent
Now we can analyze the behaviour of the normalized $N$-th Landau level for broading as follows
\begin{equation} \label{5.15}
    \Lambda_N(k_x) \equiv
    \frac{\qty(1/\tau)^{00}_{N}}{\qty(1/\tau)^{00}_{0}\big|_{E=0}} =
    \frac{\Gamma^{00}_{N}(\varepsilon,k_x)}{\Gamma^{00}_{0}(\varepsilon,k_x)\big|_{E=0}}
\end{equation}
and this will be
\begin{equation} \label{5.16}
    \Lambda_N (k_x) =
    \qty[
    \frac
    {\int_{-\infty}^{\infty} d {k}_1 \;
    J_0^2\qty(\lambda_1[{k}_x - {k}_1])
    \qty|
    \int_{-\infty}^{\infty} d{k}_2 \;
    \tilde{\chi}_{N}\qty(\lambda_2 k_2)
    \tilde{\chi}_{N}\qty(\lambda_2 \qty[{k}_1 - {k}_2 - {k}_x])|^2}
    {\int_{-\infty}^{\infty} d {k}_1 \;
    \qty|
    \int_{-\infty}^{\infty} d{k}_2 \;
    \tilde{\chi}_{0}\qty(\lambda_2 k_2)
    \tilde{\chi}_{0}\qty(\lambda_2 \qty[{k}_1 - {k}_2 - {k}_x])|^2}
    ]^{1/2}.
\end{equation}

\noindent
Lets calculate these constants for GaAs-based quantum well with following given physical constants and system external paramters.

\begin{table}[ht!]
\begin{center}
\begin{tabular}{ |l|c|l| }
 \hline
 \textbf{Physical constant name} & \textbf{Symbol} & \textbf{Value in SI-units} \\ [0.5ex] \hline\hline
 Electron charge  & e & $1.602\times10^{-19} \;\text{C}$ \\ \hline
 Electron mass & $m$ & $9.109\times10^{-31}\; \text{kg}$ \\ \hline
 Reduced Planck's constant &  $\hbar$ & $1.054\times10^{-34} \;\text{kg}\text{m}^2\text{s}^{-1}$ \\ \hline
 Speed of light & $c$ & $2.998\times10^8\; \text{ms}^{-1}$ \\ \hline
 Vacuum permittivity & $\varepsilon_0$ & $8.854\times10^{-12}\; \text{C}^{2}\text{s}^2\text{kg}^{-1}\text{m}^{-3}$ \\ \hline
\end{tabular}
\caption {\label{tab:5.1}Physical constant values in SI-units}
\end{center}
\end{table}

\begin{table}[ht!]
\begin{center}
\begin{tabular}{ |l|c|l| }
 \hline
 \textbf{External paramter name} & \textbf{Symbol} & \textbf{Value in SI-units} \\ [0.5ex] \hline\hline
 Average intensity  & $I$ & $\tilde{I} \times 100\;\text{W}/\text{cm}^{2} = \tilde{I}\times10^6\;\text{W}/\text{m}^{2}$ \\ \hline
 Magnetic field & $B$ & $1.2\;\text{T}$ \\ \hline
 Driving frequency & $\omega$ & $2\times10^{12} \;\text{rad}\text{s}^{-1}$ \\ \hline
 Effective mass &  $m_e$ & $0.071\times m = 6.467 \times10^{-32} \; \text{kg}$ \\ \hline
\end{tabular}
\caption {\label{tab:5.2}System external paramter values. (${\tilde{I}}$ is a dimentionless value.)}
\end{center}
\end{table}

\noindent
Therfore we can calculate following values
\begin{equation} \label{5.17}
  \omega_0 = \frac{eB}{m_e} = 2.97265 \times 10^{12}\; \text{s}^{-1}
\end{equation}
\begin{equation} \label{5.18}
  \sigma = \frac{\hbar}{eB} = 5.4851 \times 10^{-16}\; \text{m}^{2}
\end{equation}
\begin{equation} \label{5.19}
  E = \sqrt{\frac{2I}{c\varepsilon_0}}
\end{equation}
\begin{equation} \label{5.20}
  g = \frac{eE\omega_0^2}{\hbar\omega(\omega_0^2 - \omega^2)}
  = \frac{e\omega_0^2}{\hbar\omega(\omega_0^2 - \omega^2)}  \sqrt{\frac{2I}{c\varepsilon_0}}
  = 3.80958 \times 10^{7}\times \sqrt{\tilde{I}}\; \text{m}^{-1}
\end{equation}
\begin{equation} \label{5.21}
  \kappa = \sqrt{\frac{m_e \omega_0}{\hbar}}
  = 4.2698 \times 10^{7}\; \text{m}^{-1}
\end{equation}
Since
\begin{equation} \label{5.22}
  \lambda_1 = g\sigma
  = 2.08959 \times 10^{-8} \times \sqrt{\tilde{I}}\; \text{m} \quad \text{and} \quad
  \lambda_2 = \kappa\sigma = 2.34203 \times 10^{-8}\; \text{m}
\end{equation}
we can choose our integral dummy variables $k_1$, $k_2$ and momentum variable $k_x$ are in one range as follows
\begin{equation} \label{5.23}
  k_x,k_1,k_2 \approx 10^{-8}\; \text{m}^{-1}
\end{equation}

\noindent
Using above values we can re-write the normalized energy broading of the $N$-th Landau level as
\begin{equation} \label{5.24}
    \Lambda_N (k_x) =
    \qty[
    \frac
    {\int_{-\infty}^{\infty} d {k}_1 \;
    J_0^2\qty(2.090\sqrt{\tilde{I}}\times[{k}_x - {k}_1])
    \qty|
    \int_{-\infty}^{\infty} d{k}_2 \;
    \tilde{\chi}_{N}\qty(2.342 \times k_2)
    \tilde{\chi}_{N}\qty(2.342 \times \qty[{k}_1 - {k}_2 - {k}_x])|^2}
    {\int_{-\infty}^{\infty} d {k}_1 \;
    \qty|
    \int_{-\infty}^{\infty} d{k}_2 \;
    \tilde{\chi}_{0}\qty(2.342 \times k_2)
    \tilde{\chi}_{0}\qty(2.342 \times \qty[{k}_1 - {k}_2 - {k}_x])|^2}
    ]^{1/2}.
\end{equation}

\noindent
To check the variability of this expression with $k_x$ value we check it with a constant intensity. Therefore let $\tilde{I}=1$ and we can graph the $\Lambda_N (k_x)$ against $k_x$ for different Landau levels ($N$) using following equation.
\begin{equation} \label{5.25}
    \Lambda_N (k_x) =
    \qty[
    \frac
    {\int_{-\infty}^{\infty} d {k}_1 \;
    J_0^2\qty(2.090\times[{k}_x - {k}_1])
    \qty|
    \int_{-\infty}^{\infty} d{k}_2 \;
    \tilde{\chi}_{N}\qty(2.342 \times k_2)
    \tilde{\chi}_{N}\qty(2.342 \times \qty[{k}_1 - {k}_2 - {k}_x])|^2}
    {\int_{-\infty}^{\infty} d {k}_1 \;
    \qty|
    \int_{-\infty}^{\infty} d{k}_2 \;
    \tilde{\chi}_{0}\qty(2.342 \times k_2)
    \tilde{\chi}_{0}\qty(2.342 \times \qty[{k}_1 - {k}_2 - {k}_x])|^2}
    ]^{1/2}.
\end{equation}















xx
