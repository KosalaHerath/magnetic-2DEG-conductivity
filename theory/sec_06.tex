\section{Current Operator in Landau Levels}

Now consider about the current density operator for $N$th Landau level. Since we have already found the extact solution for our time depenedent Hamiltonian and we have identify them as Floquet states with quesi energies. From these solutions we can identify the \textit{Floquet modes} as given in Eq. \eqref{3.15} and using quantum numbers we can represent those states as follows
\begin{equation} \label{6.1}
  \ket{\phi_{\alpha}} = \ket{\phi_{n,m}} \equiv \ket{n,k_x} \quad
  \text{where} \quad
  k_x = m \frac{2\pi}{L_x}
\end{equation}
Using above complete set of eigenstates of Floquet Hamiltonian we can represent the single particle current operator's matrix element as
\begin{equation} \label{6.2}
  \qty(\mb{j})_{nm,n'm'} = \mel{n,k_x}{\;\hat{\mb{j}}\;}{n',k'_x}
\end{equation}
where particle current operator for this system will be
\begin{equation} \label{6.3}
  \hat{\mb{j}} = \frac{1}{m} \qty(\hat{\mb{P}} - e\qty[\mb{A}_s + \mb{A}_d(t)]).
\end{equation}
Howver, we are only consider the transverse conductivity in $x$ direction we can identify that $x$ directional current operator as
\begin{equation} \label{6.4}
  \hat{j}_x = \frac{1}{m} \qty(-i\hbar\pdv{x} + eBy).
\end{equation}
Now we can calculate the matrix elements of $x$ directional current operator's matrix in Floquet mode basis as
\begin{equation} \label{6.5}
  \qty({j}_x)_{nm,n'm'} = \mel{n,k_x}{\;\hat{j}_x\;}{n',k'_x} =
  \mel{n,k_x}{\;\frac{1}{m} \qty(-i\hbar\pdv{x} + eBy)\;}{n',k'_x}
\end{equation}
and we can evaluate these using Floquet modes derived in Eq.\eqref{3.15} as follows
\begin{equation} \label{6.6}
  \begin{aligned}
    \qty({j}_x)_{nm,n'm'} & =
    \int dx \int dy \;
    \frac{1}{\sqrt{L_x}} \chi_{n}\big(y - y_0 - \zeta(t)\big)
    \exp(-ik_x x) \\
    & \times
    \frac{1}{m} \qty(-i\hbar\pdv{x} + eBy)
    \frac{1}{\sqrt{L_x}} \chi_{n'}\big(y - y_0 - \zeta(t)\big)
    \exp(i k'_x x)
  \end{aligned}
\end{equation}
and this can be simplified as
\begin{equation} \label{6.7}
  \begin{aligned}
    \qty({j}_x)_{nm,n'm'} & =
    \frac{1}{{mL_x}}
    \int dx \exp(-i(k_x-k'_x) x)
    \int dy \;
     \chi_{n}\big(y - y_0 - \zeta(t)\big) \\
    & \times
    \qty(\hbar k'_x + eBy)
    \chi_{n'}\big(y - y_0 - \zeta(t)\big)
  \end{aligned}
\end{equation}
and
\begin{equation} \label{6.8}
  \begin{aligned}
    \qty({j}_x)_{nm,n'm'} & =
    \frac{1}{{m}}
    \delta_{k_x,k'_x}
    \int dy \;
    \qty(\hbar k'_x + eBy)
     \chi_{n}\big(y - y_0 - \zeta(t)\big)
    \chi_{n'}\big(y - y_0 - \zeta(t)\big).
  \end{aligned}
\end{equation}
Now let $y - y_0 - \zeta(t) = \bar{y}$ and we will get
\begin{equation} \label{6.9}
  \begin{aligned}
    \qty({j}_x)_{nm,n'm'} & =
    \frac{1}{{m}}
    \delta_{k_x,k'_x}
    \int d\bar{y} \;
    \qty(\hbar k'_x + eB\bar{y} + eBy_0 + eB\zeta(t))
     \chi_{n}(\bar{y})
    \chi_{n'}(\bar{y}).
  \end{aligned}
\end{equation}
using definition of $y_0$ given in Eq. \eqref{1.11} this will be modiofied to
\begin{equation} \label{6.10}
  \begin{aligned}
    \qty({j}_x)_{nm,n'm'} & =
    \frac{1}{{m}}
    \delta_{k_x,k'_x}
    \int d\bar{y} \;
    \qty(\hbar k'_x + eB\bar{y} -\hbar k'_x + eB\zeta(t))
     \chi_{n}(\bar{y})
    \chi_{n'}(\bar{y})
  \end{aligned}
\end{equation}
and using integral indentities of Gauss-Hermite functions
\begin{equation} \label{6.11}
    \int d{y} \;
    \chi_{n}({y})
    \chi_{n'}({y}) =
    \delta_{n',n}
\end{equation}
\begin{equation} \label{6.12}
    \int dy \;
    y
    \chi_{n}({y})
    \chi_{n'}({y}) =
    \qty(\sqrt{\frac{n+1}{2}} \delta_{n',n+1} + \sqrt{\frac{n}{2}}
    \delta_{n',n-1})
\end{equation}
this becomes
\begin{equation} \label{6.13}
  \begin{aligned}
    \qty({j}_x)_{nm,n'm'} & =
    \frac{1}{{m}}
    \delta_{k_x,k'_x}
    eB
    \qty[
    \qty(\sqrt{\frac{n+1}{2}} \delta_{n',n+1} + \sqrt{\frac{n}{2}}
    \delta_{n',n-1})
    + \zeta(t) \delta_{n',n}
    ]
  \end{aligned}
\end{equation}
\textcolor{red}{Due to complexity we can only consider the constant contribution and we allows only the one-cycle averaged current flow} and then we can derive the $s=0$ components of the Fourier series as
\begin{equation} \label{6.14}
  \begin{aligned}
    \qty({j}^x_{s=0})_{nm,n'm'} & =
    \frac{1}{T} \int_0^T dt \;
    \frac{1}{{m}}
    \delta_{k_x,k'_x}
    eB
    \qty[
    \qty(\sqrt{\frac{n+1}{2}} \delta_{n',n+1} + \sqrt{\frac{n}{2}}
    \delta_{n',n-1})
    + \frac{eE}{m(\omega_0^2 - \omega^2)}\sin(\omega t) \delta_{n',n}
    ]
  \end{aligned}
\end{equation}
and this can be evaluate and get
\begin{equation} \label{6.15}
  \begin{aligned}
    \qty({j}^x_{s=0})_{nm,n'm'} & =
    \frac{eB}{{m}}
    \delta_{k_x,k'_x}
    \qty(\sqrt{\frac{n+1}{2}} \delta_{n',n+1} + \sqrt{\frac{n}{2}}
    \delta_{n',n-1})
  \end{aligned}
\end{equation}

\noindent
For electric current operator we can introduce the electron's charge and effective mass
\begin{equation} \label{6.15}
  \begin{aligned}
    \qty({j}^x_{s=0})_{nm,n'm'} & =
    \frac{e^2B}{{m}}
    \delta_{k_x,k'_x}
    \qty(\sqrt{\frac{n+1}{2}} \delta_{n',n+1} + \sqrt{\frac{n}{2}}
    \delta_{n',n-1})
  \end{aligned}
\end{equation}
\hfill$\blacksquare$
