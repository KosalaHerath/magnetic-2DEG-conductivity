\section{Floquet-Drude Conductivity in Quantum Hall Systems}

The general expression for the conductivity [*Ref: Martin Wackerl Thesis 1.250] with the disorder averaging can be represent as follows
\begin{equation} \label{7.1}
  \begin{aligned}
    \lim_{\omega \to 0}
    \text{Re}[{\sigma}^{xx}(0,\omega)] &=
    \frac{-1}{4\pi\hbar A}
    \int_{\lambda-\hbar\Omega/2}^{\lambda+ \hbar\Omega/2} d\varepsilon
    \qty(
    -\frac{\partial f}{\partial \varepsilon})
    \\
    & \times
    \tr
    \qty[
    {j}^x_0
    \qty(
    \mb{G}^{r} (\varepsilon) - \mb{G}^{a} (\varepsilon)
    )
    {j}^x_0
    \qty(
    \mb{G}^{r}_0 (\varepsilon) - \mb{G}^{a}_0 (\varepsilon)
    )
    ].
  \end{aligned}
\end{equation}
where ${j}^x_0$ and $\mb{G}^{r,a} (\varepsilon)$ are $x$ directional current operator matrix and white noise disorder averaged Green function matrix respectvely defined against to the \textit{Floquet modes} of the system. Here we have assumed that only $s=0$ Fourier component of the current operator is contributing to the conductivity.

\noindent
Now this can be expand in off resonant regime ($\omega\tau_0 \gg 1$)using only central entry Fourier components ($l=l'=0$) of \textit{Floquet modes} mentioned in Eq. \eqref{6.1} as
\begin{equation} \label{7.2}
  \begin{aligned}
    \lim_{\omega \to 0}
    \text{Re}[{\sigma}^{xx}(0,\omega)] & =
    \frac{-1}{4\pi\hbar A}
    \int_{\lambda-\hbar\Omega/2}^{\lambda+ \hbar\Omega/2} d\varepsilon
    \qty(
    -\frac{\partial f}{\partial \varepsilon})
    \\
    & \times
    \frac{1}{V_{k_x}} \sum_{k_x}
    \sum_{n}
    \mel{n,k_x}{
    {j}^x_0
    \qty(
    \mb{G}^{r} (\varepsilon) - \mb{G}^{a} (\varepsilon)
    )
    {j}^x_0
    \qty(
    \mb{G}^{r}_0 (\varepsilon) - \mb{G}^{a}_0 (\varepsilon)
    )
    }
    {n,k_x}
  \end{aligned}
\end{equation}
and one can evaluate these matrix elements as follows
\begin{equation} \label{7.3}
  \begin{aligned}
    \lim_{\omega \to 0}
    \text{Re}[{\sigma}^{xx}(0,\omega)] & =
    \frac{-1}{4\pi\hbar A}
    \int_{\lambda-\hbar\Omega/2}^{\lambda+ \hbar\Omega/2} d\varepsilon
    \qty(
    -\frac{\partial f}{\partial \varepsilon})
    \frac{1}{V_{k_x}} \sum_{k_x} \sum_{n}
    \frac{1}{{L_x}^3} \sum_{{k_x}_1,{k_x}_2,{k_x}_3}
    \sum_{n_1,n_2,n_3}
    \\
    & \times
    \mel{n,k_x}{
    {j}^x_0}
    {n_1,{k_x}_1}
    \mel{{n_1,{k_x}_1}}{
    \qty(
    \mb{G}^{r} (\varepsilon) - \mb{G}^{a} (\varepsilon)
    )}
    {n_2,{k_x}_2} \\
    & \times
    \mel{n_2,{k_x}_2}{
    {j}^x_0}
    {n_3,{k_x}_3}
    \mel{n_3,{k_x}_3}{
    \qty(
    \mb{G}^{r} (\varepsilon) - \mb{G}^{a} (\varepsilon)
    )
    }
    {n,k_x}
  \end{aligned}
\end{equation}
Since we can diagonalize the impurity averaged Green's function using unitary trasnformation ($\mb{T} = \ket{n,k_x}$) [*Ref: Martin Wackerl - Paper] and we can evaluate the matrix element of differece between retarded and advanced Green's function as follows
[*Ref: My report 2.535]
\begin{equation} \label{7.4}
  \mel{{n_1,{k_x}_1}}{
  \mb{T}^{\dagger}
  \qty(
  \mb{G}^{r} (\varepsilon) - \mb{G}^{a} (\varepsilon)
  )\mb{T}}
  {n_2,{k_x}_2} =
  \qty[
  \frac{2i \text{Im}\qty(\mb{T}^{\dagger} \sum^r \mb{T})
  \delta_{n_1,n_2}\delta_{{k_x}_1,{k_x}_2}}
  {
  \qty(
  \frac{1}{\hbar}\varepsilon -
  \frac{1}{\hbar}\varepsilon_{n_1}
  )^2
  + \qty[\text{Im}\qty(\mb{T}^{\dagger} \sum^r \mb{T})]^2
  }]
\end{equation}
and
\begin{equation} \label{7.5}
  \mel{{n_3,{k_x}_3}}{
  \mb{T}^{\dagger}
  \qty(
  \mb{G}^{r} (\varepsilon) - \mb{G}^{a} (\varepsilon)
  )\mb{T}}
  {n,{k_x}} =
  \qty[
  \frac{2i \text{Im}\qty(\mb{T}^{\dagger} \sum^r \mb{T})
  \delta_{n_3,n}\delta_{{k_x}_3,{k_x}}}
  {
  \qty(
  \frac{1}{\hbar}\varepsilon -
  \frac{1}{\hbar}\varepsilon_{n}
  )^2
  + \qty[\text{Im}\qty(\mb{T}^{\dagger} \sum^r \mb{T})]^2
  }]
\end{equation}

\noindent
Then appying the results we derived in previous section \eqref{6.17} we can calculate the conductivity
\begin{equation} \label{7.6}
  \begin{aligned}
    \lim_{\omega \to 0}
    \text{Re}[{\sigma}^{xx}(0,\omega)] & =
    \frac{-1}{4\pi\hbar A}
    \int_{\lambda-\hbar\Omega/2}^{\lambda+ \hbar\Omega/2} d\varepsilon
    \qty(
    -\frac{\partial f}{\partial \varepsilon})
    \frac{1}{V_{k_x}} \sum_{k_x} \sum_{n}
    \frac{1}{{V_{k_x}}^3} \sum_{{k_x}_1,{k_x}_2,{k_x}_3}
    \sum_{n_1,n_2,n_3}
    \\
    & \times
    \frac{e^2B}{{m_e}}
    \delta_{k_x,{k_x}_1}
    \qty(\sqrt{\frac{n+1}{2}} \delta_{n_1,n+1} + \sqrt{\frac{n}{2}}
    \delta_{n_1,n-1})
    \qty[
    \frac{2i \text{Im}\qty(\mb{T}^{\dagger} \sum^r \mb{T})
    \delta_{n_1,n_2}\delta_{{k_x}_1,{k_x}_2}}
    {
    \qty(
    \frac{1}{\hbar}\varepsilon -
    \frac{1}{\hbar}\varepsilon_{n_1}
    )^2
    + \qty[\text{Im}\qty(\mb{T}^{\dagger} \sum^r \mb{T})]^2
    }] \\
    & \times
    \frac{e^2B}{{m_e}}
    \delta_{{k_x}_2,{k_x}_3}
    \qty(\sqrt{\frac{n_2+1}{2}} \delta_{n_3,n_2+1} + \sqrt{\frac{n_2}{2}}
    \delta_{n_3,n_2-1})
    \qty[
    \frac{2i \text{Im}\qty(\mb{T}^{\dagger} \sum^r \mb{T})
    \delta_{n_3,n}\delta_{{k_x}_3,{k_x}}}
    {
    \qty(
    \frac{1}{\hbar}\varepsilon -
    \frac{1}{\hbar}\varepsilon_{n}
    )^2
    + \qty[\text{Im}\qty(\mb{T}^{\dagger} \sum^r \mb{T})]^2
    }]
  \end{aligned}
\end{equation}
and this will be modified to
\begin{equation} \label{7.7}
  \begin{aligned}
    \lim_{\omega \to 0}
    \text{Re}[{\sigma}^{xx}(0,\omega)] & =
    \frac{-1}{4\pi\hbar A}
    \int_{\lambda-\hbar\Omega/2}^{\lambda+ \hbar\Omega/2} d\varepsilon
    \qty(
    -\frac{\partial f}{\partial \varepsilon})
    \frac{1}{V_{k_x}} \sum_{k_x} \sum_{n}
    \sum_{n_1,n_2}
    \\
    & \times
    \frac{e^2B}{{m_e}}
    \qty(\sqrt{\frac{n+1}{2}} \delta_{n_1,n+1} + \sqrt{\frac{n}{2}}
    \delta_{n_1,n-1})
    \qty[
    \frac{2i \text{Im}\qty(\mb{T}^{\dagger} \sum^r \mb{T})
    \delta_{n_1,n_2}}
    {
    \qty(
    \frac{1}{\hbar}\varepsilon -
    \frac{1}{\hbar}\varepsilon_{n_1}
    )^2
    + \qty[\text{Im}\qty(\mb{T}^{\dagger} \sum^r \mb{T})]^2
    }] \\
    & \times
    \frac{e^2B}{{m_e}}
    \qty(\sqrt{\frac{n_2+1}{2}} \delta_{n,n_2+1} + \sqrt{\frac{n_2}{2}}
    \delta_{n,n_2-1})
    \qty[
    \frac{2i \text{Im}\qty(\mb{T}^{\dagger} \sum^r \mb{T})
    }
    {
    \qty(
    \frac{1}{\hbar}\varepsilon -
    \frac{1}{\hbar}\varepsilon_{n}
    )^2
    + \qty[\text{Im}\qty(\mb{T}^{\dagger} \sum^r \mb{T})]^2
    }]
  \end{aligned}
\end{equation}
and the only non-zero term would be
\begin{equation} \label{7.8}
  \begin{aligned}
    \lim_{\omega \to 0}
    \text{Re}[{\sigma}^{xx}(0,\omega)] & =
    \frac{-1}{4\pi\hbar A}
    \frac{e^4B^2}{{{m_e}^2}}
    \int_{\lambda-\hbar\Omega/2}^{\lambda+ \hbar\Omega/2} d\varepsilon
    \qty(
    -\frac{\partial f}{\partial \varepsilon})
    \frac{1}{V_{k_x}} \sum_{k_x} \sum_{n}
    \qty(n+1)
    \\
    & \times
    \qty[
    \frac{2i \text{Im}\qty(\mb{T}^{\dagger} \sum^r \mb{T})_{\varepsilon_{n+1}}
    }
    {
    \qty(
    \frac{1}{\hbar}\varepsilon -
    \frac{1}{\hbar}\varepsilon_{n+1}
    )^2
    + \qty[\text{Im}\qty(\mb{T}^{\dagger} \sum^r \mb{T})_{\varepsilon_{n+1}}]^2
    }]
    \qty[
    \frac{2i \text{Im}\qty(\mb{T}^{\dagger} \sum^r \mb{T})_{\varepsilon_{n}}
    }
    {
    \qty(
    \frac{1}{\hbar}\varepsilon -
    \frac{1}{\hbar}\varepsilon_{n}
    )^2
    + \qty[\text{Im}\qty(\mb{T}^{\dagger} \sum^r \mb{T})_{\varepsilon_{n}}]^2
    }]
  \end{aligned}
\end{equation}
\hfill$\blacksquare$

\noindent
Then using the following identity derived in [*Ref: My report 2.509]
\begin{equation} \label{7.9}
  \qty(\frac{1}{\tau(\varepsilon,k_x)})_{ll} =
  -2\text{Im}\qty[\qty(\mb{T}^{\dagger} {\sum}^r \mb{T})_{\varepsilon}]_{ll}
\end{equation}
using central element of the inverse scattering time matrix we can modify our result as
\begin{equation} \label{7.10}
  \begin{aligned}
    \lim_{\omega \to 0}
    \text{Re}[{\sigma}^{xx}(0,\omega)] & =
    \frac{1}{4\pi\hbar A}
    \frac{e^4B^2}{{{m_e}^2}}
    \int_{\lambda-\hbar\Omega/2}^{\lambda+ \hbar\Omega/2} d\varepsilon
    \qty(
    -\frac{\partial f}{\partial \varepsilon})
    \frac{1}{V_{k_x}} \sum_{k_x} \sum_{n}
    \qty(n+1)
    \\
    & \times
    \qty[
    \frac{\qty(\frac{1}{\tau(\varepsilon_{n+1},k_x)})
    }
    {
    \qty(
    \frac{1}{\hbar}\varepsilon -
    \frac{1}{\hbar}\varepsilon_{n+1}
    )^2
    + \qty(\frac{1}{2\tau(\varepsilon_{n+1},k_x)})^2
    }]
    \qty[
    \frac{\qty(\frac{1}{\tau(\varepsilon_{n},k_x)})
    }
    {
    \qty(
    \frac{1}{\hbar}\varepsilon -
    \frac{1}{\hbar}\varepsilon_{n}
    )^2
    + \qty(\frac{1}{2\tau(\varepsilon_{n},k_x)})^2
    }]
  \end{aligned}
\end{equation}

\noindent
We have identited that the inverse scattering time matrix's central element is not $k_x$ dependent we can get the sum over all available momentum space in $x$ direction. However by considering the condition that the center of the force of the oscillator $y_0$ must physically liw within the system $-L_y/2 < y_0 < L_y/2$, one can derive that
\begin{equation} \label{7.11}
 -\frac{m_e\omega_0 Ly}{2\hbar} \leq k_x \leq \frac{m_e\omega_0 Ly}{2\hbar}
\end{equation}
and we can derive that
\begin{equation} \label{7.12}
    \frac{1}{V_{k_x}}\sum_{k_x} = \frac{m_e\omega_0 Ly}{\hbar V_{k_x}} = 1
\end{equation}
Thefore Eq. \eqref{7.10} modified to
\begin{equation} \label{7.13}
  \begin{aligned}
    \lim_{\omega \to 0}
    \text{Re}[{\sigma}^{xx}(0,\omega)] & =
    \frac{e^2 \omega_0^2}{4\pi\hbar A}
    \int_{\lambda-\hbar\Omega/2}^{\lambda+ \hbar\Omega/2} d\varepsilon
    \qty(
    -\frac{\partial f}{\partial \varepsilon})
    \sum_{n}
    \qty(n+1)
    \\
    & \times
    \qty[
    \frac{\qty(\frac{1}{\tau(\varepsilon_{n+1})})
    }
    {
    \qty(
    \frac{1}{\hbar}\varepsilon -
    \frac{1}{\hbar}\varepsilon_{n+1}
    )^2
    + \qty(\frac{1}{2\tau(\varepsilon_{n+1})})^2
    }]
    \qty[
    \frac{\qty(\frac{1}{\tau(\varepsilon_{n})})
    }
    {
    \qty(
    \frac{1}{\hbar}\varepsilon -
    \frac{1}{\hbar}\varepsilon_{n}
    )^2
    + \qty(\frac{1}{2\tau(\varepsilon_{n})})^2
    }]
  \end{aligned}
\end{equation}

\noindent
Then using Fermi-Dirac distribution as our partical distribution function ($f$) for this system
\begin{equation} \label{7.14}
  f(\varepsilon) = \frac{1}{\qty[\exp(\varepsilon - \varepsilon_F)/k_B T]+1}
\end{equation}
where $k_B$ is Botlzmann constant, $T$ is absolute tempurature and $\varepsilon_F$ is Fermi energy of the system. Using above distribution, for extreamly low tempuratures we can appromixate that
\begin{equation} \label{7.15}
  - \pdv{f(\varepsilon)}{\varepsilon} \approx \delta(\varepsilon - \varepsilon_F)
\end{equation}
and this will mpre simplify our derivation of conductivity as
\begin{equation} \label{7.16}
  \begin{aligned}
    \lim_{\omega \to 0}
    \text{Re}[{\sigma}^{xx}(0,\omega)] & =
    \frac{e^2 \omega_0^2}{4\pi\hbar A}
    \sum_{n}
    \qty(n+1)
    \qty[
    \frac{\qty(\frac{1}{\tau(\varepsilon_{n+1})})
    }
    {
    \qty(
    \frac{1}{\hbar}\varepsilon_F -
    \frac{1}{\hbar}\varepsilon_{n+1}
    )^2
    + \qty(\frac{1}{2\tau(\varepsilon_{n+1})})^2
    }]
    \qty[
    \frac{\qty(\frac{1}{\tau(\varepsilon_{n})})
    }
    {
    \qty(
    \frac{1}{\hbar}\varepsilon_F -
    \frac{1}{\hbar}\varepsilon_{n}
    )^2
    + \qty(\frac{1}{2\tau(\varepsilon_{n})})^2
    }]
  \end{aligned}
\end{equation}

\noindent
Now introduce a new paramter with a physical meaning od scattering-induced broading of the Landau level as follows
\begin{equation} \label{6.17}
  \Gamma_n \equiv\Gamma(\varepsilon_n) \equiv \qty(\frac{\hbar }{2\tau(\varepsilon_n)})
\end{equation}
and then we can re-write Eq. \eqref{7.16} as follows
\begin{equation} \label{7.18}
  \begin{aligned}
    \lim_{\omega \to 0}
    \text{Re}[{\sigma}^{xx}(0,\omega)] & =
    \frac{e^2 (\hbar\omega_0)^2}{\pi\hbar A}
    \sum_{n}
    \qty(n+1)
    \qty[
    \frac{\Gamma(\varepsilon_{n+1})
    }
    {
    \qty(
    \varepsilon_F - \varepsilon_{n+1}
    )^2
    + \Gamma^2(\varepsilon_{n+1})
    }]
    \qty[
    \frac{\Gamma(\varepsilon_{n})
    }
    {
    \qty(
    \varepsilon_F - \varepsilon_{n}
    )^2
    + \Gamma^2(\varepsilon_{n})
    }]
  \end{aligned}
\end{equation}
\begin{equation} \label{7.19}
  \begin{aligned}
    \lim_{\omega \to 0}
    \text{Re}[{\sigma}^{xx}(0,\omega)] & =
    \frac{e^2 (\hbar\omega_0)^2}{\pi\hbar A}
    \sum_{n}
    \qty(n+1)
    \qty[
    \frac{\Gamma_{n+1}
    }
    {
    \qty(
    \varepsilon_F - \varepsilon_{n+1}
    )^2
    + \Gamma^2_{n+1}
    }]
    \qty[
    \frac{\Gamma_{n}
    }
    {
    \qty(
    \varepsilon_F - \varepsilon_{n}
    )^2
    + \Gamma^2_{n}
    }]
  \end{aligned}
\end{equation}

\noindent
Now use new dimentionless paramters
\begin{equation} \label{7.20}
  X_F \equiv \frac{\varepsilon_F}{\hbar \omega_0} -\frac{1}{2}
\end{equation}
and
\begin{equation} \label{7.21}
  \gamma_n \equiv \frac{\Gamma_n}{\hbar \omega_0}.
\end{equation}
Therefore the Eq. \eqref{7.19} leads to
\begin{equation} \label{7.22}
  \begin{aligned}
    \lim_{\omega \to 0}
    \text{Re}[{\sigma}^{xx}(0,\omega)] & =
    \frac{e^2}{\hbar}
    \frac{1}{\pi A}
    \sum_{n}
    \qty(n+1)
    \qty[
    \frac{\gamma_{n+1}
    }
    {
    \qty(
    X_F - n - 1
    )^2
    + \gamma^2_{n+1}
    }]
    \qty[
    \frac{\gamma_{n}
    }
    {
    \qty(
    X_F - n
    )^2
    + \gamma^2_{n}
    }]
  \end{aligned}
\end{equation}
and
\begin{equation} \label{7.23}
  \begin{aligned}
    \lim_{\omega \to 0}
    \text{Re}[{\sigma}^{xx}(0,\omega)] & =
    \frac{e^2}{\hbar}
    \frac{1}{\pi A}
    \sum_{n}
    \frac{\qty(n+1)}{\gamma_{n}\gamma_{n+1}}
    \qty[
      \frac{1}
      {
        1 + \qty(\frac{X_F - n -1}{\gamma_{n+1}})^2
      }
    ]
    \qty[
      \frac{1}
      {
        1 + \qty(\frac{X_F - n}{\gamma_{n}})^2
      }
    ]
  \end{aligned}
\end{equation}
\hfill$\blacksquare$

\noindent
Same as above derivation we can derive the transverse conductivity in $y$ direction by using the current operator derived in Eq. \eqref{6.27} as follows
\begin{equation} \label{7.24}
  \begin{aligned}
    \lim_{\omega \to 0}
    \text{Re}[{\sigma}^{yy}(0,\omega)] & =
    \frac{1}{4\pi\hbar A}
    \frac{e^2\hbar^2}{{m^2}}
    \int_{\lambda-\hbar\Omega/2}^{\lambda+ \hbar\Omega/2} d\varepsilon
    \qty(
    -\frac{\partial f}{\partial \varepsilon})
    \frac{1}{V_{k_x}} \sum_{k_x} \sum_{n}
    -\qty(n+1)
    \\
    & \times
    \qty[
    \frac{2i \text{Im}\qty(\mb{T}^{\dagger} \sum^r \mb{T})_{\varepsilon_{n+1}}
    }
    {
    \qty(
    \frac{1}{\hbar}\varepsilon -
    \frac{1}{\hbar}\varepsilon_{n+1}
    )^2
    + \qty[\text{Im}\qty(\mb{T}^{\dagger} \sum^r \mb{T})_{\varepsilon_{n+1}}]^2
    }]
    \qty[
    \frac{2i \text{Im}\qty(\mb{T}^{\dagger} \sum^r \mb{T})_{\varepsilon_{n}}
    }
    {
    \qty(
    \frac{1}{\hbar}\varepsilon -
    \frac{1}{\hbar}\varepsilon_{n}
    )^2
    + \qty[\text{Im}\qty(\mb{T}^{\dagger} \sum^r \mb{T})_{\varepsilon_{n}}]^2
    }]
  \end{aligned}
\end{equation}
and same as above derivation this can be simplified into
\begin{equation} \label{7.25}
  \begin{aligned}
    \lim_{\omega \to 0}
    \text{Re}[{\sigma}^{yy}(0,\omega)] & =
    \frac{e^2}{\hbar}
    \frac{1}{\pi A}
    \frac{1}{e^2B^2}
    \sum_{n}
    \frac{\qty(n+1)}{\gamma_{n}\gamma_{n+1}}
    \qty[
      \frac{1}
      {
        1 + \qty(\frac{X_F - n -1}{\gamma_{n+1}})^2
      }
    ]
    \qty[
      \frac{1}
      {
        1 + \qty(\frac{X_F - n}{\gamma_{n}})^2
      }
    ]
  \end{aligned}
\end{equation}













\hfill$\blacksquare$
